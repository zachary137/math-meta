\documentclass[11pt]{article}

\usepackage{kernel_of_truth}
\usepackage{normal_setup}

\renewcommand{\A}{\mathcal{A}}
\renewcommand{\L}{\mathbb{L}} % cotangent complex
\renewcommand{\phi}{\varphi}
\newcommand{\W}{\mathcal{W}} % saturated completion

\DeclareMathOperator{\cn}{cn} % connective
\DeclareMathOperator{\conj}{conj} % conjugate
\DeclareMathOperator{\perf}{perf} % perfection (coperfection by Kedlaya's conventions)
\DeclareMathOperator{\qrsp}{qrsp} % quasiregular semiperfect
\DeclareMathOperator{\qsyn}{qsyn} % quasisyntomic

\begin{document}
\title{More Comparison}
\author{Zachary Gardner}
\date{\texttt{zachary.gardner@bc.edu}}
\maketitle

I want to begin by unpacking the proof of a result mentioned last time. Recall that we are concerned with $p$-complete objects $A(-)\in\CAlg(D(\Fun(\CAlg_{\F_p}^{\reg},\Ab)))$ equipped with $u_0: \Omega_{(-)}\xto{\sim}\F_p\Ltensor A(-)$ an isomorphism of commutative algebra objects ``mod $p$.'' Recall as well that such $A(-)$ admits a left Kan extension $\ov{A}: \CAlg_{\F_p}^{\qsyn}\to\D_p(\Z)$ which is itself a commutative algebra object, where we have implicitly viewed $A$ as an object of the $\infty$-category $\Fun(\CAlg_{\F_p}^{\reg},\D_p(\Z))$.

\begin{lemma}
The functor $\ov{A}: \CAlg_{\F_p}^{\qsyn}\to\D_p(\Z)$ is a right Kan extension of its restriction to $\CAlg_{\F_p}^{\qrsp}$, with the latter by definition the category of quasiregular semiperfect $\F_p$-algebras.
\end{lemma}

Where does such a result come from? Certainly, this makes precise the vague idea that quasiregular semiperfect $\F_p$-algebras are relatively abundant among all quasisyntomic $\F_p$-algebras. Our key input is the following result from BMS II.

\begin{theorem}
Let $S\in\CRing$ be a base ring, $n\geq0$, and $f: B\to C$ a faithfully flat map of $S$-algebras. Then, the natural map $\Wedge^n\L_{B/S}\to\Tot(\Wedge^n\L_{C^{\bullet}/S})$ is an isomorphism in $\D(S)$.\footnote{Said another way, the functor $B\mapsto\Wedge_B^n\L_{B/S}$ is an ($\infty$-categorical) fpqc sheaf.}
\end{theorem}

What does this mean? By considering the tensor powers $C,C\tensor_BC,C\tensor_BC\tensor_BC$, etc. we may associate to $C$ a cosimplicial $S$-algebra $C^{\bullet}$ which can be precisely identified with the Cech nerve of $f: B\to C$ (\textcolor{red}{Isn't this backwards?}). From this we obtain a cosimplicial object $\Wedge^n\L_{C^{\bullet}/S}$ whose (homotopy) limit we denote by $\Tot(\Wedge^n\L_{C^{\bullet}/S})$. Let's apply this to our context of interest. Let $R\in\CAlg_{\F_p}^{\qsyn}$ and choose a set $\{x_i\}_{i\in I}$ of $\F_p$-algebra generators for $R$. To this we may associate
$$R^0:=R\tensor_{\F_p[\{x_i\}]}\F_p[\{x_i\}]_{\perf},$$
which is weakly initial in $\CAlg_R^{\qrsp}$ (the failure of uniqueness stems from the fact that we could have chosen different generators).\footnote{The notation $(\cdot)_{\perf}$ indicates taking the perfection -- i.e., the filtered colimit over the power map $(\cdot)^p$. For example, given $k$ a perfect field of characteristic $p$, we have $k[x]_{\perf}\iso k[x^{1/p^{\infty}}]$.} In the above setup, we may take $S=\F_p,B=R,C=R^0$ and form $R^{\bullet}$ from $R$-tensor powers of $R^0$ as above. Given $n\geq0$, we conclude by the theorem that the natural map $\Wedge^n\L_{R/\F_p}\to\Tot(\Wedge^n\L_{R^{\bullet}/\F_p})$ is an isomorphism in $\D(\F_p)$. By assumption we have $\Omega_{(-)}\simeq\F_p\Ltensor A(-)$ and so may upgrade this to obtain $\L\Omega_{(-)}\simeq\F_p\Ltensor\ov{A}(-)$. We want to conclude that the corresponding natural map $\ov{A}(R)\to\Tot(\ov{A}(R^{\bullet}))$ is a qis. There are several components to this.
\begin{itemize}
\item $\ov{A}(-)$ is $p$-complete.

\item Looking at the conjugate filtration we have $\gr_n(\Fil_{\bullet}^{\conj}\L\Omega_R)\simeq(\Wedge^n\L_{R^{(1)}/\F_p})[-n]$.

\item We know that $H^n(\ov{A}(R))$ vanishes for $n<0$ and is $p$-torsion-free for $n=0$.
\end{itemize}
\textcolor{red}{Why does this to establish the final property that $\ov{A}$ is supposed to have as a right Kan extension?} 

Our next item of business is establishing the following result.

\begin{proposition}
Let $\theta: \Omega_{(-)}\to\Omega_{(-)}$ be a morphism of commutative algebra objects in $\D(\Fun(\CAlg_{\F_p}^{\reg},\Vect_{\F_p}))\simeq\Fun(\CAlg_{\F_p}^{\reg},\D(\F_p))$ such that 
\begin{center}
\begin{tikzcd}
\Omega_{(-)} \arrow[rr, "\theta"] \arrow[rd, "\eps^{\dR}"] & & \Omega_{(-)} \arrow[ld, "\eps^{\dR}"] \\
& \id &
\end{tikzcd}
\end{center}
commutes. Then, $\theta=\id$.
\end{proposition}

We begin by extending $\theta$ to get $\ov{\theta}: \L\Omega_{(-)}\to\L\Omega_{(-)}$ an endomorphism of commutative algebra objects in $\Fun(\CAlg_{\F_p}^{\qsyn},\D(\F_p))$. Inside $\CAlg_{\F_p}$ we may consider the full subcategory $\CAlg_{\F_p}^{\std}$ of standard $\F_p$-algebras, defined by the fact that they are finite tensor products over $\F_p$ of algebras of the form $\F_p[x]_{\perf}$ or $\F_p[x]_{\perf}/(x)$. Such algebras are quasiregular semiperfect, and working as before one can show that the restriction of $\L\Omega_{(-)}$ to $\CAlg_{\F_p}^{\poly}$ is a right Kan extension of the restriction to $\CAlg_{\F_p}^{\std}$. Assuming that $\ov{\theta}$ restricts to $\id$ on $\CAlg_{\F_p}^{\std}$, we conclude that $\ov{\theta}$ restricts to $\id$ on $\CAlg_{\F_p}^{\poly}$ (by the right Kan extension property) and thus that it restricts to $\id$ on $\CAlg_{\F_p}^{\qsyn}$ (by the left Kan extension property). Hence, $\theta$ restricts to $\id$ as well! So, we need to prove the first restriction claim for $\ov{\theta}$. Let $R\in\CAlg_{\F_p}^{\std}$ and consider $\ov{\theta}_R: \L\Omega_R\to\L\Omega_R$, which fits into a commutative diagram
\begin{center}
\begin{tikzcd}
\L\Omega_R \arrow[rr, "\ov{\theta}_R"] \arrow[rd, "\eps_R^{\dR}"] & & \L\Omega_R \arrow[ld, "\eps_R^{\dR}"] \\
& R &
\end{tikzcd}
\end{center}
We may work with $R$ one tensor factor at a time, allowing us to immediately dispose of the case $R=\F_p[x]_{\perf}$ and thus only need to deal with the case $R=\F_p[x]_{\perf}/(x)$. Assume for now that we have the following result.

\begin{lemma}
The endomorphism $\theta_R: \Omega_R\to\Omega_R$ is an isomorphism on cohomology for any $R\in\CAlg_{\F_p}^{\reg}$.
\end{lemma}

Our aim is to decompose $\L\Omega_R$ in terms of the associated graded pieces of its conjugate filtration. To do this, we first need to enlarge $\F_p$ to $k$ perfect containing $t$ not algebraic over $\F_p$ (this modification is harmless because of faithfully flat extension of scalars). So, we assume $R=k[x]_{\perf}/(x)$. The $k$-algebra automorphism $x\mapsto tx$ of $k[x]$ extends to an automorphism of $k[x]_{\perf}$ sending the ideal $(x)$ to itself and so induces an automorphism $\tau$ of $\L\Omega_R$. By the lemma, $\ov{\theta}_R$ preserves the conjugate filtration on $\L\Omega_R$ and induces the identity on associated graded terms. Moreover, the automorphism $\tau$ acts semisimply on the associated graded terms of $\L\Omega_R$ with disjoint eigenvalues for each term. We thereby obtain a canonical splitting 
$$\L\Omega_R\simeq\bigoplus_{n\geq0}\gr_n(\Fil_{\bullet}^{\conj}\L\Omega_R)$$
and conclude that $\ov{\theta}_R$ is $\id$. This leaves three things to be established.
\begin{enum}{\arabic}
\item The underlying linear algebra result.

\item The semisimplicity of the action of $\tau$.

\item The proof of the lemma.
\end{enum}
\end{document}