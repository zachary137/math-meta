\documentclass[11pt]{article}

\usepackage{kernel_of_truth}
\usepackage{normal_setup}

\usepackage[backend=biber,style=alphabetic]{biblatex}
\addbibresource{DAG_references.bib}

% Put tikzcd diagrams in displaystyle (i.e., math appears as though it were on its own line)
\tikzcdset{
  cells={font=\everymath\expandafter{\the\everymath\displaystyle}},
}

\renewcommand{\AA}{\mathcal{A}}
\renewcommand{\C}{\mathcal{C}}
\newcommand{\FF}{\mathcal{F}}
\newcommand{\GG}{\mathcal{G}}
\newcommand{\HH}{\mathcal{H}}
\newcommand{\J}{\mathcal{J}}
\renewcommand{\L}{\mathbb{L}}
\renewcommand{\P}{\mathbb{P}}
\renewcommand{\phi}{\varphi}
\newcommand{\UU}{\mathscr{U}}

\begin{document}
\title{Categorified Algebraic Geometry}
\author{Zachary Gardner}
\date{\texttt{zachary.gardner@bc.edu}}
\maketitle

\thispagestyle{fancydate}

\section{Introduction}
The goal of this talk is to present a slightly unorthodox view of usual algebraic geometry which may nonetheless be helpful in getting a feel for derived algebraic geometry. Given any ($1$-)category $\C$, we let $\Pre(\C):=\Fun(\C^{\op},\Set)$ be its corresponding category of presheaves of sets. Our main reference is \cite{Ras18}, with inspiration from \cite{Toe} and \cite{Str00}. All of these are in turn inspired by \cite{DG70} and \cite{LM00}.

\section{Spaces and Affine Schemes}
Most objects of interest will live in the functor category $\Fun(\Alg^{\heart},\Set)$, sometimes called the category of \emph{spaces} and denoted $\Space$.\footnote{Note that this notation and terminology can be confusing given the context.} This category is natural to consider because, letting $\Aff^{\cl}$ denote the usual (classical) category of affine schemes, we have $(\Aff^{\cl})^{\op}\simeq\CAlg^{\heart}$ and so $\Aff^{\cl}$ embeds into $\Fun((\Aff^{\cl})^{\op},\Set)\simeq\Fun(\CAlg^{\heart},\Set)$ via its Yoneda embedding. This suggests \emph{defining} the category $\Aff$ of affine schemes to be the full subcategory of $\Fun(\CAlg^{\heart},\Set)$ spanned by the co-representable objects.\footnote{Actually, the most ``morally correct'' definition would be to look at representable objects in $\Fun((\CAlg^{\heart})^{\op},\Set)$. This does not change any of the formal details but is much more notationally cumbersome.} Explicitly, these objects are denoted $\Spec A$ and are characterized by 
$$(\Spec A)(B):=\Hom_{\CAlg^{\heart}}(A,B).$$
Yoneda's lemma tells us that $\Aff^{\op}\simeq\CAlg^{\heart}$ and thus $\Aff\simeq\Aff^{\cl}$. The category of spaces inherits many of the nice properties of the category $\Set$ -- e.g., it is both complete and cocomplete, with limits and colimits defined ``pointwise.'' As an explicit example, given $X,Y\in\Space$ and $A\in\CAlg^{\heart}$, $(X\times Y)(A):=X(A)\times Y(A)$. Conveniently (if we allow $0$ to be a ring), $\Space$ has a co-representable terminal object $\Spec\Z$ and co-representable initial object $\emptyset=\Spec0$. In this way, we see that all spaces are based in the sense that they all naturally live over $\Spec\Z$ -- i.e., $\Space\simeq\Space_{/\Spec\Z}$. This is great since, given any base space $S\in\Space$ and $S\to T$ a map of spaces, we have a base change functor $T\times_S\cdot: \Space_{/S}\to\Space_{/T}$. 

\begin{remark}
One of the advantages of this formalism is that group objects are easy to reason about. It isn't hard to see that the category of group objects in $\Space$ (simply called \emph{group spaces}) is equivalent (via Yoneda) to the full subcategory of $\Fun(\Space^{\op},\Grp^{\heart})$ spanned by representable objects. In fact, though, the latter is easily seen to be equivalent to $\Fun(\CAlg^{\heart},\Grp^{\heart})$. Given $X\in\Space$ equipped with the action of a group space $G$, we can readily make sense of the quotient $X/G$ as a space.
\end{remark}

Analogous to usual algebraic geometry, we understand many spaces by bootstrapping from the affine setting. With this in mind, a map $f: X\to Y$ of spaces is said be \textbf{affine} if it pulls back affine schemes to affine schemes. Explicitly, given a map $\Spec B\to Y$ of spaces, we have that $f^{-1}(\Spec B):=\Spec B\times_YX$ is affine. We say that a map $\Spec B\to\Spec A$ of affine schemes is a \emph{closed embedding} if the associated map $A\to B$ of commutative rings is surjective or, equivalently, $B\iso A/I$ for some ideal $I\normal A$. We define the \textbf{vanishing locus} of $I\normal A$ to be $V(I):=\Spec A/I$. This comes equipped with a canonical closed embedding $V(I)\to\Spec A$. The closed embedding condition can then be defined for any map $f: X\to Y$ of spaces by requiring that $f$ is affine and any pullback of $f$ to a map of affine schemes is itself a closed embedding. One checks that this is consistent with the previous notion.

With this notion of closed embedding we can can then make sense of open embeddings. First, though, given a map $Z\to X$ of spaces we want to make sense of the \emph{complement} $X\setminus Z$ as a space. Intuitively speaking, the ``points'' of $X\setminus Z$ should be ``points'' of $X$ that don't ``intersect'' $Z$. The following definition makes this rigorous.

\begin{definition}
Let $Z\to X$ be a subspace and $A\in\CAlg^{\heart}$. Define $(X\setminus Z)(A)$ to be the set of $x\in X(A)$ such that, after identifying $x$ with $x\in\Hom_{\Space}(\Spec A,X)$ using Yoneda's lemma, the diagram
\begin{center}
\begin{tikzcd}
\emptyset \arrow[r] \arrow[d] & Z \arrow[d] \\
\Spec A \arrow[r] & X
\end{tikzcd}
\end{center}
is a Cartesian square. Equivalently, $\Spec A\times_XZ=\emptyset$.
\end{definition}

This construction is functorial in $A$ and so defines a complementary space to $Z$. Our interest will primarily be in complements of closed embeddings $Z\to X$. Note that $Z$ is necessarily a \emph{subspace} of $X$ in the sense that $Z\to X$ is a monomorphism in $\Space$. This is usually expressed by saying that $Z$ is a \textbf{subfunctor} of $X$ and emphasized with the notation $Z\inj X$. In the affine setting, we need only consider $V(I)\inj\Spec A$ for $I\normal A$. The resulting complement $D(I):=\Spec A\setminus V(I)$ is a subspace of $\Spec A$ and the associated map $D(I)\inj\Spec A$ is a canonical example of an \textbf{open embedding}. Note that $D(I)$ need not be affine in general though, in the special case $I=(f)$ is principal, we naturally have $D(f)\iso\Spec A_f$. More generally, we say that a map $U\to X$ of spaces is an \textbf{open embedding} if, given any map $\Spec B\to X$ of spaces, the induced map $\Spec B\times_XU\to\Spec B$ is an open embedding. As with closed embeddings, one checks that this agrees with the previous notion. 

\begin{remark}
Let $X\in\Space$. Here are a few things to keep in mind.
\begin{itemize}
\item The map to $X$ from a closed or open subspace of $X$ is part of the data.

\item The complement of a closed embedding $Z\inj X$ is always an open subspace of $X$.

\item Every open embedding into $X$ is a monomorphism.

\item Not every open subspace of $X$ is the complement of a closed subspace.

\item The complement of an open subspace of $X$ need not be closed. Even in the simplest case we have $\Spec A\setminus D(f)\iso\fcolim_{n\geq1}\Spec A/f^n$. This follows from the fact that both sides have $B$-points given by $\{\phi\in\Hom_{\CAlg}(A,B) : \phi(f)\in\nil(B)\}$.
\end{itemize}
\end{remark}

\section{Schemes}
Now that we have a notion of open embedding we can make sense of the notion of a \textbf{(Zariksi) open covering} of $X\in\Space$. This is the data of a collection $\UU=\{(U,i_U: U\inj X)\}$ of open embeddings such that, for every map $\Spec B\to X$ of spaces with $\Spec B$ nonempty, we have $\Spec B\times_XU\neq\emptyset$ for some $U\in\UU$. This is in turn allows us to make sense of various Zariski sites associated to (appropriate classes of) spaces and hence topoi (one must of course check the axioms for a Grothendieck topology). 

\begin{remark}
In a nutshell, the (Zariski) sheaf condition boils down to the following. Given $X,Y\in\Space$ and $\UU$ any open covering of $Y$, the natural map 
$$\Hom_{\Space}(Y,X)\to\{(f_U\in\Hom_{\Space}(U,X))_{U\in\UU}\}$$
is injective with image consisting precisely of tuples $(f_U\in\Hom_{\Space}(U,X))_{U\in\UU}$ agreeing on overlaps in the sense that $f_U|_{U\cap V}=f_V|_{U\cap V}$ for all $U,V\in\UU$.
\end{remark}

Our primary interest is in \textbf{schemes}, which are defined to be spaces covered by affine schemes and satisfying the (Zariski) sheaf condition, spanning a full subcategory $\Sch\subset\Space$. We readily see that affine schemes are themselves schemes (which is basically a descent statement) and that $\Sch$ is closed under arbitrary fiber products (which is basically already needed above to get a site). Returning to the setting of relative algebraic geometry, fix a base space $S\in\Space$. We define the category $\Sch_S$ of \textbf{$S$-schemes} to be the full subcategory of spaces $X/S\in\Space_{/S}$ such that $T\times_SX\in\Sch$ for every map of spaces $T\to S$ with $T$ affine. Similarly, we define the category $\Aff_S$ of \textbf{$S$-affine schemes} to be the full subcategory of spaces $X/S\in\Space_{/S}$ such that $T\times_SX\in\Aff$ for every map of spaces $T\to S$ with $T$ affine.

\begin{theorem}
Let $S\in\Sch$. Then, the natural inclusion $\Sch_{/S}\inj\Sch_S$ is an equivalence of categories (i.e., it is essentially surjective).\footnote{In fact, the two are isomorphic as categories!} Equivalently, if $X/S\in\Space_S$ satisfies $T\times_SX\in\Sch$ for every $T/S\in\Aff_{/S}$ then $X\in\Sch$.
\end{theorem}

In other words, schemes over $S$ are always $S$-schemes and the two are the same thing if $S\in\Sch$.

\begin{remark}
Assuming $S\in\Sch$, this result says that objects of $\Aff_S$ are exactly affine maps $X\to S$ of schemes. As we will see after discussing quasicoherent sheaves, there is a canonical equivalence $\Aff_S\simeq\CAlg^{\heart}(\QCoh^{\heart}(S))$ between the category of $S$-affine schemes and the category of commutative algebra objects in $\QCoh^{\heart}(S)$.
\end{remark}

\section{Quasicoherent Sheaves}
\subsection{Basics}
Fix a scheme $X\in\Sch$. How should we define a module over $X$? By definition, a \textbf{quasicoherent sheaf} is an assignment from objects in $\Aff_{/X}$ to $\Mod_{\Z}^{\heart}$ such that, for every $(f: \Spec A\to X)\in\Aff_{/X}$, the abelian group $\FF_f$ is equipped with a natural (left) $A$-module structure (with morphisms of quasicoherent sheaves respecting this natural module structure).\footnote{Recall that there is a forgetful functor $\Mod_A^{\heart}\to\Mod_{\Z}^{\heart}$ obtained by remembering only the additive structure of a (left) $A$-module.} As it turns out, pinning down the precise meaning of ``natural'' here is a bit tricky. Given $g: \Spec B\to\Spec A$, we obtain both the $B$-modules $\FF_{f\circ g}$ and $B\tensor_A\FF_f$. A priori there is no reason that these two should be related, so we must specify an isomorphism
$$\alpha_{f,g}\in\Isom_{\Mod_B^{\heart}}(\FF_{f\circ g},B\tensor_A\FF_f).$$
Consider now a string of maps\footnote{This should be thought of as a composable triple in $\Aff_{/X}$.}
\begin{center}
\begin{tikzcd}
\Spec C \arrow[r, "h"] & \Spec B \arrow[r, "g"] & \Spec A \arrow[r, "f"] & X
\end{tikzcd}
\end{center}
We want our above construction to be ``associative.'' The precise meaning of this statement is that the diagram
\begin{center}
\begin{tikzcd}
& C\tensor_B(B\tensor_A\FF_f) & \\
C\tensor_A\FF_f \arrow[ru, "\sigma_{f,g,h}^{-1}"'] & & C\tensor_B\FF_{f\circ g} \arrow[lu, "\id_C\tensor\alpha_{f,g}"] \\
\FF_{f\circ(g\circ h)} \arrow[rr, equals] \arrow[u, "\alpha_{f,g\circ h}"'] & & \FF_{(f\circ g)\circ h} \arrow[u, "\alpha_{f\circ g,h}"]
\end{tikzcd}
\end{center}
commutes, where $\sigma_{f,g,h}: C\tensor_B(B\tensor_A\FF_f)\xto{\sim}C\tensor_A\FF_f$ is the natural isomorphism obtained by base change.\footnote{Even though $g,h$ don't explicitly appear in the domain and codomain of $\theta_{f,g,h}$ they are lurking in the background.} This weak associativity condition is called the \textbf{cocycle condition}. We call each isomorphism $\alpha_{f,g}$ a \textbf{cocycle}. 

What about morphisms of quasicoherent sheaves? The data of $\phi\in\Hom_{\QCoh^{\heart}(X)}(\FF,\GG)$ is the data of compatible linear maps $\phi_f: \FF_f\to\GG_f$ for every $(f: \Spec A\to X)\in\Aff_{/X}$. Given $(f: \Spec A\to X)\in\Aff_{/X}$, any morphism to this object in $\Aff_{/X}$ is uniquely encoded by a commutative diagram\footnote{In the $\infty$-categorical setting, ``commutativity'' of a diagram is data and not a condition.}
\begin{center}
\begin{tikzcd}
\Spec B \arrow[r, "g"] \arrow[rd, dotted, "f\circ g"'] & \Spec A \arrow[d, "f"] \\
& X
\end{tikzcd}
\end{center}
Letting $\beta_{f,g}$ be the relevant cocycle for $\GG$, the compatibility condition mentioned above is that
\begin{center}
\begin{tikzcd}
\FF_{f\circ g} \arrow[r, "\phi_{f\circ g}"] \arrow[d, "\alpha_{f,g}"'] & \GG_{f\circ g} \arrow[d, "\beta_{f,g}"] \\
B\tensor_A\FF_f \arrow[r, "\id_B\tensor\phi_f"'] & B\tensor_A\GG_f
\end{tikzcd}
\end{center}
commutes and that $\phi_{f,g}$ in turn satisfies the cocycle condition (relative to any $h: \Spec C\to\Spec B$). The resulting category $\QCoh^{\heart}(X)$ can be viewed as a non-full subcategory of $\Fun(\Aff_{/X}^{\op},\Mod_{\Z}^{\heart})$. Using the above setup, to a given $\FF$ we may associate $\un{\FF}\in\Fun(\Aff_{/X}^{\op},\Mod_{\Z}^{\heart})$ defined by $\un{\FF}(f):=\FF_f$, with the map from $f$ to $f\circ g$ in $\Fun(\Aff_{/X}^{\op},\Mod_{\Z}^{\heart})$ given by the composition 
\begin{center}
\begin{tikzcd}
\FF_f \arrow[r, "1_B\tensor\id"] & B\tensor_A\FF_f \arrow[r, "\alpha_{f,g}^{-1}"] & \FF_{f\circ g}
\end{tikzcd}
\end{center}
We then complete the picture by noting that
\begin{center}
\begin{tikzcd}
\FF_f \arrow[r, "1_B\tensor\id"] \arrow[d, "\phi_f"'] & B\tensor_A\FF_f \arrow[r, "\alpha_{f,g}^{-1}"] & \FF_{f\circ g} \arrow[d, "\phi_{f\circ g}"] \\
\GG_f \arrow[r, "1_B\tensor\id"'] & B\tensor_A\GG_f \arrow[r, "\beta_{f,g}^{-1}"'] & \GG_{f\circ g}
\end{tikzcd}
\end{center}
commutes and so $\phi$ encodes a natural transformation from $\un{\FF}$ to $\un{\GG}$. The resulting functor out of $\QCoh^{\heart}(X)$ is evidently faithful but not full because morphisms in $\Fun(\Aff_{/X}^{\op},\Mod_{\Z}^{\heart})$ need not satisfy the cocycle condition.

Why have we chosen to write the cocycle condition using a trapezoidal shape instead of the usual square? There are two reasons for this. The first is to call into question the assumption that $\FF_{f\circ(g\circ h)}$ and $\FF_{(f\circ g)\circ h}$ are really the same thing. Of course, in the classical $1$-categorical setting this question does not uncover anything interesting since, of course, $f\circ(g\circ h)$ and $(f\circ g)\circ h$ are the same map by the usual associativity of composition. Despite the definite strength of assuming this ``strong'' form of associativity, time has shown that various weakened forms of associativity are also useful (and necessary to consider). As a simple illustration of this, note that $(M\tensor_AN)\tensor_AP$ and $M\tensor_A(N\tensor_AP)$ are not literally the same even both satisfy the same universal property (and so are the ``same'' for most algebraic intents and purposes).

The second reason we write the diagram in a funny way is to call into question whether we actually need the diagram to commute ``on the nose.'' Basically, instead of requiring the above cocycle diagrams to commute we can require that they commute up to a specified natural isomorphism, say $\eta_{\FF}$ relative to $\FF$ in the above setup. To $\GG$ and the triple $(f,g,h)$ we may similarly associate $\eta_{\GG}$. We could then require $\phi$ to send $\eta_{\FF}$ to $\eta_{\GG}$. This would give us a new category $\QCoh'(X)$, right? Not quite. The issue is that $\phi$ is being asked to do too much. As a functor $\phi$ only knows how to act on objects and morphisms -- in particular, it does not know how to act on the functors $\FF,\GG$ and the natural transformations $\eta_{\FF},\eta_{\GG}$. The classical (``stacky'') solution to this problem is to think of $\QCoh'(X)$ not as a usual $1$-category but instead as a $2$-category! We then think of $\phi$ as some generalization of a $1$-functor.

In the classical setting there is a bit of a miracle that occurs -- i.e., there is a general procedure to relate any $2$-category via equivalence to a special kind of $2$-category called a \emph{strict} $2$-category. Applying this procedure to $\QCoh'(X)$ recovers $\QCoh^{\heart}(X)$ (interpreted as a strict $2$-category in the obvious way). Both $\QCoh^{\heart}(X)$ and $\QCoh'(X)$ yield exactly the same answers for whatever we classically want to do regarding sheaves (e.g., computing cohomology). Part of this comes from the fact that most of the distinctions between the two fade away when passing to essential images. In the future we will think of both $\QCoh^{\heart}(X)$ and $\QCoh'(X)$ as ``shadows'' of some $\infty$-category $\QCoh(X)$ of quasicoherent sheaves on $X$.

\subsection{Descent}
We start by discussing sections of quasicoherent sheaves.

\begin{theorem}
Let $X\in\Spec A$. Then, the \textbf{global section} functor 
$$\Gamma(X,\cdot): \QCoh^{\heart}(X)\to\Mod_A^{\heart},\qquad \FF\mapsto\FF_{\id_X}$$
is an equivalence of categories with inverse given by $M\mapsto\FF_M$ for $\FF_M$ sending $f: \Spec B\to X$ to $B\tensor_AM$ (with cocycles defined by base change).
\end{theorem}

That is, the data of a quasicoherent sheaf over an affine scheme is simply the data of a module (with the rest of the information automatically accounted for by base change). It is common to denote $\FF_M$ by $\twid{M}$. Given $X\in\Sch$, we may associate the \textbf{structure sheaf} $\O_X\in\QCoh^{\heart}(X)$ defined by sending $f: \Spec A\to X$ to $A$. In particular, $\O_{\Spec A}=\twid{A}\in\QCoh^{\heart}(\Spec A)$ and so $\Gamma(\Spec A,\O_{\Spec A})\iso A$. It follows that there is a natural isomorphism $\Gamma(X,\FF)\iso\Hom_{\QCoh^{\heart}(X)}(\O_X,\FF)$ of abelian groups when $X$ is affine and so we can \emph{define} $\Gamma(X,\FF):=\Hom_{\QCoh^{\heart}}(\O_X,\FF)$ for general $X\in\Sch$.

\begin{remark}
We will see more generally that $\QCoh(\Spec A)\simeq\Mod_A$ in terms of $\infty$-categories.
\end{remark}

Given a map $\rho: Y\to X$ of schemes, there is a natural pullback functor $\rho^*: \QCoh^{\heart}(X)\to\QCoh^{\heart}(Y)$ defined on $\FF\in\QCoh^{\heart}(X)$ via
$$\rho^*\FF: \Aff_{/Y}^{\op}\to\Mod_{\Z}^{\heart},\qquad f\mapsto\FF_{\rho\circ f}.$$
Assuming $\rho$ is qcqs (which holds, e.g., if $\rho$ is affine), there is also a natural pushforward functor $\rho_*: \QCoh^{\heart}(Y)\to\QCoh^{\heart}(X)$ which is right adjoint to $\rho^*$. Both of these constructions are appropriately functorial. Moreover, given any map $\rho: \Spec B\to\Spec A$ of affine schemes and $M\in\Mod_A^{\heart}$, there is a canonical isomorphism $\FF_{B\tensor_AM}\xto{\sim}\rho^*\FF_M$ in $\QCoh^{\heart}(\Spec B)$. 

\begin{remark}
Given an open embedding $i: U\inj X$ and $\FF\in\QCoh^{\heart}(X)$, we let $\FF|_U:=i^*\FF$.
\end{remark}

Letting $\UU$ be a Zariski open covering of $X\in\Sch$, define $\QCoh^{\heart}(X;\UU)$ to be the category whose objects consist of tuples $(\FF_U)_{U\in\UU}$ with $\FF_U\in\QCoh^{\heart}(U)$ and cocycles 
$$\alpha_{U,V}\in\Isom_{\QCoh^{\heart}(U\cap V)}(\FF_U|_{U\cap V},\FF_V|_{U\cap V})$$
satisfying the cocycle condition defined analogously to before.

\begin{theorem}[Serre]
Let $X\in\Sch$ and $\UU$ a Zariski open covering of $X$. The functor
$$\QCoh^{\heart}(X)\to\QCoh^{\heart}(X;\UU),\qquad \FF\mapsto(\FF|_U)_{U\in\UU}$$
is an equivalence of categories.
\end{theorem}

In particular, if $\UU=\{\Spec A_i\}$ is an affine open covering then this tells us that $\QCoh^{\heart}(X)$ is built in a precise way from the categories $\Mod_{A_i}^{\heart}$.

\begin{remark}
We will see more generally that 
$$\QCoh(X)\simeq\flim_{\Spec A\inj X}\Mod_A$$ 
in terms of $\infty$-categories, sampling over affine open subspaces.
\end{remark}

\subsection{Global Spec}
Let $\AA\in\QCoh^{\heart}(X)$ be a commutative algebra object. There are at least three equivalent ways to make sense of this statement.
\begin{enum}{\roman}
\item $\AA$ is a commutative ring object in $\QCoh^{\heart}(X)$ -- i.e., $\AA$ behaves like a categorified commutative ring.

\item $\AA$ is a (tensorial) commutative algebra over $\O_X$ -- i.e., $\AA$ is equipped with a compatible associative commutative multiplication map $m: \AA\tensor_{\O_X}\AA\to\AA$ and algebra unit map $1_{\AA}: \O_X\to\AA$.

\item $\AA$ is a quasicoherent assignment of commutative algebras. This one requires a bit more explanation. First of all, the $A$-module $\AA_f$ associated to $f: \Spec A\to X$ should actually be a commutative $A$-algebra. Second, the cocycle $\alpha_{f,g}\in\Isom_{\Mod_B^{\heart}}(\AA_{f\circ g},B\tensor_A\AA_f)$ associated to $g: \Spec B\to\Spec A$ should actually be an isomorphism of commutative $B$-algebras. No change needs to be made to the cocycle condition since the constructions involving modules work just as well for algebras.\footnote{In more general settings the cocycle condition may need to be modified.}  
\end{enum}
We denote the resulting category of such objects by $\CAlg^{\heart}(\QCoh^{\heart}(X))$. Our goal is to define a generalized version of $\Spec$ relative to $\AA$, fittingly called \textbf{global $\Spec$} and denoted $\Spec_X\AA$.\footnote{Some sources write $\un{\Spec}_X\AA$ for extra emphasis.} If the name is anything to go by then we should have $\Spec_X\AA\in\Aff_X$. Let's first describe $\Spec_X\AA$ as a space. Fixing $B\in\CAlg^{\heart}$, given any $x: \Spec B\to X$ we can consider $x^*\AA$. Since $\AA\in\QCoh^{\heart}(X)$ we have $x^*\AA\in\QCoh^{\heart}(\Spec B)$ and so $x^*\AA$ is equivalent to the data of the $B$-algebra $\Gamma(x^*\AA)$. To every section of the structure map $B\to\Gamma(x^*\AA)$ we may uniquely associate a map of spaces $\rho: \Spec B\to\Spec\Gamma(x^*\AA)$. That is, $\rho$ arises from a ring map $\sigma: \Gamma(x^*\AA)\to B$ such that the composition $B\to\Gamma(x^*\AA)\xto{\sigma}B$ is $\id_B$. We may then define $(\Spec_X\AA)(B)$ to consist of all pairs $(x,\rho)$. This construction is contravariantly functorial in $\AA$ and comes equipped with a map of spaces $\Spec_X\AA\to X$ obtained by forgetting the section $\rho$. Given $x: \Spec B\to X$, there is a canonical isomorphism $\Spec(x^*\AA)\xto{\sim}\Spec B\times_X\Spec_X\AA$ and hence the map $\Spec_X\AA\to X$ is affine. A similar argument shows that $(\Spec_X\AA\to X)\in\Sch_X\iso\Sch_{/X}$ and so $\Spec_X\AA\in\Sch$.

\begin{theorem}
The functor $\Spec_X: \CAlg^{\heart}(\QCoh^{\heart}(X))^{\op}\to\Aff_X$ is an equivalence of categories.
\end{theorem}

\begin{proof}
We will (sketchily) show that $\Spec_X$ is essentially surjective and leave the rest as an exercise to the reader. So, let $\rho: Y\to X$ be affine. We claim that $\rho_*\O_Y\in\CAlg^{\heart}(\QCoh^{\heart}(X))$, leaving it as an exercise that there is a canonical isomorphism from $\Spec_X\rho_*\O_Y$ to $Y$ in $\Aff_X$. We can take our algebra unit $1: \O_X\to\rho_*\O_Y$ to be the map induced by the natural isomorphism $\rho^*\O_X\xto{\sim}\O_Y$. How do we get our multiplication map? Using lots of adjunction! Consider the composition
$$\rho^*(\rho_*\O_Y\tensor_{\O_X}\rho_*\O_Y)\xto{\sim}\rho^*\rho_*\O_Y\tensor_{\O_Y}\rho^*\rho_*\O_Y\to\O_Y\tensor_{\O_Y}\O_Y\xto{\sim}\O_Y,$$
where we have made use of the counit $\eps_{\rho}$. Applying $\rho_*$ to the above and using the unit $\eta_{\rho}$ gives the desired multiplication map $m: \rho_*\O_Y\tensor_{\O_X}\rho_*\O_Y\to\rho_*\O_Y$.
\end{proof}

\begin{remark}
The right way to understand and generalize this result is by thinking about so-called \emph{monads} and \emph{comonads}. The \emph{Barr-Beck theorem} is of particular relevance.
\end{remark}

One rich and important class of affine maps to $X$ is provided by closed embeddings $j: Z\inj X$. What can we say about $\Spec_X^{-1}(j)\in\CAlg^{\heart}(\QCoh^{\heart}(X))$? The key comes from looking more at categorified algebra. A monomorphism in $\QCoh^{\heart}(X)$ is precisely the data of $\phi\in\Hom_{\QCoh^{\heart}(X)}(\FF,\GG)$ such that $\ker\phi=0$. We may then think of $\FF$ as a submodule of $\GG$ and write $\FF\inj\GG$. In particular, taking a submodule of $\O_X$ itself recovers the notion of a \textbf{quasicoherent ideal sheaf} on $X$.

\begin{proposition}
Let $X\in\Sch$.
\begin{enum}{\alph}
\item Let $\J\inj\O_X$ be a quasicoherent ideal sheaf. Then, $\O_X/\J:=\coker(\J\inj\O_X)$ is naturally a commutative algebra object in $\QCoh^{\heart}(X)$ and the canonical map $\Spec_X\O_X/\J\to X$ is a closed embedding.

\item Quasicoherent ideal sheaves on $X$ correspond bijectively with closed embeddings into $X$.\footnote{Given a closed embedding $j: Z\inj X$, the key is to consider $\ker(\O_X\to j_*\O_Z)$.}
\end{enum}
\end{proposition}

With this in mind we define $V(\J):=\Spec_X\O_X/\J$ for $\J\inj\O_X$ a quasicoherent ideal sheaf and call this the \textbf{vanishing locus} of $\J$ (we saw above that this is naturally a closed subscheme of $X$). We can also define the \textbf{nonvanishing locus} $D(\J)$ to be $X\setminus V(\J)$.\footnote{Recall that $X\iso\Spec_X\O_X$ canonically in $\Sch_{/X}$.} Then, the canonical map $D(\J)\to X$ is an open embedding.

\begin{remark}
This formalism is really easy to use! Let $f\in\Gamma(X,\O_X)=\End_{\QCoh^{\heart}(X)}(\O_X)$ be a global section. Then, the nonvanishing locus $X_f\inj X$ can simply be obtained as $\Spec_X\O_X[f^{-1}]$ and we readily see that this is an open subspace of $X$.
\end{remark}

\section{Topological Concerns}
The category $\Sch$ of schemes as we have defined them and the category $\Sch^{\cl}$ of usual (classical) schemes are the ``same'' in a manner that goes far beyond a mere equivalence of categories. We won't go into too much detail on this here but we will address a few potential ``topological'' concerns. Throughout we fix $X\in\Sch$. 
\begin{itemize}
\item The notions of quasicompact (qc), quasiseparated (qs), connected, and such make sense for (most) sites and so we can port them over to our setting without any issues. Moreover, things work as you would expect -- e.g., affine schemes are qc and open subschemes of the form $D(I)\inj\Spec A$ are qc if and only if $I$ is a finitely generated ideal of $A$. 

\item Given $\UU$ a collection of open subspaces of $X$, $\UU$ is an open covering if and only if $X(A)\iso\colim_{U\in\UU}U(A)$ for every \emph{local} ring $A\in\CAlg^{\heart}$.\footnote{This colimit is basically just a union of sets.}

\item How can we recover an underlying topological space $|X|$ from $X$? Consider the set $E(X)$ of pairs $(k,x)$ with $k$ a field and $x\in X(k)$ a $k$-point. Define an equivalence relation $\sim$ on $E(X)$ by $(k,x)\sim(k',x')$ if there exists a field $L$ and maps of rings $i: k\inj L$ and $i': k'\inj L$ such that $i(x)=i'(x')$ in $X(L)$. The quotient set $|X|:=E(X)/\!\sim$ is then functorial in $X$ and naturally in bijection with the collection of prime ideals of $A$ if $X=\Spec A$. We can equip $|X|$ with a topology by declaring $V\subset|X|$ to be \emph{open} if there exists an open subspace $U\inj X$ such that $V=|U|$. This is basically the same as the Zariski topology, and we obtain a functor $\abs{\cdot}: \Sch\to\Top$.

\item The \textbf{reduction} $X_{\red}$ of $X$ can be obtained in several ways. Consider the following universal problems.
\begin{enum}{\alph}
\item Find the smallest closed subscheme $Z\inj X$ such that $X\setminus Z=\emptyset$.

\item Find $Z\in\Aff_X$ reduced with the universal property that any $Y\in\Sch_{/X}$ with $Y$ reduced uniquely factors through $Z\to X$.
\end{enum}
The solution to both universal problems is to take $X_{\red}:=V(\nil(\O_X))=\Spec_X\O_X/\nil(\O_X)$, where $\nil(\O_X)$ is the nilradical of $\O_X$ and $\Spec_X$ is global Spec.

\item Density and closure are somewhat touchy notions since we are not working with a genuine topology in the setting of spaces. The basic fix (sweeping under the rug some subtlety) is to only think about scheme-theoretic closure and density. This works nicely and agrees with more ``topological'' approaches when working with qcqs morphisms (especially qc open embeddings).
\end{itemize}

\newpage
\printbibliography[title={References}]
\end{document}