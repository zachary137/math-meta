\documentclass[11pt]{article}

\usepackage{kernel_of_truth}
\usepackage{normal_setup}

\newcommand{\Cart}{\textup{Cart}} % Cartier
\newcommand{\DA}{\mathsf{DA}} % Dieudonn\'{e} algebra(s)
\newcommand{\DC}{\mathsf{DC}} % Dieudonn\'{e} complex(es)
\newcommand{\sat}{\textup{sat}} % saturated
\newcommand{\Sat}{\textup{Sat}} % saturation
\newcommand{\str}{\textup{str}} % strict
\newcommand{\W}{\mathcal{W}}
\newcommand{\X}{\mathfrak{X}}

\renewcommand{\L}{\mathbb{L}}
\renewcommand{\phi}{\varphi}

\begin{document}
\title{BLM}
\author{Zachary Gardner}
\date{}
\maketitle

\section{Some Calculations}
Let's apply our theory to the torus $\G_{m,\Z_p}^n$.\footnote{You might think it's more universal to look at $\G_{m,\Z}^n$, but as we shall see replacing $\Z$ by $\Z_p$ makes no material difference. This is perhaps not so surprising since our theory is local to $p$.} With that in mind, let $R:=\Z_p[x_1^{\pm1},\ldots,x_n^{\pm1}]$. Then, we have a cdga isomorphism $\Omega_R^{\bullet}\iso\Wedge_R^{\bullet}[d\log x_1,\ldots,d\log x_n]$, where the generators are in degree $1$ and
$$d\log x_i:=\df{dx_i}{x_i},\qquad d(x_i^a)=ax_i^a\cdot d\log x_i,\qquad d(d\log x_i)=0.$$
The $p$-torsion-free ring $R$ comes equipped with a mod $p$ Frobenius lift $\phi$ which is the identity on $\Z_p$ and satisfies $\phi(x_i)=x_i^p$. It follows that $(R,\phi)$ is a good ring and so there is a natural way to extend $\phi$ to all of $\Omega_R^{\bullet}$ making the latter a Dieudonn\'{e} algebra. Using the above isomorphism we may transport the $\DA$ structure from $\Omega_R^{\bullet}$ to $\Wedge_R^{\bullet}[d\log x_1,\ldots,d\log x_n]$, which is characterized by
$$F(x_i)=x_i^p,\qquad F(d\log x_i)=d\log x_i.$$
We have
$$R[\phi^{-1}]\iso\Z_p[x_1^{\pm1/p^{\infty}},\ldots,x_n^{\pm1/p^{\infty}}]=:R_{\infty}$$
and so 
$$\Omega_R^{\bullet}[F^{-1}]\iso\Wedge_{R_{\infty}}^{\bullet}[d\log x_i,\ldots,d\log x_n]$$
since $F$ fixes the generators $d\log x_i$.

Let $M\in\DC$ be $p$-torsion-free with $F$ injective on $M$. We may describe the iterated d\'{e}calage $\eta_p^rM\subset M[p^{-1}]$ by 
$$(\eta_p^rM)^n=\{x\in p^{rn}M^n : dx\in p^{r(n+1)}M^{n+1}\}.$$
Recall that 
$$M[F^{-1}]:=\colim(M\xto{F}M\xto{F}\cdots).$$ 
We want to identify $\Sat(M)$ inside of $M[F^{-1}]$. 

\begin{remark}
Intuitively, $M[F^{-1}]$ should be thought of as the union $\bigcup_{r\geq0}F^{-r}M$. Interesting phenomena only arise at the ``infinite'' level since $\colim(M\xto{F}M)\iso M$. This isomorphism arises from the fact that the data of a commutative diagram
\begin{center}
\begin{tikzcd}
M \arrow[r, "F"] \arrow[rd] & M \arrow[d] \\
& N
\end{tikzcd}
\end{center}
is equivalent to the data of just a map $M\to N$.
\end{remark}

We have a commutative diagram of graded abelian groups
\begin{center}
\begin{tikzcd}
M \arrow[r, "\alpha_F"] \arrow[d, equals] & \eta_pM \arrow[r, "\eta_p\alpha_F"] \arrow[d, hookrightarrow, "\theta_1"] & \eta_p^2M \arrow[r, "\eta_p^2\alpha_F"] \arrow[d, hookrightarrow, "\theta_2"] & \cdots \\
M \arrow[r, hookrightarrow, "F"] \arrow[d, equals] & M \arrow[r, hookrightarrow, "F"] \arrow[d, hookrightarrow, "F^{-1}"] & M \arrow[r, hookrightarrow, "F"] \arrow[d, hookrightarrow, "F^{-2}"] & \cdots \\
M \arrow[r, hookrightarrow] & F^{-1}M \arrow[r, hookrightarrow] & F^{-2}M \arrow[r, hookrightarrow] & \cdots \\
\end{tikzcd}
\end{center}
with
$$\theta_r: (\eta_p^rM)^n\inj M^n,\qquad x\mapsto p^{-rn}x.$$
It follows that there is an induced injection $\theta: \Sat(M)\inj M[F^{-1}]$.\footnote{The natural induced map is injective because filtered colimits commute with finite limits.} What is the image of $\theta$? We have 
$$(\im\theta)^n=\bigcup_{r\geq0}(\im\theta_r)^n\cap F^{-r}M^n$$
with
$$(\im\theta_r)^n\cap F^{-r}M^n=\{x\in F^{-r}M^n : d(F^rx)\in p^rM^{n+1}\}.$$
It follows that may identify $\Sat(M)$ as a graded abelian group via
$$\Sat(M)\iso\{x\in M^{\bullet}[F^{-1}] : d(F^rx)\in p^rM^{\bullet+1}\textrm{ for }r\gg0\}.$$
In this setup the differential $d$ on $\Sat(M)$ is described by
$$x\mapsto p^{-r}F^{-r}d(F^rx)$$
for $r\gg0$.\footnote{The ordering of $p^{-r}$ and $F^{-r}$ doesn't matter if we first invert $p$ and extend $F$. However, it does matter if we work more simply.} Viewed another way, we obtain $\twid{d}: M^{\bullet}[F^{-1}]\to M^{\bullet+1}[F^{-1}]\tensor_{\Z}\Z[p^{-1}]$ using the same formula and $\Sat(M)$ is precisely the stuff stable under $\twid{d}$. Let's now apply this to understand $\Sigma:=\Sat(\Omega_R^{\bullet})$. For simplicity, let $y_i:=d\log x_i$. While it suffices to describe each $\Omega_R^m[F^{-1}]$ entirely in terms of homogeneous forms, some care must be taken for $\Sigma^m$. Let's first compute $\twid{d}: \Omega_R^{\bullet}[F^{-1}]\to\Omega_R^{\bullet+1}[F^{-1}]\tensor_{\Z}\Z[p^{-1}]$. We obtain $\Omega_R^{\bullet}[F^{-1}]$ as the $\Z_p$-linear span of homogeneous forms
$$\omega=x_1^{a_1}\cdots x_n^{a_n}\underbrace{y_{k_1}\wedge\cdots\wedge y_{k_m}}_{=:\eta},$$ 
where we could have $m=0$. Then,
\begin{align*}
\twid{d}\omega
&=p^{-r}F^{-r}d(F^r\omega) \\
&=p^{-r}F^{-r}d(x_1^{p^ra_1}\cdots x_n^{p^ra_n}\eta) \\
&=p^{-r}F^{-r}d(p^rx_1^{p^ra_1}\cdots x_n^{p^ra_n}(a_1y_1+\cdots+a_ny_n)\wedge\eta) \\
&=x_1^{a_1}\cdots x_n^{a_n}(a_1y_1+\cdots+a_ny_n)\wedge\eta \\
&=(a_1y_1+\cdots+a_ny_n)\wedge\omega.
\end{align*}
Now we turn our attention to $\Sigma^0$. Of course,
$$\Omega_R^0[F^{-1}]\iso R_{\infty}=\Span_{\Z_p}\{x_1^{a_1}\cdots x_n^{a_n} : a_i\in\Z[p^{-1}]\}.$$
Let $\alpha:=\lambda x_1^{a_1}\cdots x_n^{a_n}$ and $\beta:=\gamma x_1^{b_1}\cdots x_n^{b_n}$ be elements of $R_{\infty}$. Then, the polynomial coefficient for $y_i$ in $\twid{d}(\alpha+\beta)$ is $a_i\alpha+b_i\beta$ and so it suffices to consider just homogeneous elements for constructing $\Sigma^0$. From the above we get 
\begin{align*}
\Sigma^0
&=\Span_{\Z_p}\{\lambda x_1^{a_1}\cdots x_n^{a_n}\in R_{\infty} : \lambda a_1,\ldots,\lambda a_n\in\Z_p\} \\
&=\Span_{\Z_p}\{\lambda x_1^{a_1}\cdots x_n^{a_n}\in R_{\infty} : v_p(\lambda)+\min\{v_p(a_i) : 1\leq i\leq n\}\geq0\}. 
\end{align*}
As before,
$$\Omega_R^1[F^{-1}]\iso\Span_{\Z_p}\{x_1^{a_1}\cdots x_n^{a_n}y_k : a_i\in\Z[p^{-1}],1\leq k\leq n\}.$$
After performing a calculation similar to before, we conclude that the right ``shape'' of elements to consider for describing $\Sigma^1$ is 
$$x_1^{a_1}\cdots x_n^{a_n}(\lambda_1y_1+\cdots+\lambda_ny_n).$$
Hitting an element like this with $\twid{d}$ gives
$$x_1^{a_1}\cdots x_n^{a_n}(a_1y_1+\cdots+a_ny_n)\wedge(\lambda_1y_1+\cdots+\lambda_ny_n)=x_1^{a_1}\cdots x_n^{a_n}\sum_{i<j}(a_i\lambda_j-a_j\lambda_i)y_i\wedge y_j$$
and so 
$$\Sigma^1=\Span_{\Z_p}\{x_1^{a_1}\cdots x_n^{a_n}(\lambda_1y_1+\cdots+\lambda_ny_n) : v_p(a_i\lambda_j-a_j\lambda_i)\geq0\textrm{ for }i<j\}.$$
The same sort of calculation goes through to describe all $\Sigma^m$ for $m\geq1$. Our next goal is to compute the completion $\W\Sigma$. For this we need the verschiebung map $V: \Sigma\to\Sigma$ characterized by $FV=p=VF$. A little thought gives 
$$V(x_1^{a_1}\cdots x_n^{a_n}y_{k_1}\wedge\cdots\wedge y_{k_m})=px_1^{a_1/p}\cdots x_n^{a_n/p}y_{k_1}\wedge\cdots\wedge y_{k_m}$$
and $V$ inherits $\Z_p$-linearity from $F$. Note that $V$ is not an algebra map for $n>1$ since
$$V(y_1\wedge y_2)=p(y_1\wedge y_2)\neq p^2(y_1\wedge y_2)=Vy_1\wedge Vy_2.$$
Note that $d$ and $V$ don't commute just as $d$ and $F$ don't commute. Indeed,
$$d(V\omega)=p^{-1}V(d\omega)\implies d(V^r\omega)=p^{-r}V^r(d\omega).$$
By definition,
$$\W\Sigma:=\flim_{r\geq0}\W_r\Sigma,\qquad \W_r\Sigma:=\Sigma/(\im(V^r)+\im(dV^r)).$$
We seek to understand $\W_r\Sigma$, so let's start with the simplest case $(\W_r\Sigma)^0=\Sigma^0/V^r\Sigma^0$. We have
\begin{align*}
V^r\Sigma^0
&=\Span_{\Z_p}\{\lambda x_1^{a_1}\cdots x_n^{a_n}\in\Sigma^0 : v_p(\lambda)\geq r\} \\
&=\Span_{\Z_p}\{\lambda x_1^{a_1}\cdots x_n^{a_n}\in R_{\infty} : v_p(\lambda)\geq r,v_p(\lambda)+\min\{v_p(a_i) : 1\leq i\leq n\}\geq0\}.
\end{align*}
By contrast,
\begin{align*}
p^r\Sigma^0
=\Span_{\Z_p}\{\lambda x_1^{a_1}\cdots x_n^{a_n}\in R_{\infty} : v_p(\lambda)\geq r,v_p(\lambda)+\min\{v_p(a_i) : 1\leq i\leq n\}\geq r\}
\end{align*}
and so the $V$-adic completion of $\Sigma^0$ is actually much easier to describe than the $p$-adic completion.\footnote{Just a precautionary note, beware that $\Z_p[t]$ is not $p$-adically complete. The same applies if you throw in more variables.} We have
\begin{align*}
(\W\Sigma)^0
\iso\Span_{\Z_p}\left\{\sum_{r\geq0}\lambda^{(r)}x_1^{a_1^{(r)}}\cdots x_n^{a_n^{(r)}} : \lambda^{(r)}x_1^{a_1^{(r)}}\cdots x_n^{a_n^{(r)}}\in V^r\Sigma^0\right\}.
\end{align*}
The coefficients satisfy $v_p(\lambda^{(r)})\to\infty$ as $r\to\infty$. We obtain $V^r\Sigma^1$ as the $\Z_p$-linear span of 
$$x_1^{a_1}\cdots x_n^{a_n}(\lambda_1y_1+\cdots+\lambda_ny_n)$$
subject to $v_p(a_i\lambda_j-a_j\lambda_i)\geq0$ for $i<j$ and $v_p(\lambda_i)\geq r$ for all $i$. At the same time, we obtain $dV^r\Sigma^0$ as the $\Z_p$-linear span of 
$$\lambda x_1^{a_1}\cdots x_n^{a_n}(a_1y_1+\cdots+a_ny_n)$$
subject to $v_p(\lambda)+\min\{v_p(a_i) : 1\leq i\leq n\}\geq0$ and $v_p(\lambda)\geq r$.
\end{document}

\section{The Case of a Cusp}
So far we've seen a bit how the saturated de Rham-Witt complex behaves in the smooth case. Our aim now is to investigate the non-smooth case by way of the mildest kind of singularity: a cusp. To that end, let $R:=\F_p[x,y]/(x^2-y^3)$ so that $\Spec R$ is a cuspidal cubic curve in $\A_{\F_p}^2$. We can resolve the cusp of $\Spec R$ by considering the normalization $\ov{R}$, obtained as the integral closure of $R$ in $\Frac(R)$ and given concretely by $\F_p[t]$ using the map $R\inj\F_p[t]$ induced by 
$$\F_p[x,y]\to\F_p[t],\qquad x\mapsto t^3,y\mapsto t^2.$$
For simplicity, we identify $R$ with its image $F_p[t^2,t^3]$ under this map and thus consider $R$ as a subring of $\F_p[t]$. Using this description, we will see that the normalization map $R\to\ov{R}$ induces an isomorphism $\W\Omega_R^{\bullet}\xto{\sim}\W\Omega_{\ov{R}}^{\bullet}$. This has two important consequences.
\begin{itemize}
\item The canonical map $W\Omega_R^{\bullet}\to\W\Omega_R^{\bullet}$ is not an isomorphism (and need not even be a qis).

\item $\W\Omega_R^{\bullet}$ is not isomorphic to $R\Gamma_{\crys}(\Spec R)$ in $D(\Z)$ and so cannot be used to compute crystalline cohomology in the same way as $W\Omega_R^{\bullet}$.
\end{itemize}

As promised, here is our first main result.

\begin{proposition}
The inclusion $R\inj\F_p[t]$ induces an isomorphism $\W\Omega_R^{\bullet}\xto{\sim}\W\Omega_{\F_p[t]}^{\bullet}$.
\end{proposition}

\begin{proof}
Let $\twid{R}:=\Z_p[t^2,t^3]\subset\Z_p[t]$, which is $p$-torsion-free and satisfies $R\iso\twid{R}/p\twid{R}$. Equip $\Z_p[t]$ with the grading in which $t$ is homogeneous of degree $1$, and equip $\twid{R}$ with the induced grading. $\Z_p[t]$ is naturally a good ring with $\phi: \Z_p[t]\to\Z_p[t]$ the $\Z_p$-algebra map given by $t\mapsto t^p$. Identifying $\phi$ with its restriction to $\twid{R}$, the pair $(\twid{R},\phi)$ is then a good ring and the inclusion $(\twid{R},\phi)\inj(\Z_p[t],\phi)$ is a graded map of good rings that is an isomorphism in degrees $\neq1$. By the subsequent lemma (statement and proof soon), the natural map $\Omega_{\twid{R}}^{\bullet}\to\Omega_{\Z_p[t]}^{\bullet}$ induces an isomorphism $\Sat(\Omega_{\twid{R}}^{\bullet})\xto{\sim}\Sat(\Omega_{\Z_p[t]}^{\bullet})$ and hence an isomorphism $\W\Sat(\Omega_{\twid{R}}^{\bullet})\xto{\sim}\W\Sat(\Omega_{\Z_p[t]}^{\bullet})$. Recall that, given $S$ a good ring, there exists a map $\w{\Omega}_S^{\bullet}\to\W\Omega_{S/pS}^{\bullet}$ of Dieudonn\'{e} algebras (unique up to certain compatibility) inducing an isomorphism $\W\Sat(\w{\Omega}_S^{\bullet})\xto{\sim}\W\Omega_{S/pS}^{\bullet}$, where $\W\Omega_{S/pS}^{\bullet}=\W\Sat(\Omega_{W(S/pS)}^{\bullet})$ assuming $S/pS$ is reduced. In our setting, this gives
$$\Omega_{\twid{R}}^{\bullet}\iso\w{\Omega}_{\twid{R}}^{\bullet}\to\W\Omega_R^{\bullet}\leadsto\W\Sat(\Omega_{\twid{R}}^{\bullet})\xto{\sim}\W\Omega_R^{\bullet}$$
and
$$\Omega_{\Z_p[t]}^{\bullet}\iso\w{\Omega}_{\Z_p[t]}^{\bullet}\to\W\Omega_{\F_p[t]}^{\bullet}\leadsto\W\Sat(\Omega_{\Z_p[t]}^{\bullet})\xto{\sim}\W\Omega_{\F_p[t]}^{\bullet}$$
We then have a commutative diagram
\begin{center}
\begin{tikzcd}
\W\Sat(\Omega_{\twid{R}}^{\bullet}) \arrow[r, "\sim"] \arrow[d, "\iso"'] & \W\Sat(\Omega_{\Z_p{[}t{]}}^{\bullet}) \arrow[d, "\iso"] \\
\W\Omega_R^{\bullet} \arrow[r] & \W\Omega_{\F_p{[}t{]}}^{\bullet}
\end{tikzcd}
\end{center}
and so the bottom map is an isomorphism.
\end{proof}

Note that the induced map $\Omega_{\twid{R}}^{\bullet}\to\Omega_{\Z_p[t]}^{\bullet}$ is \textbf{not} an isomorphism since, e.g., $dt\in\Omega_{\Z_p[t]}^1$ is not in the image. Fortunately, what matters is that $F^n(dt)$ is in the image for some $n$. In fact, letting $x=t^3,y=t^2$ gives
\begin{align*}
p\geq5:\qquad&F(dt)=t^{p-1}dt=\df{1}{2}xy^{(p-5)/2}dy, \\
p=3:\qquad&F^2(dt)=t^8dt=\df{1}{2}xy^2dy, \\
p=2:\qquad&F^3(dt)=t^7dt=\df{1}{3}xydx.
\end{align*}
Now for the promised lemma.

\begin{lemma}
Let $f: (R,\phi)\to(R',\phi')$ be a graded map of good rings equipped with non-negative gradings.\footnote{Recall that $(R,\phi)$ is a good ring if $R$ is $p$-torsion-free and $\phi$ is a mod $p$ lift of Frobenius. Here, the map $f$ is required to satisfy the compatibility condition $f\circ\phi=\phi'\circ f$.} Suppose that 
$$\phi(R_k)\subset R_{pk},\qquad \phi(R'_k)\subset R'_{pk}$$
and that $f$ induces an isomorphism $R_k\xto{\sim}R'_k$ for $k=0$ and $k\geq N$ for some $N\gg0$. Then, the natural map $\Omega_R^{\bullet}\to\Omega_{R'}^{\bullet}$ induces an isomorphism $\Sat(\Omega_R^{\bullet})\xto{\sim}\Sat(\Omega_{R'}^{\bullet})$.
\end{lemma}

Before we tackle the proof, we revisit the saturation process. Given $M\in\DC$, recall that we first replace $M$ with $M/M[p^{\infty}]$ if necessary and then take 
$$\Sat(M):=\colim(M\xto{\alpha_F}\eta_pM\xto{\eta_p(\alpha_F)}\eta_p(\eta_pM)\xto{\eta_p(\eta_p(\alpha_F))}\cdots).$$
With this in mind, assume that $M$ is $p$-torsion-free and $F$ is injective on $M$, which we can achieve by modding out by $M[F^{\infty}]:=\{x\in M : F^nx=0\textrm{ for }n\gg0\}$. Consider
$$M[F^{-1}]:=\colim(M\xto{F}M\xto{F}\cdots),$$
whose elements look like formal expressions $F^{-n}x$ for $n\in\Z$ and $x\in M$.\footnote{This is very similar to the process of (ordinary) localization, and in fact the two can be placed on equal footing in the appropriate category.} Using this, we have the description
$$\Sat(M)=\{x\in M[F^{-1}] : d(F^nx)\in p^nM\textrm{ for }n\gg0\},$$
which concretely means that if $x=F^{-m}y$ with $m>0$ and $y\in M$ then $d(F^{n-m}y)\in p^nM$ for $n\gg m$. We obtain $F$ on $\Sat(M)$ as the restriction of the induced map $F: M[F^{-1}]\to M[F^{-1}]$, and $d$ looks like $x\mapsto F^{-n}p^{-n}d(F^nx)$ for $n\gg0$. Viewed another way, $d$ extends to a collection of maps
$$\twid{d}: M^{\bullet}[F^{-1}]\to M^{\bullet+1}[F^{-1}]\tensor_{\Z}\Z[1/p],\qquad F^{-n}x\mapsto p^{-n}F^{-n}dx$$
and then 
$$\Sat(M)=\{y=F^{-n}x\in M[F^{-1}] : \twid{d}y\in M[F^{-1}]\}.$$

\begin{proof}
Note first of all that $\Omega_R^{\bullet}$ naturally carries a bigrading with pieces $(\Omega_R^n)_k$ such that, if $x_0,x_1,\ldots,x_n\in R$ are homogeneous of degrees $d_0,d_1,\ldots,d_n$, then $x_0dx_1\wedge\cdots\wedge dx_n$ is homogeneous of degree $d_0+d_1+\cdots+d_n$. Under our assumptions, if $\omega\in\Omega_R^n$ is homogeneous of degree $k$ then $F(w)\in\Omega_R^n$ is naturally homogeneous of degree $pk$. The map $f: (R,\phi)\to(R',\phi')$ induces a map $f^*: \Omega_R^{\bullet}\to\Omega_{R'}^{\bullet}$ of Dieudonn\'{e} algebras and thus a map $\theta: \Omega_R^{\bullet}/\Omega_R^{\bullet}[p^{\infty}]\to\Omega_{R'}^{\bullet}/\Omega_{R'}^{\bullet}[p^{\infty}]$. By the above, it suffices to show that the induced map 
$$\theta[F^{-1}]: (\Omega_R^{\bullet}/\Omega_R^{\bullet}[p^{\infty}])[F^{-1}]\to(\Omega_{R'}^{\bullet}/\Omega_{R'}^{\bullet}[p^{\infty}])[F^{-1}]$$
is an isomorphism or, equivalently, that $\ker\theta$ and $\coker\theta$ are both annihilated by $F^r$ for $r\gg0$.\footnote{Let $\alpha: (M,F)\to(N,G)$ be a map of $R$-modules equipped with endomorphisms, which induces a module map $\twid{\alpha}: M[F^{-1}]\to N[G^{-1}]$. This is injective if and only if $\ker\alpha\subset M[F^{\infty}]$, and surjective if and only if $G^k(N)\subset\im f$ for $k\gg0$.} The first inkling of this comes from looking in ``degree zero,'' where we get an isomorphism $f[F^{-1}]: R[F^{-1}]\xto{\sim} R'[F^{-1}]$ (this uses the fact that $R,R'$ are both $p$-torsion-free). Since $f$ is graded, $\ker f$ is naturally a graded ideal of $R$ -- letting $f_k: R_k\to R'_k$ be the $k$th graded piece of $f$, we have $(\ker f)_k=\ker f_k$. Since $F$ restricts to $\phi$ on $R$ and $\phi'$ on $R'$, we are simply localizing at $\phi$ and $\phi'$. Let $x\in\ker f$ homogeneous of degree $k$. If $k=0$ then $x\in\ker f_0$ and so $x=0$ by assumption. If $k>0$ then choose $r\gg0$ so that $p^rk>N$. Then, $\phi^r(x)$ lands in $\ker f$ and is homogeneous of degree $p^rk$, hence vanishes by assumption. This shows that $f[F^{-1}]$ is injective. A similar argument shows that $f[F^{-1}]$ is surjective. We immediately lift this result to get injectivity of $\theta[F^{-1}]$ using the identity $p^{rn}F^r=(\phi^r)^*$ on $\Omega_R^n$ (where the key is to take $r\gg0$ so that $p^r\geq N$).\footnote{One directly verifies this identity for $n=0,1$ and then the result extends basically by construction.} For surjectivity of $\theta[F^{-1}]$, it suffices to work in degree $1$ and to disregard $p^{\infty}$-torsion. That is, we wish to show that $f$ induces a surjection $\Omega_R^1[F^{-1}]\surj\Omega_{R'}^1[F^{-1}]$ or, equivalently, that the natural map $R'\tensor_R\Omega_R^1\to\Omega_{R'}^1$ is surjective once we localize at $F$. We have a natural exact sequence
\begin{center}
\begin{tikzcd}
R'\tensor_R\Omega_R^1 \arrow[r] & \Omega_{R'}^1 \arrow[r] & \Omega_{R'/R}^1 \arrow[r] & 0
\end{tikzcd}
\end{center}
in $\Mod_{R'}$ and, intuitively, localizing at $F$ is flat so that
\begin{center}
\begin{tikzcd}
(R'\tensor_R\Omega_R^1){[}F^{-1}{]} \arrow[r] & \Omega_{R'}^1{[}F^{-1}{]} \arrow[r] & \Omega_{R'/R}^1{[}F^{-1}{]} \arrow[r] & 0
\end{tikzcd}
\end{center}
is also exact. As noted earlier, $f$ induces an isomorphism $R[F^{-1}]\xto{\sim}R'[F^{-1}]$ and so 
$$\Omega_{R'/R}^1[F^{-1}]\iso\Omega_{R'[F^{-1}]/R[F^{-1}]}^1=0.$$
\end{proof}

\begin{proposition}
Let $R:=\F_p[t^2,t^3]\subset\F_p[t]$. Then, the comparison map $c_R: W\Omega_R^{\bullet}\to\W\Omega_R^{\bullet}$ is not an isomorphism.
\end{proposition}

\begin{proof}
We claim that $W\Omega_R^2\neq0$ while $\W\Omega_R^2=0$. To see that $\W\Omega_R^2=0$, note that 
$$\W\Omega_R^{\bullet}\iso\W\Omega_{\F_p[t]}^{\bullet}\iso\W\Sat(\Omega_{\Z_p[t]}^{\bullet})$$
and the latter complex is concentrated in degrees $0$ and $1$.\footnote{Given any $M\in\DC$, if $M^n=0$ then $\Sat(M)^n=0$. Similarly, if $M\in\DC_{\sat}$ with $M^n=0$ then $\W(M)^n=0$.} To see that $W\Omega_R^2\neq0$, let $x:=t^3,y:=t^2$ so that $\Omega_R^1$ is the $R$-module with generators $dx,dy$ and relation $2xdx=3y^2dy$. This relation disappears if we mod out by $x$ and $y$, which means that $dx,dy$ have linearly independent images in the $\F_p$-vector space $R/(x,y)\tensor_R\Omega_R^1$ and so $dx\wedge dy$ is nonzero as an element of $\Omega_R^2$. Since $W\Omega_R^{\bullet}\surj\Omega_R^{\bullet}$ it follows that $W\Omega_R^2\neq0$.
\end{proof}

In fact, $c_R$ need not even be a qis! To see this, note that we have a commutative diagram
\begin{center}
\begin{tikzcd}
W\Omega_R^{\bullet} \arrow[r, "c_R"] \arrow[d] & \W\Omega_R^{\bullet} \arrow[d, "\iso"] \\
W\Omega_{\F_p{[}t{]}}^{\bullet} \arrow[r, "\sim"] & \W\Omega_{\F_p{[}t{]}}^{\bullet}
\end{tikzcd}
\end{center}
If $c_R$ were a qis then the left vertical map would be a qis and we would have an induced isomorphism 
$$W\Omega_R^{\bullet}\Ltensor_{\Z}\F_p\xto{\sim}W\Omega_{\F_p[t]}^{\bullet}\Ltensor_{\Z}\F_p$$
in $D(\Z)$. This fits into a commutative diagram
\begin{center}
\begin{tikzcd}
W\Omega_R^{\bullet}\Ltensor_{\Z}\F_p \arrow[r] \arrow[d] & W\Omega_{\F_p{[}t{]}}^{\bullet}\Ltensor_{\Z}\F_p \arrow[d, "\iso"] \\
\Omega_R^{\bullet} \arrow[r] & \Omega_{\F_p{[}t{]}}^{\bullet}
\end{tikzcd}
\end{center}
in $D(\Z)$, where the right vertical map is an isomorphism in $D(\Z)$ since $\F_p[t]$ is a smooth $\F_p$-algebra. Hence, the natural map $\Omega_R^{\bullet}\to\Omega_{\F_p[t]}^{\bullet}$ must be a surjection on cohomology and so we should have $d(t^p)=0$ in $\Omega_R^1$ (i.e., we can lift the $0$-cocycle $[t^p]\in H^0(\Omega_{\F_p[t]}^{\bullet})$). This is demonstrably false for $p\leq7$ since 
$$\Omega_R^1\iso(Rdx\oplus Rdy)/R(2xdx-3y^2dy)$$
and
\begin{align*}
d(t^2)&=dy, \\
d(t^3)&=dx, \\
d(t^5)&=d(xy)=ydx+xdy, \\
d(t^7)&=d(xy^2)=y^2dx+2xydy.
\end{align*}

The following shows that $\W\Omega_R^{\bullet}$ is not isomorphic to $R\Gamma_{\crys}(\Spec R)$ in $D(\Z)$. 

\begin{proposition}
$R\Gamma_{\crys}(\Spec R)\Ltensor_{\Z_p}\F_p$ has nonvanishing cohomology in degree $2$.
\end{proposition}

To see that $\W\Omega_R^{\bullet}$ is not isomorphic to $R\Gamma_{\crys}(\Spec R)$ in $D(\Z)$, suppose otherwise. Then, 
\begin{align*}
R\Gamma_{\crys}(\Spec R)\Ltensor_{\Z_p}\F_p
&\simeq\W\Omega_R^{\bullet}\Ltensor_{\Z_p}\F_p \\
&\simeq\W\Omega_R^{\bullet}/p\W\Omega_R^{\bullet} \\
&\simeq\W\Omega_{\F_p[t]}^{\bullet}/p\W\Omega_{\F_p[t]}^{\bullet} \\
&\simeq\Omega_{\F_p[t]}^{\bullet}.
\end{align*}
This is impossible since the latter is supported in cohomological degrees $0$ and $1$.

\begin{proof}
The geometry of the situation is determined by the fact that $R$ is a simple example of a local complete intersection and that $R$ admits a flat lift to $\Z/p^2\Z$ with a compatible lift of Frobenius. It follows that there is a splitting
$$R\Gamma_{\crys}(\Spec R)\Ltensor_{\Z_p}\F_p\simeq\bigoplus_i(\Wedge^i\L_{R/\F_p})[-i],$$
where $\L_{R/\F_p}$ denotes the (derived) cotangent complex of $R/\F_p$. It follows that $H^2(R\Gamma_{\crys}(\Spec R)\Ltensor_{\Z_p}\F_p)$ contains 
$$H^2((\Wedge^2\L_{R/\F_p})[-2])\iso H^0(\Wedge^2\L_{R/\F_p})\iso\Omega_{R/\F_p}^2\neq0$$
as a direct summand and so is nonzero.
\end{proof}
\end{document}

%%%%%

$\CRing$ denotes the category of commutative (unital) rings. Given $R\in\CRing$, $\Mod_R$ denotes the category of (left) $R$-modules, $\Ch(R)$ the category of cochain complexes enriched over $\Mod_R$, and $\CAlg_R$ the category of commutative (associative, left-) $R$-algebras. Fix a prime $p>0$. $\Ch(R)^{\tf}$ denotes the full subcategory of $p$-torsion-free complexes. We write qis as an abbreviation for quasi-isomorphism. $D(R)$ denotes the (classical) derived category formed by formally inverting qis's in $\Ch(R)$. $W(R)$ denotes the ring of Witt vectors over $R$ (relative to $p$), which arises from the Witt endofunctor $W: \CRing\to\CRing$ (that is particularly well-behaved on rings that are either perfect $\F_p$-algebras or $p$-torsion-free). The abbreviation (c)dga is short for (commutative) differential graded algebra.

\section{Introduction}
Let $X$ be an algebraic variety defined over a perfect field $k$ of characteristic $p$. Let $W\Omega_X^{\bullet}$ denote the classical de Rham-Witt complex of $X$, which when $X=\Spec R$ is characterized up to unique isomorphism as the initial object of the category of so-called $R$-framed $V$-pro-complexes. Here are two key results.

\begin{theorem}
Suppose $X$ is smooth. Then, there exists a canonical map $W\Omega_X^{\bullet}\to\Omega_X^{\bullet}$ inducing a qis $W\Omega_X^{\bullet}/pW\Omega_X^{\bullet}\to\Omega_{X/k}^{\bullet}$.
\end{theorem}

\begin{theorem}
Let $\X$ be a smooth formal scheme over $\Spf(W(k))$ with special fiber $X:=\Spec k\times_{\Spf(W(k))}\X$. Suppose that the Frobenius map $\phi_X: X\to X$ extends to a map of formal schemes $\phi_{\X}: \X\to\X$. Then, there exists a natural qis $\Omega_{\X/W(k)}^{\bullet}\to W\Omega_X^{\bullet}$ which depends on the choice of $\phi_{\X}$ but is independent of this choice on the level of derived categories.
\end{theorem}

Our goal is to construct a complex $\W\Omega_X^{\bullet}$, called the saturated de Rham-Witt complex, which agrees with $W\Omega_X^{\bullet}$ when $X$ is smooth and more naturally satisfies the above results. The key comes from extrapolating properties of $\Omega_{R/k}^{\bullet}$ when $X=\Spec R$ is smooth. The construction of $\W\Omega_R^{\bullet}$ will proceed in two stages, the first stage providing us with a relevant Verschiebung map $V$ through a saturation process and the second stage roughly forcing completeness with respect to $V$. In the interest of time, we will skip right to the algebra.

\section{Dieudonn\'{e} Complexes}
\begin{definition}
Let $\DC$ denote the category of \textbf{Dieudonn\'{e} complexes}, which are triples $(M,d,F)$ with $(M,d)$ a cochain complex of abelian groups and $F: M\to M$ the \textbf{Frobenius} map, a homomorphism of graded abelian groups satisfying 
$$dF(x)=pF(dx)$$
for every $x\in M$. In practice, we will often omit $d$ and $F$ when they are clear from context. A morphism $f: (M,d,F)\to(M',d',F')$ of Dieudonn\'{e} complexes is a map of cochain complexes $f: (M,d)\to(M',d')$ such that $F'\circ f=f\circ F$. Note that $\DC$ has a natural symmetric monoidal structure given by tensor product.
\end{definition}

\begin{remark}
The function $F$ is not a map of complexes from $(M/pM,d)$ to itself. It does, however, give a map of complexes from $(M/pM,0)$ to $(M/pM,d)$. This becomes part of the initial inspiration for considering Dieudonn\'{e} complexes when paired with the Cartier isomorphism (to be described later).
\end{remark}

Any $p$-torsion-free complex $(M,d)$ may be regarded as a subcomplex of the localization $M[p^{-1}]$. Given such a complex, define the \textbf{d\'{e}calage} $\eta_pM\subset M[p^{-1}]$ via
$$(\eta_pM)^n:=\{x\in p^nM^n : dx\in p^{n+1}M^{n+1}\}.$$
Given $(M,d,F)\in\DC$ a $p$-torsion-free complex, we have a well-defined cochain map $\alpha_F: M\to\eta_pM$ defined on $n$th components by $x\mapsto p^nF(x)$. Conversely, given $(M,d)$ simply a $p$-torsion-free complex and $\alpha\in\Hom(M,\eta_pM)$, we may define a Frobenius map $F: M\to M$ on $n$th components via $x\mapsto p^{-n}\alpha(x)$. These two constructions are inverse to one another.

\begin{definition}
$(M,d,F)\in\DC$ is \textbf{saturated} if $M$ is $p$-torsion-free and for every $n\in\Z$ the map $F$ induces a group isomorphism $M^n\xto{\sim}\{x\in M^n : dx\in pM^{n+1}\}$ (equivalently, $\alpha_F$ is an isomorphism).\footnote{To get an idea of what is going on here, think of multiplication by $m\in\Z$ inducing an isomorphism of $\Z$ onto its subgroup $m\Z$ and thus an injective map $\Z\inj\Z$.} We obtain a full subcategory $\DC_{\sat}\subset\DC$. Since $F: M\to M$ is necessarily injective with image containing $pM$, we obtain a unique \textbf{Verschiebung} map $V: M\to M$ satisfying $F(Vx)=px$ for every $x\in M$.
\end{definition}

\begin{proposition}
Given $M\in\DC_{\sat}$, the Verschiebung map $V: M\to M$ is injective and satisfies
\begin{itemize}
\item $F\circ V=V\circ F=p\cdot\id$;

\item $F\circ d\circ V=d$;

\item $p\cdot(d\circ V)=V\circ d$.
\end{itemize}
\end{proposition}

\begin{comment}
\begin{proposition}
Let $M\in\DC_{\sat}$ and $r\in\Z^{\geq0}$. Then, $F^r$ induces an isomorphism of graded abelian groups $M^{\bullet}\xto{\sim}\{x\in M^{\bullet} : dx\in p^rM^{\bullet+1}\}$.
\end{proposition}
\end{comment}

\begin{definition}
A morphism $f\in\Hom_{\DC}(M,N)$ \textbf{exhibits $N$ as a saturation of $M$} if $N$ is saturated and, for every $K\in\DC_{\sat}$, $f$ induces a bijection $\Hom_{\DC}(N,K)\xto{\sim}\Hom_{\DC}(M,K)$. If it exists, the data of $(N,f)$ is unique up to unique isomorphism. We call it the \textbf{saturation} of $M$ and write $\Sat(M)$.
\end{definition}

\begin{proposition}
Saturations exist. 
\end{proposition}

\begin{proof}
Let $(M,d,F)\in\DC$. Consider the subcomplex $M[p^{\infty}]\subset M$ defined by 
$$M[p^{\infty}]:=\{x\in M : p^nx=0\textrm{ for }n\gg0\}.$$
Replacing $M$ by $M/M[p^{\infty}]$ if necessary, we may assume WLOG that $M$ is $p$-torsion-free. The obstruction to $M$ being saturated is measured by $\alpha_F: M\to\eta_pM$ failing to be an isomorphism. So, we just force things:
$$\Sat(M):=\colim(M\xto{\alpha_F}\eta_pM\xto{\eta_p(\alpha_F)}\eta_p(\eta_pM)\xto{\eta_p(\eta_p(\alpha_F))}\cdots).$$
Then, $\Sat(M)$ is saturated since $\eta_p$ commutes with filtered colimits.\footnote{Why does $\DC$ admit filtered colimits? By construction, $\DC$ is a non-full subcategory of $\Ch(\Z)$. $\Ch(\Z)$ admits filtered colimits and working termwise with suitable Frobenii induces a suitable Frobenius map on the filtered colimit in $\Ch(\Z)$ that makes it an object of $\DC$.} Moreover, the tautological map $M\to\Sat(M)$ exhibits $\Sat(M)$ as a saturation of $M$.
\end{proof}

\begin{comment}
consider 
$$M[F^{-1}]:=\colim(M\xto{F}M\xto{F}M\xto{F}\cdots),$$
with colimit taken in $\DC$.\footnote{If this feels somewhat alien then consider that we obtain $M[f^{-1}]$ for $M$ an abelian group and $f\in\Z$ by performing the same procedure using the multiplication map induced by $f$.} 
\end{comment}

\begin{corollary}
$\Sat$ is left adjoint to $\DC_{\sat}\inj\DC$.
\end{corollary}

Given $M\in\DC$ and $x\in M$, we have $d(Fx)=pF(dx)$ and so the image of $Fx$ in $M/pM$ is a cycle. Hence, we obtain a graded map $M\to H^{\bullet}(M/pM)$ which necessarily factors through $M/pM$.

\begin{definition}
$M\in\DC_{\sat}$ is of \textbf{Cartier type} if $M$ is $p$-torsion-free and $F$ induces a graded isomorphism $M/pM\xto{\sim}H^{\bullet}(M/pM)$. We will often abbreviate this by simply saying that $M$ is \textbf{Cartier}.
\end{definition}

As it turns out, the Cartier type condition is the right condition to place on $M$ so that we can control the behavior of the saturated de Rham-Witt complex (which is not yet defined). The first inkling of this is the following.

\begin{theorem}[Cartier Criterion]
Let $M$ be Cartier. Then, the canonical map $M\to\Sat(M)$ induces a qis $M/pM\to\Sat(M)/p\Sat(M)$.
\end{theorem}

How do we prove this result? The key is d\'{e}calage. Given $M$ a $p$-torsion-free complex, there is a map $\ov{\gamma}: \eta_pM\to H^{\bullet}(M/pM)$ of graded abelian groups defined on $n$th components by $x\mapsto[p^{-n}x]$. This factors uniquely as 
\begin{center}
\begin{tikzcd}
\eta_pM \arrow[r] & (\eta_pM)/p(\eta_pM) \arrow[r, "\gamma"] & H^{\bullet}(M/pM)
\end{tikzcd}
\end{center}
As we saw in BMS I, $\gamma$ is a qis when $H^{\bullet}(M/pM)$ is equipped with the cochain complex structure coming from the Bockstein operator $\beta: H^{\bullet}(M/pM)\to H^{\bullet+1}(M/pM)$ induced by the short exact sequence
\begin{center}
\begin{tikzcd}
0 \arrow[r] & M/pM \arrow[r, "p"] & M/p^2M \arrow[r] & M/pM \arrow[r] & 0
\end{tikzcd}
\end{center}
From this we conclude that, given $f: M\to N$ a map of $p$-torsion-free complexes that is a qis mod $p$, the induced map $(\eta_pM)/p(\eta_pM)\to(\eta_pN)/p(\eta_pN)$ is a qis. Using the presentation
$$\Sat(M)=\colim(M\xto{\alpha_F}\eta_pM\xto{\eta_p(\alpha_F)}\eta_p(\eta_pM)\xto{\eta_p(\eta_p(\alpha_F))}\cdots),$$
to show that the natural map $M/pM\to\Sat(M)/p\Sat(M)$ is a qis it suffices to show that 
$$(\eta_p^kM)/p(\eta_p^kM)\to(\eta_p^{k+1}M)/p(\eta_p^{k+1}M)$$
is a qis for each $k\geq0$. But by what just showed it suffices to check the case $k=0$, which follows since
\begin{center}
\begin{tikzcd}
M/pM \arrow[r, "\alpha_F"] \arrow[d, "\iso", "F"'] & (\eta_pM)/p(\eta_pM) \arrow[ld, "\gamma", "\textrm{qis}"'] \\
H^{\bullet}(M/pM)
\end{tikzcd}
\end{center}
commutes and so $\alpha_F$ is a qis.

\begin{definition}
Given $M\in\DC_{\sat}$ and $r\in\Z^{\geq0}$, define 
$$\W_r(M):=M/(\im(V^r)+\im(dV^r)),$$
which is a quotient complex of $M$. This comes equipped with natural restriction maps 
$$\Res: \W_{r+1}(M)\to\W_r(M).$$
Using this we define the \textbf{completion} of $M$ to be 
$$\W(M):=\flim_{r\geq0}\W_r(M).$$
\end{definition}

We may uniquely complete each diagram
\begin{center}
\begin{tikzcd}
M \arrow[r, "F"] \arrow[d] & M \arrow[d] \\
\W_r(M) \arrow[r, dotted, "\exists!\,F"'] & \W_{r-1}(M)
\end{tikzcd}
\end{center}
and together these induce the Frobenius $F: \W(M)\to\W(M)$. Similarly, we obtain the Verschiebung $V: \W(M)\to\W(M)$ from unique maps $V: \W_r(M)\to\W_{r+1}(M)$. This naturally defines $\W(M)$ as an object of $\DC$, and the association $M\mapsto\W(M)$ is functorial in $M$.\footnote{Note that $\W_0(M)=0$ and so it contributes nothing to the inverse limit $\W(M)$.} Moreover, there is a canonical map $\rho_M: M\to\W(M)$ of Dieudonn\'{e} complexes that is itself functorial in $M$. 

\begin{definition}
$M\in\DC_{\sat}$ is \textbf{strict} if $\rho_M$ is an isomorphism. We obtain a full subcategory $\DC_{\str}\subset\DC_{\sat}$.
\end{definition}

Given $M\in\DC_{\sat}$, $\W(M)$ is $p$-adically complete.\footnote{In fact, we have the stronger condition that each of the terms of $\W(M)$ is itself $p$-adically complete. Technically speaking, $p$-adic completion of general objects in $\Ch(\Z)$ only works this way when the object in question is $p$-torsion-free.} Hence, if $M$ is strict then $M$ itself is $p$-adically complete. It follows that $M$ is $\l$-torsion-free for $\l\neq p$ and thus that $M$ is torsion-free since being $p$-torsion-free is part of being saturated.

\begin{example}
Regard $M\in\Mod_{\Z}$ as a complex concentrated in degree zero. Then, any endomorphism $F: M\to M$ endows $M$ with the structure of a Dieudonn\'{e} complex. Using this, we deduce that $M$ is saturated if and only if $M$ is $p$-torsion-free and $F$ is an automorphism. Assuming $M$ is saturated, $M$ is strict if and only if it is $p$-adically complete.
\end{example}

It isn't hard to see that $\W(M)$ is saturated given $M\in\DC_{\sat}$. In fact, though, more is true.

\begin{theorem}
Let $M\in\DC_{\sat}$. Then, $\W(M)\in\DC_{\str}$.
\end{theorem}

Completion also satisfies a useful universal property.

\begin{proposition}
Let $M,N\in\DC_{\sat}$ with $N$ strict. Then, $\rho_M$ induces a natural bijection 
$$\Hom_{\DC}(\W(M),N)\xto{\sim}\Hom_{\DC}(M,N).$$
\end{proposition}

Where this comes from is, given $M\in\DC_{\sat}$ and $r\in\Z^{\geq0}$, $F^r: M\to M$ induces an isomorphism of graded abelian groups $\W_r(M)\xto{\sim}H^{\bullet}(M/p^rM)$ and so the associated quotient map $M/p^rM\to\W_r(M)$ is a qis. From this, we conclude that, given $f\in\Hom_{\DC_{\sat}}(M,N)$, the induced map $M/p^rM\to N/p^rN$ is a qis if and only if the induced map $\W_r(M)\to\W_r(N)$ is an isomorphism. Moreover, we need only check at the $r=1$ level to get equivalence for every $r$.

\begin{corollary}
$\W$ is left adjoint to $\DC_{\str}\inj\DC_{\sat}$.
\end{corollary}

\begin{proposition}
Let $M\in\DC_{\sat}$. 
\begin{enum}{\alph}
\item Given $r\in\Z^{\geq0}$, $\rho_M$ induces a qis $M/p^rM\to\W(M)/p^r\W(M)$.

\item $\rho_M$ exhibits $\W(M)$ as the $p$-completion of $M$ in $D(\Z)$.
\end{enum}
\end{proposition}

Applying the Cartier Criterion, we conclude the following.

\begin{corollary}
Let $M$ be Cartier with each $M^n$ $p$-adically complete. Then, the canonical map $M\to\W\Sat(M)$ is a qis.
\end{corollary}

\section{Dieudonn\'{e} Algebras}
\begin{definition}
Let $\DA$ denote the category of \textbf{Dieudonn\'{e} algebras}, which are triples $(A,d,F)$ with $(A,d)$ a cdga\footnote{Recall that this means that $(A,d)$ is a cochain complex with graded ring structure such that multiplication on $A$ is graded-commutative and $d$ satisfies the Leibniz rule. Additionally, we require that if $x\in A^n$ is homogeneous of odd degree then $x^2=0$ in $A^{2n}$.} and \textbf{Frobenius} $F: M\to M$ a homomorphism of graded abelian groups such that 
\begin{itemize}
\item $A^n=0$ for $n>0$;

\item given $x\in A^0$, $Fx\equiv x^p\pmod{p}$;

\item given $x\in A$, $dF(x)=pF(dx)$.
\end{itemize}
Morphisms in $\DA$ are cochain maps with the expected compatibilities.
\end{definition}

Roughly speaking, we think of Dieudonn\'{e} algebras as Dieudonn\'{e} complexes equipped with a ring structure compatible with Frobenius and the differential. Formally, we have a forgetful functor $\DA\to\DC$ and may view $A\in\DA$ as a commutative algebra object in $\DC$ such that 
\begin{itemize}
\item $A^n=0$ for $n>0$;

\item given $x\in A^0$, $Fx\equiv x^p\pmod{p}$;

\item given $x\in A$ homogeneous of odd degree, $x^2=0$.
\end{itemize}

\begin{definition}
$A\in\DA$ is \textbf{saturated} if it saturated as a Dieudonn\'{e} complex. In this case, $\W(A)\in\DC_{\str}$ is naturally a Dieudonn\'{e} algebra and the tautological map $\rho_A: A\to\W(A)$ is a morphism in $\DA$. We say that $A$ is \textbf{strict} if $\rho_A$ is an isomorphism in $\DA$, which holds if and only if $A$ is strict regarded as an object of $\DC$. From this we obtain full subcategories $\DA_{\str}\subset\DA_{\sat}\subset\DA$.
\end{definition}

As with Dieudonn\'{e} complexes, saturations and completions of Dieudonn\'{e} algebras make sense and exist (and have the expected universal properties). In fact, they are obtained by placing suitable algebra structures on the relevant constructions for Dieudonn\'{e} complexes.

\begin{example}
Let $R\in\CAlg_{\F_p}$ and regard $W(R)$ as a cdga concentrated in degree zero. Then, $W(R)$ equipped with its Witt vector Frobenius $F$ is naturally a Dieudonn\'{e} algebra which is saturated if and only if $R$ is perfect. In the saturated case, $W(R)$ is automatically strict. Recall that, in the case that $R$ is perfect, $W(R)$ is $p$-adically complete with canonical isomorphism $W(R)/pW(R)\xto{\sim}R$ and satisfies the universal property that, given $p$-adically complete $A\in\CRing$ equipped with a ring map $R\to A/pA$, we may uniquely complete the diagram
\begin{center}
\begin{tikzcd}
W(R) \arrow[r, "\exists!"] \arrow[d, twoheadrightarrow] & A \arrow[d, twoheadrightarrow] \\
R \arrow[r] & A/pA
\end{tikzcd}
\end{center}
\end{example}

Fix $p$-torsion-free $R\in\CRing$. We will see shortly that $\Omega_R^{\bullet}$ is naturally a Dieudonn\'{e} algebra (under an additional constraint). Before that, we briefly discuss the notion of $\delta$-rings. 

\begin{definition}
Let $\CRing_{\delta}$ denote the category of \textbf{(commutative) $\delta$-rings}, which are pairs $(A,\delta)$ with $A\in\CRing$ and $\delta: A\to A$ a \textbf{$p$-derivation} satisfying
\begin{itemize}
\item $\delta(xy)=x^p\delta(y)+\delta(x)y^p+p\delta(x)\delta(y)$;

\item $\delta(x+y)=\delta(x)+\delta(y)-\displaystyle\sum_{0<i<p}\df{(p-1)!}{i!(p-i)!}x^iy^{p-i}$;\footnote{This forces $\delta(0)=0$. }

\item $\delta(1)=0$
\end{itemize}
for all $x,y\in A$. A morphism $f: (A,\delta_A)\to(B,\delta_B)$ in $\CRing_{\delta}$ is a ring map $f: A\to B$ such that $f\circ\delta_A=\delta_B\circ f$.
\end{definition}

Given $(A,\delta)\in\CRing_{\delta}$, the map
$$\phi: R\to R,\qquad x\mapsto x^p+p\delta(x)$$
is a mod $p$ lift of Frobenius. Conversely, given $p$-torsion-free $A\in\CRing$ with $\phi: A\to A$ a mod $p$ lift of Frobenius, the map
$$\delta: A\to A,\qquad x\mapsto\df{\phi(x)-x^p}{p}$$
is a $p$-derivation on $A$ and so $(A,\delta)\in\CRing_{\delta}$.\footnote{Note that this is not a full converse since we had to impose a $p$-torsion-free hypothesis. There is a way around this.}

Returning to $\Omega_R^{\bullet}$, suppose that we have chosen $\phi: R\to R$ a mod $p$ lift of Frobenius (which need not exist in general). Then, as above we have an associated $p$-derivation $\delta_{\phi}: R\to R$. 

\begin{proposition}
There exists a unique ring homomorphism $F: \Omega_R^{\bullet}\to\Omega_R^{\bullet}$ such that
\begin{itemize}
\item $F=\phi$ on $R=\Omega_R^0$;

\item given $x\in R$,
$$F(dx)=\underbrace{x^{p-1}dx+d\delta_{\phi}(x)}_{=:\rho_{\phi}(x)}.$$
\end{itemize}
Moreover, the triple $(\Omega_R^{\bullet},d,F)$ is a Dieudonn\'{e} algebra.
\end{proposition}

\begin{proof}
Uniqueness is clear since $\Omega_R^{\bullet}$ is generated by elements of the form $x$ and $dx$ for $x\in R$. The main idea behind proving existence is to obtain $F$ using the universal property of $\Omega_R^{\bullet}$. The first step is to verify that $\rho_{\phi}: R\to\Omega_R^1$ is a group homomorphism and a $\phi$-linear derivation in the sense that
$$\rho_{\phi}(xy)=\phi(y)\rho_{\phi}(x)+\phi(x)\rho_{\phi}(y)$$
for all $x,y\in R$. The universal property of $\Omega_R^{\bullet}$ then gives that there is a unique $\phi$-semilinear map $F: \Omega_R^1\to\Omega_R^1$ such that $F\circ d=\rho_{\phi}$. This then extends uniquely to a ring homomorphism $F: \Omega_R^{\bullet}\to\Omega_R^{\bullet}$ satisfying the desired properties. To see that $(\Omega_R^{\bullet},d,F)$ is a Dieudonn\'{e} algebra, the key is to show that the graded subgroup
$$A:=\{\omega\in\Omega_R^{\bullet} : dF(\omega)=pF(d\omega)\}$$
is closed under multiplication and contains $x$ and $dx$ given $x\in R$, hence is equal to all of $\Omega_R^{\bullet}$.
\end{proof}

Unless otherwise stated, we will hereafter assume that $\Omega_R^{\bullet}$ is equipped with the Dieudonn\'{e} algebra structure described by the proposition. This structure possesses a useful universal property.

\begin{proposition}
Given $p$-torsion-free $A\in\DA$, the restriction map
$$\Hom_{\DA}(\Omega_R^{\bullet},A)\to\Hom_{\CRing}(R,A^0)$$
is injective with image $\{f\in\Hom_{\CRing}(R,A^0) : f\circ\phi=F_A\circ f\textrm{ on }R\}$.
\end{proposition}

\begin{remark}
The Dieudonn\'{e} algebra structure we placed on $\Omega_R^{\bullet}$ depends on the choice of mod $p$ Frobenius lift $\phi$, so let's use the notation $\Omega_{R,\phi}^{\bullet}$ to emphasize this fact. Given another mod $p$ Frobenius lift $\phi'$, the above result says that restriction induces
$$\Hom_{\DA}(\Omega_{R,\phi}^{\bullet},\Omega_{R,\phi'}^{\bullet})\inj\Hom_{\CRing}(R,R)$$
with image $\{f : f\circ\phi=\phi'\circ f\}$. This provides concrete means to compare $\Omega_{R,\phi}^{\bullet}$ and $\Omega_{R,\phi'}^{\bullet}$ as Dieudonn\'{e} algebras.
\end{remark}

In order to write down the Cartier isomorphism, we need to expand on the above construction. Fix $R\in\CRing$. The \textbf{completed de Rham complex} of $R$ is 
$$\w{\Omega}_R^{\bullet}:=\flim_n\Omega_R^{\bullet}/p^n\Omega_R^{\bullet}\iso\flim_n\Omega_{R/p^nR}^{\bullet},$$
which naturally satisfies $\w{\Omega}_R^{\bullet}/p^n\w{\Omega}_R^{\bullet}\iso\Omega_{R/p^nR}^{\bullet}$ and is $p$-adically complete. The completed de Rham complex admits a unique multiplication such that the tautological map $\Omega_R^{\bullet}\to\w{\Omega}_R^{\bullet}$ is a homomorphism of dga's. Return now to the assumption that $R$ is $p$-torsion-free with $\phi$ a choice of mod $p$ lift of Frobenius (which we encapsulate by saying that $(R,\phi)$ or simply $R$ is \textbf{good}). Then, the morphism $F: \Omega_R^{\bullet}\to\Omega_R^{\bullet}$ induces $F: \w{\Omega}_R^{\bullet}\to\w{\Omega}_R^{\bullet}$ endowing $\w{\Omega}_R^{\bullet}$ with the structure of a Dieudonn\'{e} algebra that has a universal property similar to the one for $\Omega_R^{\bullet}$.

\begin{proposition}
Given $p$-torsion-free and $p$-adically complete $A\in\DA$, the restriction map
$$\Hom_{\DA}(\Omega_R^{\bullet},A)\to\Hom_{\CRing}(R,A^0)$$
is injective with image $\{f\in\Hom_{\CRing}(R,A^0) : f\circ\phi=F_A\circ f\textrm{ on }R\}$.
\end{proposition}

Consider now the following.

\begin{proposition}
Let $A\in\CAlg_{\F_p}$. Then, there exists a unique homomorphism of graded algebras $\Cart=\Cart_A: \Omega_A^{\bullet}\to H^{\bullet}(\Omega_A^{\bullet})$, called the \textbf{Cartier map}, such that 
$$\Cart(x)=[x^p],\qquad \Cart(dy)=[y^{p-1}dy]$$
for all $x,y\in A$.
\end{proposition}

This map is necessarily given by 
$$x_0dx_1\wedge\cdots\wedge dx_n\mapsto[x_0^p(x_1^{p-1}dx_1)\wedge\cdots\wedge(x_n^{p-1}dx_n)]$$
for $x_0,x_1,\ldots,x_n\in R$.

\begin{theorem}[Cartier Isomorphism]
Let $k\in\CAlg_{\F_p}$ be perfect and $A\in\CAlg_k$ smooth. Then, $\Cart: \Omega_A^{\bullet}\to H^{\bullet}(\Omega_A^{\bullet})$ is an isomorphism.\footnote{This is a nontrivial result that requires some work done later in the BLM paper.}
\end{theorem}

What does this have to do with our previous setting? The map $F: \w{\Omega}_R^{\bullet}\to\w{\Omega}_R^{\bullet}$ induces the composition
\begin{center}
\begin{tikzcd}
\Omega_{R/pR}^{\bullet} \arrow[r, "\sim"] & \w{\Omega}_R^{\bullet}/p\w{\Omega}_R^{\bullet} \arrow[r, "F"] & H^{\bullet}(\w{\Omega}_R^{\bullet}/p\w{\Omega}_R^{\bullet}) \arrow[r, "\sim"] & H^{\bullet}(\Omega_{R/pR}^{\bullet})
\end{tikzcd}
\end{center}
One can check that this agrees with the Cartier map and so we see that the composition is independent of the choice of $\phi$.

\begin{corollary}
Let $R\in\CRing$ be good and suppose there exists $k\in\CAlg_{\F_p}$ perfect such that $R/pR$ is a smooth $k$-algebra. Then, $\w{\Omega}_R^{\bullet}$ is Cartier and the canonical map $\w{\Omega}_R^{\bullet}\to\W\Sat(\w{\Omega}_R^{\bullet})$ is a qis.
\end{corollary}

The only non-obvious part of the above result is that $\w{\Omega}_R^{\bullet}$ is $p$-torsion-free, which follows from the fact that each $\w{\Omega}_R^i$ is a projective module of finite rank over the completion $\w{R}=\flim_nR/p^nR$.

\section{The Saturated de Rham-Witt Complex}
We begin by elaborating on the universal property of Witt vectors.

\begin{proposition}
Let $A\in\DA_{\str}$ and $R:=A^0/VA^0$. Then, there exists a unique ring isomorphism $u: A^0\xto{\sim}W(R)$ compatible with the identification $R=A^0/VA^0$ and Frobenius -- i.e., the diagrams 
\begin{center}
\begin{tikzcd}
A^0 \arrow[r, "u"] \arrow[d, twoheadrightarrow] & W(R) \arrow[d, twoheadrightarrow] \\
VA^0 \arrow[r, "\id"'] & R
\end{tikzcd}
\end{center}
and
\begin{center}
\begin{tikzcd}
A^0 \arrow[r, "u"] \arrow[d, "F"'] & W(R) \arrow[d, "F"] \\
A^0 \arrow[r, "u"'] & W(R)
\end{tikzcd}
\end{center}
commute.
\end{proposition}

Note that $A^0/VA^0$ is automatically reduced for $A\in\DA_{\sat}$.

\begin{proposition}
Let $A\in\DA_{\str}$, $R\in\CAlg_{\F_p}$, and $f_0\in\Hom_{\CAlg_{\F_p}}(R,A^0/VA^0)$. Then, there exists a unique ring map $f: W(R)\to A^0$ such that the diagrams
\begin{center}
\begin{tikzcd}
W(R) \arrow[r, "f"] \arrow[d, twoheadrightarrow] & A^0 \arrow[d, twoheadrightarrow] \\
R \arrow[r, "f_0"'] & A^0/VA^0
\end{tikzcd}
\end{center}
and
\begin{center}
\begin{tikzcd}
W(R) \arrow[r, "f"] \arrow[d, "F"'] & A^0 \arrow[d, "F"] \\
W(R) \arrow[r, "f"'] & A^0
\end{tikzcd}
\end{center}
commute.
\end{proposition}

The existence of $f$ comes from considering $W(f_0): W(R)\to W(A^0/VA^0)$ and using the strictness of $A$ to get $W(A^0/VA^0)\iso A^0$.

\begin{definition}
Let $A\in\DA_{\str}$ and $R\in\CAlg_{\F_p}$. We say that $f\in\Hom_{\CAlg_{\F_p}}(R,A^0/VA^0)$ \textbf{exhibits $A$ as a saturated de Rham-Witt complex} of $R$ if, given $B\in\DA_{\str}$, 
$$\Hom_{\DA}(A,B)\xto{\sim}\Hom_{\CAlg_{\F_p}}(R,B^0/VB^0).$$
The pair $(A,f)$ is unique up to unique isomorphism if it exists, so we call $A$ the \textbf{saturated de Rham-Witt complex} of $R$ and write $\W\Omega_R^{\bullet}$.
\end{definition}

\begin{proposition}
Saturated de Rham-Witt complexes exist!
\end{proposition}

In more detail, we construct $\W\Omega_R^{\bullet}$ by first reducing to the case that $R$ is reduced and then taking $\W\Sat(\Omega_{W(R)}^{\bullet})$.

\begin{proposition}
Let $R\in\CRing$ be good and $A\in\DA_{\str}$. Then,
$$\Hom_{\DA}(\w{\Omega}_R^{\bullet},A)\xto{\sim}\Hom_{\CAlg_{\F_p}}(R,A^0/VA^0).$$
\end{proposition}
\end{document}