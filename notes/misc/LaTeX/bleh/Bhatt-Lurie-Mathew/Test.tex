\documentclass[11pt]{article}

\usepackage{kernel_of_truth}
\usepackage{normal_setup}

\newcommand{\DA}{\mathsf{DA}} % Dieudonn\'{e} algebra(s)
\newcommand{\DC}{\mathsf{DC}} % Dieudonn\'{e} complex(es)
\newcommand{\free}{\textup{free}} % free
\newcommand{\Isom}{\textup{Isom}} % isomorphism(s)
\newcommand{\pf}{\textup{pro-free}} % free
\newcommand{\sat}{\textup{sat}} % saturated
\newcommand{\Sat}{\textup{Sat}} % saturation
\newcommand{\str}{\textup{str}} % strict

\renewcommand{\L}{\mathbb{L}}
\renewcommand{\phi}{\varphi}

\begin{document}
Let $\Ch(\Z)$ denote the category of cochain complexes of abelian groups, $\Ch(\Z)^{\free}$ and $\Ch(\Z)^{\tf}$ the full subcategories of objects whose entries are free and $p$-torsion-free, and $D(\Z)$ the derived category. Since $\Z$ has projective dimension $1$, this is given by
$$D(\Z)\simeq h\Ch(\Z)^{\free}\simeq\Ch(\Z)^{\tf}[\textrm{qis}^{-1}].$$
The category $h\Ch(\Z)^{\free}$ is the homotopy category of $\Ch(\Z)^{\free}$, whose objects are $X,Y\in\Ch(\Z)^{\free}$ and morphisms are cochain homotopy classes of maps in $[X,Y]$. Note that our construction of the derived category requires choosing free or $p$-torsion-free resolutions for general cochain complexes.

\begin{remark}
Let's talk a bit about homotopy. The abelian category $\Ch(\Z)$ comes equipped with a natural tensor product $\tensor$ giving it the structure of a symmetric monoidal category. There is an internal Hom functor $\un{\Hom}: \Ch(\Z)^{\op}\times\Ch(\Z)\to\Ch(\Z)$ which of course satisfies
$$\Hom(Z\tensor X,Y)\iso\Hom(Z,\un{\Hom}(X,Y))$$
functorial in $X,Y,Z\in\Ch(\Z)$.\footnote{$\un{\Hom}$ is sometimes referred to as the \textbf{mapping class group}.} Explicitly, 
$$\un{\Hom}^n(X,Y)=\prod_{i\in\Z}\Hom(X^i,Y^{i+n})$$
with differential $df=d_Y\circ f-(-1)^nf\circ d_X$ for $f$ homogeneous of degree $n$.\footnote{Be warned that I might be confusing homological and cohomological conventions.} One of the nice things about $\un{\Hom}$ is that it captures homotopy. There is a natural group isomorphism
$$\Hom(X,Y)\iso Z^0(\un{\Hom}(X,Y))=\{f\in\un{\Hom}^0(X,Y) : df=0\}$$
that identifies the nullhomotopies of a fixed $f\in\Hom(X,Y)$ with 
$$B^0(\un{\Hom}(X,Y))=\{h\in\un{\Hom}^{-1}(X,Y) : dh=f\}.$$ 
It follows that $H^0(\un{\Hom}(X,Y))\iso[X,Y]$ and taking higher cohomology captures higher homotopy. Also note that if $X,Y\in\Mod_{\Z}$ then $\un{\Hom}(X,Y)$ is just $\Hom(X,Y)$ concentrated in degree $0$.\footnote{Hence, $\un{\Hom}$ extends the inner $\Hom$ on $\Mod_{\Z}$ (which is represented by the na\"{i}ve $\Hom$ that is automatically enriched over $\Z$). Note that $\un{\Hom}$ gives rise to $\un{\Ext}^i$ on $\Ch(\Z)$. For $X,Y\in\Mod_{\Z}$ it seems reasonable that we would have $\un{\Ext}^i(X,Y)$ is just $\Ext^i(X,Y)$ in degree $0$. More generally I expect there to be some spectral sequence relating the two notions.}
\end{remark}

\begin{definition}
\textbf{Classical $p$-completion} is the functor
$$\w{\cdot}: \Mod_{\Z}\to\Mod_{\Z},\qquad X\mapsto\flim_{n\geq1}X/p^nX.$$
We say $X\in\Mod_{\Z}$ is \textbf{classically $p$-complete} if the natural map $X\to\w{X}$ is an isomorphism. On a somewhat related note, $X\in D(\Z)$ is \textbf{derived $p$-complete} if $\Hom_{D(\Z)}(Y,X)=0$ for every $Y\in D(\Z)$ such that $p: Y\xto{\sim}Y$. Such objects span a full subcategory $D_p(\Z)\subset D(\Z)$.
\end{definition}

\begin{proposition}
The inclusion $D_p(\Z)\inj D(\Z)$ admits a left adjoint $\w{\cdot}: D(\Z)\to D_p(\Z)$ called the \textbf{derived $p$-completion} given by choosing a representative in $\Ch(\Z)$ and applying classical $p$-completion in each degree.
\end{proposition}

We extend derived notions to $\Mod_{\Z}$ by thinking of abelian groups as complexes concentrated in degree $0$. Given $X\in\Mod_{\Z}$, the classical $p$-completion of $X$ represents the derived $p$-completion of $X$ and so we may identify the two.

\begin{proposition}
Let $X\in\Mod_{\Z}$. 
\begin{enum}{\alph}
\item $X$ is derived $p$-complete if and only if $\Hom(\Z[p^{-1}],X)$ and $\Ext^1(\Z[p^{-1}],X)$ are both contractible. This holds if and only if every short exact sequence
\begin{center}
\begin{tikzcd}
0 \arrow[r] & X \arrow[r] & M \arrow[r] & \Z{[}p^{-1}{]} \arrow[r] & 0
\end{tikzcd}
\end{center}
admits a unique splitting.

\item $X$ is classically $p$-complete if and only if it is $p$-adically separated and derived $p$-complete.

\item $X$ is \textbf{pro-free} (i.e., the $p$-completion of a free abelian group) if and only if it is derived $p$-complete and $p$-torsion-free.
\end{enum}
\end{proposition}

\begin{remark}
I'm not entirely sure about the content of the above proposition. I know that $\Ext^1(\Z[p^{-1}],X)$ classifies extensions of $\Z[p^{-1}]$ by $X$, with the zero element corresponding to the trivial extension. Any splitting is by definition an extension isomorphic to the trivial extension as a short exact sequence. So it makes sense that if $\Ext^1$ vanishes then there is a unique splitting (up to ismorphism). But what if there is only a weak equivalence to $0$?
\end{remark}

Complexes of pro-free abelian groups span a full subcategory $\Ch(\Z)^{\pf}\subset\Ch(\Z)$. The following result shows that this subcategory lets us get at $D_p(\Z)$.

\begin{theorem}
The functor $\Ch(\Z)^{\pf}\to D(\Z)$ obtained by passing to (formal) qis classes has essential image $D_p(\Z)$ and induces an equivalence $h\Ch(\Z)^{\pf}\xto{\sim}D_p(\Z)$. In more detail, given $X,Y\in\Ch(\Z)^{\pf}$, $\Hom_{\Ch(\Z)}(X,Y)\surj\Hom_{D(\Z)}(X,Y)$ and $f,g\in\Hom_{\Ch(\Z)}(X,Y)$ have the same image if and only if $f\simeq g$.
\end{theorem}

Our goal now is to start tying in fixed points. Our first stop is deriving $\eta_p$.

\begin{proposition}
There is an essentially unique functor $L\eta_p: D(\Z)\to D(\Z)$ such that
\begin{center}
\begin{tikzcd}
\Ch(\Z)^{\tf} \arrow[r, "\eta_p"] \arrow[d] & \Ch(\Z)^{\tf} \arrow[d] \\
D(\Z) \arrow[r, dotted, "\exists!\; L\eta_p"'] & D(\Z)
\end{tikzcd}
\end{center}
commutes up to natural isomorphism.\footnote{In particular, there is no nontrivial homotopical coherence introduced at this stage in the game.}
\end{proposition}

\begin{definition}
Let $\mc{C}$ be a category and $T: \mc{C}\to\mc{C}$ an endofunctor. The \textbf{fixed point} category $\mc{C}^T$ of $\mc{C}$ with respect to $T$ is the category whose objects are pairs $(X,\phi)$ with $X\in\mc{C}$ and $\phi\in\Isom_{\mc{C}}(X,TX)$. The data of a morphism $f: (X,\phi)\to(X',\phi')$ is $f\in\Hom_{\mc{C}}(X,X')$ such that
\begin{center}
\begin{tikzcd}
X \arrow[r, "f"] \arrow[d, "\phi"'] & X' \arrow[d, "\phi'"] \\
TX \arrow[r, "Tf"'] & TX'
\end{tikzcd}
\end{center}
commutes.
\end{definition}

Basically by definition, we immediately see that there is an equivalence $\DC_{\sat}\simeq(\Ch(\Z)^{\pf})^{\eta_p}$. In fact, more is true.

\begin{theorem}
$\DC_{\sat}\simeq(\Ch(\Z)^{\pf})^{\eta_p}$ restricts to an equivalence $\DC_{\str}\xto{\sim}D_p(\Z)^{L\eta_p}$.
\end{theorem}


\end{document}