\documentclass[11pt]{article}

\usepackage{kernel_of_truth}
\usepackage{normal_setup}

\renewcommand{\AA}{\mathcal{A}}
\newcommand{\BB}{\mathcal{B}}
\newcommand{\CC}{\mathcal{C}}
\newcommand{\EE}{\mathcal{E}}
\newcommand{\FF}{\mathcal{F}}
\newcommand{\G}{\mathbb{G}} % multiplicative group scheme
\newcommand{\GG}{\mathcal{G}}
\newcommand{\HH}{\mathcal{H}}
\newcommand{\J}{\mathcal{J}}
\renewcommand{\P}{\mathbb{P}}
\renewcommand{\phi}{\varphi}

\DeclareMathOperator{\DDelta}{\mathsf{\Delta}} % simply category
\DeclareMathOperator{\DK}{DK} % Dold-Kan (as in Dold-Kan correspondence)
\DeclareMathOperator{\iinj}{inj} % injective

\begin{document}
\title{Arithmetic Intersection Theory}
\author{Zachary Gardner}
\date{\texttt{zachary.gardner@bc.edu}}
\maketitle

The goal of these notes is to develop a solid understanding of arithmetic intersection theory, from the perspective of both theory and application. We use the term ``closed embedding'' instead of ``closed immersion.'' Given $Y,X\in\Sch$, we say that $Y$ is a closed subscheme of $X$ if there is an isomorphism $Y\iso Z$ for $Z\in\Sch$ equipped with a closed embedding $Z\inj X$. We will often ignore this distinction in practice, though we note that several constructions which are non-canonical for closed subschemes become canonical for closed embeddings.

\section{Introduction}
Recall that $X\in\Sch$ is regular if $\O_X(U)$ is regular Noetherian for every $U\in\Aff\Op(X)$; a Noetherian local ring $(A,\m)$ is regular if $\m$ can be generated by $\dim A$ elements. A choice of generators forms a regular sequence in $A$ (I think; the choice of order does matter here). \textcolor{red}{TO DO: This can be stated in a more functorial manner. We are imposing a geometric condition on the closed points of $X$, which themselves encode geometric information. The key is to understand both notions of dimension entirely functorially. Some notion of global dimension enters the picture. Is there a natural way to see why the Noetherian hypothesis enters the picture?}

If $X$ is regular then it is locally Noetherian (by definition). For a general scheme, the structure sheaf need not be coherent (over itself) but this is not an issue for us since the fact that $X$ is locally Noetherian implies that $\FF\in\QCoh(X)$ is coherent if and only if $\FF_f$ is a finitely generated $A$-module for every $(f: \Spec A\to X)\in\Aff\Sch_{/X}$ or, equivalently, $\FF$ is a finitely generated $\O_X$-module. In particular, $\O_X$ is trivially coherent and so the category $\Loc\Free(X)\subset\QCoh(X)$ of finitely generated locally free sheaves on $X$ is naturally a full subcategory of $\Coh(X)$. 

Given a closed subscheme $Y\subset X$, let $\Coh(X,Y)\subset\Coh(X)$ denote the full subcategory of sheaves supported on $Y$, which by definition are sheaves $\FF$ on $X$ such that $\supp(\FF):=\{x\in\abs{X} : \FF_x\neq0\}$ is contained in $Y$. 

\textcolor{red}{\begin{remark}
How do the following notions compare?
\begin{itemize}
\item $\FF$ is supported on $Y$.
\item $\FF$ is acyclic outside $Y$ -- i.e., $\FF$ is cohomologically supported on $Y$ in the sense that $H^i(X\setminus Y,\FF)=0$ for every $i\geq0$.
\item $\FF|_{X\setminus Y}=0$.
\end{itemize}
These seems to be tied to the notion of cohomology with supports in $Y$, given by 
$$\Gamma_Y(X,\FF):=\{s\in\FF(X) : \supp(s)\subset Y\}.$$
\end{remark}}

We have an analogously defined full subcategory $\Loc\Free(X,Y)\subset\Loc\Free(X)$. All categories under consideration are exact, though it's worth noting that $\Loc\Free(X)$ and $\Loc\Free(X,Y)$ are generally not abelian. Applying the Grothendieck group construction $K_0$ produces several groups of interest.
\begin{align*}
K_0(X)&:=K_0(\Loc\Free(X)), \\
G_0(X)&:=K_0(\Coh(X)), \\
K_0(X,Y)&:=K_0(\Loc\Free(X,Y)). \\
\end{align*}
Given $A\in\CRing$ regular Noetherian, we define $K_0(A):=K_0(\Spec A)$.

\textcolor{red}{\begin{remark}
Why not use $\Vect(X)$ in place of $\Loc\Free(X)$? The difference between the two is that objects in the former category are required to have globally constant rank. Passing to connected components shows that this doesn't really matter (I think).
\end{remark}}

Using the above setup, we may define $K_0^Y(X)$ by applying $K_0$ to $\Ch_{\geq0}^b(\Loc\Free(X,Y))$ and then modding out by the subgroup generated by classes of acyclic complexes.

\textcolor{red}{\begin{remark}
How does $K_0(X,Y)$ compare with $K_0^Y(X)$? I think we may identify $K_0(X,Y)$ with the set of classes in $K_0^Y(X)$ represented by complexes homologically concentrated in degree $0$ (this is the homology of a chain complex of sheaves, which has nothing to do with the cohomology of any sheaves in the complex).
\end{remark}}

Recall that, given an exact category $\CC$, $K_0(\CC)$ is the group completion of the set\footnote{For a general exact category passing to isomorphism classes of objects may not yield a set (in technical terms, the skeleton may not be small). We will, however, follow the time-honored tradition of ignoring such set-theoretic issues. Note that this is not an issue in our setting anyway since we do get sets.} of isomorphism classes $[F]$ of objects $F$ in $\CC$, modulo the relation $[F]=[F']+[F'']$ for short exact sequences
\begin{center}
\begin{tikzcd}
0 \arrow[r] & F' \arrow[r] & F \arrow[r] & F'' \arrow[r] & 0
\end{tikzcd}
\end{center}
This formulation accounts for the fact that not every short exact sequence may be split.

\begin{example}
Let $A\in\CRing$ be a local integral domain (e.g., a field). Then, every finitely generated projective $A$-module is locally free hence free and so is classified up to isomorphism by its rank. It follows that $K_0(A)\iso\Z$ obtained as the group completion of $(\Z^{\geq0},+)$.
\end{example}

We have at our disposal two techniques for comparing different Grothendieck groups. The first uses resolutions while the latter, known as d\'{e}vissage, uses filtrations.

\begin{lemma}
Let $\BB\subset\AA$ be additive categories with $\AA$ abelian. Suppose that every object in $\AA$ admits a finite resolution by projective objects in $\BB$. Then, the natural map $K_0(\BB)\to K_0(\AA)$ is a group isomorphism, with inverse given by 
$$K_0(\AA)\to K_0(\BB),\qquad [A]\mapsto\sum_{i=0}^n(-1)^i[P_i]$$
for $P_{\bullet}\to A$ a choice of finite projective resolution in $\BB$.
\end{lemma}

\begin{lemma}
Let $\BB\subset\AA$ be abelian categories with $\BB$ an exact subcategory closed under taking subobjects and quotient objects. Suppose that every object in $\AA$ admits a finite filtration by objects in $\AA$ whose successive quotients are in $\BB$. Then, the natural map $K_0(\BB)\to K_0(\AA)$ is a group isomorphism.
\end{lemma}

\begin{proof}
Let $A\in\AA$. By assumption, there exists a finite filtration 
$$A=A_0\supset A_1\supset\cdots\supset A_{n-1}\supset A_n=0$$
with objects in $\AA$ such that each $A_i/A_{i+1}\in\BB$. Define $\phi: K_0(\AA)\to K_0(\BB)$ by 
$$[A]\mapsto\sum_{i=0}^{n-1}[A_i/A_{i+1}].$$ 
This is evidently an inverse to the desired map. One then checks that $\phi$ is well-defined, the fact that it is a group homomorphism following automatically. To do this, let $\{A_i\}_{0\leq i\leq n}$ and $\{A'_j\}_{0\leq j\leq m}$ be filtrations of $\AA$ of the desired form. Consider the refinements
\begin{align*}
A_{i,j}&:=(A_i\cap A'_j)+A_{i+1}, \\
A'_{j,i}&:=(A'_j\cap A_i)+A'_{j+1},
\end{align*}
which have the property that 
$$A_{i,j}/A_{i,j+1}\iso\df{A_i\cap A'_j}{(A_i\cap A'_{j+1})+(A_{i+1}\cap A'_j)}\iso A'_{j,i}/A'_{j,i+1}.$$
It follows that all subquotients of the latter refinement are isomorphic to some subquotient of the former refinement and vice versa. By induction we see that $\phi$ is well-defined.
\end{proof}

The resolution result is evidently useful because of the natural map $K_0(X)\to G_0(X)$ (called the Cartan map) induced by the inclusion $\Loc\Free(X)\subset\Coh(X)$. 

\begin{theorem}
Let $X\in\Sch$ be regular.\footnote{Do we also need $X$ to be separated?} Then, the natural Cartan map $K_0(X)\to G_0(X)$ is a group isomorphism.
\end{theorem}

\begin{theorem}
Let $X\in\Sch$ be regular and $j: Y\inj X$ a closed embedding. Then, the natural map $K_0^Y(X)\to G_0(Y)$ is a group isomorphism.
\end{theorem}

\begin{remark}
How does $K_0^X(X)$ compare with $K_0(X)$?
\end{remark}

\begin{proof}
Let $\J\normal\O_X$ be the quasicoherent ideal sheaf corresponding to $j: Y\inj X$, so $\J=\ker(\O_X\to j_*\O_Y)$. The natural map $\phi: K_0^Y(X)\to G_0(Y)$ is given by 
$$[\FF_{\bullet}]\mapsto\sum_{i\geq0}(-1)^i\sum_{k\geq0}[\J^k\HH_i(\FF_{\bullet})/\J^{k+1}\HH_i(\FF_{\bullet})],$$
where $\HH_i(\FF_{\bullet})$ is the $i$th homology sheaf of $\FF_{\bullet}$ (which is a complex of sheaves). This is a finite sum since $\J^n$ annihilates $\HH_i(\FF_{\bullet})$ for $n\gg0$ (Why?). The inverse of this map is given by 
$$\psi: G_0(Y)\to K_0^Y(X),\qquad [\EE]\mapsto[\FF_{\bullet}(\EE)],$$
where $\FF_{\bullet}(\EE)$ is a choice of finite locally free resolution of $j_*\EE$ (which exists since $X$ is regular).
\end{proof}

\begin{theorem}
Let $X\in\Sch$ be Noetherian, $Y\subset X$ a closed subscheme, and $U:=X\setminus Y$. 
\begin{enum}{\alph}
\item $\Coh(X,Y)$ is a Serre subcategory of $\Coh(X)$.

\item There is a natural equivalence of categories $\Coh(X)/\Coh(X,Y)\simeq\Coh(U)$.

\item There is a natural exact sequence
\begin{center}
\begin{tikzcd}
G_0(Y) \arrow[r] & G_0(X) \arrow[r] & G_0(U) \arrow[r] & 0
\end{tikzcd}
\end{center}
\end{enum}
\end{theorem}

\begin{lemma}
Let $Y,Z\inj X$ be closed subschemes. Then,
$$K_0^Y(X)\times K_0^Z(X)\to K_0^{Y\cap Z}(X),\qquad ([\FF],[\GG]\mapsto[\FF\tensor\GG])$$
is a well-defined $\Z$-bilinear map.
\end{lemma}

\begin{proof}
The content of this result is that, given $[\FF]\in K_0^Y(X)$ and $[\GG]\in K_0^Z(X)$, each sheaf in the complex $\FF\tensor\GG$ is cohomologically supported on $Y\cap Z$. \textcolor{red}{We evidently need to be careful here since the tensor product of acyclic sheaves is not generally acyclic.}
\end{proof}

\begin{remark}
This should descend to a statement involving $G_0$, which should recover Serre's $\Tor$ intersection formula.
\end{remark}

Let $\pi: X\to S$ be a morphism of separated Noetherian schemes (for simplicity). We wish to discuss for a moment the functoriality of the constructions of $K_0$ and $G_0$. The pullback $\pi^*: \QCoh(S)\to\QCoh(X)$ always makes sense and sends $\Loc\Free(S)$ to $\Loc\Free(X)$. I think this works just as well for coherent sheaves, though we may need flatness. We obtain $\pi^*: K_0(S)\to K_0(X)$.

The pushforward $\pi_*: \QCoh(X)\to\QCoh(S)$ exists under some fairly mild conditions on $\pi$ (e.g., if it is qcqs). If $\pi$ is proper (which is equivalent to a valuative criterion assuming $\pi$ is FT and qs) then $\pi_*$ sends $\Coh(X)$ to $\Coh(S)$ and so we have $\pi_*: G_0(X)\to G_0(S)$. Under our assumptions this induces a map $K_0(X)\to K_0(S)$, though I don't believe this arises from $\pi_*$ acting on $\Loc\Free(X)$.

\begin{proposition}[Projection Formula]
Let $\pi: X\to S$ be as above, $[\FF]\in K_0(S)$, and $[\GG]\in G_0(X)$. Then,
$$\pi_*(\pi^*[\FF][\GG])=[\FF]\pi_*[\GG].$$
\end{proposition}

This is a consequence of the push-pull isomorphism $R^i\pi_*(\pi^*\FF\tensor\GG)\iso\FF\tensor R^i\pi_*\GG$. Before turning our attention to homotopical matters we will compute a few instances of $K_0$ and $G_0$ to see how this theory unfolds in practice.

\begin{example}
\begin{enum}{\arabic}
\item Let $X=\Spec\Z/p^n$ for $p$ prime. Then, $K_0(X)\iso\Z$ with generator $[\Z/p^n]=n[\Z/p]$. Meanwhile, $G_0(X)\iso\Z$ with generator $[\Z/p]$ and isomorphism $[\FF]\mapsto\log_p(\#\FF)$. This shows that the Cartan map need not be an isomorphism in general.

\item Let $X\in\Sch$ be Noetherian with reduction $\pi: X_{\red}\to X$. We know that $\pi$ is a closed embedding with nilpotent associated ideal sheaf $\J$. It follows that $\J^n=0$ for some $n>0$ since $X$ is Noetherian. We have a filtration
$$\FF\supset\J^2\FF\supset\cdots\supset\J^{n-1}\FF\supset\J^n\FF=0$$
whose successive quotients are killed by $\J$ and so live in $\Coh(X_{\red})$. It follows that the natural map $G_0(X_{\red})\to G_0(X)$ induced by $\pi_*: \Coh(X_{\red})\to\Coh(X)$ is an isomorphism. In particular, $G_0(\Z/p)\iso G(\Z/p^n)$ as we saw previously.

\item Let $k$ be a field. We claim that $G_0(\P_k^n)\iso\Z^{n+1}$ with generators given by the classes of $\O,\O(1),\ldots,\O(n)$. One first shows that $G_0(\A_k^n)\iso G_0(\Spec k)\iso\Z$. Choose now a hyperplane in $\P_k^n$, which we identify with $\P_k^{n-1}$ and whose complement we identify with $\A_k^n$. We obtain an exact sequence 
\begin{center}
\begin{tikzcd}
G_0(\P_k^{n-1}) \arrow[r] & G_0(\P_k^n) \arrow[r] & G_0(\A_k^n) \arrow[r] & 0
\end{tikzcd}
\end{center}
and so by induction we see that $G_0(\P_k^n)$ is generated by at most $n+1$ elements. Noting that $G_0(\P_k^n)\iso K_0(\P_k^n)$, consider the function 
$$\phi: K_0(\P_k^n)\times K_0(\P_k^n)\to\Z,\qquad ([\FF],[\GG])\mapsto\chi(\FF\tensor\GG^{\vee})=\sum_{i=0}^n\dim_kH^i(\P_k^n,\FF\tensor\GG^{\vee}),$$
which is well-defined since $\FF,\GG\in\Loc\Free(X)$ are flat. Associated to this is the $(n+1)\times(n+1)$-matrix whose $i,j$-entry is $\phi(\O(i),\O(j))$, which is an upper triangular matrix with $1$'s on the diagonal. It follows that the classes of $\O,\O(1),\ldots,\O(n)$ are $\Z$-linearly independent and generate the entire integral lattice $\Z^{n+1}$.
\end{enum}
\end{example}

\section{Dold-Kan}
Our goal in this section is to understand the statement as well as some of the consequences of Dold-Kan. Let $\DDelta$ denote the simplex category. By definition, $\DDelta$ is the category of totally ordered finite sets with nondecreasing maps. We often identify this category with its skeleton, whose objects look like $[n]:=\{0<1<\cdots<n\}$. Inside $\DDelta$ is the non-full semisimplex subcategory $\DDelta_{\iinj}$ whose morphisms are strictly increasing maps. Given $n\geq1$ and $0\leq i\leq n$, we have functions $\delta^i: [n-1]\inj[n]$ and $\sigma^i: [n+1]\surj[n]$ given by 
\begin{equation*}
\delta^i(j):=
\begin{cases}
j, & j<i, \\
j+1, & j\geq i,
\end{cases}
\end{equation*}
and
\begin{equation*}
\sigma^i(j):=
\begin{cases}
j, & j\leq i, \\
j-1, & j>i.
\end{cases}
\end{equation*}
Note that $\sigma^0: [1]\surj[0]$ makes sense as well.

Given any category $\CC$ we may associate the category of simplicial objects $\CC_{\DDelta}:=\Fun(\DDelta^{\op},\CC)$ and the category of semisimplicial objects $\CC_{\DDelta_{\iinj}}:=\Fun(\DDelta_{\iinj}^{\op},\CC)$ (clearly, every simplicial object determines a unique semisimplicial object by restriction). It is common to denote $\CC_{\DDelta}$ by $\simp\CC$. Given $X\in\CC_{\DDelta}$ and $0\leq i\leq n$, the morphism $\delta^i: [n-1]\to[n]$ induces the $i$th face map $d_i: X_n\to X_{n-1}$ and the morphism $\sigma^i: [n+1]\to[n]$ induces the $i$th degeneracy map $s_i: X_n\to X_{n+1}$ (we also have $s_0: X_0\to X_1$ arising from $\sigma^0: [1]\surj[0]$). Given $\alpha\in\Hom_{\DDelta}([n],[m])$, it is common to write the induced map from $X_m$ to $X_n$ as $X(\alpha)$ or $\alpha^*$.

In its simplest form, Dold-Kan yields an equivalence of categories between $\simp\Mod_A$ and $\Ch_{\geq0}(\Mod_A)$ for $A\in\CRing$. In ``non-additive'' situations, $\simp\Mod_A$ often serves as a better-behaved replacement for $\Ch_{\geq0}(\Mod_A)$. Let $X$ be a semisimplicial $A$-module. To this we may associate the Moore complex $M(X)\in\Ch_{\geq0}(\Mod_A)$ given by 
\begin{equation*}
M(X)_n:=
\begin{cases}
X_n, & n\geq0, \\
0, & n<0,
\end{cases}
\end{equation*}
with differential $\partial: X_n\to X_{n-1}$ given by $\partial:=\sum_{i=0}^n(-1)^id_i$. Inside of this is the degenerate subcomplex $D(X)\subset M(X)$ given for $n\geq0$ by 
$$D(X)_n:=\ip{\im(s_i: X_{n-1}\to X_n) : 0\leq i\leq n-1}.$$
This yields the normalized Moore complex $N(X):=M(X)/D(X)$. At the same time we have the subcomplex $\twid{N}(X)\subset M(X)$ given by $\twid{N}(X)_0:=X_0$ and 
$$\twid{N}(X)_n:=\bigcap_{i=1}^n\ker(d_i: X_n\to X_{n-1})$$
for $n>0$, which satisfies $\partial|_{\twid{N}(X)}=d_0$. All of these constructions are functorial in $X$.

\begin{lemma}
Let $X\in\simp\Mod_A$ and $n\geq0$. Then, the map 
$$\bigoplus_{\alpha: [n]\surj[m]}\twid{N}(X)_m\to X_n,\qquad (x_{\alpha})\mapsto\sum_{\alpha}X(\alpha)(x_{\alpha})$$
is an isomorphism of $A$-modules, where $\alpha$ is nondecreasing and $0\leq m\leq n$.
\end{lemma}

\begin{theorem}
Let $X\in\simp\Mod_A$.
\begin{enum}{\alph}
\item The composition $\twid{N}(X)\inj M(X)\surj N(X)$ is an isomorphism and $M(X)$ splits as the direct sum of $\twid{N}(X)$ and $D(X)$.

\item The quotient map $M(X)\surj N(X)$ is a homotopy equivalence and hence a qis.

\item The inclusion map $\twid{N}(X)\inj M(X)$ is a qis and hence $D(X)$ is acyclic.
\end{enum}
\end{theorem}

\begin{theorem}[Dold-Kan]
The functor $N: \simp\Mod_A\to\Ch_{\geq0}(\Mod_A)$ is an equivalence of categories with quasi-inverse $K: \Ch_{\geq0}(\Mod_A)\to\simp\Mod_A$ given by 
$$K(F)_n:=\Hom_{\Ch_{\geq0}(\Mod_A)}(N(\Delta^n;A),F),$$
where $N(\Delta^n;A):=N(A[\Delta^n])$ and $A[\Delta^n]$ is the free simplicial $A$-module generated by $\Delta^n$ (or, equivalently, by its nondegenerate simplices).
\end{theorem}

The natural isomorphism $\twid{N}\xto{\sim}N$ allows us to make this much more computationally explicit.

\begin{theorem}
The functor $\twid{N}: \simp\Mod_A\to\Ch_{\geq0}(\Mod_A)$ is an equivalence of categories with quasi-inverse $\DK: \Ch_{\geq0}(\Mod_A)\to\simp\Mod_A$ given by 
$$DK(F)_n:=\bigoplus_{\alpha: [n]\surj[m]}F_m,$$
where $\alpha$ is nondecreasing and $0\leq m\leq n$. Moreover, $\twid{N}\circ\DK$ is the identity functor on $\Ch_{\geq0}(\Mod_A)$.
\end{theorem}

Given $\beta\in\Hom_{\DDelta}([q],[n])$ we still need to describe the induced morphism $\DK(F)_n\to\DK(F)_q$. This is the data of maps $\DK(F)_{\alpha,\gamma}: F_m\to F_p$ for every $\alpha: [n]\surj[m]$ and $\gamma: [q]\surj[p]$ with $0\leq m\leq n$ and $0\leq p\leq q$. Only the data of $F$, which carries a differential $\partial^F$, is relevant here. Namely,
\begin{equation*}
\DK(F)_{\alpha,\gamma}:=
\begin{cases}
\id_{F_m}, & p=m, \\
\partial_m^F, & p=m-1\textrm{ and }\alpha\circ\beta=\delta^0\circ\alpha', \\
0, & \textrm{otherwise},
\end{cases}
\end{equation*}
where $\delta^0: [m-1]\inj[m]$ and $\alpha': [q]\surj[m-1]$.

\begin{example}
Let $F\in\Ch_{\geq0}(\Mod_A)$. Our goal is to describe the first few terms of $\DK(F)$. As an $A$-module, we always have 
$$\DK(F)_n=\bigoplus_{i=0}^nF_i^{\oplus\binom{n}{i}}.$$
In particular,
\begin{align*}
\DK(F)_0&=F_0, \\
\DK(F)_1&=F_0\oplus F_1, \\
\DK(F)_2&=F_0\oplus F_1\oplus F_1\oplus F_2.
\end{align*}
Next up is describing the associated face maps. To begin, we have 
\begin{align*}
d_0: F_0\oplus F_1\to F_0,\qquad &(e_0^0,e_1^1)\mapsto e_0^0+\partial_1^F(e_1^1), \\
d_1: F_0\oplus F_1\to F_0,\qquad &(e_0^0,e_1^1)\mapsto e_0^0.
\end{align*}
Continuing, we have 
\begin{align*}
d_0: F_0\oplus F_1\oplus F_1\oplus F_2\to F_0\oplus F_1,\qquad &(e_0^0,e_1^1,e_1^2,e_2^1,)\mapsto, \\
d_1: F_0\oplus F_1\oplus F_1\oplus F_2\to F_0\oplus F_1,\qquad &(e_0^0,e_1^1,e_1^2,e_2^1,)\mapsto, \\
d_2: F_0\oplus F_1\oplus F_1\oplus F_2\to F_0\oplus F_1,\qquad &(e_0^0,e_1^1,e_1^2,e_2^1,)\mapsto, \\
\end{align*}
\end{example}

Let $A\in\CRing$. Given simplicial $A$-modules $X,Y\in\simp\Mod_A$, we obtain a simplicial $A$-module $X\wedge Y\in\simp\Mod_A$ by taking $(X\wedge Y)_n:=X_n\wedge Y_n$ and sending $\alpha\in\Hom_{\DDelta}([n],[m])$ to 
$$X(\alpha)\wedge Y(\alpha): X_m\wedge Y_m\to X_n\wedge Y_n.$$ 
In other words, we transport the bifunctor $\wedge: \Mod_A\times\Mod_A\to\Mod_A$ to $\simp\Mod_A=\Fun(\DDelta^{\op},\Mod_A)$ in the natural way. Let $F,G\in\Ch_{\geq0}(\Mod_A)$. We want to understand $F\wedge G$ and, by induction, $\Wedge^kF$. By definition, $F\wedge G:=\twid{N}(\DK(F)\wedge\DK(G))$.

Given $a\in A$, let $\Kos(a)\in\Ch_{\geq0}(\Mod_A)$ denote the Koszul complex
\begin{center}
\begin{tikzcd}
\cdots \arrow[r] & 0 \arrow[r] & A \arrow[r, "a"] & A \arrow[r] & 0
\end{tikzcd}
\end{center}
concentrated in degrees $0,1$.

\begin{lemma}
Let $a\in A$ and $k\geq1$. Then, 
$$\Wedge^k\Kos(a)\iso\Kos(a)[1-k]$$
\end{lemma}
\end{document}
\begin{proof}

\end{proof}

\section{Results}
\begin{theorem}
Let $X$ be a finite dimensional regular scheme and $Y\subset X$ a closed subscheme. Then, there is a group isomorphism $\CH_Y^p(X)_{\Q}\xto{\sim}\gr^pK_0^Y(X)_{\Q}$ for every $p\geq0$.
\end{theorem}

This is one precise sense in which $K$-theory ``computes'' rational Chow groups.

Given $A\in\CRing$ and $a\in A$, let $\Kos(a)\in\Ch_{\geq0}(\Mod_A)$ denote the Koszul complex
\begin{center}
\begin{tikzcd}
\cdots \arrow[r] & 0 \arrow[r] & A \arrow[r, "a"] & A \arrow[r] & 0
\end{tikzcd}
\end{center}
concentrated in degrees $0,1$.

\begin{theorem}
Let $X\in\Sch$ be regular and $\twid{K}_0(X):=\bigoplus_{Y\inj X}K_0^Y(X)$ defined by summing over closed embeddings $Y\inj X$. Then, $\twid{K}_0(X)$ has a natural $\lambda$-ring structure $\{\lambda^k: \twid{K}_0(X)\to\twid{K}_0(X)\}_{k\geq0}$ with $\lambda^k$ sending each $K_0^Y(X)$ to itself. Moreover, if $X=\Spec A$ and $Y=\Spec A/(a)$ then $\psi^k([\Kos(a)])=k[\Kos(a)]$ for every $k\geq0$.
\end{theorem}

\begin{lemma}
Let $X\in\Sch$ be regular and $Y\subset X$ a closed subscheme of codimension $m$. Then, there exists an exact sequence 
\begin{center}
\begin{tikzcd}
0 \arrow[r] & F^{m+1}K_0^Y(X) \arrow[r] & K_0^Y(X) \arrow[r] & \displaystyle\bigoplus_{x\in Y\cap X^{(m)}}K_0^{\{x\}}(\O_{X,x})
\end{tikzcd}
\end{center}
\end{lemma}
\end{document}