\documentclass[11pt]{article}

\usepackage{kernel_of_truth}
\usepackage{normal_setup}

\usepackage{footnote}
\makesavenoteenv{tabular} % allow footnotes in tabular environment

\renewcommand{\AA}{\mathcal{A}}
\newcommand{\BB}{\mathcal{B}}
\newcommand{\CC}{\mathcal{C}}
\renewcommand{\F}{\mathcal{F}}
\newcommand{\EE}{\mathcal{E}}
\newcommand{\FF}{\mathbb{F}}
\newcommand{\G}{\mathcal{G}}
\newcommand{\GG}{\mathbb{G}}
\newcommand{\I}{\mathcal{I}}
\newcommand{\J}{\mathcal{J}}
\renewcommand{\L}{\mathcal{L}}
\newcommand{\M}{\mathcal{M}}
\renewcommand{\P}{\mathbb{P}}
\renewcommand{\phi}{\varphi}
\newcommand{\T}{\mathcal{T}}
\newcommand{\U}{\mathscr{U}}
\newcommand{\V}{\mathscr{V}}
\newcommand{\W}{\mathscr{W}}

\DeclareMathOperator{\rig}{rig} % rigid(-ified)

\begin{document}
\title{Algebraic Geometry}
\author{Zachary Gardner}
\date{\texttt{zachary.gardner@bc.edu}}
\maketitle

The following table indicates basic categories we will encounter.
\begin{center}
\begin{tabular}{|l|l|l|}
\hline
Category & Objects & Morphisms \\
\hline
$\Ab$ & abelian groups & group homomorphisms \\
$\CAlg_A$ & commutative (associative, left, unital) $A$-algebras & $A$-algebra homomorphisms\footnote{We require a map of $A$-algebras $B\to C$ to send $1_B$ to $1_C$.} \\
$\CRing$ & commutative (unital) rings & ring homomorphisms \\
$\Grp$ & groups & group homomorphisms \\
$\Mod_A$ & (left) $A$-modules & $A$-module homomorphisms \\
$\Set$ & sets & functions \\
\hline
\end{tabular}
\end{center}

Note that $A\in\CRing$ denotes a fixed commutative ring in the above table. Note also the following equivalences.
\begin{itemize}
\item $\Ab\simeq\Mod_{\Z}$.

\item $\CRing\simeq\CAlg_{\Z}$.

\item $\CAlg_A\simeq\CRing_{A/}$.
\end{itemize}

We use the acronym ``TFAE'' as shorthand for ``the following are equivalent.'' We use ``RAPL'' for ``right adjoints preserve limits'' (this has dual ``LAPC'' for ``left adjoints preserve colimits''). The words ``natural'' and ``canonical'' usually mean the same thing (and are not just filler!), capturing in intuitive language the categorical idea that there is only one way to do a certain thing. The term ``$\oblv$'' is often used to indicate that a functor is forgetful (for extra emphasis).

\newpage

\tableofcontents

\newpage

\section{Introduction}
Let's start off by briefly discussing the classical point of view on algebraic geometry, using the lens of affine varieties. Fix a field $k$ (not necessarily algebraically closed) and define affine $n$-space (over $k$) to be $\A^n=\A_k^n:=k^n$. This has associated coordinate ring $I(\A^n):=k[t_1,\ldots,t_n]$. There is a natural way to evaluate elements of $I(\A^n)$ at points of $\A^n$, allowing us to define the vanishing set
$$V(I):=\{\alpha\in\A^n : f(\alpha)=0\textrm{ for every }f\in I\}$$
given $I\normal k[t_1,\ldots,t_n]$ (which we can always write as $I=(f_1,\ldots,f_r)$ using Hilbert's Basis Theorem). We get a topology on $\A^n$, called the Zariski topology, by declaring the closed sets to be of the form $V(I)$. Note the following important information.
\begin{align*}
&V(0)=\A^n, \\
&V(k[t_1,\ldots,t_n])=\emptyset, \\
&V(I)\cup V(J)=V(IJ), \\
&V(I)\cap V(J)=V(I+J).
\end{align*}
One of the main disadvantages of the classical theory of (affine) varieties is that there are natural geometric processes we want to perform that take us outside the land of (affine) varieties. The first remedy to this is to expand our scope, working with all commutative rings instead of just finitely generated polynomial algebras over fields. With this in mind, let $A\in\CRing$ and define $\Spec A$ to be the collection of prime ideals in $A$. We call $\Spec A$ the \textbf{spectrum} of $A$ and say that $A$ is an \textbf{affine scheme}.

\begin{example}
Fix a field $k$.
\begin{itemize}
\item $\Spec k=\{(0)\}$.

\item $\Spec k[t]/(t^2)=\{(t)\}$.

\item $\Spec\Z=\{(0),(2),(3),\ldots\}$.

\item $\Spec\Z_p=\{(0),(p)\}$, for $p\in\Z$ prime.
\end{itemize}
\end{example}

As before, we get a topology on $\Spec A$, still called the Zariski topology, by declaring the closed sets to be the vanishing loci
$$V(I):=\{\mf{p}\in\Spec A : I\subset\mf{p}\}$$
for $I\normal A$. Analogous to before we have\footnote{Note that ideal sums are defined for infinite collections of ideals by taking sums of finitely many elements at a time.}
\begin{align*}
&V(0)=\Spec A, \\
&V(A)=\emptyset, \\
&V(I)\cup V(J)=V(IJ), \\
&V(I)\cap V(J)=V(I+J).
\end{align*}
Any $\phi\in\Hom_{\CRing}(A,B)$ induces a map $\Spec\phi: \Spec B\to\Spec A$ defined by pullback.

\begin{exercise}
Show that $\Spec\phi$ is continuous with respect to the Zariski topology on $\Spec A$ and $\Spec B$.
\end{exercise}

Given $I\normal A$ and $f\in A$, we have natural maps 
$$A\surj A/I\leadsto\Spec A/I\inj\Spec A$$
and
$$A\to A_f\leadsto\Spec A_f\inj\Spec A,$$
where $A_f$ is defined to be the localization of $A$ at (the multiplicative set generated by) $f$. We see immediately that the image of $\Spec A/I\inj\Spec A$ is $V(I)$. We also define 
$$D(f):=\im(\Spec A_f\inj\Spec A).$$
This is called a \textbf{principal open} subset of $\Spec A$. The map $V(I)\inj\Spec A$ is a prototypical example of a closed embedding, while the map $D(f)\inj\Spec A$ is a prototypical example of an open embedding.

\begin{exercise}
Fix $A\in\CRing$.
\begin{enum}{\alph}
\item Show that the principal open subsets $D(f)$ are open in $\Spec A$ and define a basis for the Zariski topology on $\Spec A$ (in particular, they give an open covering).\footnote{This allows us to view open subsets of $\Spec A$ as built up from principal open subsets instead of as complements of vanishing loci.} For this reason $D(f)$ is often called a \textbf{basic} open subset.

\item Show that $D(f)\cap D(g)=D(fg)$ and so the collection of principal open subsets is closed under finite intersections.

\item Is $D(f+g)$ related to $D(f)\cup D(g)$? Be careful not to confuse $(f+g)$ with $(f,g)$.
\end{enum}
\end{exercise}

The next step, outlined in a book like Vakil's, is to equip $\Spec A$ with a so-called structure sheaf $\O_{\Spec A}$. We will comment more on this later. For now we turn to the setting of spaces.

\section{Spaces}
\begin{definition}
Define the category of \textbf{spaces} to be $\Space:=\Fun(\CRing,\Set)$.\footnote{Be warned that this is not the same notion as the term \emph{algebraic space}, which means something different in algebraic geometry. The terminology we use here is not super common as far as I can tell, but it works well enough for our purposes.} We refer to elements of $X(A)$ as \textbf{$A$-valued points} of $X$. Intuitively, if we think of $X$ as some kind of geometric space then the points of $X$ should be the elements of $X(A)$ for varying $A$.
\end{definition}

It follows immediately that $\Space^{\op}\simeq\Pre(\CRing)$, with underlying identifications
$$\Fun(\CRing^{\op},\Set)\simeq\Fun(\CRing,\Set)^{\op}\simeq\Fun(\CRing,\Set^{\op}).$$ 
Given $A\in\CRing$, define 
$$\Spec A:=\Hom_{\CRing}(A,\cdot)\in\Space.$$ 
We call this space an \textbf{affine scheme}, and all such spaces span a full subcategory $\Aff\Sch\subset\Space$.

\begin{example}
\hfill
\begin{itemize}
\item $(\Spec\Z)(A)=\Hom_{\CRing}(\Z,A)=\{*\}$.

\item $(\Spec\Z[t])(A)=\Hom_{\CRing}(\Z[t],A)\iso A$.\footnote{This is closely related to additive group schemes.}

\item $(\Spec\Z[t^{\pm1}])(A)=\Hom_{\CRing}(\Z[t^{\pm1}],A)\iso A^{\times}$.\footnote{This is closely related to multiplicative group schemes.}
\end{itemize}
\end{example}

Yoneda's Lemma tells us that the functor 
$$\CRing\to\Pre(\CRing),\qquad A\mapsto\Hom_{\CRing}(\cdot,A)$$
is a (covariant) embedding. Likewise, Yoneda's Lemma also tells us that the functor
$$\CRing^{\op}\to\Pre(\CRing^{\op}),\qquad A\mapsto\Hom_{\CRing^{\op}}(\cdot,A)=\Hom_{\CRing}(A,\cdot)$$
is a (covariant) embedding. Using the identifications
$$\Pre(\CRing^{\op})\simeq\Pre(\CRing)^{\op}\simeq\Space,$$
we obtain the following result.

\begin{theorem}
The functor $\Spec: \CRing^{\op}\to\Space$ is an embedding and thus is an equivalence onto its (essential) image $\Aff\Sch$. In particular, given any $A,B\in\CRing$,
$$\Hom_{\Space}(\Spec B,\Spec A)\iso\Hom_{\CRing}(A,B).$$
\end{theorem}

Given $X\in\Space$ and $A\in\CRing$, there is a canonical map 
$$\Hom_{\Space}(\Spec A,X)\to X(A),\qquad (F: \Spec A\to X)\mapsto F(A)(\id_A).$$
which is a bijection since that's precisely what Yoneda's Lemma tells us.\footnote{The ``shape'' of this equation suggests that there is some adjunction at work.} In fact, investigating such maps is exactly how one proves Yoneda's Lemma.

\begin{exercise}
Suppose you are given an isomorphism $F\in\Isom_{\Space}(\Spec A,X)$. Does there naturally exist $B\in\CRing$ such that $X=\Spec B$? If this statement is true then the image and essential image of $\Spec$ coincide. In practice we want $\Aff\Sch$ to be closed under isomorphism and so often implicitly identify $\Aff\Sch$ with either the image and essential image of $\Spec$ depending on the situation (if they are in fact different).
\end{exercise}

Thanks to the previous theorem, if we want to study affine schemes then it's natural to study spaces. One of the advantages of the category $\Space=\Fun(\CRing,\Set)$ is that $\Set$ is complete and cocomplete so $\Space$ is as well, with (co-)limits given by ``pointwise'' evaluation using (co-)limits in $\Set$.\footnote{By definition, a category is (co-)complete if it is admits all small (co-)limits. We won't trouble ourselves too much over this smallness distinction.} For example,
$$(X\times_ZY)(A)=X(A)\times_{Z(A)}Y(A)$$
defines the fiber product $X\times_ZY$, which by definition fits into a Cartesian square
\begin{center}
\begin{tikzcd}
X\times_ZY \arrow[r] \arrow[d] & X \arrow[d] \\
Y \arrow[r] & Z
\end{tikzcd}
\end{center}
The functor $\Spec$ is contravariant and so takes limits/colimits in $\CRing$ to colimits/limits in $\Set$. This warrants a closer look at $\CRing$.

\begin{exercise}
Show that $\CRing$ has product given by the Cartesian product $\times$ and coproduct given by the tensor product $\tensor=\tensor_{\Z}$. Show also that $\CRing$ has initial object $\Z$ and terminal object $0$ (the ring with one element).
\end{exercise}

It follows that $\Spec A\coprod\Spec B\iso\Spec(A\times B)$ and $\Spec A\times\Spec B\iso\Spec(A\tensor B)$. Additionally, $\Space$ has terminal object $\Spec\Z$ and initial object $\emptyset:=\Spec0$. The significance of the former statement is that every space is naturally defined ``over'' $\Spec\Z$ (this will be significant later). Many statements about affine schemes break if the affine scheme in question is $\emptyset$, so we often implicitly assume that affine schemes are nonempty.

Since it will be useful later, it's good to say a bit more about pushouts in $\CRing$ as these correspond to pullbacks in $\Aff\Sch$. The data of the diagram 
$$\Spec A\rightarrow\Spec C\leftarrow\Spec B$$
in $\Aff\Sch$ is the same as the data of the diagram
$$A\leftarrow C\rightarrow B$$
in $\CRing$. These ring homomorphisms equip $A$ and $B$ with the structure of $C$-algebras, inducing a $C$-algebra structure on $A\tensor_CB$ via $(a\tensor b)(a'\tensor b')=aa'\tensor bb'$. We conclude that
$$\Spec A\times_{\Spec C}\Spec B\iso\Spec(A\tensor_CB).$$
Let's describe this isomorphism more explicitly. Suppose that we have structure maps $\phi: C\to A$ and $\psi: C\to B$. Given $D\in\CRing$, we need to define a bijection
$$\Hom_{\CRing}(A,D)\times_{\Hom_{\CRing}(C,D)}\Hom_{\CRing}(B,D)\iso\Hom_{\CRing}(A\tensor_CB,D)$$
functorial in $D$. The LHS is given explicitly by 
$$\{(f,g)\in\Hom_{\CRing}(A,D)\times\Hom_{\CRing}(B,D) : f\circ\phi=g\circ\psi\}.$$
Let $(f,g)$ be an element of the LHS and consider the $C$-bilinear map 
$$A\times B\to D,\qquad (a,b)\mapsto f(a)g(b).$$
Using the universal property of the tensor product, we obtain a unique map $\Phi(f,g): A\tensor_CB\to D$ such that
\begin{center}
\begin{tikzcd}
A\times B \arrow[r, "\tensor"] \arrow[rd] & A\tensor_CB \arrow[d, dotted, "\exists!\,\Phi(f{,}g)"] \\
& D
\end{tikzcd}
\end{center}
commutes and readily deduce that $\Phi(f,g)$ is a ring map. Conversely, let $h\in\Hom_{\CRing}(A\tensor_CB,D)$ and consider the assignment
$$\Psi(h):=(a\mapsto h(a\tensor1_B),b\mapsto h(1_A\tensor b)),$$
which we readily see defines an element of $\Hom_{\CRing}(A,D)\times_{\Hom_{\CRing}(C,D)}\Hom_{\CRing}(B,D)$. One can then check that $\Phi$ and $\Psi$ are inverse functions which are functorial in $D$ (and, in fact, in $A,B,C$ as well).

\begin{exercise}
We can in fact say more about pushouts. The fiber product $X\times_ZY$ comes equipped with projection maps $\pr_1: X\times_ZY\to X$ and $\pr_2: X\times_ZY\to Y$ defined as you would expect. In the affine case $X=\Spec A$, $Z=\Spec C$, and $Y=\Spec B$, the projection maps correspond to ring maps from $A,B$ to $A\tensor_CB$. Show that $\pr_1$ corresponds to $a\mapsto a\tensor1_B$ and $\pr_2$ corresponds to $b\mapsto1_A\tensor b$.
\end{exercise}

\section{Topology}
Now that we've got the basic algebraic and categorical setup out of the way, our goal is to bring in geometry by introducing something like a topology. Back at the beginning we had an identification between $V(I)$ and $\Spec A/I$. Why not just take this to be a definition in the land of spaces? By the First Isomorphism Theorem, the class of projections $A\surj A/I$ encodes the same data as the larger class of all surjective homomorphisms $A\surj B$. This prompts the following definition.

\begin{definition}
A map $\Spec B\to\Spec A$ of affine schemes is a \textbf{closed embedding} if the associated homomorphism $A\to B$ is surjective or, equivalently, $B\iso A/I$ for some $I\normal A$.\footnote{The former condition is more natural in a certain sense, while the latter condition is more useful for computations.} We define the \textbf{vanishing locus} of $I$ to be $V(I):=\Spec A/I$. This comes equipped with a canonical closed embedding $V(I)\to\Spec A$.
\end{definition}

\begin{exercise}
Fix $A\in\CRing$ and $I,J\normal A$. Show that 
$$V(I)\times_{\Spec A}V(J)\iso V(I+J).$$
\end{exercise}

For our purposes it is advantageous to extend the notion of closed embedding from maps of affine schemes to maps of arbitrary spaces. To do this, we borrow intuition from the category of topological spaces. Continuous maps of topological spaces are defined by the property that they pull back open subsets to open subsets, with open subsets acting as the building blocks of any given topological space. In our setting, we want the basic building blocks to be affine schemes. This prompts the following definition.

\begin{definition}
A map $f: X\to Y$ of spaces is \textbf{affine} if, for every $g\in\Hom_{\Space}(\Spec A,Y)$ with $A\in\CRing$, the induced space $f^{-1}(\Spec A):=\Spec A\times_YX$ is affine. Stated simply, $f$ pulls back affine schemes to affine schemes.
\end{definition}

\begin{example}
Let $f: \Spec B\to\Spec C$ be any map of affine schemes. Then, we claim that $f$ is affine. This is simply because, given any $g\in\Hom_{\Space}(\Spec A,\Spec C)$, 
$$\Spec A\times_{\Spec C}\Spec B\iso\Spec(A\tensor_CB).$$
\end{example}

We now extend our earlier definition.

\begin{definition}
A map $f: X\to Y$ of spaces is a \textbf{closed embedding} if it is affine and, for every $g\in\Hom_{\Space}(\Spec A,Y)$ with $A\in\CRing$, the induced map of spaces $f^{-1}(\Spec A)=\Spec A\times_YX\to\Spec A$ is a closed embedding.\footnote{What we have called a closed embedding is often also called a closed immersion in practice.}
\end{definition}

\begin{proposition}
This new notion of closed embedding extends the previous notion (i.e., is well-defined).
\end{proposition}

\begin{proof}
Let $f: \Spec B\to\Spec C$ be a closed embedding in the old sense. Given $g: \Spec A\to\Spec C$ a map of spaces, we already know that $f$ is affine and so we just need to show that $\Spec(A\tensor_CB)\to\Spec A$ is a closed embedding in the old sense. But this is clear since $B\iso C/I$ for some $I\normal C$ and so 
$$A\surj A/IA\iso A\tensor_CC/I\iso A\tensor_CB.$$
\end{proof}

\begin{remark}
Recall that, given a category $\mc{C}$ and a morphism $f\in\Hom_{\mc{C}}(X,Y)$, $f$ is a \emph{monomorphism} if it is left-cancellative in the sense that, given $g,h\in\Hom_{\mc{C}}(Z,X)$, if $f\circ g=f\circ h$ then $g=h$. In $\Set$ the monomorphisms are precisely the injective maps. In $\Space$, the monomorphisms are precisely $Z\to X$ with $Z(A)\to X(A)$ injective for every $A\in\CRing$. We call these \textbf{subspaces}. If the map $Z\to X$ is clear from context then we often just say that $Z$ is a \textbf{subspace} of $X$. Note that being a subspace is equivalent to the natural map $\Hom_{\Space}(\Spec A,Z)\to\Hom_{\Space}(\Spec A,X)$ being injective for every $A\in\CRing$. That is, every map of spaces $\Spec A\to Z$ extends uniquely to $X$.\footnote{In other words, it suffices to know that a monomorphism is left-cancellative only on maps out of affine schemes.}
\end{remark}

\begin{proposition}
Show that every closed embedding $Z\to X$ of affine schemes is a subspace.\footnote{This helps justify the use of the term ``embedding.''} Is it true that general closed embeddings are subspaces?
\end{proposition}

Let $Z\to X$ be any map of schemes. We would like to make sense of the \textbf{complementary space} or \textbf{complement} $X\setminus Z\in\Space$. How should this be defined? One obvious guess is to take the $A$-points of $X\setminus Z$ to be $X(A)\setminus Z(A)$. This has several flaws, listed here in order of severity.
\begin{enum}{\arabic}
\item The complement $X(A)\setminus Z(A)$ need not even be well-defined!

\item Even assuming $Z$ is a subspace of $X$, the assignment $A\mapsto X(A)\setminus Z(A)$ need not be functorial.

\item This definition fails to capture the right underlying geometry.
\end{enum}

Intuitively speaking, the ``points'' of $X\setminus Z$ should be ``points'' of $X$ that don't ``intersect'' $Z$. The following definition makes this rigorous.

\begin{definition}
Let $Z\to X$ be a subspace and $A\in\CRing$. Define $(X\setminus Z)(A)$ to be the set of $x\in X(A)$ such that, after identifying $x$ with $x\in\Hom_{\Space}(\Spec A,X)$ using Yoneda's Lemma, the diagram
\begin{center}
\begin{tikzcd}
\emptyset \arrow[r] \arrow[d] & Z \arrow[d] \\
\Spec A \arrow[r] & X
\end{tikzcd}
\end{center}
is a Cartesian square. Equivalently, $\Spec A\times_XZ=\emptyset$.
\end{definition}

\begin{exercise}
Check that the above construction is functorial in $A$ hence defines $X\setminus Z$ as a space. Is the complement of a subspace necessarily a subspace?
\end{exercise}

With this in hand, we define an \textbf{open embedding} $U\inj X$ to be the complement of a closed embedding $Z\inj X$, with the caveat that $X$ is assumed to be affine. More generally, an \textbf{open embedding} $U\inj X$ with $X\in\Space$ is a morphism $U\to X$ of spaces such that, for every $f\in\Hom_{\Space}(\Spec B,X)$, the induced map $\Spec B\times_XU\to\Spec B$ is an open embedding. Note that we do not assume that $U\inj X$ is affine.

\begin{exercise}
Check that this agrees with the original definition in the case that $X$ is affine. Are open embeddings actually subspaces in the affine case? How about more generally?
\end{exercise}

Note that this allows us to make rigorous sense of the terms open and closed subspace, even though open and closed embeddings may not in general technically give rise to subspaces. 

\begin{exercise}
The notions of open and closed embedding both make sense for general spaces. Is it necessarily the case that every open embedding in $\Space$ is the complement of a closed embedding?
\end{exercise}

\begin{exercise}
Let $X\in\Space$ be affine. We know that the complements of closed embeddings into $X$ give open embeddings. Do complements of open embeddings yield closed embeddings?
\end{exercise}

\begin{exercise}
Show that the classes of affine morphisms, closed embeddings, and open embeddings are all closed under composition.
\end{exercise}

\begin{exercise}
Fix $A\in\CRing$ and $f\in A$. Show that there is a canonical isomorphism 
$$\Spec A_f\iso\Spec A\setminus\Spec A/f.$$
\end{exercise}

Geometrically, we think of $\Spec A/f$ as the vanishing locus of $f$ on $\Spec A$ and $\Spec A_f$ as the nonvanishing locus of $f$ on $\Spec A$. In connection with earlier work, we define 
$$D(f):=\Spec A\setminus\Spec A/f\iso\Spec A_f$$
and refer to this as a \textbf{principal} open subspace of $\Spec A$. We can extend the notation to encompass any ideal $I\normal A$ via
$$D(I):=\Spec A\setminus\Spec A/I.$$
We think of this as the nonvanishing locus of $I$ on $\Spec A$. 

\begin{remark}
We see from the above that $\Spec A_f\inj\Spec A$ is an example of an open subspace of $\Spec A$. Note, however, that not all open subspaces of $\Spec A$ look like this.
\end{remark}

\begin{exercise}
Let $X=\Spec A\in\Aff\Sch$ and $f,g\in A$. Show that there is a canonical isomorphism
$$D(fg)\iso D(f)\times_XD(g),$$
where the fiber product is computed using $D(f)\inj X$ and $D(g)\inj X$. It may help to recall that localization and tensor product commute with each other in an appropriate sense. Given ideals $I,J\normal A$, is it necessarily true that $D(I)\times_XD(J)\iso D(IJ)$?
\end{exercise}

\begin{exercise}
Let $A,B\in\CRing$ with $I\normal A$. Show that 
$$\Hom_{\Space}(\Spec B,\Spec A\setminus\Spec A/I)\iso\{\phi\in\Hom_{\CRing}(A,B) : \phi(I)B=B\}.$$
Note that $\phi(I)B$ is commonly written as just $IB$, the action of $\phi$ being left implicit (we choose to make the action of $\phi$ explicit in the above equation for clarity).
\end{exercise}

\begin{exercise}
Let $X=\Spec A\in\Aff\Sch$. Let $Z\inj X$ be a closed embedding (so $Z=\Spec A/I$) and $f\in\Hom_{\Space}(\Spec B,X)$. Show that
$$\Spec B\setminus(\Spec B\times_XZ)\iso\Spec B\times_X(X\setminus Z).$$
Does this still work for general $X\in\Space$?
\end{exercise}

Since this material can be a little dense the first time around, we include the solution of the first half of this exercise.

\begin{proof}
Note first of all that $\Spec B\times_XZ\iso\Spec B/IB$ and so $\Spec B\setminus(\Spec B\times_XZ)$ is well-defined. By Yoneda's Lemma, it suffices to show that 
$$\Hom_{\Space}(\Spec C,\Spec B\setminus(\Spec B\times_XZ))\iso\Hom_{\Space}(\Spec C,\Spec B\times_X(X\setminus Z))$$
for any given $C\in\CRing$. The LHS looks like
$$\Hom_{\Space}(\Spec C,\Spec B\setminus\Spec B/IB)\iso\{\phi\in\Hom_{\CRing}(B,C) : \phi(IB)C=C\}.$$
The RHS looks like 
\begin{align*}
&\Hom_{\Space}(\Spec C,\Spec B\times_X(X\setminus Z)) \\
&\iso\Hom_{\Space}(\Spec C,\Spec B)\times_{\Hom_{\Space}(\Spec C,X)}\Hom_{\Space}(\Spec C,X\setminus Z) \\
&\iso\Hom_{\CRing}(B,C)\times_{\Hom_{\CRing}(A,C)}\{\psi\in\Hom_{\CRing}(A,C) : \psi(I)C=C\},
\end{align*}
where we have made use of the universal property of the fiber product. The result then follows after unwinding the definition of the fiber product in $\Set$. Note that you need to use the structure map $A\to B$ arising from the map $f: \Spec B\to\Spec A$ of spaces.
\end{proof}

\section{Open Coverings of Affine Schemes}
So far we've only discussed affine schemes. Although these contain a lot of geometric richness in their own right, moving forward we want to construct more general schemes by ``gluing together'' affine schemes in some appropriate sense. Our first step towards this goal is the following.

\begin{definition}
Let $X\in\Space$. A \textbf{(Zariski) open covering} of $X$ is a collection of (Zariski) open embeddings $\U=\{(U,i_U: U\inj X)\}$ such that, for every $f\in\Hom_{\Space}(\Spec B,X)$ with $\Spec B$ nonempty, we have $\Spec B\times_XU\neq\emptyset$ for some $U\in\U$.
\end{definition}

The name ``Zariski'' appears here in connection with the Zariski topology.

\begin{exercise}
Intuitively, we should be able to obtain a space by ``gluing together'' the constituents of any open covering. In our situation, the correct way to functorially encode this is by taking colimits. Is it necessarily true that a space $X\in\Space$ is isomorphic to the colimit of any given open covering $\U$ of $X$?\footnote{This question is a little ambiguous as phrased. Letting $\U=\{U,V\}$ for simplicity, there could be a difference between the colimit of $U,V$ by themselves and the colimit of $U\rightarrow X\leftarrow V$. Indeed, the latter is canonically $X$ and so our question concerns the former.}
\end{exercise}

\begin{proposition}
Let $X=\Spec A\in\Aff\Sch$ be nonempty and $\U=\{(U,i_U: U\inj X)\}$ a collection of open embeddings. TFAE:
\begin{enum}{\roman}
\item $\U$ is an open covering.

\item There exists a finite subcollection $\U'\subset\U$ such that $\U'$ is an open covering (we say that $\U'$ is a \textbf{(Zariski) open subcovering}).

\item Let $x\in\Hom_{\Space}(\Spec k,X)$ for $k$ a field. Then, there exists $U\in\U$ such that $x$ factors through $i_U$.

\item For each $U\in\U$, write $U=X\setminus Z_U$ for $Z_U=\Spec A/I_U$ with $I_U\normal A$. Then, $\sum_{U\in\U}I_U=A$.
\end{enum}
\end{proposition}

Note that it is very important in this result that we assume $X$ is affine -- i.e., this result is very much false if $X$ is \textbf{not} affine! Before discussing the proof, we first comment on the geometric significance of this result.
\begin{itemize}
\item Point (ii) says that affine schemes satisfy a condition similar to compactness for topological spaces. Note, however, that this does not mean that affine schemes behave like compact topological spaces.\footnote{The ``true'' condition that mimics compactness is called \emph{properness}. We will discuss this more later.} This condition is often useful for proving that a space is \textbf{not} affine.

\item Point (iii) says that affine schemes of the form $\Spec k$ for $k$ a field ``behave like points.''

\item Point (iv) says that open coverings of affine schemes admit partitions of unity.\footnote{Intuition for the geometric significance behind this comes from the theory of manifolds.} This condition is often the easiest to check in practice.
\end{itemize}

\begin{proof}
We begin by making some important observations. Let $f\in\Hom_{\Space}(\Spec B,X)$ and $U\in\U$. Then, $\Spec B\times_XU\iso\Spec B\setminus\Spec B/I_UB$ and so 
$$\Hom_{\Space}(\Spec C,\Spec B\times_XU)\iso\{\phi\in\Hom_{\CRing}(B,C) : \phi(I_UB)C=C\}$$
given $C\in\CRing$. Given any $g\in\Hom_{\Space}(Y,X)$, to say that $g$ factors through $i_U$ is to say that we have a commutative diagram
\begin{center}
\begin{tikzcd}
Y \arrow[r, dotted, "\exists"] \arrow[rd, "g"'] & U \arrow[d, hookrightarrow, "i_U"] \\
& X
\end{tikzcd}
\end{center}
If $Y=\Spec C$ then $g$ corresponds to some $\psi\in\Hom_{\CRing}(A,C)$ and $g$ factoring through $i_U$ corresponds to the condition $\psi(I_U)C=C$.
\begin{enumerate}
\item[(i)$\implies$(iv)] Let $B:=A/\sum_{U\in\U}I_U$. We claim that $B=0$, which is equivalent to $\Spec B=\emptyset$. Given $U\in\U$, we have $\Spec B\times_XU\iso\Spec B\setminus\Spec B/I_UB$. By construction, $I_UB=0$ and so $\Spec B/I_UB\iso\Spec B$. Hence, $\Spec B\times_XU=\emptyset$ and so $\Spec B=\emptyset$ since $\U$ is an open covering of $X$ by assumption.

\item[(iv)$\implies$(iii)] Let $k$ be a field and $x\in\Hom_{\Space}(\Spec k,X)$. The point $x$ corresponds to a ring map $\phi: A\to k$ and so we need to show that $\phi(I_U)k=k$ for some $U\in\U$. Since $k$ is a field it suffices merely to show that $\phi(I_U)\neq0$ for some $U\in\U$. This follows from the fact that $\phi(1_A)=1_k$ and the assumption that $A=\sum_{U\in\U}I_U$.

\item[(iii)$\implies$(i)] Let $f\in\Hom_{\Space}(\Spec B,X)$ with $\Spec B$ nonempty. Since $B\neq0$, $B$ has a maximal ideal $\m$ giving rise to a closed point $\Spec B/\m\to\Spec B$. We will show that 
$$\Hom_{\Space}(\Spec B/\m,\Spec B\times_XU)\iso\{\phi\in\Hom_{\CRing}(B,B/\m) : \phi(I_UB)B/\m=B/\m\}$$
is nonempty and hence that $\Spec B\times_XU$ is nonempty as well. The natural projection $B\surj B/\m$ yields a $B/\m$-point
\begin{center}
\begin{tikzcd}
\Spec B/\m \arrow[r] & \Spec B \arrow[r, "f"] & \Spec A
\end{tikzcd}
\end{center}
which corresponds to $\psi\in\Hom_{\CRing}(A,B/\m)$ factoring through some $\phi\in\Hom_{\CRing}(A,B)$. Since the above $B/\m$-point factors through some $i_U$, we have $\psi(I_U)B/\m=B/\m$. It follows that $\phi(I_UB)B/\m=B/\m$ since $\psi(I_U)=\phi(I_UB)$.

\item[(iv)$\implies$(ii)] We have $\sum_{U\in\U}I_U=A$ and so $1_A=f_1+\cdots+f_n$ for $f_i\in I_{U_i}$ with $U_1,\ldots,U_n\in\U$. Then, $\sum_{i=1}^nI_{U_i}=A$ and so $\U':=\{U_1,\ldots,U_n\}$ is an open covering of $X$.
\end{enumerate}
Finally, the implication (ii)$\implies$(i) is obvious.
\end{proof}

Combining (ii) and (iv) says that every open covering of $\Spec A$ is characterized by finitely many ideals $I_1,\ldots,I_r\normal A$. That is, every open covering of $\Spec A$ contains a subcovering of the form $\{D(I_1),\ldots,D(I_r)\}$. None of these ideals need be finitely generated. However, amalgamating a set of generators for each ideal yields a set $\{f_t\}_{t\in T}\subset A$ such that $\sum_{t\in T}f_tA=A$. Then, $\{D(f_t) : t\in T\}$ is an open covering of $\Spec A$ and so contains a finite subcovering $\{D(f_1),\ldots,D(f_n)\}$. Given any space $S\in\Space$, we refer to any finite open covering of $S$ by principal open subspaces as a \textbf{principal open covering}. Dropping the finiteness assumption yields a \textbf{big principal open covering}. Our above analysis shows that any open covering of an affine scheme contains a subcovering that itself gives rise to a principal open covering.

\begin{exercise}
Show that in condition \textup{(iii)} of the proposition it is sufficient to consider only fields of the form $A/\m$ for $\m$ a maximal ideal of $A$.
\end{exercise}

Given any space $X\in\Space$, a \textbf{closed point} of $X$ is a closed embedding $\Spec k\inj X$ with $k$ a field. Since (nonzero) commutative rings always have at least one maximal ideal by Zorn's lemma, (nonempty) affine schemes always have at least one closed point. This need not be the case for general spaces.

\begin{exercise}
Fix $X\in\Space$ and consider the set $X^0$ of closed points of $X$ (which may be empty).
\begin{enum}{\arabic}
\item Is $X^0$ functorial in $X$? Can you make sense of $X^0$ as a space?

\item Assuming $X=\Spec A$, $X^0$ can be identified with the set of maximal ideals of $A$ (typically denoted $\MaxSpec A$ and called the \textbf{maximal spectrum} of $A$). Compute $X^0$ for various affine schemes.

\item Given $X,Y\in\Space$, when are $X^0$ and $Y^0$ isomorphic?

\item Under what conditions can we recover $X$ from $X^0$? 
\end{enum}
\end{exercise}

More generally, we may also consider the set of \textbf{points} of $X$, denoted $|X|$, defined to be the restriction of $X$ to the full subcategory of fields $\Field\subset\CRing$.\footnote{These are also called \emph{field-valued points} to eliminate any potential for confusion.}

\begin{proposition}
Given $A\in\CRing$, $\abs{\Spec A}$ is exactly the set of prime ideals of $A$.
\end{proposition}

\begin{proof}
For the sake of this argument define $\Spec'A$ to be the set of prime ideals of $A$. As stated, it is not technically true that $\abs{\Spec A}$ and $\Spec'A$ are in bijection because of a dumb set theoretic issue (we will, however, ignore this). Given $\mf{p}\in\Spec'A$, consider $\kappa(\mf{p}):=\Frac(A/\mf{p})$, the fraction field of $A/\mf{p}$ (which is an integral domain by definition of prime ideal). This comes equipped with a canonical map $\phi_{\mf{p}}$ given by the composition $A\surj A/\mf{p}\inj\kappa(\mf{p})$. Conversely, given a ring map $\phi: A\to k$ with $k$ a field, $(0)$ is a prime ideal of $k$ and so $\ker\phi\in\Spec'A$. By construction we have $\ker\phi_{\mf{p}}=\mf{p}$. Conversely, given $\phi: A\to k$, the universal property of localization provides a ring map $\Frac(A/\ker\phi)\to k$ which must necessarily be injective and so we can identify $\Frac(A/\ker\phi)$ with a subfield of $k$. The image of $\phi$ is the same as the image of the induced map $A/\ker\phi\to k$. Moreover, the latter is contained inside $\Frac(A/\ker\phi)$ since $k$ is a field. This shows that the original map $\phi: A\to k$ is uniquely obtained from $\Frac(A/\ker\phi)\inj k$ by extending by zero. This shows that there is a bijection between $\Spec'A$ and the collection of all $\Hom_{\CRing}(A,k)$ for $k$ sampling over all isomorphism classes of fields.
\end{proof}

\begin{exercise}
Let $X\in\Sch$. Assuming $X$ is affine, we saw already that a collection $\U$ of open subspaces of $X$ is an open covering if and only if $X(k)=\bigcup_{U|in\U}U(k)$ for every field $k$. Does the same hold true for general $X$? What if we expand our focus to allow local rings instead of just fields?
\end{exercise}

\section{Relative Algebraic Geometry - Relative Spaces}
Given a space $S\in\Space$ (often called a \textbf{base space}), we define the category $\Space_S$ of \textbf{spaces over $S$} or \textbf{$S$-spaces} to be the overcategory $\Space_{/S}$. Explicitly, objects of $\Space_S$ are morphisms $X\to S$ in $\Space$ (often written $X/S$) and morphisms are commutative diagrams
\begin{center}
\begin{tikzcd}
X \arrow[rr] \arrow[rd] & & \arrow[ld] Y \\
& S &
\end{tikzcd}
\end{center}
It is customary to think of $S$ as an object of $\Space_S$ using $\id_S$, in which case $S$ is the terminal object of $\Space_S$. Intuitively, we think of objects of $\Space_S$ both as families of spaces over $S$ and spaces living inside $S$ (the latter is especially true for subspaces). Note that $\Space_{\Spec\Z}\simeq\Space$ since $\Spec\Z$ is the terminal object of $\Space$. The product in $\Space_S$ is given by the fiber product $\times_S$ in $\Space$. Especially when $S$ is clear from context, we write $\cap$ for the product in $\Space_S$ and $\cup$ for the coproduct.\footnote{The notation is chosen to be geometrically suggestive.} Note that there are canonical isomorphisms $U\cap V\iso V\cap U$ and $U\cup V\iso V\cup U$ which are, in practice, often simply treated as equalities.

\begin{remark}
If $S=\Spec A$ then we often write $\times_A$ instead of $\times_{\Spec A}$.
\end{remark}

\textbf{\un{Slogan}:} ``Algebraic geometry relative to $S$ is geometry over $\Space_S$.''

\textbf{\textcolor{red}{The rest of this section can (and probably should) be skipped on a first reading.}}

\hrule

Any map $T\to S$ of spaces induces a \textbf{base change}\footnote{Even though the map $T\to S$ is suppressed from the notation, the resulting functor depends heavily on the choice of map.} functor 
$$T\times_S\cdot: \Space_S\to\Space_T,\qquad (X\to S)\mapsto(T\times_SX\to T).$$ 
Let $P_O$ be a property of objects\footnote{``O'' is short for ``object.''} of $\Space_S$ that also makes sense for objects of $\Space_T$ (this happens, e.g., if $P_O$ makes sense for all of $\Space$). We say that $P_O$ is \textbf{stable under base change to $T$} if, given $(X\to S)\in\Space_S$ with property $P_O$, $(T\times_SX\to T)\in\Space_T$ has property $P_O$. Conversely, we say that $P_O$ \textbf{descends to $S$} if, given $(Y\to T)\in\Space_T$ with property $P_O$, $(Y\to T\to S)\in\Space_S$ has property $P_O$. These relative notions can be extended to absolute notions for a property $P_O$ on $\Space_S$ as follows. We say that $P_O$ is \textbf{stable under base change} if it is stable under base change to $T$ for every map of spaces $T\to S$. Conversely, we say that $P_O$ \textbf{descends absolutely} if it descends to $S'$ for every map of spaces $S\to S'$. The absolute notion of base change is generally used when $S=\Spec\Z$ to make sense of things on $\Space$.

\begin{remark}
Our use here of the notion of ``descent'' is not standard terminology. Note that when algebraic geometers discuss descent they are often talking about something analogous to the sheaf condition (an example of which will be defined soon).
\end{remark}

\begin{remark}
Generally speaking, we want ``descends absolutely'' to be the same as ``descends to $\Spec\Z$'' since $\Spec\Z$ is the terminal object of $\Space$. This isn't always the case but can often be made true with the appropriate modifications.
\end{remark}

\begin{exercise}
Show that the base change functor $T\times_S\cdot: \Space_S\to\Space_T$ is left adjoint to the ``forgetful'' functor 
$$\Space_T\to\Space_S,\qquad(Y\to T)\mapsto(Y\to T\to S).$$
This allows for easy proof of many categorical facts about base change.
\end{exercise}

\begin{exercise}
Try proving that the property of being affine is stable under base change. You should hit a snag that will be remedied later when we discuss general schemes.\footnote{In a nutshell, the problem is that the property of being affine is defined only in terms of pulling back affine schemes, rather than general spaces.} This illustrates that stability under base change and descent are \emph{geometric} rather than purely categorical phenomena.
\end{exercise}

\begin{exercise}
Let $P_M$ be a property that makes sense for morphisms in both $\Space_S$ and $\Space_T$. 
\begin{itemize}
\item Define precisely what it means for $P_M$ to be stable under base change to $T$ and to descend to $S$.

\item By definition $P_M$ corresponds to some class of morphisms in an appropriate category containing both $\Space_S$ and $\Space_T$. What general properties are desirable for a class of morphisms in $\Space_S$ to have (e.g., closure under composition and isomorphism)? It might be helpful to play around with triples of spaces $T'\to T\to S$. 

\item Does this yield any new information that cannot be described using the language of properties $P_O$ of objects?
\end{itemize}
\end{exercise}

\begin{exercise}
Fix $A\in\CRing$. Show that there is a canonical equivalence $\Space_{\Spec A}\simeq\Fun(\CAlg_A,\Set)$.
\end{exercise}

\section{Zariski Sheaves}
Let $X,S\in\Space$ and $\U=\{(U,i_U: U\inj S\}$ an open covering of $S$. Let $U,V\in\U$ with arbitrary maps $f_U: U\to X$ and $f_V: V\to X$. Note first of all that we have a Cartesian square
\begin{center}
\begin{tikzcd}
U\cap V \arrow[r, "i_{V,U}"] \arrow[d, "i_{U,V}"'] & U \arrow[d, hookrightarrow, "i_U"] \\
V \arrow[r, hookrightarrow, "i_V"'] & S
\end{tikzcd}
\end{center}
Both $i_{V,U}$ and $i_{U,V}$ are monomorphisms, which means that the composition 
$$i_U\times i_V:=i_U\circ i_{V,U}=i_V\circ i_{U,V}: U\cap V\to S$$
is as well.\footnote{The notation $i_{V,U}$ suggests that $i_{V,U}$ has target $U$ and arises from a map with source $V$.} Define $f_U|_{U\cap V}: U\cap V\to X$ to be the composition
\begin{center}
\begin{tikzcd}
U\cap V \arrow[r, hookrightarrow, "i_{V,U}"] & U \arrow[r, "f_U"] & X
\end{tikzcd}
\end{center}
Similarly, define $f_V|_{U\cap V}: U\cap V\to X$ to be the composition
\begin{center}
\begin{tikzcd}
U\cap V \arrow[r, hookrightarrow, "i_{U,V}"] & V \arrow[r, "f_V"] & X
\end{tikzcd}
\end{center}
Thinking of $X$ as some kind of manifold and elements of $\U$ as coordinate charts on $X$, it is natural to consider the analogue of change of coordinates. This leads us to consider the set 
$$\Psi(S,\U,X):=\{\{f_U\in\Hom_{\Space}(U,X)\}_{U\in\U} : f_U|_{U\cap V}=f_V|_{U\cap V}\textrm{ for every }U,V\in\U\}.$$
This vague analogy doesn't quite work since $\U$ is an open covering of $S$ and not $X$. We bridge the gap by considering elements of $\Hom_{\Space}(S,X)$, which we think of as picking out manageable chunks of $X$. The map\footnote{The codomain of this map is a collection of collections. There is potentially a lot of data here!}
$$\Hom_{\Space}(S,X)\to\{\{f_U\in\Hom_{\Space}(U,X)\}_{U\in\U}\},\qquad f\mapsto\{f\circ i_U\}_{U\in\U}$$
is natural in $X$ and factors through $\Psi(S,\U,X)$ since
\begin{align*}
(f\circ i_U)|_{U\cap V}
&=f\circ i_U\circ i_{V,U} \\
&=f\circ i_V\circ i_{U,V} \\
&=(f\circ i_V)|_{U\cap V}.
\end{align*}
As before it is natural to define $f|_U:=f\circ i_U$ for $U\in\U$.\footnote{In fact, it is natural to use restriction notation for any precomposition in $\Space$.} We call $f|_U$ the \textbf{section} of $f$ over $U$.

\begin{definition}
A space $X\in\Space$ is a \textbf{Zariski sheaf} if for every open covering $\U$ of every affine space $S:=\Spec A$ the natural map 
\begin{align*}
\Hom_{\Space}(S,X)\to\Psi(S,\U,X)
\end{align*}
is a bijection. Equivalently, for every $\{f_U: U\to X\}_{U\in\U}\in\Psi(S,\U,X)$, there exists a unique $f: S\to X$ such that $f|_U=f_U$ for every $U\in\U$.
\end{definition}

Informally, a Zariski sheaf is a space with existence and uniqueness of gluings for compatible collections of sections of $X$ over suitable open subspaces. If we are thinking about only a single cover $\U$ then we say $X$ \textbf{satisfies the Zariski sheaf condition} with respect to $\U$. This makes sense for general $S\in\Space$. For convenience we let $\Shv_{\Zar}\subset\Space$ denote the full subcategory spanned by Zariksi sheaves. 

\begin{example}
Let $X,Y\in\Shv_{\Zar}$. It is immediate that $X\times Y\in\Shv_{\Zar}$. In fact, the same argument shows that arbitrary products of Zariski sheaves are themselves Zariski sheaves. 
\end{example}

\begin{example}
Consider $\A^1=\A_{\Z}^1:=\Spec\Z[t]\in\Space$. We will see shortly that $\A^1$ is a Zariski sheaf. It then follows that 
$$\A^n=\A_{\Z}^n:=\Spec\Z[t_1,\ldots,t_n]\iso(\A^1)^{\times n}$$ 
is a Zariski sheaf for every $n\geq1$.
\end{example}

\begin{exercise}
Show that the association $X(A):=\{f\in A : f\in A^{\times}\textrm{ or }1-f\in A^{\times}\}$ defines a space $X\in\Space$ that is not a Zariski sheaf.
\end{exercise}

Just as we would like to think about multiple open coverings at once for a manifold, we would like to do the same for spaces. With this in mind, fix a base space $S\in\Space$ and an open covering $\U=\{(U,i_U: U\inj S)\}$ of $S$. We can naturally view $\U$ as an object in some category $\Cov(S)$. Given $\V=\{(V,j_V: V\inj S)\}\in\Cov(S)$, we say that $\V$ is a \textbf{refinement} of $\U$ and write $\V\subset\U$ if, for every $U\in\U$, the collection
$$\V_U:=\{(U\cap V,j_{V,U}: U\cap V\to U)\}$$
is an open covering of $U$. Here, we are using the Cartesian square
\begin{center}
\begin{tikzcd}
U\cap V \arrow[r] \arrow[d, "j_{V,U}"'] & V \arrow[d, hookrightarrow, "j_V"] \\
U \arrow[r, hookrightarrow, "i_U"'] & S
\end{tikzcd}
\end{center}
associated to every pair $(U,V)\in\U\times\V$. For convenience we define $i_U\times j_V:=i_U\circ j_{V,U}: U\cap V\to S$. Informally, a refinement is a covering that ``covers'' another covering.

\begin{exercise}
Fix $\U,\V\in\Cov(S)$.
\begin{enum}{\alph}
\item What is $\Hom_{\Cov(S)}(\U,\V)$?

\item Show that $\U\times\V:=\{(U\cap V,i_U\times j_{V,U}: U\cap V\to S)\}$ is the categorical product of $\U$ and $\V$ in $\Cov(S)$.

\item Suppose that $\U$ refines $\V$ and $\V$ refines $\U$. Is it true that $\U\iso\V$?
\end{enum}
\end{exercise}

The following lemma says that we can bootstrap from refinements.

\begin{lemma}
Let $X\in\Space$ and $\U,\V\in\Cov(S)$ with $\V$ refining $\U$. Suppose that $X$ satisfies the Zariski sheaf condition with respect to $\V$. Suppose further that every $U\in\U$ satisfies the Zariski sheaf condition with respect to $\V_U$. Then, $X$ satisfies the Zariski sheaf condition with respect to $\U$.
\end{lemma}

Explicitly, if $\Hom_{\Space}(S,X)\xto{\sim}\Psi(S,\V,X)$ and $\Hom_{\Space}(S,U)\xto{\sim}\Psi(S,\V_U,U)$ for every $U\in\U$ then $\Hom_{\Space}(S,X)\xto{\sim}\Psi(S,\U,X)$.

\begin{exercise}
Prove the lemma!
\end{exercise}

\begin{corollary}
Let $X\in\Shv_{\Zar}$. Then, $X$ satisfies the Zariksi sheaf condition with respect to every $\U\in\Cov(S)$ for every $S\in\Space$.
\end{corollary}

The reason why we single out only $S$ affine in the definition of Zariski sheaves is that affine schemes should be the crux of the geometry of general schemes. Our simplification also makes the condition easier to check, which is important because we want this formalism to be usable in practice.

\section{Schemes}
\begin{definition}
A \textbf{scheme} is a space which is a Zariski sheaf and admits an open covering by affine schemes. These span a full subcategory $\Sch\subset\Space$.
\end{definition}

\begin{exercise}
Show that the Zariski sheaf condition is preserved by open and closed embeddings. That is, open and closed subspaces of Zariski sheaves are themselves Zariski sheaves.
\end{exercise}

\begin{proof}
To give the reader an idea of how these sorts of things go, we sketch the first half of the exercise. Let $j: Z\inj X$ be a closed embedding with $X$ a Zariski sheaf. Let $S=\Spec A$ be an affine scheme and $\U:=\{(U,i_U: U\inj S)\}$ an open covering of $S$. We claim that the natural map $\Hom_{\Space}(S,Z)\to\Psi(S,\U,Z)$ is a bijection. By assumption we know that the natural map $\Hom_{\Space}(S,X)\to\Psi(S,\U,X)$ is a bijection. Let $\{f_U: U\to Z\}_{U\in\U}\in\Psi(S,\U,Z)$. Then, $\{j\circ f_U: U\to X\}_{U\in\U}\in\Psi(S,\U,X)$ and so there exists a unique $g\in\Hom_{\Space}(S,X)$ such that $g|_U=j\circ f_U$ for every $U\in\U$. Fixing $U\in\U$, we have a commutative diagram
\begin{center}
\begin{tikzcd}
U \arrow[rrd, bend left, "f_U"] \arrow[rd, dotted, "\exists!"] \arrow[rdd, hookrightarrow, bend right, "i_U"'] & & \\
& S\times_XZ \arrow[r] \arrow[d] & Z \arrow[d, hookrightarrow, "j"] \\
& S \arrow[r, "g"'] & X
\end{tikzcd}
\end{center}
where the dotted arrow is induced by the universal property of the fiber product since $g\circ i_U=j\circ f_U$ by assumption. Our goal is to produce a morphism $f: S\to Z$ such that $g=j\circ f$. Geometrically, the way this works is clear. Since $j: Z\inj X$ is a closed embedding, the fiber product $S\times_XZ$ is isomorphic to $\Spec A/I$ for some ideal $I\normal A$. To say that $g$ factors through $Z$ is to say that $g$ kills $I$. The matter of whether $g$ kills $I$ is equivalent to whether every $g|_U$ kills $I$. This follows since each $f_U$ factors through $\Spec A/I$ and hence $g|_U=j\circ f_U$ kills $I$. We may thus fill in the above diagram to get
\begin{center}
\begin{tikzcd}
U \arrow[rrd, bend left, "f_U"] \arrow[rd] \arrow[rdd, hookrightarrow, bend right, "i_U"'] & & \\
& S\times_XZ \arrow[r] \arrow[d] & Z \arrow[d, hookrightarrow, "j"] \\
& S \arrow[r, "g"'] \arrow[ru, dotted, "\exists\,f"] & X
\end{tikzcd}
\end{center}
from which we immediately conclude that $f\in\Hom_{\Space}(S,Z)$ is a lift of $\{f_U\}_{U\in\U}\in\Psi(S,\U,Z)$. The uniqueness of this lift follows from the uniqueness of $g$.

Algebraically, what is happening in the affine case where $Z\inj X$ corresponds to $\Spec B/J\inj\Spec B$ and $g: \Spec A\to X$ corresponds to $\phi: B\to A$ is that, writing each $U\in\U$ as $D(I_U)$, we have $A=\sum_{U\in\U}I_U$ and so 
$$\phi(J)=\phi(J)\cap\sum_{U\in\U}I_U=\sum_{U\in\U}\phi(J)\cap I_U=\sum_{U\in\U}0=0.$$
Hence, we have a factorization
\begin{center}
\begin{tikzcd}
& B/J \arrow[ld, dotted, "\exists!"'] \\
A & B \arrow[l, "\phi"] \arrow[u, twoheadrightarrow]
\end{tikzcd}
\end{center}
inducing a factorization
\begin{center}
\begin{tikzcd}
& Z \arrow[d, hookrightarrow, "j"] \\
S \arrow[r, "g"'] \arrow[ru, dotted, "\exists!\,f"] & X
\end{tikzcd}
\end{center}
as desired.
\end{proof}

\begin{theorem}
Let $X\in\Aff\Sch$. Then, $X$ is a scheme.
\end{theorem}

It follows that $\Aff\Sch$ is a full subcategory of $\Sch$, justifying the name and the notation. The crux of the matter is proving that $X$ is a Zariski sheaf. Before diving into the proof, we introduce an important slogan that will guide us in our journey through algebraic geometry.

\textbf{\un{Slogan}:} ``Think geometrically and prove algebraically.''

Let's try to put this slogan into action.

\begin{exercise}
Note that $\A^n$ has a closed subspace $0:=\Spec\Z[t_1,\ldots,t_n]/(t_1,\ldots,t_n)\iso\Spec\Z$ inducing an open embedding $\A^n\setminus0\inj\A^n$.
\begin{enum}{\alph}
\item Show that $\A^1\setminus0$ is affine.

\item Show that $\A^n\setminus0$ is not affine for $n>1$.

\item Given $A\in\CRing$, show that $(\A^n\setminus0)(A)$ is equivalent to 
$$\left\{(a_1,\ldots,a_n)\in A^n : \sum_{i=1}^na_ix_i=1\textrm{ has a solution}\right\}.$$

\item Does this match your geometric intuition?
\end{enum}
\end{exercise}

How do we aim to prove our theorem? The key is to break things into several steps and do some bootstrapping. Here are the steps.
\begin{enum}{\arabic}
\item Show that $\A^1$ is a Zariski sheaf.

\item Let $T$ be any set. We immediately conclude that $\A^T\iso\Spec\Z[\{x_t : t\in T\}]$ is a Zariski sheaf.

\item Let $X=\Spec A$ be any affine scheme. Then, $A$ can be identified with the quotient of some polynomial algebra $\Z[\{x_t : t\in T\}]$ by an ideal of relations. Hence, we have a closed embedding $X\inj\A^T$ and so $X$ is a Zariski sheaf since closed embeddings preserve the Zariski sheaf condition.
\end{enum}

At this point, the only unresolved step is step \textup{(1)}.

\begin{definition}
Given $X\in\Space$, let $\Func(X):=\Hom_{\Space}(X,\A^1)$ denote the collection of \textbf{functions} on $X$.
\end{definition}

\begin{example}
We have 
\begin{align*}
\Func(\A^1)
=\Hom_{\Space}(\A^1,\A^1)
\iso\Hom_{\CRing}(\Spec\Z[t],\Spec\Z[t])
\iso\Z[t].
\end{align*}
This explains the use of the term ``function.'' More generally, given $A\in\CRing$, $\Func(\Spec A)$ simply recovers the underlying set of $A$. We therefore think of elements of $\Func(\Spec A)$ as functions on $A$, which by Yoneda's Lemma correspond to maps $\Spec A\to\A^1$.
\end{example}

\begin{exercise}
\hfill
\begin{enum}{\alph}
\item Show that we obtain a functor $\Func: \Space^{\op}\to\CRing$ satisfying
$$\Func(X)\iso\lim_{\Spec A\to X}A$$
as rings.

\item Another way of viewing this is to note that $\A^1$ is naturally a commutative ring object in $\Space$ -- i.e., $\A^1\in\CAlg(\Space)\simeq\Fun(\CRing,\CRing)$. Can you relate the above to the identity functor $\id_{\CRing}$?\footnote{Hint: Consider the restriction of $\Func$ to $\Aff\Sch^{\op}\simeq\CRing$.}
\end{enum}
With all of this in mind it is common to refer to $\Func(X)$ as the \textbf{ring of functions on $X$}.
\end{exercise}

\begin{exercise}
Fix $A\in\CRing$ and $f\in A$. Thinking of $f$ as a function on $A$, show that 
$$\Spec A\times_{\A^1}(\A^1\setminus0)\iso D(f).$$
In other words, there is a natural Cartesian diagram
\begin{center}
\begin{tikzcd}
D(f) \arrow[r, "*"] \arrow[d, "*"'] & \Spec A \arrow[d, "f"] \\
\A^1\setminus0 \arrow[r, hookrightarrow] & \A^1
\end{tikzcd}
\end{center}
Note that the arrows marked with $*$ need to be defined since the goal is to show that $D(f)$ is the pullback. This helps cement our geometric intuition that $D(f)$ is the nonvanishing locus of $f$.
\end{exercise}

\begin{exercise}
Let $A\in\CRing$ and $I\normal A$. Then, $\{D(f) : f\in I\textrm{ nonzero}\}$ is an open covering of $D(I)$. Note that $D(I)$ need not be affine and that part of this exercise involves explicitly constructing open embeddings $D(f)\inj D(I)$ for $f\in I$ nonzero.
\end{exercise}

\begin{proposition}
Let $S=\Spec A\in\Aff\Sch$ and $\U\in\Cov(S)$. Then, there exists a big principal open covering of $S$ refining $\U$.
\end{proposition}

\begin{proof}
Each $U\in\U$ looks like $D(I_U)$ for some $I_U\normal A$ and so we know that $\sum_{U\in\U}I_U=A$. Consider the collection
$$\V:=\bigcup_{U\in\U}\{D(f) : f\in I_U\textrm{ nonzero}\}.$$
This is an open covering of $S$ since $I_U$ is generated by its nonzero elements. Moreover, it is immediate from the previous exercise that $\V$ is a refinement of $\U$.
\end{proof}

\begin{exercise}
Let $A\in\CRing$ and $S\subset A$ a multiplicative subset. Show that localization at $S$ is exact -- i.e., that $S^{-1}A$ is flat as an $A$-module. Note that, given $M\in\Mod_A$, $S^{-1}M\iso S^{-1}A\tensor_AM$ as $A$-modules.
\end{exercise}

Suppose now that $S=\Spec A\in\Aff\Sch$ and $\U=\{D(f_i)\}_{i\in T}\in\Cov(S)$ is a big principal open covering. We will show that $\A^1$ satisfies the Zariski sheaf condition with respect to $\U$ and hence that $\A^1$ is a Zariski sheaf by prior work. We need to show that the natural map $\Func(S)\to\Psi(S,\U,\A^1)$ is a bijection. Using our earlier comments on functions, this is equivalent to the natural map $\Phi$ from $A$ to the collection of $\{g_i\in A_{f_i}\}_{i\in T}$ such that $g_i$ and $g_j$ have the same image in $A_{f_if_j}\iso A_{f_jf_i}$ for all $i,j\in T$. Both the domain and codomain of $\Phi$ are naturally $A$-modules and $\Phi$ itself is an $A$-module homomorphism. There are two maps
$$\prod_{i\in T}A_{f_i}\rightrightarrows\prod_{i,j\in T}A_{f_if_j}$$
and the codomain of $\Phi$ can naturally be expressed as the kernel of the difference of these maps. Our key input is the following exercise.

\begin{exercise}
Let $M,N\in\Mod_A$ and $\phi\in\Hom_{\Mod_A}(M,N)$. Then, $\phi$ is injective (resp., surjective) if and only if $\phi_{f_i}: M_{f_i}\to N_{f_i}$ is injective (resp., surjective) for every $i\in T$.\footnote{Hint: Use the facts that localization at any fixed $f_i$ is exact and that $f_1,\ldots,f_n\in A$ generate $A$ if and only if $f_1^r,\ldots,f_n^r$ generate $A$ for some $r\geq1$.}
\end{exercise}

This immediately tells us that $\Phi$ is injective. To see that $\Phi$ is surjective, one first verifies this in the case that $\U$ is finite using the result of the previous exercise. To handle the general case, the key is that only finitely many of the $f_i$ are needed to generate $A$.

\begin{exercise}
Fill in the details to finish this proof.
\end{exercise}

\section{Relative Algebraic Geometry - Fiber Products and Relative Schemes}
\begin{theorem}
$\Sch\subset\Space$ is closed under fiber products.
\end{theorem}

Given $S\in\Space$, we define once and for all
$$\A_S^n:=S\times_{\Spec\Z}\A_{\Z}^n.$$
If $S$ is a scheme then $\A_S^n$ is a scheme by the theorem and if $S=\Spec A$ is affine then $\A_S^n\iso\Spec A[t_1,\ldots,t_n]$ hence is affine. For ease of notation we write $\A_A^n$ instead of $\A_{\Spec A}^n$.

\begin{proof}
Let $X\to Z$ and $Y\to Z$ be maps of schemes. We claim that $X\times_ZY$ is a scheme. The key is to break the argument into steps characterized by the following assumptions.
\begin{enum}{\arabic}
\item $X,Y,Z$ are affine.

\item $X,Z$ are affine (same as $Y,Z$ affine by symmetry).

\item $Z$ is affine.

\item $X,Y$ are affine.

\item No affine assumptions.
\end{enum}
Steps \textup{(1-3)} are easy and so we leave them to the reader. Step \textup{(5)} also follows easily from step \textup{(4)} after taking affine open coverings of $X$ and $Y$. So, let's tackle step \textup{(4)}. Let $\W$ be an affine open covering. Given $W\in\W$, consider the commutative diagram
\begin{center}
\begin{tikzcd}[row sep=scriptsize, column sep=scriptsize]
& W' \arrow[dl] \arrow[rr] \arrow[dd, dotted, blue] & & W\times_ZY \arrow[dl] \arrow[dd, green] \\
X\times_ZW \arrow[rr, crossing over, near end] \arrow[dd, green] & & W \\
& X\times_ZY \arrow[dl] \arrow[rr] & & Y \arrow[dl] \\
X \arrow[rr] & & Z \arrow[from=uu, crossing over, hookrightarrow]\\
\end{tikzcd}
\end{center}
Here, the bottom, top, right, and front faces are all Cartesian by assumption. The blue arrow is induced by the universal property of $W'$. It is a general categorical fact that the left and back faces are Cartesian as well, giving us a Cartesian cube.\footnote{The fact that the diagram commutes is roughly equivalent to the ``associativity'' of fiber products.} The Green Lemma tells us that the green arrows are open embeddings, thus $X\times_ZW$ and $W\times_ZY$ are both quasiaffine hence schemes.\footnote{By definition, a quasiaffine scheme is an open subscheme of an affine scheme. Earlier results show that quasiaffine schemes are covered by affine schemes and inherit the Zariski sheaf condition hence are themselves schemes.} The Blue Lemma tells us that the blue arrow $W'\inj X\times_ZY$ is an open embedding. It follows that the collection of these open embeddings is an open covering of $X\times_ZY$ by schemes and hence that $X\times_ZY$ is itself a scheme.
\end{proof}

The following exercises are all useful for understanding the technical workings of the above proof.

\begin{exercise}
Let $\mc{C}$ be a category and 
\begin{center}
\begin{tikzcd}
X \arrow[r] \arrow[d] & Y \arrow[d] \\
Z \arrow[r] \arrow[d] & W \arrow[d] \\
U \arrow[r] & V
\end{tikzcd}
\end{center}
a commutative diagram in $\mc{C}$ such that the top and bottom squares are Cartesian. Show that the outer square is Cartesian.
\end{exercise}

\begin{exercise}
Let $U\to X$ be a map of schemes and $\V\in\Cov(U)$ such that the composition $V\inj U\to X$ is an open embedding for every $V\in\V$. Then, $U\to X$ is an open embedding. Think of this as a kind of recognition principle for open embeddings of schemes.
\end{exercise}

\begin{exercise}
Show that a space with an open covering by schemes is a scheme.
\end{exercise}

\begin{exercise}[Green Lemma]
Let $W\inj Z$ be an open embedding and $X\to Z$ a map of spaces with $W,X\in\Aff\Sch$ and $Z\in\Sch$. Then, the induced map $X\times_ZW\to W$ is an open embedding.
\end{exercise}

\begin{exercise}[Blue Lemma]
Let $T\inj X$ be an open embedding and $S\to X$ a map of spaces with $X\in\Aff\Sch$ and $T\in\Sch$. Then, the induced map $T\times_XS\to S$ is an open embedding.
\end{exercise}

\begin{definition}
Fix a base space $S\in\Space$. We define the category $\Sch_S$ of \textbf{$S$-schemes} to be the full subcategory of spaces $X/S\in\Space_S$ such that $T\times_SX\in\Sch$ for every map of spaces $T\to S$ with $T$ affine. Similarly, we define the category $\Aff\Sch_S$ of \textbf{$S$-affine schemes} to be the full subcategory of spaces $X/S\in\Space_S$ such that $T\times_SX\in\Aff\Sch$ for every map of spaces $T\to S$ with $T$ affine.
\end{definition}

\begin{remark}
If $S=\Spec A$ then we often write $\Space_A$ and $\Sch_A$ instead of $\Space_{\Spec A}$ and $\Sch_{\Spec A}$. Similarly, we use the names \emph{$A$-space} and \emph{$A$-scheme}.
\end{remark}

By definition, objects of $\Aff\Sch_S$ are exactly affine maps $X\to S$ of spaces. The category $\Sch_S$ should not be confused with the overcategory $\Sch_{/S}$, even though both are full subcategories of $\Space_S$. If $S\in\Sch$ and $X/S\in\Sch_{/S}$ then $T\times_SX\in\Sch$ for every $T/S\in\Sch_{/S}$ (in particular, the affine ones), so the inclusion $\Sch_{/S}\inj\Space_S$ factors through $\Sch_S$, giving a natural inclusion $\Sch_{/S}\inj\Sch_S$. 

\begin{theorem}
Let $S\in\Sch$. Then, the natural inclusion $\Sch_{/S}\inj\Sch_S$ is an equivalence of categories (i.e., it is essentially surjective).\footnote{In fact, the two are isomorphic as categories!} Equivalently, if $X/S\in\Space_S$ satisfies $T\times_SX\in\Sch$ for every $T/S\in\Aff\Sch_{/S}$ then $X\in\Sch$.
\end{theorem}

In other words, schemes over $S$ are always $S$-schemes and the two are the same thing if $S\in\Sch$.

\begin{exercise}
Prove the theorem!
\end{exercise}

\begin{remark}
Assuming $S\in\Sch$, this result says that objects of $\Aff\Sch_S$ are exactly affine maps $X\to S$ of schemes. As we will see later after discussing quasicoherent sheaves, there is a canonical equivalence $\Aff\Sch_S\simeq\CAlg(\QCoh(S))$.
\end{remark}

With all of this in mind, let's revise our earlier slogan.

\textbf{\un{Slogan}:} ``Algebraic geometry relative to $S$ is geometry over $\Sch_S$.''

\begin{remark}
To give more credit to this slogan, nearly all of our previous work involving open coverings (including the definition of open covering) readily extends to the setting of $\Space_S$ and $\Sch_S$.\footnote{The reader should of course verify this!} Thus, the geometry extends as well.
\end{remark}

With this modification we can suitably adapt the earlier notions of base change and descent. We obtain two different versions for schemes and general spaces, with the former weaker than the latter. If no specification is made then the reader should assume that we are working with schemes and not general spaces.

\begin{example}
We claim that the property of being affine is stable under base change. Explicitly, this translates to the statement that, given $Y\to X$ an affine map of schemes and $Z\to X$ any map of schemes, the induced map $Z\times_XY\to Z$ is affine. So, let $\Spec A\to Z$ be a map of schemes. We need to show that $W:=\Spec A\times_Z(Z\times_XY)$ is affine. This follows since $W\iso\Spec A\times_XY$ is affine because $Y\to Z$ is affine.
\end{example}

\begin{exercise}
Show that the property of being a closed or open embedding is stable under base change.
\end{exercise}

\begin{exercise}
Assume $S\in\Sch$. 
\begin{enum}{\alph}
\item Is it true that the natural inclusion $\Sch_{/S}\inj\Sch_S$ induces a natural inclusion $\Aff\Sch_{/S}\inj\Aff\Sch_S$? This is certainly true if $S$ is affine but what about more generally?

\item Assuming we have an inclusion $\Aff\Sch_{/S}\inj\Aff\Sch_S$, when do we get an equivalence? This is once again true when $S$ is affine but how about more generally?
\end{enum}
\end{exercise}

In general, we must be careful to distinguish between $S$-affine schemes and affine $S$-schemes.

\section{Zoology of Morphisms -- ``Topological'' Conditions}
Our goal in this section is to introduce and study several important properties of maps of schemes. Many of the properties defined here make sense for general maps of spaces but aren't nearly as well-behaved. For this reason we choose not to work in full generality. Whenever you encounter a new property of morphisms, you should should always ask yourself if the associated class of morphisms is closed under (at least) the following processes.
\begin{itemize}
\item Composition

\item Isomorphism

\item Base change
\end{itemize}

\begin{remark}
To give another random example of a process that might be worth considering, think about commutative triangles
\begin{center}
\begin{tikzcd}
X \arrow[rr, "f"] \arrow[rd, "g"'] & & Y \arrow[ld, "h"] \\
& Z &
\end{tikzcd}
\end{center}
It's natural to ask whether, supposing two of $f,g,h$ have a given property, must the third morphism have that property as well? For example, closure under composition says that $f,h$ having a certain property implies that $g$ also has that property. In general, if you can attach geometric meaning to a particular ``$2$-out-of-$3$'' process then that process is probably worth studying.
\end{remark}

\begin{definition}
Let $X\in\Space$. We say $X$ is
\begin{itemize}
\item \textbf{quasiaffine} if it is an open subspace of an affine scheme;

\item \textbf{quasicompact} or \textbf{qc} if every open covering of $X$ admits a finite subcovering;

\item \textbf{quasiseparated} or \textbf{qs} if the intersection of any two affine open subspaces of $X$ is qc;

\item \textbf{qcqs} if it is both qc and qs.
\end{itemize}
\end{definition}

\begin{example}
Affine schemes are qcqs. To see this, note that we know from an earlier proposition that affine schemes are qc. We also know that intersections of affine open subspaces of an affine scheme are themselves affine hence qc.
\end{example}

\begin{exercise}
Show that a quasiaffine space admits an open covering by affine schemes hence is a scheme. Give an example of a quasiaffine space with an affine open covering that admits no finite subcovering.\footnote{Hint: Consider a polynomial ring in a countably infinite number of variables.}
\end{exercise}

\begin{exercise}
Let $X\in\Sch$ be qc. Show that $X$ admits a closed point.
\end{exercise}

\begin{exercise}
By assumption, every scheme admits an affine open covering and so every qc scheme admits a finite affine open covering. Show that the converse is true as well. That is, a scheme admitting a finite affine open covering is qc. It follows that a scheme is qc if and only if it admits a finite open covering by qc schemes.
\end{exercise}

This result has an immediate application. Let $A\in\CRing$ and $I\normal A$. Then, $D(I)$ admits a big principal open covering $\{D(f_i)\}$ by any collection of $f_i\in A$ such that $\sum f_iA=I$. It follows immediately that $D(I)$ is not qc (and hence not affine) if $I$ is not finitely generated. If $I$ is finitely generated then it admits a principal open covering $\{D(f_1),\ldots,D(f_n)\}$. Each $D(f_i)$ is affine hence qc and so $D(I)$ is qc.

\begin{exercise}
Construct an example of a qc scheme that is not qs. This shows that a scheme being qs is not equivalent to admitting an open covering by qs schemes.\footnote{This follows since every scheme admits an open covering by affine schemes, which are themselves qcqs.}
\end{exercise}

\begin{exercise}
Given $A\in\CRing$ and $I\normal A$, when is $D(I)$ qs? Affine?
\end{exercise}

We now transfer the above to properties of morphisms.

\begin{definition}
We say that a map of spaces $\pi: X\to S$ is \textbf{quasiaffine} (resp., \textbf{qc}, \textbf{qs}) if, given any map of spaces $\Spec B\to S$, the induced fiber product $\pi^{-1}(\Spec B)=\Spec B\times_SX$ is quasiaffine (resp., qc, qs).
\end{definition}

\begin{exercise}
Show that $X\in\Space$ is quasiaffine (resp., qc, qs) if and only if $X\to\Spec\Z$ is quasi-affine (resp., qc, qs). 
\end{exercise}

\begin{exercise}
Show that the properties of being quasiaffine, qc, and qs are stable under base change.
\end{exercise}

\begin{exercise}
Show that a map of schemes is qc if and only if it pulls back qc schemes to qc schemes.
\end{exercise}

\begin{exercise}
Show that closed embeddings are qc. Are affine maps of schemes always qc?
\end{exercise}

Let $P$ be a property of maps of schemes. There are various condition we can impose on $P$ that dictate how it interacts with the underlying topology. We say that $P$ is \textbf{local on the target} if, given any map of schemes $\pi: X\to S$ and $\U\in\Cov(S)$, $\pi$ has $P$ if and only if the induced map $\pi^{-1}(U)\to X$ has $P$ for every $U\in\U$. We say that $P$ is \textbf{local on the source} if, given any map of schemes $\pi: X\to S$ and $\V\in\Cov(X)$, $\pi$ has $P$ if and only if the composition $V\to X\to S$ has $P$ for every $V\in\V$. We obtain the notions of \emph{affine-local on the target} and \emph{affine-local on the source} by restricting attention only to affine coverings. 

\begin{remark}
In general, being affine-local is a weaker condition than being local since we are restricting attention to a smaller class of coverings. Across all of algebraic geometry there are many types of ``weakened locality'' that are useful in practice.
\end{remark}

\begin{exercise}
Show that the following properties are affine-local on the target.\footnote{For intuition on how to go about this skip ahead to when we discuss the Affine Communication Lemma.}
\begin{enum}{\arabic}
\item Qc

\item Qs

\item Affine

\item Closed embedding
\end{enum}
\end{exercise}

\begin{exercise}
Are open embeddings affine-local on the target?
\end{exercise}

\begin{exercise}
Construct examples showing that the properties of being qc, affine, or a closed embedding are not affine-local on the source.
\end{exercise}

All of this is well and good but is there a more geometric way that we can think about quasiseparatedness? The affirmative answer comes from looking at diagonals. Let $\pi: X\to S$ be a map of schemes. Associated to this is the diagonal morphism $\Delta=\Delta_{X/S}=\Delta_{\pi}: X\to X\times_SX$ obtained by taking 
$$\Delta(A): X(A)\to(X\times_SX)(A)=X(A)\times_{S(A)}X(A)$$
to be the diagonal map.\footnote{Note that this construction is functorial in $A$.} 

\begin{exercise}
If $X=\Spec B$ and $S=\Spec A$ then we have 
$$\Delta: \Spec B\to\Spec B\times_{\Spec A}\Spec B\iso\Spec(B\tensor_AB)$$
and so $\Delta$ corresponds to a ring map $\rho: B\tensor_AB\to B$. Check that $\rho$ is exactly the multiplication map $x\tensor y\mapsto xy$.
\end{exercise}

\begin{exercise}
Show that $\pi\in\Hom_{\Sch}(X,S)$ is qs if and only if $\Delta_{\pi}$ is qc.
\end{exercise}

Since closed embeddings are qc, we can strengthen the condition on the diagonal morphism from being qc to being a closed embedding. This gives us the notion of a \textbf{separated} morphism. Separated schemes are analogous to Hausdorff topological spaces.\footnote{Expanding on this analogy further, let $X$ be a topological space and equip $X\times X$ with the product topology. Then, the diagonal in $X\times X$ is closed if and only if $X$ is Hausdorff.}

\begin{example}
The classic example of a non-Hausdorff topological space is the line with two origins. We can mimic this example in the algebraic setting by taking two copies of $\A^1$ and gluing them along $\A^1\setminus0$. As a space, this is given by taking the pushout of
\begin{center}
\begin{tikzcd}
\A^1\setminus0 \arrow[r, hookrightarrow] \arrow[d, hookrightarrow] & \A^1 \\
\A^1 &
\end{tikzcd}
\end{center}
This corresponds to the diagram of rings
\begin{center}
\begin{tikzcd}
\Z{[}z^{\pm1}{]} & \arrow[l, "x\mapsto z"'] \Z{[}x{]} \\
\Z{[}y{]} \arrow[u, "y\mapsto z"] &
\end{tikzcd}
\end{center}
\end{example} 

Note that there is some subtlety here since general pushouts don't exist in $\Sch$. In particular, one needs to show that the above pushout space is actually a scheme. We will talk more about such gluing constructions later on.

\begin{exercise}
Show that the algebraic line with two origins is not a separated space. Can you come up with a geometric procedure to test if a space is separated?
\end{exercise}

\begin{exercise}
Suppose that we instead glue two copies of $\A^1$ along $\A^1\setminus0$ using the identifications
\begin{center}
\begin{tikzcd}
\Z{[}z^{\pm1}{]} & \arrow[l, "x\mapsto z"'] \Z{[}x{]} \\
\Z{[}y{]} \arrow[u, "y\mapsto z^{-1}"] &
\end{tikzcd}
\end{center}
Describe the resulting space. Is it a scheme? Is it separated?
\end{exercise}

\begin{comment}
We can glue schemes along closed subschemes.
More generally, when can we say that pushouts exist in $\Sch$?
What other properties do qc, qs, and separated morphisms have? (see Stacks Project)
\end{comment}

\section{Bridging the Gap}
Our goal in this section is to connect our theory of schemes with the more traditional theory of schemes. With this in mind, let $\Op(X)$ denote the category of open subschemes of $X$ and $\Aff\Op(X)$ the full subcategory of affine open subschemes. Explicitly, morphisms in $\Op(X)$ are commutative triangles
\begin{center}
\begin{tikzcd}
V \arrow[r, hookrightarrow] \arrow[rd, hookrightarrow] & U \arrow[d, hookrightarrow] \\
& X
\end{tikzcd}
\end{center}
with all arrows open embeddings. Note that we need not assume that $V\to U$ is an open embedding. Let's look at this in the affine setting. Let $X=\Spec A$ and consider ideals $J\subset I\subset A$. Then, there is a natural isomorphism of rings
$$(A/J)/(I/J)\xto{\sim}A/I$$
by the Third Isomorphism Theorem and so we obtain a canonical closed embedding $V(I)\inj V(J)$. This in turn induces a natural map $D(J)\to D(I)$ such that the composition $D(f)\inj D(J)\to D(I)$ is an open embedding for every $f\in J$. It follows that the natural map $D(J)\to D(I)$ is itself an open embedding.

\begin{exercise}
Reduce to the affine setting.
\end{exercise}

\begin{exercise}
Is it true that the map $V\to U$ making the diagram commute is unique up to unique isomorphism?
\end{exercise}

This shows that $\Op(X)$ is a full subcategory of $\Sch_{/X}$ and so $\Aff\Op(X)$ is a full subcategory of $\Aff\Sch_{/X}$. We may then consider the composition 
\begin{center}
\begin{tikzcd}
\Aff\Op(X) \arrow[r, "\oblv"] & \Aff\Sch \arrow[r, "\Spec^{-1}"] & \CRing^{\op}
\end{tikzcd}
\end{center}
where the first functor forgets the map to $X$. Taking the opposite of the essential image of this composition gives a full subcategory $\CRing_X\subset\CRing$. 

\begin{exercise}
Show that, as a space, $X$ is completely and uniquely determined by its restriction to $\CRing_X$.\footnote{Hint: Consider first the affine case and show that there is a canonical equivalence $\CRing_{\Spec A}\simeq\CAlg_A$.}
\end{exercise}

\begin{theorem}[Affine Communication Lemma]
Let $P$ be a class of objects in $\Aff\Op(X)$ with the following properties.
\begin{enum}{\roman}
\item Suppose $(\Spec A\inj X)\in P$. Then, the induced map $D(f)\inj\Spec A\to X$ is contained in $P$ for every $f\in A$.

\item Let $(\Spec A\inj X)\in\Aff\Op(X)$. Suppose $f_1,\ldots,f_n\in A$ such that $(f_1,\ldots,f_n)=A$ and each composition $D(f_i)\inj\Spec A\inj X$ is contained in $P$. Then, $\Spec A\inj X$ is contained in $P$.

\item There exists an affine open covering of $X$ with objects contained in $P$.
\end{enum}
Then, $P$ contains every object in $\Aff\Op(X)$.
\end{theorem}

This is an excellent tool for analyzing properties of maps of schemes. In particular, the Affine Communication Lemma lets us easily check if a property is affine-local on the target. However, we don't want to rely on this tool too much since we want to keep our intuition to be in line with general spaces.

\begin{exercise}
\hfill
\begin{enum}{\alph}
\item Prove the Affine Communication Lemma.

\item Does a class of objects in $\Aff\Sch_{/X}$ satisfying the Affine Communication Lemma necessarily contain every object in $\Aff\Sch_{/X}$?

\item Investigate the situation for general maps of schemes to $X$.
\end{enum}
\end{exercise}

Recall that the collection $\abs{\Spec A}$ of field-valued points of $\Spec A$ is closely linked to the collection $\Spec'A$ of prime ideals of $A$. It isn't quite correct to say the two contain the same data because the former also ``knows about'' the quotient maps $A\surj A/\mf{p}$ associated to prime ideals.

\begin{exercise}
Does $\abs{\Spec A}$ contain enough data to recover $\Spec'A$, viewed as a topological space equipped with the Zariski topology, up to homeomorphism?
\end{exercise}

Note that we much prefer to work with $\Spec A$ because of its good functorial properties. Note as well that the data of $X$ is certainly not the the same as the data of $|X|$ for a general space $X$. 

\begin{exercise}
Let $\Sch'$ denote the category of schemes viewed as suitably defined locally ringed spaces, and let $\Aff\Sch'$ denote the corresponding full subcategory of affine schemes. Given $X\in\Sch'$, consider the functor 
$$h^X: \CRing\to\Set,\qquad A\mapsto\Hom_{\Sch'}(\Spec'A,X).$$
Show that the functor 
$$\Sch'\inj\Space,\qquad X\mapsto h^X$$
induces an equivalence of categories $\Sch'\xto{\sim}\Sch$ restricting to an equivalence $\Aff\Sch'\xto{\sim}\Aff\Sch$.
\end{exercise}

There's a lot more that could be said about the equivalence $\Sch'\xto{\sim}\Sch$ but let's leave it at that for now.

\section{Quasicoherent Sheaves}
\subsection{Basics}
Fix a scheme $X\in\Sch$. How should we define a module over $X$? By definition, a \textbf{quasicoherent sheaf}\footnote{It's common for ``quasicoherent'' to be abbreviated to ``QC.'' We will try to avoid this terminology to minimize potential confusion with the abbreviation ``qc.''} is an assignment from objects in $\Aff\Sch_{/X}$ to $\Ab$ such that, for every $(f: \Spec A\to X)\in\Aff\Sch_{/X}$, the abelian group $\F_f$ is equipped with a natural (left) $A$-module structure (with morphisms of quasicoherent sheaves respecting this natural module structure).\footnote{Recall that there is a forgetful functor $\Mod_A\to\Ab$ obtained by remembering only the additive structure of a (left) $A$-module.} As it turns out, pinning down the precise meaning of ``natural'' here is a bit tricky. Given $g: \Spec B\to\Spec A$, we obtain both the $B$-modules $\F_{f\circ g}$ and $B\tensor_A\F_f$. A priori there is no reason that these two should be related, so we must specify an isomorphism
$$\alpha_{f,g}\in\Isom_{\Mod_B}(\F_{f\circ g},B\tensor_A\F_f).$$
Consider now a string of maps
\begin{center}
\begin{tikzcd}
\Spec C \arrow[r, "h"] & \Spec B \arrow[r, "g"] & \Spec A \arrow[r, "f"] & X
\end{tikzcd}
\end{center}
We want our above construction to be ``associative.'' The precise meaning of this statement is that the diagram
\begin{center}
\begin{tikzcd}
& C\tensor_B(B\tensor_A\F_f) & \\
C\tensor_A\F_f \arrow[ru, "\sigma_{f,g,h}^{-1}"'] & & C\tensor_B\F_{f\circ g} \arrow[lu, "\id_C\tensor\alpha_{f,g}"] \\
\F_{f\circ(g\circ h)} \arrow[rr, equals] \arrow[u, "\alpha_{f,g\circ h}"'] & & \F_{(f\circ g)\circ h} \arrow[u, "\alpha_{f\circ g,h}"]
\end{tikzcd}
\end{center}
commutes, where $\sigma_{f,g,h}: C\tensor_B(B\tensor_A\F_f)\xto{\sim}C\tensor_A\F_f$ is the natural isomorphism obtained by base change.\footnote{Even though $g,h$ don't explicitly appear in the domain and codomain of $\theta_{f,g,h}$ they are lurking in the background.} This weak associativity condition is called the \textbf{cocycle condition}. We call each isomorphism $\alpha_{f,g}$ a \textbf{cocycle}. 

What about morphisms of quasicoherent sheaves? The data of $\phi\in\Hom_{\QCoh(X)}(\F,\G)$ is the data of compatible linear maps $\phi_f: \F_f\to\G_f$ for every $(f: \Spec A\to X)\in\Aff\Sch_{/X}$. Given $(f: \Spec A\to X)\in\Aff\Sch_{/X}$, any morphism to this object in $\Aff\Sch_{/X}$ is uniquely encoded by a commutative diagram
\begin{center}
\begin{tikzcd}
\Spec B \arrow[r, "g"] \arrow[rd, dotted, "f\circ g"'] & \Spec A \arrow[d, "f"] \\
& X
\end{tikzcd}
\end{center}
Letting $\beta_{f,g}$ be the relevant cocycle for $\G$, the compatibility condition mentioned above is that
\begin{center}
\begin{tikzcd}
\F_{f\circ g} \arrow[r, "\phi_{f\circ g}"] \arrow[d, "\alpha_{f,g}"'] & \G_{f\circ g} \arrow[d, "\beta_{f,g}"] \\
B\tensor_A\F_f \arrow[r, "\id_B\tensor\phi_f"'] & B\tensor_A\G_f
\end{tikzcd}
\end{center}
commutes and that $\phi_{f,g}$ in turn satisfies the cocycle condition (relative to any $h: \Spec C\to\Spec B$).

\begin{exercise}
Prove that $\QCoh(X)$ is indeed a category (i.e., verify the category axioms).
\end{exercise}

It is possible to realize $\QCoh(X)$ as a non-full subcategory of $\Fun(\Aff\Sch_{/X}^{\op},\Ab)$.\footnote{In fact, our argument shows that $\QCoh(X)$ is a non-full subcategory of the more ``universal'' category $\Fun(\Aff\Sch_{/X},\Set)$.} Using the above setup, to a given $\F$ we may associate $\un{\F}\in\Fun(\Aff\Sch_{/X}^{\op},\Ab)$ defined by $\un{\F}(f):=\F_f$, with the map from $f$ to $f\circ g$ in $\Fun(\Aff\Sch_{/X}^{\op},\Ab)$ given by the composition 
\begin{center}
\begin{tikzcd}
\F_f \arrow[r, "1_B\tensor\id"] & B\tensor_A\F_f \arrow[r, "\alpha_{f,g}^{-1}"] & \F_{f\circ g}
\end{tikzcd}
\end{center}
We then complete the picture by noting that
\begin{center}
\begin{tikzcd}
\F_f \arrow[r, "1_B\tensor\id"] \arrow[d, "\phi_f"'] & B\tensor_A\F_f \arrow[r, "\alpha_{f,g}^{-1}"] & \F_{f\circ g} \arrow[d, "\phi_{f\circ g}"] \\
\G_f \arrow[r, "1_B\tensor\id"'] & B\tensor_A\G_f \arrow[r, "\beta_{f,g}^{-1}"'] & \G_{f\circ g}
\end{tikzcd}
\end{center}
commutes and so $\phi$ encodes a natural transformation from $\un{\F}$ to $\un{\G}$. The resulting functor out of $\QCoh(X)$ is evidently faithful but not full because morphisms in $\Fun(\Aff\Sch_{/X}^{\op},\Ab)$ need not satisfy the cocycle condition.

\textbf{\textcolor{red}{A first-time reader should skip the following discussion enclosed in horizontal lines. Be warned that this discussion uses notation and concepts defined later on.}}

\hrule

Why have we chosen to write the cocycle condition using a trapezoidal shape instead of the usual square? There are two reasons for this. The first is to call into question the assumption that $\F_{f\circ(g\circ h)}$ and $\F_{(f\circ g)\circ h}$ are really the same thing. Of course, in the classical $1$-categorical setting this question does not uncover anything interesting since, of course, $f\circ(g\circ h)$ and $(f\circ g)\circ h$ are the same map by the usual associativity of composition. Despite the definite strength of assuming this ``strong'' form of associativity, time has shown that various weakened forms of associativity are also useful (and necessary to consider). As a simple illustration of this, note that $(M\tensor_AN)\tensor_AP$ and $M\tensor_A(N\tensor_AP)$ are not literally the same even both satisfy the same universal property (and so are the ``same'' for most algebraic intents and purposes). This brief discussion is one jumping off point for the theory of $\infty$-categories.

The second reason we write the diagram in a funny way is to call into question whether we actually need the diagram to commute ``on the nose'' (this is one jumping off point for the theory of stacks). Basically, instead of requiring the above cocycle diagrams to commute we can require that they commute up to a specified natural isomorphism, say $\eta_{\F}$ relative to $\F$ in the above setup. To $\G$ and the triple $(f,g,h)$ we may similarly associate $\eta_{\G}$. We could then require $\phi$ to send $\eta_{\F}$ to $\eta_{\G}$. This would give us a new category $\QCoh'(X)$, right? Not quite. The issue is that $\phi$ is being asked to do too much. As a functor $\phi$ only knows how to act on objects and morphisms -- in particular, it does not know how to act on the functors $\F,\G$ and the natural transformations $\eta_{\F},\eta_{\G}$. The solution to this problem is to think of $\QCoh'(X)$ not as a usual $1$-category but instead as a $2$-category! We then think of $\phi$ as a $2$-functor rather than a usual $1$-functor.

Fortunately for us, there is a bit of a miracle that occurs. There is a general procedure to turn $1$-categories into special kinds of $2$-categories, called \emph{strict} $2$-categories. Applying this procedure to $\QCoh(X)$ gives us a strict $2$-category which we will also denote $\QCoh(X)$. Both $\QCoh(X)$ and $\QCoh'(X)$ are then equivalent, not as $1$-categories but as $2$-categories. This is not all that amazing on its own. The amazing stuff comes when we look at sheaves. In a nutshell, both $\QCoh(X)$ and $\QCoh'(X)$ yield exactly the same answers for whatever we want to do regarding sheaves (e.g., computing cohomology). Part of this comes from the fact that most of the distinctions between the two fade away when passing to essential images. For this reason we will not really distinguish between $\QCoh(X)$ and $\QCoh'(X)$ -- in fact, we will mostly only be concerned with a sort of $1$-categorical ``shadow'' of $\QCoh'(X)$.

\begin{exercise}
In our definition of quasicoherent sheaves we have thought only about composable triples $(f,g,h)$. What if we played the same game with composable quadruples instead?\footnote{If you want to go down this rabbit hole then look up the \emph{nerve} of a (small) category.} Is the resulting (1-)category equivalent to $\QCoh(X)$?
\end{exercise}

All of the above indicates that our definition of quasicoherent sheaves is rather ``rigid.''

\begin{exercise}
Fix $\F\in\QCoh(X)$. What can be said about $\Aut_{\QCoh(X)}(\F)$?
\end{exercise}

\hrule

Altogether this is obviously a lot of data to keep track of and so it's nice to know which data is ``essential.''

\begin{theorem}
Let $X\in\Spec A$. Then, the \textbf{global section} functor 
$$\Gamma(X,\cdot): \QCoh(X)\to\Mod_A,\qquad \F\mapsto\F_{\id_X}$$
is an equivalence of categories with inverse given by $M\mapsto\F_M$ for $\F_M$ sending $f: \Spec B\to X$ to $B\tensor_AM$ (with cocycles defined by base change).
\end{theorem}

That is, the data of a quasicoherent sheaf over an affine scheme is simply the data of a module (with the rest of the information automatically accounted for by base change). It is common to denote $\F_M$ by $\twid{M}$.

\begin{example}
Given $X\in\Sch$, we may associate the \textbf{structure sheaf} $\O_X\in\QCoh(X)$ defined by sending $f: \Spec A\to X$ to $A$. In particular, $\O_{\Spec A}=\twid{A}\in\QCoh(\Spec A)$ and so $\Gamma(\Spec A,\O_{\Spec A})\iso A$.
\end{example}

\subsection{Pullback and Pushforward}
Given $\pi\in\Hom_{\Sch}(X,S)$, it is natural to ask how $\pi$ relates quasicoherent sheaves on $X$ and on $S$. Let's first think about going from $S$ to $X$. Given $\F\in\QCoh(S)$, we obtain a functor $\pi^*\F: \Aff\Sch_{/X}^{\op}\to\Ab$ via $(\pi^*\F)_f:=\F_{\pi\circ f}$.

\begin{exercise}
Show that $\pi^*\F\in\QCoh(X)$ and hence that there is an induced functor 
$$\pi^*: \QCoh(S)\to\QCoh(X).$$
\end{exercise}

We call $\pi^*$ the \textbf{pullback functor}. The following exercise describes what this functor looks like ``locally.''

\begin{exercise}
Let $\pi\in\Hom_{\Sch}(\Spec B,\Spec A)$ and $M\in\Mod_A$. Show that there is a canonical isomorphism $\F_{B\tensor_AM}\xto{\sim}\pi^*\F_M$ in $\QCoh(\Spec B)$.
\end{exercise}

\begin{exercise}
Let $\rho: S\to T$ be any map of schemes. Show that $(\rho\circ\pi)^*=\pi^*\circ\rho^*$ as functors from $\QCoh(T)$ to $\QCoh(X)$. Can you describe $(\cdot)^*$ itself as a contravariant functor?\footnote{To get started, the domain should probably be $\Map(\Sch)$, the so-called \emph{mapping category} of $\Sch$ whose objects are morphisms in $\Sch$ and morphisms are commutative squares.}
\end{exercise}

What about going from $S$ to $X$? Viewing $\QCoh(S)$ as a non-full subcategory of $\Fun(\Aff\Sch_{/S}^{\op},\Ab)$, the natural thing to do is construct a functor from $\Fun(\Aff\Sch_{/X}^{\op},\Ab)$ to $\Fun(\Aff\Sch_{/S}^{\op},\Ab)$. Given $\F\in\Fun(\Aff\Sch_{/X}^{\op},\Ab)$ and $(f: \Spec A\to S)\in\Aff\Sch_{/S}^{\op}$, we can consider the Cartesian square
\begin{center}
\begin{tikzcd}
\pi^{-1}(\Spec A) \arrow[r] \arrow[d] & \Spec A \arrow[d, "f"] \\
X \arrow[r, "\pi"'] & S
\end{tikzcd}
\end{center}
and define $(\pi_*\F)_f$ to be the induced map $\pi^{-1}(\Spec A)\to X$. The issue is that $\pi^{-1}(\Spec A)$ need not be affine! 

\begin{remark}
The above analysis shows that there is a well-defined functor 
$$\pi_*: \Fun(\Aff\Sch_{/X}^{\op},\Ab)\to\Fun(\Aff\Sch_S^{\op},\Ab).$$
Unfortunately, given general $S\in\Sch$ there is not even an inclusion $\Aff\Sch_{/S}\inj\Aff\Sch_S$ since non-qs schemes exist. Despite this difficulty, it may still be possible to view $\QCoh(S)$ as a non-full subcategory of $\Fun(\Aff\Sch_S^{\op},\Ab)$. Assuming this is possible, the question then becomes whether $\pi_*$ restricted to $\QCoh(X)$ factors through $\QCoh(S)$.
\end{remark}

\begin{exercise}
The previous remark shows that one way of getting around $\pi_*: \QCoh(X)\to\QCoh(S)$ not being defined is to place $\QCoh(S)$ inside of some larger category. Another method is to place conditions on $\pi$ or on the domain $X$ and codomain $S$ themselves. Play around with this. Are the various methods of obtaining $\pi_*$ compatible with each other?
\end{exercise}

We will return to this matter after addressing an important question.

\subsection{Quasicoherent Sheaves Are ``Sheafy''}
\textbf{\un{Question}:} What is ``sheafy'' about quasicoherent sheaves?

Let $X\in\Sch$ and $\U\in\Cov(X)$. Define $\QCoh(X;\U)$ to be category whose objects consist of collections $\{\F_U\}_{U\in\U}$ with $\F_U\in\QCoh(U)$ and cocycles 
$$\alpha_{U,V}\in\Isom_{\QCoh(U\cap V)}(\F_U|_{U\cap V},\F_V|_{U\cap V})$$
satisfying the cocycle condition defined analogously to before.

\begin{theorem}[Serre]
Let $X\in\Sch$ and $\U=\{(U,i_U: U\inj X)\}\in\Cov(X)$. The functor
$$\QCoh(X)\xto{\sim}\QCoh(X;\U),\qquad \F\mapsto\{i_U^*\F\}_{U\in\U}$$
is an equivalence of categories.\footnote{Note that in the case of Zariski open coverings there is a ``maximal'' choice of open covering. This need not be the case for general sites.}
\end{theorem}

The proof of this theorem is very similar to the proof that affine schemes are schemes. As in that setting the crucial step is establishing the affine case.

\begin{exercise}
Let $X=\Spec A\in\Aff\Sch$ and $\U:=\{D(f_i)\}_{i\in T}\in\Cov(X)$ be a big principal open covering. Given $M\in\Mod_A$, show that the natural map
$$M\to\eq\paren{\prod_{i\in T}M_{f_i}\rightrightarrows\prod_{i,j\in T}M_{f_if_j}}$$
is an isomorphism of $A$-modules.
\end{exercise}

Serre's theorem is incredibly useful because it gives us a much easier way to check if a given construction is quasicoherent.

\begin{exercise}
Finish the proof of Serre's theorem.
\end{exercise}

\begin{example}
Let $X:=\A^2\setminus0$, which is given exactly by $D(I)$ for $I:=(x,y)\normal\Z[x,y]$. Geometrically, $X$ should be covered by $(\A^1\setminus0)\times\A^1$ and $\A^1\times(\A^1\setminus0)$. In fact, it isn't hard to rigorously show that $D(x)\iso\Spec\Z[x^{\pm1},y]$ and $D(y)\iso\Spec\Z[x,y^{\pm1}]$ form an open covering of $X$. The data of a quasicoherent sheaf on $X$ is then the data of a $\Z[x^{\pm1},y]$-module $M$ and a $\Z[x,y^{\pm1}]$-module $N$ together with a coherent isomorphism $M[y^{\pm1}]\xto{\sim}N[x^{\pm1}]$ of $\Z[x^{\pm1},y^{\pm1}]$-modules.
\end{example}

This opens us up to think more about sections of quasicoherent sheaves. So, let $\F\in\QCoh(X)$. Given $i_U: U\inj X\in\Aff\Op(X)$, the sheaf $\F|_U=i_U^*\F\in\QCoh(U)$ is equivalent to the data of a unique module -- namely, the $\Gamma(U,\O_U)$-module $\Gamma(U,\F|_U)$. If we sample over $U$ in some affine open covering $\U\in\Cov(X)$ then these modules should be compatible in some precise sense and so we should be able to glue them together to define $\Gamma(X,\F)$. This approach is logical (and necessary in practice for doing computations), but has a few drawbacks.
\begin{itemize}
\item It's not a priori clear that $\Gamma(X,\F)$ is independent of $\U$.

\item Building off of the previous point, it's not immediately clear that $\Gamma(X,\F)$ is functorial in $X$ and $\F$ (which we certainly want if possible).

\item We would like to be able to directly define sections over any $U\in\Op(X)$.
\end{itemize}
The way around this is simple. Given $\F\in\QCoh(X)$ and $U\in\Op(X)$ (e.g., $X$ itself), define
$$\F(U)=\Gamma(U,\F):=\Hom_{\QCoh(U)}(\O_X|_U,\F|_U).$$
This is naturally an abelian group and clearly only depends on $\F|_U$. In fact, the whole thing is local with respect to $U$.

\begin{exercise}
Show that there is a canonical isomorphism $\O_X|_U\iso\O_U$ in $\QCoh(U)$.
\end{exercise}

\begin{exercise}
Let $A\in\CRing$ and $M\in\Mod_A$. Show that both notions of $\Gamma(\Spec A,\twid{M})$ agree canonically and so our new notion extends the previous one.
\end{exercise}

We extract from this the valuable functor $\Gamma_{\F}(\cdot):=\Gamma(\cdot,\F): \Op(X)^{\op}\to\Ab$. We also have the functor $\Gamma_X(\cdot):=\Gamma(X,\cdot): \QCoh(X)\to\Ab$.

\begin{exercise}
Fix $\F\in\QCoh(X)$ and $\U\in\Cov(X)$. Show that $\Gamma_{\F}$ is uniquely determined by its values on $\U$.
\end{exercise}

\begin{exercise}
How much data is needed to uniquely determine $\Gamma_X$?
\end{exercise}

It's natural to ask how $\Gamma_X$ changes with $X$. Given $\rho: Y\to X$ a map of schemes, there is a canonically induced functor $\rho^{-1}: \Op(X)\to\Op(Y)$ which we may equivalently view as a functor from $\Op(X)^{\op}$ to $\Op(Y)^{\op}$. At the same time, we have $\rho^*: \QCoh(X)\to\QCoh(Y)$. 

\begin{exercise}
Fix $\F\in\QCoh(X)$. We have a triangle
\begin{center}
\begin{tikzcd}
\Op(X)^{\op} \arrow[r, "\rho^{-1}"] \arrow[rd, "\Gamma_{\F}"'] & \Op(Y)^{\op} \arrow[d, "\Gamma_{\rho^*\F}"] \\
& \Ab
\end{tikzcd}
\end{center}
Does this triangle commute? Does it encode any interesting information about $\rho$?
\end{exercise}

\subsection{Pushforward Revisited and Adjunction}
Let's now revisit how to make sense of $\pi_*: \QCoh(X)\to\QCoh(S)$ given $\pi\in\Hom_{\Sch}(X,S)$ (by imposing some constraints). Our earlier discussion shows that
$$\pi_*: \Fun(\Aff\Sch_{/X}^{\op},\Ab)\to\Fun(\Aff\Sch_{/S}^{\op},\Ab),\qquad \F\mapsto(Y/S\mapsto\F(\pi^{-1}(Y)/X))$$
is a well-defined functor if (and only if) $\pi$ is affine. Asking that the same construction send $\QCoh(X)$ to $\QCoh(S)$ is a weaker condition.

\begin{exercise}
\hfill
\begin{enum}{\alph}
\item Suppose that $\pi$ is qcqs. Show that $\pi_*$ sends $\QCoh(X)$ to $\QCoh(S)$.

\item Suppose that $\pi_*$ sends $\QCoh(X)$ to $\QCoh(S)$. Is it necessarily true that $\pi$ is qcqs?

\item Let $\rho: S\to T$ be a map of schemes such that $\rho_*: \QCoh(S)\to\QCoh(T)$ is defined. Does it follow that $(\rho\circ\pi)_*=\rho_*\circ\pi_*$ as functors from $\QCoh(X)$ to $\QCoh(T)$?
\end{enum}
\end{exercise}

\begin{exercise}
Suppose that $\pi: X\to S$ is affine. 
\begin{enum}{\alph}
\item Show that $\pi^*\dashv\pi_*$ is an adjoint pair.\footnote{Hint: Reduce to the affine case and use the adjunction for extension/restriction of scalars.} That is, given $\G\in\QCoh(S)$ and $\F\in\QCoh(X)$, there is a natural bijection
$$\Hom_{\QCoh(X)}(\pi^*\G,\F)\iso\Hom_{\QCoh(S)}(\G,\pi_*\F).$$

\item Show that this remains true if $\pi$ is merely qcqs.
\end{enum}
\end{exercise}

The description of $\pi_*$ as right adjoint to $\pi^*$ characterizes it as a functor up to unique natural isomorphism. This provides another method to establish existence criteria for $\pi_*$.

\subsection{Tensor Products, Kernels, and Cokernels (Oh My!)}
Let's revisit the structure sheaf $\O_X\in\QCoh(X)$. 

\begin{exercise}
Show that $0$ is the zero object in $\QCoh(X)$.\footnote{By definition, this means that every quasicoherent sheaf on $X$ both receives a unique map from and admits a unique map to $0$.}
\end{exercise}

On $\QCoh(X)$ we can make sense of the tensor product $\tensor_{\O_X}$. 

\begin{exercise}
Let $\F,\G\in\QCoh(X)$. 
\begin{enum}{\arabic}
\item Show that there is a well-defined, uniquely\footnote{``Unique'' here (as elsewhere) means unique up to unique isomorphism.} determined quasicoherent sheaf $\F\tensor_{\O_X}\G$ with the property that 
$$(\F\tensor_{\O_X}\G)|_U=\F|_U\tensor_{\O_U}\G|_U$$ 
for every $U\in\Op(X)$.

\item Show that this defines a bifunctor $\tensor_{\O_X}: \QCoh(X)\times\QCoh(X)\to\QCoh(X)$.

\item Show that $(\F\tensor_{\O_X}\G)(U)\iso\F(U)\tensor_{\O_X(U)}\G(U)$ for every $U\in\Aff\Op(X)$.

\item Is it true that $(\F\tensor_{\O_X}\G)(U)\iso\F(U)\tensor_{\O_X(U)}\G(U)$ for every $U\in\Op(X)$?
\end{enum}
\end{exercise}

We can learn more about the bifunctor $\tensor_{\O_X}: \QCoh(X)\times\QCoh(X)\to\QCoh(X)$ by constructing an internal $\Hom$ for $\QCoh(X)$. 

\begin{exercise}
Let $\F,\G\in\QCoh(X)$. 
\begin{enum}{\arabic}
\item Show that there is a well-defined, uniquely determined quasicoherent sheaf $\HHom(\F,\G)=\HHom_{\O_X}(\F,\G)$ with the property that 
$$\HHom(\F,\G)|_U=\Hom_{\QCoh(U)}(\F|_U,\G|_U)$$ 
for every $U\in\Op(X)$.

\item Show that this defines a bifunctor $\HHom: \QCoh(X)^{\op}\times\QCoh(X)\to\QCoh(X)$.

\item Is it true that $\HHom(\F,\G)(U)\iso\Hom_{\Mod_{\O_X(U)}}(\F(U),\G(U))$ for $U\in\Aff\Op(X)$? How about general $U\in\Op(X)$?

\item Show that $\tensor_{\O_X}$ and $\HHom$ satisfy a generalized tensor-$\Hom$ adjunction. That is, letting $\mc{H}\in\QCoh(X)$, there is a bijection
$$\Hom_{\QCoh}(X)(\F\tensor_{\O_X}\G,\mc{H})\iso\Hom_{\QCoh}(X)(\F,\HHom(\G,\mc{H}))$$
functorial in $\F,\G,\mc{H}$.
\end{enum}
\end{exercise}

From the above we know that 
$$\cdot\tensor_{\O_X}\F\dashv\HHom(\F,\cdot)$$
as (covariant) endofunctors on $\QCoh(X)$ and so $\cdot\tensor_{\O_X}\F$ preserves colimits while $\HHom(\F,\cdot)$ preserves limits. We can also extract a contravariant tensor-$\Hom$ adjunction. Stating this precisely requires us to tread carefully.

\begin{remark}
Let $A,B,C$ be (not necessarily commutative) rings and let $\Mod_{(A,B)}$ (etc.) denote the category of $(A,B)$-bimodules. Let $M\in\Mod_{(A,B)}$, $N\in\Mod_{(B,C)}$, and $P\in\Mod_{(A,C)}$. Then, there are natural isomorphisms 
$$\Hom_C(M\tensor_BN,P)\iso\Hom_B(M,\Hom_C(N,P))$$
in $\Mod_{(A,A)}$ and
$$\Hom_A(M\tensor_BN,P)\iso\Hom_B(N,\Hom_A(M,P))$$
in $\Mod_{(C,C)}$.
\end{remark} 

With this in mind, we see that we have a contravariant adjunction 
\begin{center}
$\F\tensor_{\O_X}\cdot$ \reflectbox{$\dashv$} $\HHom(\cdot,\F)$
\end{center}
and so $\HHom(\cdot,\F)$ sends limits to colimits.

\begin{exercise}
Show that $\tensor_{\O_X}$ equips $\QCoh(X)$ with a strict symmetric monoidal structure with unit object $\O_X$.
\end{exercise}

\begin{exercise}
Let $\phi\in\Hom_{\QCoh(X)}(\F,\G)$.\footnote{Serre's theorem may be of use in this exercise.}
\begin{enum}{\alph}
\item Show that $\ker\phi$ exists in $\QCoh(X)$ and satisfies $(\ker\phi)|_U=\ker(\phi|_U: \F|_U\to\G|_U)$ for every $U\in\Op(X)$. Recall that $\ker\phi$ is by definition the limit of 
\begin{center}
\begin{tikzcd}
\F \arrow[r, "\phi"] & \G & 0 \arrow[l]
\end{tikzcd}
\end{center}

\item Show that $\coker\phi$ exists in $\QCoh(X)$ and satisfies $(\coker\phi)|_U=\coker(\phi|_U: \F|_U\to\G|_U)$ for every $U\in\Op(X)$. Recall that $\coker\phi$ is by definition the colimit of 
\begin{center}
\begin{tikzcd}
0 & \F \arrow[l] \arrow[r, "\phi"] & \G
\end{tikzcd}
\end{center}

\item Show that there are canonical isomorphisms
\begin{center}
$(\ker\phi)(U)\iso\ker(\phi(U): \F(U)\to\G(U))$ and $(\coker\phi)(U)\iso\coker(\phi(U): \F(U)\to\G(U))$ 
\end{center}
for every $U\in\Aff\Op(X)$. How about general $U\in\Op(X)$?
\end{enum}
\end{exercise}

\begin{exercise}
\hfill
\begin{enum}{\alph}
\item Show that $\QCoh(X)$ admits a biproduct $\oplus$.

\item Show that $\QCoh(X)$ is an abelian category.

\item Explore how $\tensor_{\O_X}$ interacts with kernels and cokernels in $\QCoh(X)$.

\item Is $\QCoh(X)$ complete? Cocomplete?
\end{enum}
\end{exercise}

\begin{remark}
In case the reader has not figured this out yet, we will give brief descriptions of how to construct (co-)kernels, tensor products, and the internal $\Hom$ in $\QCoh(X)$. Let $f:\Spec A\to X$ and $\F,\G\in\QCoh(X)$. We take $(\F\tensor_{\O_X}\G)_f$ to be $\F_f\tensor_A\G_f$. Similarly, we take $(\HHom(\F,\G))_f$ to be $\Hom_{\Mod_A}(\F_f,\G_f)$. Given $\phi: \F\to\G$ we take $(\ker\phi)_f$ to be $\ker\phi_f$ and $(\coker\phi)_f$ to be $\coker\phi_f$. What to do with cocycles and morphisms is left to the reader.
\end{remark}

\subsection{Global Spec}
\begin{exercise}
Show that the data of a commutative algebra object $\AA$ in $\QCoh(X)$ is equivalent to the data of $\AA$ (as a quasicoherent sheaf) together with an associative commutative multiplication map $m: \AA\tensor_{\O_X}\AA\to\AA$ and algebra unit map $1_{\AA}: \O_X\to\AA$.
\end{exercise}

Feel free to take this as the definition of a commutative algebra object in $\QCoh(X)$. This gives us a convenient way to view $\CAlg(\QCoh(X))$, which then necessarily contains $\O_X$ as its initial object.\footnote{This is very similar to how $\Z$ is the initial object in $\CRing$ and, more generally, $A$ is the initial object in $\CAlg_A$.}

\begin{remark}
The above way of thinking about commutative algebra objects in $\QCoh(X)$ envisions them as commutative $\O_X$-algebras. There are least two other ways to make sense of commutative algebra objects in $\QCoh(X)$. One of these is to take general ring objects in $\QCoh(X)$.\footnote{If this seems funny to you then recall that every commutative ring is naturally a commutative $\Z$-algebra.} Another way is to revisit the definition of a quasicoherent sheaf. We want to say that $\AA\in\QCoh(X)$ is a commutative algebra object if, first of all, the $A$-module $\AA_f$ associated to $f: \Spec A\to X$ is actually a (commutative) $A$-algebra. Second, the cocycle $\alpha_{f,g}\in\Isom_{\Mod_B}(\AA_{f\circ g},B\tensor_A\AA_f)$ associated to $g: \Spec B\to\Spec A$ should actually be an isomorphism of (commutative) $B$-algebras. No change needs to be made to the cocycle condition since the constructions involving modules work just as well for algebras.\footnote{In more general settings (which we will not concern ourselves with us) the cocycle condition does need to be modified.} 
\end{remark}

\begin{exercise}
Let $X=\Spec A$. Show that the restriction of $\Gamma_X: \QCoh(X)\to\Mod_A$ to $\CAlg(\QCoh(X))$ induces an equivalence of categories $\CAlg(\QCoh(X))\xto{\sim}\CAlg_A$. In particular, this restriction factors through $\CRing$.
\end{exercise}

This shows that, given $\AA\in\CAlg(\QCoh(X))$, $\AA(U)$ is a commutative $\O_X(U)$-algebra for every $U\in\Aff\Op(X)$ and, in fact, this data plus some compatibility conditions on overlaps completely and uniquely determines $\AA$.\footnote{With a bit more work one can show that we only need to know the sections over a given affine open covering.} In other words, a commutative algebra over $X$ is the same thing as a compatible collection of commutative rings associated to the affine open subschemes of $X$. This perspective helps unify the two above perspectives on commutative algebra objects.

\begin{exercise}
Check directly that the three perspectives on commutative algebra objects yield equivalent categories.\footnote{In fact, these categories are very nearly isomorphic since they are all characterized by a sort of universal property.}
\end{exercise}

Switching gears a bit, it turns out that $\CAlg(\QCoh(X))$ encodes a lot of rich geometric information and is actually equivalent to a different category we have encountered previously. As a rough heuristic, imagine that we want to ``glue together'' the functors $\Gamma_{\Spec A}$ for varying $A\in\CRing$. With this in mind, let $\rho: Y\to X$ be any map of schemes.

\begin{exercise}
\hfill
\begin{enum}{\alph}
\item Show that there is a natural map $\rho^*\O_X\to\O_Y$. Show, moreover, that this map is an isomorphism. 

\item Assume that $\rho_*$ is well-defined. Is it necessarily true that the induced map $\O_X\to\rho_*\O_Y$ is an isomorphism?

\item Let $\F,\G\in\QCoh(X)$. Show that there is a natural isomorphism $\rho^*(\F\tensor_{\O_X}\G)\xto{\sim}\rho^*\F\tensor_{\O_Y}\rho^*\G$.
\end{enum}
\end{exercise}

One interesting consequence of this is that the assignment 
$$X\mapsto\Gamma(X,\O_X)=\End_{\QCoh(X)}(\O_X)$$
from $\Sch$ to $\CRing$ is contravariantly functorial in $X$ and so defines a functor $\Gamma: \Sch^{\op}\to\CRing$. Is this related to the functor $\Spec: \CRing\to\Sch$?

\begin{exercise}
Show that there is a contravariant adjunction between $\Gamma$ and $\Spec$. Which is the left adjoint and which is the right adjoint? Is $\Gamma(X)$ related to the category $\CRing_X$?
\end{exercise}

Assume now that $\rho_*$ is defined (which holds, e.g., if $\rho$ is affine). Then, the adjunction $\rho^*\dashv\rho_*$ provides counit $\eps_{\rho}: \rho^*\rho_*\to\id_{\QCoh(Y)}$ and unit $\eta_{\rho}: \id_{\QCoh(X)}\to\rho_*\rho^*$. Hence, we have\footnote{This is the canonical structure map for $\rho_*\O_Y$ as an $\O_X$-module.} 
$$\O_X\to\rho_*\rho^*\O_X\xto{\sim}\rho_*\O_Y$$
Let $\AA\in\CAlg(\QCoh(X))$. Our goal is to define a generalized version of $\Spec$ relative to $\AA$, fittingly called \textbf{global $\Spec$} and denoted $\Spec_X\AA$.\footnote{Some sources write $\un{\Spec}_X\AA$ for extra emphasis.} If the name is anything to go by then we should have $\Spec_X\AA\in\Aff\Sch_X$. Let's first describe $\Spec_X\AA$ as a space. Fixing $B\in\CRing$, given any $x: \Spec B\to X$ we can consider $x^*\AA$. Since $\AA\in\QCoh(X)$ we have $x^*\AA\in\QCoh(\Spec B)$ and so $x^*\AA$ is equivalent to the data of the $B$-algebra $\Gamma(x^*\AA)$. To every section of the structure map $B\to\Gamma(x^*\AA)$ we may uniquely associate a map of spaces $\rho: \Spec B\to\Spec\Gamma(x^*\AA)$. That is, $\rho$ arises from a ring map $\sigma: \Gamma(x^*\AA)\to B$ such that the composition $B\to\Gamma(x^*\AA)\xto{\sigma}B$ is $\id_B$. We may then define $(\Spec_X\AA)(B)$ to consist of all pairs $(x,\rho)$.

\begin{exercise}
Check that this construction is contravariantly functorial in $\AA$.
\end{exercise}

This comes equipped with a map of spaces $\Spec_X\AA\to X$ obtained by forgetting the section $\rho$.

\begin{exercise}
Given $x: \Spec B\to X$, show that there is a canonical isomorphism $\Spec(x^*\AA)\xto{\sim}\Spec B\times_X\Spec_X\AA$ and hence that the map $\Spec_X\AA\to X$ is affine. Using a similar argument, show that $(\Spec_X\AA\to X)\in\Sch_X\iso\Sch_{/X}$ and so $\Spec_X\AA\in\Sch$.
\end{exercise} 

\begin{theorem}
The functor $\Spec_X: \CAlg(\QCoh(X))^{\op}\to\Aff\Sch_X$ is an equivalence of categories.
\end{theorem}

\begin{proof}
We will (sketchily) show that $\Spec_X$ is essentially surjective and leave the rest as an exercise to the reader. So, let $\rho: Y\to X$ be affine. We claim that $\rho_*\O_Y\in\CAlg(\QCoh(X))$, leaving it as an exercise that there is a canonical isomorphism from $\Spec_X\rho_*\O_Y$ to $Y$ in $\Aff\Sch_X$. We can take our algebra unit $1: \O_X\to\rho_*\O_Y$ to be the map discussed above. How do we get our multiplication map? Using lots of adjunction! Consider the composition
$$\rho^*(\rho_*\O_Y\tensor_{\O_X}\rho_*\O_Y)\xto{\sim}\rho^*\rho_*\O_Y\tensor_{\O_Y}\rho^*\rho_*\O_Y\to\O_Y\tensor_{\O_Y}\O_Y\xto{\sim}\O_Y,$$
where we have made use of the counit $\eps_{\rho}$. Applying $\rho_*$ to the above and using the unit $\eta_{\rho}$ gives the desired multiplication map $m: \rho_*\O_Y\tensor_{\O_X}\rho_*\O_Y\to\rho_*\O_Y$.
\end{proof}

\begin{remark}
The algebra behind this result can be greatly generalized by thinking about adjoint functor pairs in terms of so-called \emph{monads} and \emph{comonads}.
\end{remark}

\begin{exercise}
Let $\rho: \Spec A\to X$ be affine. Can you give an explicit description of $\AA:=\rho_*\O_{\Spec A}$ and hence $\Spec_X\AA$? 
\end{exercise}

One rich and important class of affine maps to $X$ is provided by closed embeddings $j: Z\inj X$. What can we say about $\Spec_X^{-1}(j)\in\CAlg(\QCoh(X))$? The key comes from looking more at categorified algebra. A monomorphism in $\QCoh(X)$ is precisely the data of $\phi\in\Hom_{\QCoh(X)}(\F,\G)$ such that $\ker\phi=0$. We may then think of $\F$ as a submodule of $\G$ and write $\F\inj\G$. In particular, taking a submodule of $\O_X$ itself recovers the notion of a \textbf{(quasicoherent) ideal sheaf} on $X$.

\begin{exercise}
Let $\J\inj\O_X$ be an ideal sheaf. Show that $\O_X/\J:=\coker(\J\inj\O_X)$ is automatically a commutative algebra object in $\QCoh(X)$.
\end{exercise}

\begin{exercise}
Let $A\in\CRing$ and $I\normal A$. Show that $\twid{A/I}\iso\twid{A}/\twid{I}$ as quasicoherent sheaves on $\Spec A$ -- i.e., 
\begin{center}
\begin{tikzcd}
0 \arrow[r] & \twid{I} \arrow[r] & \twid{A} \arrow[r] & \twid{A/I} \arrow[r] & 0
\end{tikzcd}
\end{center}
is exact.
\end{exercise}

\begin{exercise}
\hfill
\begin{enum}{\alph}
\item Let $\J\inj\O_X$ be an ideal sheaf. Show that the canonical map $\Spec_X\O_X/\J\to X$ is a closed embedding.

\item Show that ideal sheaves on $X$ correspond bijectively with closed embeddings into $X$ (i.e., closed subschemes of $X$).\footnote{Hint: Given a closed embedding $j: Z\inj X$, consider $\ker(\O_X\to j_*\O_Z)$.}
\end{enum}
\end{exercise}

With this in mind we define $V(\J):=\Spec_X\O_X/\J$ for $\J\inj\O_X$ an ideal sheaf and call this the \textbf{vanishing locus} of $\J$ (we saw above that this is naturally a closed subscheme of $X$). We can also define the \textbf{nonvanishing locus} $D(\J)$ to be $X\setminus V(\J)$.\footnote{Recall that $X\iso\Spec_X\O_X$ canonically in $\Sch_{/X}$.}

\begin{exercise}
Show that the canonical map $D(\J)\to X$ is an open embedding.
\end{exercise}

\begin{exercise}
Let $\phi\in\Hom_{\CAlg(\QCoh(X))}(\AA,\BB)$. Show that $\phi$ is an epimorphism if and only if the induced map $\Spec_X\phi: \Spec_X\BB\to\Spec_X\AA$ is a closed embedding.\footnote{Hint: In the affine case, use the fact that epimorphisms of modules are exactly surjective module homomorphisms.}
\end{exercise}

\subsection{More On Nonvanishing Loci}
Given any $X\in\Sch$, choose a global section $f\in\Gamma(X)=\End_{\QCoh(X)}(\O_X)$. We would like to make sense of $X_f$, the nonvanishing locus of $f$, as a scheme. This ought to come with an open embedding $X_f\inj X$ and there should be a natural map $\Gamma(X)_f\to\Gamma(X_f)$. In the affine case $X=\Spec A$ what we ought to do is clear since we have a natural isomorphism of rings $A\iso\Gamma(\Spec A)$ and so can take $(\Spec A)_f$ to be $D(f)$. What we want to do more generally is glue together these loci.

\begin{exercise}
Let $f\in\Gamma(X)$ and $\U=\{U_i\}_{i\in I}\in\Cov(X)$ an affine open covering. Define $f_i:=f|_{U_i}$. 
\begin{enum}{\alph}
\item Show that the open subschemes $D(f_i)\inj U_i\inj X$ glue together to define an open subscheme $X_f\inj X$ with the desired properties.\footnote{Hint: One way to do this is to note that the colimit of the compositions $D(f_i)\inj X$ exists in $\Space$ and that a space covered by schemes is a scheme.}

\item Show that the above construction of $X_f$ is independent of the choice of $\U$.\footnote{More precisely, choosing a different affine open covering $\U'$ should yield a nonvanishing locus $X_f'$ and an explicit natural isomorphism between $X_f$ and $X_f'$.}

\item Suppose that $X$ is qcqs. Show that the natural map $\Gamma(X)_f\to\Gamma(X_f)$ is an isomorphism of rings.
\end{enum}
\end{exercise}

Note that it is possible to give a more intrinsic description of $X_f$ by ``categorifying'' the localization process.

\begin{exercise}
Let $A\in\CRing$ and $f\in A$. Show that there is a canonical $A$-module isomorphism
$$A_f\iso\colim(A\xto{f}A\xto{f}\cdots)$$
which is in fact an isomorphism of (commutative) $A$-algebras. Beware that the maps $f: A\to A$ are not $A$-algebra maps if $f\neq1$.
\end{exercise}

The key in the above is that, given any $B\in\CRing$ and $g\in B$, $g\in B^{\times}$ if and only if the multiplication map $g: B\to B$ is an isomorphism of rings.

Let now $\AA\in\CAlg(\QCoh(X))$ and $f\in\End_{\AA}(\AA)$.\footnote{The notation indicates that $f$ is an $\AA$-linear endomorphism, with $\AA$ viewed as an $\AA$-module over itself.} We define the \emph{localization} $\AA[f^{-1}]$ to be an object in $\CAlg(\QCoh(X))$ equipped with a map $\AA\to\AA[f^{-1}]$ such that we may uniquely complete the diagram
\begin{center}
\begin{tikzcd}
\AA \arrow[r, "\phi"] \arrow[d] & \BB \\
\AA{[}f^{-1}{]} \arrow[ru, dotted, "\exists!"'] &
\end{tikzcd}
\end{center}
in $\CAlg(\QCoh(X))$ if and only if $\phi(f)\in\Aut_{\AA}(\BB)$.

\begin{exercise}
\hfill
\begin{enum}{\alph}
\item Show that $\AA[f^{-1}]$ exists and satisfies 
$$\AA[f^{-1}]\iso\colim(\AA\xto{f}\AA\xto{f}\cdots).$$

\item Show that the canonical map $\Spec_X\AA[f^{-1}]\to X$ is an open embedding.

\item Letting $\AA=\O_X$, show that $X_f\iso\Spec_X\O_X[f^{-1}]$.\footnote{This implicitly uses the fact that every element of $\End_{\QCoh(X)}(\O_X)$ is automatically a map of commutative algebra objects.}
\end{enum}
\end{exercise}

There is another way to make sense of localization that is highly specific to the setting of quasicoherent sheaves. So that we don't have to change up earlier notational conventions we will consider $s\in\End_{\AA}(\AA)$. Given $f: \Spec A\to X$, we have $s_f\in\End_{\AA_f}(\AA_f)$ and so we identify $s_f$ with its value at the (multiplicative) identity. This suggests that we define $(\AA[s^{-1}])_f:=\AA_f[s_f^{-1}]$.

\begin{exercise}
Given $g:\Spec B\to\Spec A$, let $\alpha_{f,g}\in\Isom_{\CAlg_B}(\AA_{f\circ g},B\tensor_A\AA_f)$ be the cocycle associated to $\AA$. Show that $\alpha_{f,g}$ canonically induces a cocycle $\alpha[s^{-1}]_{f,g}$ associated to $\AA[s^{-1}]$ satisfying the cocycle condition.
\end{exercise}

\begin{exercise}
Show that $\AA[s^{-1}]$ satisfies the expected universal property. Show moreover that localization at $s$ (for $s$ a fixed global section of $\O_X$) defines an endofunctor on $\CAlg(\QCoh(X))$. In this latter case, show that there is a canonical isomorphism $\O_X[s^{-1}]\tensor_{\O_X}\AA\xto{\sim}\AA[s^{-1}]$ of commutative algebras for every $\AA\in\CAlg(\QCoh(X))$
\end{exercise}

\begin{exercise}
The section $s\in\End_{\QCoh(X)}(\AA)$ corresponds to a commutative diagram
\begin{center}
\begin{tikzcd}
\Spec_X\AA \arrow[rr, "\twid{s}"] \arrow[rd] & & \Spec_X\AA \arrow[ld] \\
& X &
\end{tikzcd}
\end{center}
of schemes. The localization process in turn gives us $\Spec_X\AA[s^{-1}]\to X$. Is there a convenient way to recover this object of $\Aff\Sch_X$ directly from $\twid{s}$?
\end{exercise}

\begin{exercise}
Let $\BB\in\CAlg(\QCoh(X))$. 
\begin{enum}{\alph}
\item Make sense of $\BB^{\times}$ as a quasicoherent sheaf on $X$.

\item Show that there is a canonical monomorphism $\BB^{\times}\inj\BB$ and that $\BB^{\times}$ is a group object.

\item Under what conditions does a map of commutative algebras $\phi: \AA\to\BB$ factor through $\BB^{\times}$?
\end{enum}
\end{exercise}

Let's now return to the matter of ideal sheaves. 

\begin{exercise}
Let $\I,\J\inj\O_X$ be ideal sheaves. 
\begin{enum}{\alph}
\item Make sense of $\I+\J$ and $\I\J$ as ideal sheaves.

\item Show that there is a canonical isomorphism $\O_X/(\I+\J)\iso\O_X/\I\tensor_{\O_X}\O_X/\J$ of commutative algebra objects.

\item Show that $V(\I+\J)\iso\lim(V(\I)\rightarrow X\leftarrow V(\J))$ as spaces. We remember this by the mnemonic $V(\I)\cap V(\J)\iso V(\I+\J)$.

\item We say that $\I$ and $\J$ are \textbf{comaximal} if $\I+\J=\O_X$. Assuming this, show that the canonical map $\O_X/\I\J\to\O_X/\I\oplus\O_X/\J$ is an isomorphism of commutative algebra objects\footnote{Note that $\oplus$ is a biproduct on $\QCoh(X)$ and so we may think of it both as a product and a coproduct.} and so the canonical map $V(\I)\coprod V(\J)\to V(\I\J)$ is an isomorphism of spaces.\footnote{Hint: Chinese Remainder Theorem.} We remember this by the mnemonic $V(\I)\cup V(\J)\iso V(\I\J)$. 

\item Show that the canonical map $\O_X/\I\J\to\O_X/\I\oplus\O_X/\J$ is an isomorphism if and only if $\I\J\iso\ker(\O_X\to\O_X/\I\oplus\O_X/\J)$.

\item Suppose that the canonical map $\O_X/\I\J\to\O_X/\I\oplus\O_X/\J$ is an isomorphism. Is it necessarily true that $\I$ and $\J$ are comaximal?

\item Give an example of $\I$ and $\J$ such that $\O_X/\I\J\to\O_X/\I\oplus\O_X/\J$ is not an isomorphism.
\end{enum}
\end{exercise}

\begin{exercise}
Let $j_1: Z_1\inj X$ and $j_2: Z_2\inj X$ be closed embeddings with associated ideal sheaves $\J_1$ and $\J_2$. When are $\J_1$ and $\J_2$ comaximal? How is $V(\J_1\J_2)$ related to $Z_1$ and $Z_2$?
\end{exercise}

\section{More on Zariski Sheaves}
Let's revisit coverings once again. A good reference for this stuff is [Stacks, Tag 00UZ]. Fix a category $\CC$. By definition, a \textbf{family of morphisms in $\CC$ with fixed target} is a collection $\{\phi_i: U_i\to U\}_{i\in I}$ of morphisms in $\CC$. We often want to distinguish a particular class\footnote{Technically, we actually require $\Cov(\CC)$ to be a set and not a proper class. Achieving this in general requires thinking about Grothendieck universes and other such things. We will sweep this under the rug.} of such families, denoted $\Cov(\CC)$, with elements called \textbf{coverings}. There are three conditions that $\Cov(\CC)$ must satisfy.
\begin{itemize}
\item[(Isomorphism)] $\Cov(\CC)$ contains all isomorphisms in $\CC$.

\item[(Locality)] Suppose $\{\phi_i: U_i\to U\}_{i\in I}\in\Cov(\CC)$ and $\{\psi_{ij}: U_{ij}\to U_i\}_{j\in I_i}\in\Cov(\CC)$ for every $i\in I$. Then, 
$$\{\phi_i\circ\psi_{ij}: U_{ij}\to U\}_{(i,j)\in\prod_{i\in I}\{i\}\times I_i}\in\Cov(\CC).$$

\item[(Base Change)] Let $\{U_i\to U\}_{i\in I}\in\Cov(\CC)$ and $V\to U$ a morphism in $\CC$. Then, $U_i\times_UV$ exists for every $i\in I$ and $\{U_i\times_UV\to V\}_{i\in I}\in\Cov(\CC)$.
\end{itemize}

Note that the first part of the base change condition often holds trivially since we often take $\CC$ to be a category admitting fiber products.\footnote{In situations where $\CC$ is \textbf{not} closed under fiber products it is often possible to (manageably) enlarge $\CC$ so that it is.} We call $\Cov(\CC)$ a \textbf{pretopology} or \textbf{Grothendieck topology} on $\CC$. The pair $(\CC,\Cov(\CC))$ is called a \textbf{site}. The advantage of sites is that they allow us to generalize the notion of open set and therefore the notion of a sheaf. 

\begin{remark}
It is helpful to say a bit about morphisms of families of morphisms with fixed target. So, let $\U:=\{\phi_i: U_i\to U\}_{i\in I}$ and $\V:=\{\psi_j: V_j\to V\}_{j\in J}$ be families of morphisms in $\CC$ with fixed target. Then, the data of a morphism from $\U$ to $\V$ is the data of a morphism $U\to V$ in $\CC$, a function $\alpha: I\to J$, and a morphism $U_i\to V_{\alpha(i)}$ for every $i\in I$ such that 
\begin{center}
\begin{tikzcd}
U_i \arrow[r] \arrow[d, "\phi_i"'] & V_{\alpha(i)} \arrow[d, "\psi_{\alpha(i)}"] \\
U \arrow[r] & V
\end{tikzcd}
\end{center}
commutes. If $I=J$, $\alpha=\id$, and $U=V$ then we get the notion of a \textbf{refinement}. In other words, given $\U:=\{\phi_i: U_i\to S\}_{i\in I}$ and $\V:=\{\psi_i: V_i\to S\}_{i\in I}$ in $\Cov(\CC)$, we have a bunch of commuting triangles
\begin{center}
\begin{tikzcd}
V_i \arrow[rr] \arrow[rd, "\psi_i"'] & & U_i \arrow[ld, "\phi_i"] \\
& S &
\end{tikzcd}
\end{center}
Note that this is (at least a priori) a weaker notion of refinement than the one we defined earlier.
\end{remark}

We have so far encountered several examples of sites. For convenience we let $S\in\Sch$, so that $\Sch_S\simeq\Sch_{/S}$ among other things.\footnote{We are again sweeping a lot of technical details under the rug. In particular, our descriptions of underlying categories are technically incorrect. Many choices are involved in constructing the actual underlying categories, and many of these choices also need to be compatible with other choices in a precise sense. Additionally, one needs to know that most of these choices do not affect the end result in any significant way.}

\begin{center}
\begin{tabular}{|l|l|l|}
\hline
Name & Notation & Underlying Category \\
\hline
Big Zariski site of spaces\footnote{This one might actually be ``too big'' to be amenable to any selection procedure.} & $\Space_{\Zar}$ & $\Space$ \\
Big Zariski site of schemes & $\Sch_{\Zar}$ & $\Sch$ \\
Big Zariski site of $S$ & $(\Sch_{/S})_{\Zar}$ & $\Sch_{/S}$ \\
Small Zariski site of $S$ & $S_{\Zar}$ & $\Op(S)$ \\
Big affine Zariski site of $S$ & $(\Aff\Sch_{/S})_{\Zar}$ & $\Aff\Sch_{/S}$ \\
Small affine Zariski site of $S$ & $S_{\Zar}'$ & $\Aff\Op(S)$ \\
\hline
\end{tabular}
\end{center}

For each of these underlying categories, the notion of (Zariski) open covering can be adapted to give us a site. We have compatible functors
\begin{center}
\begin{tikzcd}
& \Op(S) \arrow[rd, hookrightarrow] & & & \\
\Aff\Op(S) \arrow[ru, hookrightarrow] \arrow[rd, hookrightarrow] & & \Sch_{/S} \arrow[r, "\oblv"] & \Sch \arrow[r, hookrightarrow] & \Space \\
& \Aff\Sch_{/S} \arrow[ru, hookrightarrow] & & &
\end{tikzcd}
\end{center}
Given $X/S\in\Sch_{/S}$, the data of a (Zariski) open covering of $X/S$ is the data of $U/S\in\Sch_{/S}$ such that the collection of $U$ forms a (Zariski) open covering of $X$. Note that $S/S$ is the terminal object in $\Sch_{/S}$ and thus in $\Op(S)$ as well (in general $S$ may not be affine).

\begin{exercise}
Show that each of the above underlying categories, when equipped with the notion of (Zariski) open covering, forms a site.
\end{exercise}

Let $\EE$ be any category and $(\CC,\Cov(\CC))$ a site. Recall that there is a presheaf category $\Pre(\CC,\EE):=\Fun(\CC^{\op},\EE)$ of $\EE$-valued presheaves on $\CC$ which comes with a full subcategory $\Pre_{\rep}(\CC,\EE)$ of representable presheaves. Recall also that if $\EE=\Set$ then we simply write $\Pre(\CC)$.

\begin{remark}
Given $\U=\{U_i\to U\}_{i\in I}\in\Cov(\CC)$, we should be able to build $U$ by ``gluing together'' the elements of $\U$ in the sense that the natural map
$$\coeq\paren{\coprod_{i,j\in I}U_i\times_UU_j\rightrightarrows\coprod_{i\in I}U_i}\to U$$
is an isomorphism in $\CC$. The trouble with this approach is that $\CC$ may not contain all (even finite) coproducts.\footnote{For example, categories of crystals typically lack initial objects.}
\end{remark}

Assuming $\EE$ is complete we may make sense of the sheaf condition.\footnote{If $\EE$ is not complete then there are still things we can do but we will not worry about this.}

\begin{definition}
Let $\F\in\Pre(\CC,\EE)$ and $\U=\{U_i\to U\}_{i\in I}\in\Cov(\CC)$. We say that $\F$ \textbf{satisfies the sheaf condition} with respect to $\U$ if the natural map 
$$\F(U)\to\eq\paren{\prod_{i\in I}\F(U_i)\rightrightarrows\prod_{i,j\in I}\F(U_i\times_UU_j)}$$
is an isomorphism in $\EE$. We say that $\F$ is a \textbf{sheaf} (on the site $(\CC,\Cov(\CC))$) if $\F$ satisfies the sheaf condition with respect to every $\U\in\Cov(\CC)$. These form a full subcategory 
$$\Shv((\CC,\Cov(\CC)),\EE)=\Shv(\CC,\EE)$$ 
of $\Pre(\CC,\EE)$. We call such a category of sheaves a \textbf{Grothendieck topos}. If $\EE=\Set$ then we omit it from the notation as for presheaves.
\end{definition}

\begin{remark}
Checking the sheaf condition for every element of $\Cov(\CC)$ is often not practical and so we typically want conditions that specify which elements of $\Cov(\CC)$ we need to check.
\end{remark}

It follows by general nonsense that if $\EE$ is (co-)complete then $\Pre(\CC,\EE)$ will be as well. We typically take $\EE$ to be either $\Set$ or an abelian category (often $\Ab$ or $\Mod_A$ since both are concrete and capture interesting algebra). In the latter case the sheaf condition can be phrased in terms of exact sequences since in an abelian category the equalizer of $f,g: X\rightrightarrows Y$ is canonically $\ker(f-g: X\to Y)$. We also have at our disposal the Yoneda embedding $\CC\inj\Pre(\CC)$, which can also give us information about $\Pre(\CC,\EE)$ at least when $\EE$ is concrete.\footnote{This is often enough to reason about any abelian category $\EE$ by results such as the Freyd-Mitchell embedding theorem.}

We know at this point that sheaves are themselves presheaves. Is it possible to go the other way and ``sheafify'' a presheaf to get a sheaf? Yes! Given $\F\in\Pre(\CC,\EE)$, its \textbf{sheafification} is the data of $\F^{\sh}\in\Shv(\CC,\EE)$ and a map of presheaves $\F\to\F^{\sh}$ such that there is a unique factorization
\begin{center}
\begin{tikzcd}
\F \arrow[r] \arrow[rd] & \F^{\sh} \arrow[d, dotted, "\exists!"] \\
& \G
\end{tikzcd}
\end{center}
for any map of presheaves $\F\to\G$ with $\G\in\Shv(\CC,\EE)$. This is precisely asking for a left adjoint $(\cdot)^{\sh}$ of the inclusion $\Shv(\CC,\EE)\inj\Pre(\CC,\EE)$. Assuming this sheafification exists, it must necessarily preserve colimits and act as the identity on $\Shv(\CC,\EE)$. There are generally two approaches to showing that sheafification exists.
\begin{enum}{\arabic}
\item Appeal to abstract existence results for left adjoints (such as the adjoint functor theorem). This has the advantage of elegance and sometimes working better for proofs involving sheafification.

\item Explicitly construct the sheafification functor and show that it is left adjoint to the inclusion $\Shv(\CC,\EE)\inj\Pre(\CC,\EE)$. This has the advantage of being useful for computations.
\end{enum}
We will take for granted the existence of sheafification in all settings of interest to us, and that we have a commutative diagram
\begin{center}
\begin{tikzcd}
\Pre(\CC,\EE) \arrow[r, "\oblv"] \arrow[d, shift left, "(\cdot)^{\sh}"] & \Pre(\CC) \arrow[d, shift left, "(\cdot)^{\sh}"] \\
\Shv(\CC,\EE) \arrow[u, shift left, hookrightarrow] \arrow[r, "\oblv"'] & \Shv(\CC) \arrow[u, shift left, hookrightarrow]
\end{tikzcd}
\end{center}
when $\EE$ is concrete. we will also take for granted the following result.

\begin{proposition}
Sheafification commutes with finite limits.
\end{proposition}

\begin{exercise}
Fix $\phi\in\Hom_{\Pre(\CC,\Ab)}(\F,\G)$.
\begin{enum}{\alph}
\item Show that $(\ker\phi)(U):=\ker(\phi(U): \F(U)\to\G(U))$ defines the categorical kernel of $\phi$ in $\Pre(\CC,\Ab)$.

\item Show that $(\coker\phi)(U):=\coker(\phi(U): \F(U)\to\G(U))$ defines the categorical cokernel of $\phi$ in $\Pre(\CC,\Ab)$.

\item Show that $\Pre(\CC,\Ab)$ is an abelian category.
\end{enum}
\end{exercise}

\begin{exercise}
Fix $\phi\in\Hom_{\Shv(\CC,\Ab)}(\F,\G)$.
\begin{enum}{\alph}
\item Show that the presheaf kernel $\ker\phi$ is the categorical kernel of $\phi$ in $\Shv(\CC,\Ab)$.

\item Show that the sheafification $(\coker\phi)^{\sh}$ is the categorical cokernel of $\phi$ in $\Shv(\CC,\Ab)$.

\item Show that $\Shv(\CC,\Ab)$ is an abelian category.
\end{enum}
\end{exercise}

\begin{exercise}
Show that the results of the previous two exercises hold true if $\Ab$ is replaced by a general abelian category.
\end{exercise}

There's a whole lot more we could say but our goal in these notes is not to write a treatise on topos theory. Switching gears, we would like to put all of this formalism to use to reason about Zariski sheaves. Recall from way back that we already defined a notion of Zariski sheaves for spaces. 

\begin{exercise}
Show that the Yoneda embedding $\Space\inj\Pre(\Space)$ restricts to an equivalence of categories $\Shv_{\Zar}\xto{\sim}\Shv(\Space_{\Zar})$.
\end{exercise}

We can in fact use this to pullback the topos structure on $\Shv(\Space_{\Zar})$ to get a topos structure on $\Shv_{\Zar}$. Part of what we can transfer over are the adjoint notions of pushforward and pullback on $\Pre(\Space)$ that give rise to analogous notions on $\Shv(\Space_{\Zar})$. We can also make sense of sheafification on $\Space$ itself.

\begin{exercise}
Let $X\in\Space$ be the non-Zariski sheaf $A\mapsto\{f\in A : f\in A^{\times}\textrm{ or }1-f\in A^{\times}\}$ of a previous exercise. Show that $X^{\sh}\iso\A^1$.
\end{exercise}

What about the other Zariski sites that we have at our disposal? Given $S\in\Sch$, we will in practice mostly be concerned with the big and small affine Zariski site of $S$ and the small Zariski site of $X$. It is \textbf{not} true that the associated topoi of (even abelian) sheaves are equivalent as categories. Nonetheless, these topoi are strongly related to one another. In particular, the data of $\Shv(S'_{\Zar})$ in some sense already encodes all of the data of both $\Shv(S_{\Zar})$ and $\Shv((\Aff\Sch_{/S})_{\Zar})$ by work we have done previously. The key is that in order to check that a space is a Zariski sheaf we need only consider its restrictions to any given affine open covering (since we can uniquely glue together sheaves satisfying a cocycle condition, similar to how we did in Serre's theorem). 

\textbf{\un{Warning}:} These statements becomes false if we replace ``restriction'' by ``section.''

From the perspective of the entire topos there is a lot going on that is difficult to keep track of. Fortunately for us, we will most of the time only need to consider ``a few'' sheaves or ``a few'' maps to $S$ at once. Here's an attempt at a useful slogan.

\textbf{\un{Slogan}:} When working with sheaves, we don't want to consider all open coverings at once. Instead, we want to consider any given open covering, whose choice does not matter.

\section{Abstract Algebra with Respect to $\O_X$}
\subsection{Basics}
Recall that the association $\F\mapsto\un{\F}$ allows us to view $\QCoh(X)$ as a non-full subcategory of $\Fun(\Aff\Sch_{/X}^{\op},\Ab)=\Pre(\Aff\Sch_{/X},\Ab)$ (from here on we will not notationally distinguish between $\F$ and $\un{\F}$). The category $\Pre(\Aff\Sch_{/X},\Ab)$ itself is rather large and unwieldy so we would like a (necessarily non-full) subcategory of $\Pre(\Aff\Sch_{/X},\Ab)$ which is nice to work with and into which $\QCoh(X)$ embeds. We already know that $\O_X$ is a (commutative) ring object in $\QCoh(X)$ and so it is also a ring object in $\Pre(\Aff\Sch_{/X},\Ab)$. 

\begin{exercise}
Show that every object in $\QCoh(X)$ is naturally an $\O_X$-module.
\end{exercise}

We therefore see that the functor $\QCoh(X)\to\Pre(\Aff\Sch_{/X},\Ab)$ factors through $\Mod_{\O_X}(\Pre(\Aff\Sch_{/X},\Ab))$, which we will temporarily denote $\Mod'_{\O_X}$ for convenience. 

\begin{exercise}
Check that the induced functor $\QCoh(X)\to\Mod'_{\O_X}$ is full hence an embedding.\footnote{The attentive reader should investigate this embedding further. Is it a left adjoint? A right adjoint?}
\end{exercise}

The basic idea is that the extra compatibility constraints imposed by $\O_X$-linearity induce a cocycle condition. Everything we've said so far is basically algebraic in nature. Geometrically, we know that quasicoherent sheaves are ``sheafy'' in the sense that $\QCoh(X)\to\Pre(\Aff\Sch_{/X},\Ab)$ actually factors through $\Shv((\Aff\Sch_{/X})_{\Zar},\Ab)$. $\O_X$ is still a ring object in this category and so we may consider the category $\Mod_{\O_X}:=\Mod_{\O_X}(\Shv((\Aff\Sch_{/X})_{\Zar},\Ab))$ of (sheaves of) \textbf{$\O_X$-modules}.

\begin{exercise}
Show that $\Mod_{\O_X}$ is the ``intersection'' of $\Mod'_{\O_X}$ and $\Shv((\Aff\Sch_{/X})_{\Zar},\Ab)$ and hence that there is an embedding $\QCoh(X)\inj\Mod_{\O_X}$.\footnote{Use the fact that sheafification commutes with finite limits to show that it takes $\Mod'_{\O_X}$ to the full subcategory $\Mod_{\O_X}$.}
\end{exercise}

As remarked upon in the previous section, from the perspective of data encoded by sheaves there is no real harm done\footnote{Though there is some complication.} in replacing $\Pre(\Aff\Sch_X,\Ab)$ and $\Shv((\Aff\Sch_{/X})_{\Zar},\Ab)$ with $\Pre(\Aff\Op(X),\Ab)$ and $\Shv(S'_{\Zar},\Ab)$, or even the slightly larger $\Pre(\Op(X),\Ab)$ and $\Shv(S_{\Zar},\Ab)$. Say we choose one of these pairs to be our model for $\Pre_X(\Ab)$ and $\Shv_X(\Ab)$. Then, the following facts hold true as above.
\begin{itemize}
\item $\O_X$ is a ring object in both $\Pre_X(\Ab)$ and $\Shv_X(\Ab)$.

\item $\Mod_{\O_X}:=\Mod_{\O_X}(\Shv_X(\Ab))$ is a full subcategory of $\Mod'_{\O_X}:=\Mod_{\O_X}(\Pre_X(\Ab))$ with sheafification taking things in the other direction. In other words, we have a commutative diagram
\begin{center}
\begin{tikzcd}
\Pre_X(\Ab) \arrow[r, "(\cdot)^{\sh}"] & \Shv_X(\Ab) \\
\Mod'_{\O_X} \arrow[u, hookrightarrow] \arrow[r, "(\cdot)^{\sh}"'] & \Mod_{\O_X} \arrow[u, hookrightarrow]
\end{tikzcd}
\end{center}
\end{itemize}
Which choice then should we make? Recall that one deficiency of the categories $\Pre(\Aff\Sch_X,\Ab)$ for varying $X\in\Sch$ is that the natural notion of pushforward may not be well-defined. This is also an issue for $\Pre(\Aff\Op(X),\Ab)$ but not so for $\Pre(\Op(X),\Ab)$. To see this, let $\pi: X\to S$ be a map of schemes. Then, 
$$\pi_*: \Pre(\Op(X),\Ab)\to\Pre(\Op(S),\Ab),\qquad \F\mapsto(U/S\mapsto\F(\pi^{-1}(U)/X))$$
is a well-defined functor since open embeddings are stable under base change.

\begin{exercise}
Show that there is a canonical map $\O_S\to\pi_*\O_X$ in $\Pre_S(\Ab)$ and hence in $\Pre_X(\Ab)$.
\end{exercise}

\begin{exercise}
Show that $\Mod'_{\O_X}$ inherits (co-)kernels from $\Pre_X(\Ab)$ and thus is abelian. Similarly, show that $\Mod_{\O_X}$ inherits (co-)kernels from $\Shv_X(\Ab)$ and thus is abelian.
\end{exercise}

\begin{remark}
It follows that (co-)kernels on $\Mod'_{\O_X}$ and $\Mod_{\O_X}$ are related by sheafification.
\end{remark}

How do we distinguish quasicoherent sheaves among all $\O_X$-modules? Recall that, given $\F\in\QCoh(X)$ and $U\in\Op(X)$, $\F(U)$ is given by $\Hom_{\QCoh(U)}(\O_U,\F|_U)\iso\Hom_{\Mod_{\O_U}}(\O_U,\F|_U)$. Moreover, $\F|_U\iso\twid{\F(U)}$ for every $U\in\Aff\Op(X)$. This makes sense of the statement that (the essential image of) $\QCoh(X)$ consists of exactly those $\O_X$-modules that are completely determined by their sections over affine open subschemes of $X$. 

\begin{exercise}
Show that the embedding $\QCoh(X)\inj\Mod_{\O_X}$ is exact and so identifies $\QCoh(X)$ with a full abelian subcategory of $\Mod_{\O_X}$.
\end{exercise}

\begin{exercise}
Let $\F\in\Mod_{\O_X}$ and $U\in\Op(X)$. Is there a natural $\Gamma(U,\O_U)$-structure on $\Hom_{\Mod_{\O_U}}(\O_U,\F|_U)$? Is it true that $\F(U)\iso\Hom_{\Mod_{\O_U}}(\O_U,\F|_U)$?
\end{exercise}

A different perspective is that quasicoherent sheaves on $X$ are exactly the $\O_X$-modules defined by generators and relations in terms of $\O_X$ itself. The following result makes this precise.

\begin{exercise}
Let $\F\in\Mod_{\O_X}$. Show that $\F\in\QCoh(X)$ if and only if there exists an exact sequence
\begin{center}
\begin{tikzcd}
\O_X^{\oplus J}|_U \arrow[r] & \O_X^{\oplus I}|_U \arrow[r] & \F|_U \arrow[r] & 0
\end{tikzcd}
\end{center}
for every $U\in\Op(X)$.\footnote{Note that the index sets $I$ and $J$ depend on $U$ and can be infinite.} Show moreover that it suffices to check this condition on any affine open covering of $X$.
\end{exercise}

This helps capture the difference between $\QCoh(X)$ and its essential image with which it has been identified. In $\Mod_{\O_X}$ we merely posit the existence of generators and relations on every affine open section. However, in $\QCoh(X)$ as originally defined, we must keep track of all of the choices of generators and relations. If we were to take the exact sequence criterion as the definition of quasicoherence then the latter part of the above exercise says that $\F\in\Mod_{\O_X}$ is quasicoherent if and only if it is \emph{locally} quasicoherent in the sense that we only know there is such an exact sequence for affine open subschemes.

\begin{exercise}
Fix $\F,\G\in\Mod'_{\O_X}$.
\begin{enum}{\alph}
\item Show that taking $\HHom(\F,\G)(U):=\Hom_{\Mod'_{\O_U}}(\F|_U,\G|_U)$ defines a bifunctor 
$$\HHom: (\Mod'_{\O_X})^{\op}\times\Mod'_{\O_X}\to\Mod'_{\O_X}.$$

\item Show that taking $(\F\tensor_{\O_X}\G)(U):=\F(U)\tensor_{\O_X(U)}\G(U)$ defines a bifunctor 
$$\tensor_{\O_X}: \Mod'_{\O_X}\times\Mod'_{\O_X}\to\Mod'_{\O_X}.$$

\item Show that $\HHom$ and $\tensor_{\O_X}$ satisfy a generalized tensor-$\Hom$ adjunction.

\item Show that $\tensor_{\O_X}$ equips $\Mod'_{\O_X}$ with the structure of a strict symmetric monoidal category.
\end{enum}
\end{exercise}

\begin{exercise}
Fix $\F,\G\in\Mod_{\O_X}$ (so both are ``sheafy''). We temporarily adopt the notation of the previous exercise.
\begin{enum}{\alph}
\item Show that the internal $\Hom$ of $\F$ and $\G$ in $\Mod_{\O_X}$ is given by $\HHom(\F,\G)$. In particular, $\HHom(\F,\G)^{\sh}\iso\HHom(\F,\G)$ canonically.

\item Show that the tensor product of $\F$ and $\G$ in $\Mod_{\O_X}$ is given by $(\F\tensor_{\O_X}\G)^{\sh}$. We abuse notation and let $\tensor_{\O_X}$ denote the tensor product on $\Mod_{\O_X}$.

\item Given an example of $\F,\G\in\Mod_{\O_X}$ such that $\F\tensor_{\O_X}\G$ and $(\F\tensor_{\O_X}\G)^{\sh}$ are not the same.

\item Show that $\HHom$ and $\tensor_{\O_X}$ satisfy a generalized tensor-$\Hom$ adjunction.

\item Show that $\tensor_{\O_X}$ equips $\Mod_{\O_X}$ with the structure of a strict symmetric monoidal category.
\end{enum}
\end{exercise}

\begin{remark}
Part of this exercise shows that the inclusion $\Mod_{\O_X}\inj\Mod'_{\O_X}$ is not symmetric monoidal.
\end{remark}

\begin{remark}
For general (Grothendieck) topoi, it is best to characterize internal $\Hom$ not in terms of an underlying symmetric monoidal structure (which may not exist) but instead in terms of products. In particular, this lets us define an internal $\Hom$ on $\Shv_X(\Ab)$ (which exists!). Note however that this internal $\Hom$ is generally not compatible with $\HHom$ on $\Mod_{\O_X}$.\footnote{This has to do with the fact that $\Mod_{\O_X}$ is generally not a full subcategory of $\Shv_X(\Ab)$.} 
\end{remark}

\begin{exercise}
Make sense of $\O_X$-bilinear maps and show that $\tensor_{\O_X}$ satisfies a generalized version of the universal property for tensor products of modules.
\end{exercise}

It's clear already that the internal $\Hom$ on $\QCoh(X)$ and $\Mod_{\O_X}$ are compatible. What about $\tensor_{\O_X}$?

\begin{exercise}
Show that the inclusion functor $\QCoh(X)\inj\Mod_{\O_X}$ is strict symmetric monoidal.\footnote{Hint: Use the fact that quasicoherent sheaves are ``sheafy.''}
\end{exercise}

Let's now revisit the pushforward comment from earlier. Let $\pi: X\to S$ be a map of schemes and $\pi_*: \Pre_X(\Ab)\to\Pre_S(\Ab)$ as before. 

\begin{proposition}
The functor $\pi_*$ sends $\Mod'_{\O_X}$ to $\Mod'_{\O_S}$.
\end{proposition}

\begin{proof}
Let $\F\in\Mod'_{\O_X}$ and $\sigma_{\F}: \O_X\times\F\to\F$ the scalar action of $\O_X$. Given $U\in\Op(X)$, we have canonical isomorphisms
\begin{align*}
(\pi_*(\O_X\times\F))(U)
&\iso(\O_X\times\F)(\pi^{-1}(U)) \\
&\iso\O_X(\pi^{-1}(U))\times\F(\pi^{-1}(U)) \\
&\iso(\pi_*\O_X)(U)\times(\pi_*\F)(U) \\
&\iso(\pi_*\O_X\times\pi_*\F)(U)
\end{align*}
and so $\pi_*(\O_X\times\F)\iso\pi_*\O_X\times\pi_*\F$.\footnote{The same argument shows that $\pi_*$ commutes with all limits.} One then checks that the composition
\begin{center}
\begin{tikzcd}
\O_S\times\pi_*\F \arrow[r] & \pi_*\O_X\times\pi_*\F \arrow[r, "\sim"] & \pi_*(\O_X\times\F) \arrow[r, "\pi_*\sigma_{\F}"] & \pi_*\F
\end{tikzcd}
\end{center}
satisfies the conditions to be a scalar action, where the first map in the composition is built from the canonical map $\O_S\to\pi_*\O_X$.
\end{proof}

Assuming $\pi_*$ sends $\QCoh(X)$ to $\QCoh(S)$, the functor $\pi_*: \QCoh(X)\to\QCoh(S)$ is evidently right adjoint to $\pi^*: \QCoh(S)\to\QCoh(X)$ as previously discussed. 

\begin{exercise}
Show that $\pi_*: \Mod'_{\O_X}\to\Mod'_{\O_S}$ admits a left adjoint $\pi^*: \Mod'_{\O_S}\to\Mod'_{\O_X}$. Does this restrict to $\pi^*: \QCoh(S)\to\QCoh(X)$?
\end{exercise}

Mimicking the argument above we obtain a pushforward functor $\pi_*: \Mod_{\O_X}\to\Mod_{\O_S}$ with left adjoint $\pi^*: \Mod_{\O_S}\to\Mod_{\O_X}$ obtained by sheafifying $\pi^*: \Mod'_{\O_S}\to\Mod'_{\O_X}$. It is then easy to see that $\pi_*: \Mod_{\O_X}\to\Mod_{\O_S}$ restricts to $\pi_*: \Mod'_{\O_X}\to\Mod'_{\O_S}$, either directly or using the fact that right adjoints commute with right adjoints.\footnote{This implicitly uses that the relevant forgetful functor is right adjoint to sheafification.}

\begin{exercise}
Note that there are also adjoint pairs $\pi^*\dashv\pi_*$ for general presheaves and sheaves. Do these relate to the above constructions?
\end{exercise}

\subsection{Categorification}
Let's begin by fixing some terminology once and for all. We refer to objects of $\CAlg(\QCoh(X))$ as \textbf{quasicoherent (commutative) $\O_X$-algebras}, which we know are basically the same thing as $X$-affine schemes via the equivalence $\Spec_X: \CAlg(\QCoh(X))\xto{\sim}\Aff\Sch_X$. Identifying $\CAlg(\QCoh(X))$ with its essential image in $\Mod_{\O_X}$, we see that $\CAlg(\QCoh(X))$ is a full subcategory of the category $\CAlg_{\O_X}:=\CAlg(\Mod_{\O_X})$ of \textbf{(commutative) $\O_X$-algebras}.

\begin{exercise}
Show that $\CAlg(\Mod_{\O_X})\simeq\CAlg_{\O_X}(\Shv_X(\Ab))$, thus eliminating any potential ambiguity coming from our naming convention.
\end{exercise}

Note that this category has a presheaf version, where many algebro-geometric constructions are easy to do and can be transferred to $\CAlg_{\O_X}$ using sheafification. One major advantage of these categories is that they allow us to generalize a great deal of the algebra we are familiar with. We've already seen this a little bit and can take things much further.

\begin{exercise}
Let $\AA\in\CAlg_{\O_X}$ and $\I\in\Mod_{\O_X}$ an ideal of $\AA$ (so $\I$ is simply an $\O_X$-submodule of $\AA$).
\begin{enum}{\alph}
\item Show that $\AA/\I$ is automatically an $\O_X$-algebra and the projection $\AA\to\AA/\I$ is a map of $\O_X$-algebras.

\item Let $\J\normal\AA$ be another ideal of $\AA$. Can you make sense of $\I+\J$ and $\I\J$ as ideals of $\AA$? Do they behave as expected?

\item Do the various isomorphism theorems for modules generalize to this setting? What about for rings?

\item Make sense of what it means for $\I$ to be maximal. Must $\AA$ necessarily have a maximal ideal assuming it is nonzero?\footnote{In the usual context of rings this is a consequence of Zorn's Lemma. Does Zorn's Lemma apply here?}
\end{enum}
\end{exercise}

\begin{exercise}
Let 
\begin{center}
\begin{tikzcd}
0 \arrow[r] & \F \arrow[r] & \G \arrow[r] & \mc{H} \arrow[r] & 0
\end{tikzcd}
\end{center}
be a short exact sequence in $\Mod_{\O_X}$.
\begin{enum}{\alph}
\item Show that if any two entries in the sequence are quasicoherent then so is the third.\footnote{Hint: Use the fact that $\QCoh(X)$ is an abelian subcategory of $\Mod_{\O_X}$.}

\item Suppose only that $\G$ is quasicoherent. Is it necessarily true that $\F$ is quasicoherent?
\end{enum}
\end{exercise}

Fix $A\in\CRing$. A priori there are several competing notions of endomorphisms of $A$. The elements of $\End_{\CRing}(A)$ are hard to describe explicitly, while $\End_{\CAlg_A}(A)$ consists of just $\id_A$. On the other hand we have $\End_{\Mod_A}(A)\iso A$, with $f\in A$ corresponding to the multiplication map $f: A\to A$. Importantly, note that the map $f$ is not a ring map if $f\neq1$. It turns out this last notion of endomorphism is the most useful in practice and so when we write $\Hom_A$ we will always mean $\Hom_{\Mod_A}$. In the same way, we will always write $\Hom_{\O_X}$ for $\Hom_{\Mod_{\O_X}}$. The same comments apply to endomorphisms.

Let's now do some categorification. We want to unpack the idea that $\End_A(A)$ should be the categorification of $A$. Given $f\in A$, recall that $f$ is
\begin{itemize}
\item idempotent if and only if $A\iso f(A)\oplus(\id_A-f)(A)$ as $A$-modules;

\item a unit if and only if $f\in\End_A(A)$ is invertible;

\item a non-zero-divisor (NZD) if and only if $f\in\End_A(A)$ has trivial kernel;

\item nilpotent if and only if $f^n=0$ in $\End_A(A)$ for some $n$ or, equivalently, $A_f=0$.
\end{itemize}
We can further categorify things by noting that $A$ is an integral domain if and only if $A$ has no nonzero NZDs. Given an ideal $I\normal A$, $I$ is prime if and only if $A/I$ is an integral domain. Finally, $f\in A$ is prime if and only if the ideal $fA$ is prime. What are some other potential properties of $A$ that might be ripe for categorification?
\begin{itemize}
\item The nilradical $\nil(A)$ of $A$ is the ideal of nilpotent elements.

\item $A$ is reduced if $\nil(A)$ is trivial. Equivalently, the reduction map $A\surj A/\nil(A)$ is an isomorphism of rings.

\item The radical $\rad(I)$ of $I$ is the preimage of $\nil(A/I)$ under $A\surj A/I$.
\end{itemize}

\begin{exercise}
Can you categorify the concept of $f\in A$ being irreducible?\footnote{Recall that $f\in A$ is irreducible if it cannot be written as a product of two nonzero nonunits.}
\end{exercise}

Given any $M\in\Mod_A$, we may consider $f\in A$ as an element of $\End_A(M)$ via scalar multiplication. This gives us a map $A\to\End_A(M)$ which is in fact a map of $A$-modules and so the annihilator 
$$\ann_A(M):=\ker(A\to\End_A(M))$$ 
is an ideal of $A$. At the same time, the $f$-torsion $M[f]:=\ker(f: M\to M)$ is an $A$-submodule of $M$. We can recover $M$ itself via the canonical isomorphism $\Hom_A(A,M)\xto{\sim}M$.

\begin{exercise}
Think about categorifying things for (commutative) $B$-algebras, which are equivalent to the data of (commutative) ring maps $A\to B$. As a challenge, can you categorify what it means for $b\in B$ to be integral over $A$?
\end{exercise}

We could keep going but this is plenty of material to get the mental juices flowing. With this in mind, let's quickly lay out the categories that will be our major players.

\begin{center}
\begin{tabular}{|l|l|l|}
\hline
Category & Generalization \\
\hline
$\Ab$ & $\Mod_{\O_X}$ \\
$\CRing$ & $\CAlg_{\O_X}$ \\
$\Mod_A$ & $\Mod_{\AA}$ \\
$\CAlg_A$ & $\CAlg_{\AA}$ \\
\hline
\end{tabular}
\end{center}

In this table, $A\in\CRing$ and $\AA\in\CAlg_{\O_X}$. As you might have guessed we have $(\CAlg_{\O_X})_{\AA/}\simeq\CAlg_{\AA}$ -- i.e., (commutative) $\AA$-algebras are the same thing as (commutative) $\O_X$-algebras equipped with a map of algebras from $\AA$. Note that each of these generalized categories has a quasicoherent version characterized by a quasicoherent cocycle condition. Note that we also have $\tensor_{\AA}$ and $\HHom_{\AA}:=\HHom_{\Mod_{\AA}}$ which behave as expected.

\begin{exercise}
Show that $\Mod_{\AA}$ is abelian (so has biproduct $\oplus$). Show that $\tensor_{\AA}$ commutes with $\oplus$.
\end{exercise}

\begin{exercise}
Investigate the relationship between $\Shv_X(\CRing)$ and $\CAlg_{\O_X}$.
\end{exercise}

\begin{exercise}
Let $\AA,\BB\in\CAlg_{\CC}$. Show that there is a canonical isomorphism 
$$\Spec_X(\AA\tensor_{\CC}\BB)\iso\Spec_X\AA\times_{\Spec_X\CC}\Spec_X\BB$$ 
in $\Aff\Sch_X$.
\end{exercise}

\begin{exercise}
Can we get an even more ``beefed up'' version of $\QCoh(X)$ by considering composable strings
\begin{center}
\begin{tikzcd}
\Spec_X\CC \arrow[r] & \Spec_X\BB \arrow[r] & \Spec_X\AA \arrow[r] & X
\end{tikzcd}
\end{center}
for $\AA,\BB,\CC\in\CAlg(\QCoh(X))$?
\end{exercise}

From here on out, for every statement or construction involving general ``algebra over $\O_X$'' you should try to work out the strictly quasicoherent version. Think of this as many exercises wrapped into one. The guiding principle here can be made into a slogan that we will flesh out later, a sort of doctrine of ``quasicoherent algebra.''

\textbf{\un{Slogan}:} If some property of $\QCoh(X)$ extends to all of $\Mod_{\O_X}$ then that property is probably special in some (precise) way.

\subsection{Finiteness Conditions}
Fix $\AA\in\CAlg_{\O_X}$. Note that, given $\BB\in\CAlg_{\AA}$, we have base change functors $\BB\tensor_{\AA}\cdot$ sending $\Mod_{\AA}$ to $\Mod_{\BB}$ and $\CAlg_{\AA}$ to $\CAlg_{\BB}$ (the latter is also referred to as extension of scalars in analogy with the classical case). The significance of this, aside from encoding interesting geometry, is that we can often just work with $\Mod_{\O_X}$ and $\CAlg_{\O_X}$ and then suitably base change. We will leave the last base change part to the reader.

\begin{remark}
$\Mod_{\O_X}$ and $\CAlg_{\O_X}$ will also be the most important to us because of their connection to the geometry of $\Aff\Sch_X$ via $\QCoh(X)$.
\end{remark}

\begin{exercise}
Show that, given any set $I$, $\O_X^{\oplus I}$ is the free object over $I$ in $\Mod_{\O_X}$. Conclude that $\AA^{\oplus I}$ is the free object over $I$ in $\Mod_{\AA}$.
\end{exercise}

What about free $\O_X$-algebras? Fortunately, $\O_X$ itself is quasicoherent! Given indeterminants $\{t_i: i\in I\}$, the associated free $\Z$-algebra is the polynomial ring $\Z[\{t_i: i\in I\}]$ and we obtain any associated free $A$-algebra (for $A\in\CRing$) canonically by base change. Given $(f: \Spec A\to X)\in\Aff\Sch_{/X}$, this lets us define $(\O_X[\{t_i: i\in I\}])_f:=A[\{t_i: i\in I\}]$.

\begin{exercise}
Show that $\O_X[\{t_i: i\in I\}]$ is quasicoherent and is the free object over $\{t_i: i\in I\}$ in $\CAlg_{\O_X}$. Conclude that $\AA[\{t_i: i\in I\}]:=\AA\tensor_{\O_X}\O_X[\{t_i: i\in I\}]$ is the free object over $\{t_i: i\in I\}$ in $\CAlg_{\AA}$.
\end{exercise}

Let's first discuss some finiteness conditions on $\O_X$-modules.

\begin{definition}
Let $\F\in\Mod_{\O_X}$. Consider the following finiteness conditions.
\begin{itemize}
\item $\F$ is \textbf{finitely generated} if there exists an epimorphism $\O_X^{\oplus n}\surj\F$. 

\item $\F$ is \textbf{locally finitely generated} or \textbf{LFG} if $\F|_U$ is finitely generated for every $U\in\Aff\Op(X)$.

\item $\F$ is \textbf{finitely presented} if there exists an exact sequence 
\begin{center}
\begin{tikzcd}
\O_X^{\oplus m} \arrow[r] & \O_X^{\oplus n} \arrow[r] & \F \arrow[r] & 0
\end{tikzcd}
\end{center}

\item $\F$ is \textbf{locally finitely presented} or \textbf{LFP} if $\F|_U$ is finitely presented for every $U\in\Aff\Op(X)$.

\item $\F$ is \textbf{coherent} if it is finitely generated and $\ker\phi$ is finitely generated for every map $\phi: \O_X^{\oplus n}|_U\to\F|_U$ with $n\geq0$ and $U\in\Op(X)$. These form a full subcategory $\Coh(X)\subset\Mod_{\O_X}$.
\end{itemize}
\end{definition}

Building off of the convention established in this definition, given some property $P$ on objects in $\Mod_{\O_X}$ we will say that $\F\in\Mod_{\O_X}$ is \textbf{locally $P$} if $\F|_U$ is $P$ for every $U\in\Aff\Op(X)$. For each property $P$, the reader should check if $P$ or its local version hold given that $P$ or its local version holds for every (or just some) affine open covering of $X$ (spoiler: all of these will be true in most cases).

\begin{remark}
Recall that an object $Y$ of a category $\CC$ is called \textbf{compact} if $\Hom_{\CC}(Y,\cdot): \CC\to\Set$ commutes with all filtered colimits.
\begin{enum}{\alph}
\item Fix $A\in\CRing$ and $M\in\Mod_A$. Show that $M$ is compact if and only if it is finitely presented.

\item Fix $X\in\Sch$ and $\F\in\Mod_{\O_X}$. Is $\F$ compact if and only if it is finitely presented?
\end{enum}
\end{remark}

\begin{remark}
The condition of being finitely generated has a ``quasicoherent'' version defined by taking $\F\in\Mod_{\O_X}$ to be \textbf{generated by global sections} if there exists an epimorphism $\O_X^{\oplus I}\surj\F$ with $I$ possibly infinite. One could then make sense of what it means to be \emph{locally generated by global sections}.
\end{remark}

\begin{remark}
Let $\F\in\Mod_{\O_X}$ be finitely generated. Then, $\F$ is automatically \emph{strongly} finitely presented in the sense that $\F|_U$ is finitely generated for every $U\in\Op(X)$. Similarly, $\F$ is automatically \emph{strongly} finitely presented if it is finitely presented.
\end{remark}

\begin{exercise}
Fix $\F\in\Mod_{\O_X}$.
\begin{enum}{\alph}
\item Suppose that $\F$ is finitely presented. Show that $\F$ is quasicoherent.

\item Is $\F$ necessarily quasicoherent if it is LFP?

\item Is $\F$ necessarily quasicoherent if it is finitely generated? LFG?
\end{enum}
\end{exercise}

\begin{exercise}
Fix $\F\in\QCoh(X)$. 
\begin{enum}{\alph}
\item Suppose that $\F$ is LFG. Show that $\F(U)$ is a finitely generated $\O_X(U)$-module for every $U\in\Aff\Op(X)$. Is the converse true? 

\item Suppose that $\F$ is LFP. Show that $\F(U)$ is a finitely presented $\O_X(U)$-module for every $U\in\Aff\Op(X)$. Is the converse true? 
\end{enum}
\end{exercise}

\begin{exercise}
Show that coherent sheaves are finitely presented hence quasicoherent and so $\Coh(X)\subset\QCoh(X)$.
\end{exercise}

\begin{exercise}
Give an example showing that $\O_X$ need not be coherent as a module over itself.
\end{exercise}

\begin{exercise}
Fix $A\in\CRing$. We know that $\Gamma: \QCoh(\Spec A)\xto{\sim}\Mod_A$. What can be said about the image $\Gamma(\Coh(\Spec A))$ consisting of \textbf{coherent} $A$-modules?
\end{exercise}

\begin{exercise}
Analyze the category $\Coh(X)$. Here are some sample questions.
\begin{itemize}
\item Does $\Coh(X)$ have (co-)kernels?

\item Is $\Coh(X)$ closed under $\tensor_{\O_X}$ and/or $\HHom$?

\item Does $\Coh(X)$ satisfy any $2$-out-of-$3$ properties for short exact sequences in $\Mod_{\O_X}$?

\item Is $\Coh(X)$ abelian?

\item Is the inclusion $\Coh(X)\inj\Mod_{\O_X}$ exact?
\end{itemize}
\end{exercise}

What about finiteness conditions on $\O_X$-algebras? Given indeterminants $\{t_i : i\in I\}$ recall that we have the free polynomial algebra $\O_X[\{t_i : i\in I\}]$. We say that $\AA\in\CAlg_{\O_X}$ is \textbf{finitely generated} (as an $\O_X$-algebra) if there exists $n\geq0$ and an ideal $\J$ of $\O_X[t_1,\ldots,t_n]$ such that $\O_X[t_1,\ldots,t_n]/\J\iso\AA$ in $\CAlg_{\O_X}$. If $\J$ is itself finitely generated as an $\O_X[t_1,\ldots,t_n]$-module then we say that $\AA$ is \textbf{finitely presented} (as an $\O_X$-algebra). 

\begin{remark}
Since there is some potential for confusion we say that $\BB\in\CAlg_{\AA}$ is \textbf{finite} if it is finitely generated as an $\AA$-module and \textbf{finite type} if it is finitely generated as an $\AA$-algebra. This still isn't great but it is convention. We obtain from this notion of being \textbf{locally finite type} or simply \textbf{LFT}.
\end{remark}

\begin{exercise}
Show that there is a canonical isomorphism between $\Spec_X\O_X[t_1,\ldots,t_n]$ and $\A_X^n=X\times_{\Spec\Z}\A^n$ in $\Aff\Sch_X$. This will be handy later for geometric reasons.
\end{exercise}

\begin{exercise}
Fix $\AA\in\CAlg(\QCoh(X))$. 
\begin{enum}{\alph}
\item Suppose that $\AA$ is LFT. Show that $\F(U)$ is a finite type $\O_X(U)$-algebra for every $U\in\Aff\Op(X)$. Is the converse true? 

\item Suppose that $\AA$ is a locally finitely presented algebra. Show that $\AA(U)$ is a locally finitely presented $\O_X(U)$-algebra for every $U\in\Aff\Op(X)$. Is the converse true? 
\end{enum}
\end{exercise}

\subsection{More Module and Algebra Theory}
Fix $X\in\Sch$. We define the \textbf{nilradical} of $\O_X$ to be the $\O_X$-module $\nil(\O_X)$ given as a quasicoherent sheaf by sending $\Spec A\to X$ to $\nil(A)$.

\begin{exercise}
Show that $\nil(\O_X)$ is a quasicoherent ideal sheaf on $X$.
\end{exercise}

We say $X$ is \textbf{reduced} if $\nil(\O_X)=0$ and call $X_{\red}:=V(\nil(\O_X))\in\Aff\Sch_X$ the \textbf{reduction} of $X$.

\begin{exercise}
Show that $X_{\red}$ is reduced and that $X_{\red}\in\Aff\Sch_X$ has the universal property that any $Y\in\Sch_{/X}$ with $Y$ reduced uniquely factors through $X_{\red}\to X$.
\end{exercise}

Given $\AA\in\CAlg_{\O_X}$ and $\I\normal\AA$ an ideal, we say that $\I$ is \textbf{nilpotent} if $\I^n=0$. 

\begin{exercise}
Let $\J\normal\O_X$ be a nilpotent quasicoherent ideal sheaf.
\begin{enum}{\alph}
\item Is $\nil(\O_X)$ nilpotent?

\item Show that the natural map $\abs{V(\J)}\to\abs{X}$ given by sending $y:\Spec k\to V(\J)$ to the composition $\Spec k\xto{y}V(\J)\to X$ is a bijection. For this reason we call $V(\J)$ a nilpotent thickening of $X$.\footnote{The name comes from the fact that, if we view $X$ as a locally ringed space, then $\abs{X}$ is essentially the underlying set.}

\item Show that the natural map $\abs{X^{\red}}\to\abs{X}$ is a bijection.
\end{enum}
\end{exercise}

\begin{exercise}
Can you define $\nil(\AA)$ for $\AA\in\CAlg(\QCoh(X))$? How about for general $\AA\in\CAlg_{\O_X}$?
\end{exercise}

We define an \textbf{ascending chain} in $\Mod_{\AA}$ to be a sequence of monomorphisms 
\begin{center}
\begin{tikzcd}
\M_1 \arrow[r, hookrightarrow] & \M_2 \arrow[r, hookrightarrow] & \M_3 \arrow[r, hookrightarrow] & \cdots
\end{tikzcd}
\end{center}
and a \textbf{descending chain} to be a sequence of monomorphisms
\begin{center}
\begin{tikzcd}
\M_1 \arrow[r, hookleftarrow] & \M_2 \arrow[r, hookleftarrow] & \M_3 \arrow[r, hookleftarrow] & \cdots
\end{tikzcd}
\end{center}
We say such a chain \textbf{stabilizes} if the monomorphisms in the chain eventually become isomorphisms. We say that $\M\in\Mod_{\AA}$ is \textbf{Noetherian} if it satisfies the ascending chain condition for $\AA$-submodules -- i.e., every ascending chain of submodules of $\M$ stabilizes. Similarly, we say that $\M\in\Mod_{\AA}$ is \textbf{Artinian} if it satisfies the descending chain condition for $\AA$-submodules -- i.e., every descending chain of submodules of $\M$ stabilizes. Given $\AA\in\CAlg_{\O_X}$, we say that $\AA$ is \textbf{Noetherian} (resp., \textbf{Artinian}) if it is Noetherian (resp., Artinian) as a module over itself. If $\M$ is quasicoherent then we instead say that $\M$ is \emph{strongly} Noetherian or \emph{strongly} Artinian. The terms ``Noetherian'' and ``Artinian'' in this context are reserved for thinking only about \emph{quasicoherent} submodules since (spoiler!) submodules of quasicoherent sheaves generally need not be quasicoherent.

\begin{exercise}
\hfill
\begin{enum}{\alph}
\item Show that $\M\in\Mod_{\AA}$ is Noetherian if and only if every $\AA$-submodule is finitely generated.

\item Show that $\AA\in\CAlg_{\O_X}$ is Noetherian if and only if every $\AA$-module is finitely generated.

\item Suppose that $\AA\in\CAlg(\QCoh(X))$ and $\M\in\Mod_{\AA}(\QCoh(X))$. Show that $\M$ is Noetherian if and only if every quasicoherent $\AA$-submodule is finitely generated. Is there another equivalent condition?
\end{enum}
\end{exercise}

\begin{exercise}
Think about what it might mean for $X$ to be Noetherian (more on this later).
\end{exercise}

Switching gears a bit, we may think of $s\in\End_{\O_X}(\O_X)$ as an element of $\End_{\O_X}(\F)$ via the commutative diagram
\begin{center}
\begin{tikzcd}
\O_X\tensor_{\O_X}\F \arrow[r, "s\tensor\id_{\F}"] \arrow[d, "\iso"'] & \O_X\tensor_{\O_X}\F \arrow[d, "\iso"] \\
\F \arrow[r, dotted, "\exists!\,s"'] & \F
\end{tikzcd}
\end{center}
Doing the same procedure with $\AA$ instead of $\F$ actually gives us $s\in\End_{\AA}(\AA)$. These constructions are functorial in the sense that we obtain an $\O_X$-module map $\O_X\to\EEnd_{\O_X}(\F)$. We may then define the \textbf{annihilator} of $\F$ to be
$$\ann_{\O_X}(\F):=\ker(\O_X\to\EEnd_{\O_X}(\F)).$$
This is an ideal of $\O_X$ by construction.

\begin{exercise}
Show that there is a canonical isomorphism $\Gamma(\ann_{\O_{\Spec A}}(\twid{M}))\iso\ann_A(M)$. Conclude that if $\F\in\QCoh(X)$ then $\ann_{\O_X}(\F)\in\QCoh(X)$ is a quasicoherent ideal sheaf on $X$.
\end{exercise}

\begin{exercise}
``Categorify'' more of the things discussed earlier.
\end{exercise}

\subsection{Some Homological Algebra}
We can extend the section functor $\Gamma$ on $\QCoh(X)$ to get a new section functor
$$\Gamma: \Op(X)^{\op}\times\Mod_{\O_X}\to\Ab,\qquad (U,\F)\mapsto\F(U).$$
Fixing $\F\in\Mod_{\O_X}$ and $U\in\Op(X)$, we obtain $\Gamma_U: \Mod_{\O_X}\to\Ab$ and $\Gamma_{\F}: \Op(X)^{\op}\to\Ab$ as expected.

\begin{exercise}
\hfill
\begin{enum}{\alph}
\item Show that $\Gamma_X: \Mod_{\O_X}\to\Ab$ is left exact.

\item Suppose that $X=\Spec A$. Show that the restriction $\Gamma_X: \QCoh(X)\to\Ab$ is exact.\footnote{This can actually be used to characterize affine schemes among an appropriately chosen larger class of schemes.}
\end{enum}
\end{exercise}

\begin{exercise}
Fix $\F\in\Mod_{\O_X}$.
\begin{enum}{\alph}
\item Show that $\Hom_{\Mod_{\O_X}}(\F,\cdot): \Mod_{\O_X}\to\Ab$ and $\Hom_{\Mod_{\O_X}}(\cdot,\F): \Mod_{\O_X}^{\op}\to\Ab$ are left exact.

\item Is it true that $\Hom_{\Mod_{\O_X}}(\F,\cdot)$ is exact if and only if $\HHom(\F,\cdot)$ is exact?\footnote{Recall that a functor is exact if and only if it is both left and right exact (this is not just a tautology!). Note that, in general, a functor is defined to be left exact if it preserves finite limits and right exact if it preserves finite colimits.} Does anything change if $\F$ is quasicoherent?

\item Is it true that $\Hom_{\Mod_{\O_X}}(\cdot,\F)$ is exact if and only if $\HHom(\cdot,\F)$ is exact?
\end{enum}
\end{exercise}

\begin{definition}
Fix $\F\in\Mod_{\O_X}$. We say that $\F$ is
\begin{itemize}
\item \textbf{free} if $\F\iso\O_X^{\oplus I}$ for some index set $I$;

\item \textbf{projective} if $\Hom_{\O_X}(\F,\cdot): \Mod_{\O_X}\to\Ab$ is exact;

\item \textbf{injective} if $\Hom_{\O_X}(\cdot,\F): \Mod_{\O_X}^{\op}\to\Ab$ is exact;

\item \textbf{flat} if $\cdot\tensor_{\O_X}\F: \Mod_{\O_X}\to\Mod_{\O_X}$ (equivalently $\F\tensor_{\O_X}\cdot$) is exact.
\end{itemize}
\end{definition}

As is the case with Noetherian and Artinian modules, if $\F$ is quasicoherent then we really should put the word \emph{strongly} in front of all of these terms (except for ``free'') and reserve the above terms for thinking only about $\Hom_{\QCoh(X)}$. We will implicitly take this as a terminology convention moving forward.

\begin{exercise}
Relate projectivity and injectivity of $\O_X$-modules to splitting of short exact sequences.
\end{exercise}

\begin{exercise}
Show that $\F\in\QCoh(X)$ is locally projective (resp., injective, flat) if and only if $\F(U)$ is projective (resp., injective, flat) as an $\O_X(U)$-module for every $U\in\Aff\Op(X)$.
\end{exercise}

\subsection{Fibers, Stalks, and Nakayama's Lemma}
Let's now inject some more geometry into the mix. Recall that a \emph{field-valued point} of $X$ is a map of schemes $x: \Spec k\to X$ with $k$ a field and that the collection of such points is denoted $|X|$. Given $j: U\inj X$ an open embedding, we say that $U$ \textbf{contains} $x$ if there is a factorization
\begin{center}
\begin{tikzcd}
\Spec k \arrow[r, "x"] \arrow[d, dotted, "\exists"'] & X \\
U \arrow[ru, hookrightarrow, "j"']
\end{tikzcd}
\end{center}
Given $\F\in\QCoh(X)$ and $x\in|X|$ as above, we call $x^*\F\in\QCoh(\Spec k)$ the \textbf{fiber} of $\F$ at $x$ and denote it by $\F|_x$.

\begin{remark}
It is also common to use the notation $\F(x)$ for the fiber. We will use $\F(x)$ to refer to the $k$-vector space $\Gamma(\Spec k,\F|_x)$, when there is no chance of confusion. The field $k$ is often called the \textbf{residue field} of $x$ and denoted by $\kappa(x)$. If $X=\Spec A$ then we can take $\kappa(x)$ to be $\Frac(A/\mf{p})$ for some prime ideal $\mf{p}\normal A$. In such a situation we will often abuse notation and say $\mf{p}\in\abs{\Spec A}$.
\end{remark}

Another local notion associated to $x$ and defined for any presheaf $\F\in\Pre_X(\Ab)$ is the (Zariski) \textbf{stalk} 
$$\F_x:=\fcolim_{x\in U}\F(U),$$
where the colimit is taken over the filtered full subcategory of $\Op(X)$ spanned by $U$ containing $x$.

\begin{remark}
In the case that $\F\in\QCoh(X)$ there are two possible meanings for the notation $\F_x$ -- the stalk of $\F$ at $x$ and the $k$-vector space $\F_x$ specified by quasicoherence. Do these two notions agree?
\end{remark}

\begin{exercise}
Show that any cofinal system of $U\in\Op(X)$ containing $x$ encodes enough data to recover the stalk $\F_x$ up to isomorphism.
\end{exercise}

\begin{exercise}
Let $\F\in\Pre_X(\Ab)$. Show that the natural map $\F\to\F^{\sh}$ induces a natural isomorphism $\F_x\to(\F^{\sh})_x$.\footnote{One approach to defining (Zariski) sheafification even makes use of stalks.}
\end{exercise}

\begin{exercise}
Show that the sequence
\begin{center}
\begin{tikzcd}
0 \arrow[r] & \F \arrow[r] & \G \arrow[r] & \mc{H} \arrow[r] & 0
\end{tikzcd}
\end{center}
in $\Shv_X(\Ab)$ is exact if and only if the induced sequence
\begin{center}
\begin{tikzcd}
0 \arrow[r] & \F_x \arrow[r] & \G_x \arrow[r] & \mc{H}_x \arrow[r] & 0
\end{tikzcd}
\end{center}
in $\Ab$ is exact for every $x\in\abs{X}$.
\end{exercise}

\begin{remark}
The previous exercise tells us that stalks are able to discern if morphisms of sheaves are monomorphisms, epimorphisms, or isomorphisms -- consequently, we say that all of these properties are \textbf{stalk-local}. However, this does not mean that two sheaves all of whose stalks are isomorphic are themselves isomorphic since there is a priori no way to ``glue together'' the stalk isomorphisms.
\end{remark}

\begin{exercise}
Let $A\in\CRing$, $x=\mf{p}\in\abs{\Spec A}$, and $M\in\Mod_A$. Show that the stalk $\twid{M}_x$ is isomorphic to $M_{\mf{p}}$, the localization of $M$ at the complement of $\mf{p}$ in $A$. Is this isomorphism canonical?\footnote{If this isomorphism is canonical then that suggests that we can use this as the jumping off point to define stalks of quasicoherent sheaves.}
\end{exercise}

\begin{theorem}[Geometric Nakayama's Lemma]
Let $\F\in\QCoh(X)$ be LFG and $x\in|X|$ such that $\F|_x=0$. Then, there exists $U\in\Op(X)$ containing $x$ such that $\F|_U=0$.
\end{theorem}

\begin{corollary}
Let $\F\in\QCoh(X)$ be LFG and $U_{\F}\to X$ the monomorphism given by taking $U_{\F}(B)$ to be $\{f: \Spec B\to X \,|\, f^*\F=0\}$. Then, $U_{\F}\to X$ is an open embedding.
\end{corollary}

\begin{exercise}
How is this connected to the annihilator $\ann_{\O_X}(\F)$ and the associated closed embedding $V(\ann_{\O_X}(\F))\inj X$?
\end{exercise}

\begin{exercise}
Let $\J$ be a quasicoherent ideal sheaf on $X$. Is $V(\J)$ related to $U_{\J}$?
\end{exercise}

\begin{comment}
Extensions and contractions of ideals
Colon ideals
Invertible ideal sheaves and invertible sheaves more generally
Vector bundles (and specifically line bundles)
Fractional ideal
Cartier divisor
Weil divisor
\end{comment}

\section{Vector Bundles}
\subsection{Basics}
\begin{definition}
We say that $\EE\in\Mod_{\O_X}$ is a \textbf{vector bundle} if $\EE$ is LFG locally free or, equivalently, if there exists an affine open covering $\U\in\Cov(X)$ such that $\EE|_U$ is free of finite rank for every $U\in\U$. Such objects span a full subcategory $\Vect(X)\subset\Mod_{\O_X}$.
\end{definition}

\begin{exercise}
Show that vector bundles on $X$ are quasicoherent and so $\Vect(X)$ is a full subcategory of $\QCoh(X)$.
\end{exercise}

\begin{exercise}
Given $\EE\in\QCoh(X)$, show that the following are equivalent.
\begin{enum}{\roman}
\item $\EE$ is a vector bundle.

\item $\EE$ is LFG locally projective.

\item $\EE$ is LFP locally flat.
\end{enum}
Do stalks play any role here?
\end{exercise}

\begin{exercise}
Show that $\Vect(X)$ is closed under $\tensor_{\O_X}$. Is it closed under taking kernels or cokernels?
\end{exercise}

\begin{exercise}
Given $\EE\in\Vect(X)$, define its \textbf{dual} to be $\EE^{\vee}:=\HHom(\EE,\O_X)$. Show that $\EE^{\vee}\in\Vect(X)$ and, moreover, that $(\cdot)^{\vee}$ is a contravariant involution on $\Vect(X)$.
\end{exercise}

\begin{exercise}
Given $\EE\in\Vect(X)$, under what conditions are $\EE$ and $\EE^{\vee}$ isomorphic?
\end{exercise}

\begin{exercise}
Show that $\EE\in\Vect(X)$ has a well-defined rank function $\rank: \abs{X}\to\Z^{\geq0}$ which is locally constant.\footnote{Some care must be taken in interpreting this statement since we haven't put any topology on $\abs{X}$.}
\end{exercise}

We say that $\EE$ is a \textbf{line bundle} if it is of constant rank $1$. In this case, we typically use the notation $\L$ instead of $\EE$.

\subsection{Projective Space}
Our goal in this section is to classify lines in a scheme $X$, which can be understood as line bundles on $X$. This is a very simple example of a so-called \emph{moduli problem}. With this in mind, we let $\Pic(X)\subset\Vect(X)$ denote the full subcategory of line bundles on $X$ and $\PPic(X)$ the set of isomorphism classes of line bundles on $X$. In fact, though, $\PPic(X)$ is a group and so we call it the \textbf{Picard group} of $X$. Let's unpack this.

\begin{definition}
An \textbf{invertible} $\O_X$-module is a unit for the symmetric monoidal structure on $\Mod_{\O_X}$ encoded by $\tensor_{\O_X}$ -- i.e., $\F\in\Mod_{\O_X}$ such that there exists $\G\in\Mod_{\O_X}$ satisfying 
$$\F\tensor_{\O_X}\G\iso\O_X\iso\G\tensor_{\O_X}\F.$$ 
We say that $\G$ is the \textbf{inverse} of $\F$ and note that $\G$ is unique up to isomorphism.\footnote{Is it unique up to unique isomorphism?}
\end{definition}

\begin{exercise}
Given $\EE,\F\in\Vect(X)$, show that there is a canonical isomorphism 
$$\EE^{\vee}\tensor_{\O_X}\F\iso\HHom_{\O_X}(\EE,\F).$$ 
In particular, $\EE^{\vee}\tensor_{\O_X}\EE\iso\HHom_{\O_X}(\EE,\EE)=\EEnd_{\O_X}(\EE)$.
\end{exercise}

Given $\L\in\Pic(X)$ this tells us precisely that the inverse of $\L$ is $\L^{\vee}$ and so $\PPic(X)$ is a group. In more detail, we can choose an affine open covering $\U\in\Cov(X)$ such that $\L|_U\iso\O_U$ for every $U\in\U$. It then follows that 
\begin{align*}
(\L^{\vee}\tensor_{\O_X}\L)(U)
&\iso\EEnd_{\O_X}(\L)(U) \\
&\iso\End_{\O_U}(\L|_U) \\
&\iso\End_{\O_U}(\O_U) \\
&\iso\End_{\O_X(U)}(\O_X(U)) \\
&\iso\O_X(U)
\end{align*}
for every $U\in\U$ and so $\L^{\vee}\tensor_{\O_X}\L\iso\O_X$.

\begin{exercise}
Let $\pi: X\to S$ be a map of schemes.
\begin{enum}{\alph}
\item Show that there is a well-defined functor $\pi^*: \Pic(S)\to\Pic(X)$.

\item Show that there is a well-defined group homomorphism $\pi^*: \PPic(S)\to\PPic(X)$.

\item Show that $\PPic(X)$ is contravariantly functorial in $X$ and so we obtain a functor $\PPic: \Sch^{\op}\to\Grp$.

\item Suppose that $\pi_*: \QCoh(X)\to\QCoh(S)$ is defined. Show that there is a well-defined functor $\pi_*: \Pic(X)\to\Pic(S)$ and group homomorphism $\pi_*: \PPic(X)\to\PPic(S)$.

\item What does the adjunction $\pi^*\dashv\pi_*$ tell us about the group homomorphisms $\pi^*$ and $\pi_*$?
\end{enum}
\end{exercise}

\begin{exercise}
Since $\Vect(X)\subset\QCoh(X)$ we can work directly with $\QCoh(X)$ instead of $\Mod_{\O_X}$. Show that $\EE\in\QCoh(X)$ is a vector bundle of rank $n$ if and only if $\EE_f$ is a finitely generated projective $A$-module of rank $n$ for every $f: \Spec A\to X$.
\end{exercise}

\begin{exercise}
Thinking of $\Vect(X)\subset\QCoh(X)$ as in the previous exercise, we may define a descent category $\Vect(X;\U)$ analogously to $\QCoh(X;\U)$ for \emph{any} $\U\in\Cov(X)$. Show that $\Vect(X)\simeq\Vect(X;\U)$. How are $\Pic(X)$ and $\PPic(X)$ related to $\Pic(X;\U)$ and $\PPic(X;\U)$?
\end{exercise}

\begin{exercise}
Given $\U\in\Cov(X)$, consider the category $\Vect_{\rig}(X,\U)$ of \textbf{$\U$-rigidified vector bundles} defined to be vector bundles trivialized on $\U$, where we have to keep track of cocycles satisfying an appropriate cocycle condition. Part of the advantage of working with the categories $\Vect_{\rig}(X,\U)$ is that we can obtain a finer invariant of $X$ then $\PPic(X)$. The key word here is stacks! Note that each $\Vect_{\rig}(X,\U)$ has a notion of a category $\Pic_{\rig}(X,\U)$ and thus a group $\PPic_{\rig}(X,\U)$.
\end{exercise}

\begin{exercise}
By the above we have $\PPic\in\Fun(\Sch^{\op},\Grp)\simeq\Pre(\Sch,\Grp)$. Via Yoneda we have $\Grp(\Sch)\simeq\Pre_{\rep}(\Sch,\Grp)$. With this in mind, when is $\PPic$ representable and hence corresponds to a group scheme?
\end{exercise}

Given $\EE\in\Vect(X)$, define its \textbf{total space} to be
$$\Theta(\EE):=\Spec_X\Sym_{\O_X}(\EE^{\vee}).$$
There are multiple reasons for taking the dual. From a purely categorical standpoint, taking the dual ensures that $\Theta$ is covariantly functorial. Total spaces arise because of their natural connection to projectivizations.

\begin{definition}
Let $X\in\Sch$ and $\EE\in\Vect(X)$. The \textbf{projectivization} $\P(\EE)$ is the space with $\P(\EE)(A)$ for $A\in\CRing$ given by isomorphism classes\footnote{We pass to isomorphism classes to ensure that we actually get a set.} of triples $(\L,x,i)$ with $\L\in\Pic(\Spec A)$, $x\in X(A)$, and $i: \L\to x^*\EE$ everywhere nonvanishing. The forgetful functor $\P(\EE)\to X$ canonically makes $\P(\EE)$ into an $X$-scheme.
\end{definition}

Here, given a base $S\in\Sch$, a map $i: \L\to\EE$ with $\L\in\Pic(S)$ and $\EE\in\Vect(S)$ is \textbf{everywhere nonvanishing} if $f^*i: f^*\L\to f^*\EE$ is nonzero for every $(f: T\to S)\in\Aff\Sch_{/S}$.

\begin{example}
Taking $X=\Spec\Z$ and $\EE=\O_X^{\oplus(n+1)}$ in $\P(\EE)$ yields $n$-dimensional \textbf{projective space} $\P^n=\P_{\Z}^n$.
\end{example}

\begin{exercise}
Show that projectivization defines a covariant functor $\P: \QCoh(X)\to\Space$.
\end{exercise}

\begin{exercise}
Let $\pi\in\Hom_{\Space}(X,S)$ and $\EE\in\Vect(S)$. Show there is a natural isomorphism $\P(\pi^*\EE)\iso\pi^{-1}\P(\EE)$ -- i.e., show there is a natural Cartesian square
\begin{center}
\begin{tikzcd}
\P(\pi^*\EE) \arrow[r] \arrow[d] & \P(\EE) \arrow[d] \\
X \arrow[r, "\pi"'] & S
\end{tikzcd}
\end{center}
where $\P(\pi^*\EE)\to\P(\EE)$ is given by sending the isomorphism class of the triple $(\L,x,i)$ to the isomorphism class of the triple $(\L,\pi\circ x,i)$.
\end{exercise}

It follows that, given any $S\in\Sch$, we unambiguously have $\P_S^n:=\P(\O_S^{\oplus(n+1)})\iso S\times\P^n$.

\begin{exercise}
Fix $X\in\Sch$.
\begin{enum}{\alph}
\item Show that there is a canonical isomorphism $\Sym_{\O_X}(\O_X^{\oplus n})\iso\O_X[t_1,\ldots,t_n]$.

\item Show that there is a non-canonical isomorphism $\Theta(\O_X^{\oplus n})\iso\Spec_X\O_X[t_1,\ldots,t_n]$.

\item Given $B\in\CRing$, show that there is a canonical isomorphism 
$$(\Spec_X\O_X[t_1,\ldots,t_n])(B)\iso X(B)\times B^n.$$

\item Conclude that there is a non-canonical isomorphism $\Theta(\O_X^{\oplus(n+1)})\iso X\times\A^{n+1}=\A_X^{n+1}$.
\end{enum}
\end{exercise}

\begin{exercise}
Fix a base $S\in\Sch$ and $i: \L\to\EE$ with $\L\in\Pic(S)$ and $\EE\in\Vect(S)$. Show that the following are equivalent.
\begin{enum}{\roman}
\item The map $i$ is everywhere nonvanishing.

\item $i^{\vee}: \EE^{\vee}\to\L^{\vee}$ is epic.

\item $\Theta(i): \Theta(\L)\to\Theta(\EE)$ is a closed embedding.
\end{enum}
\end{exercise}

\begin{exercise}
Copy the setup of the previous exercise. Show that the following are equivalent.
\begin{enum}{\roman}
\item The map $i$ is everywhere nonvanishing.

\item The map $i$ is everywhere nonvanishing with respect to $\Spec k\to S$ for $k$ a field.

\item Assuming $\EE$ has rank $r$, choose a simultaneous trivializing open $U\in\Aff\Op(S)$ such that $\L|_U\iso\O_U$ and $\EE|_U\iso\O_U^{\oplus r}$. Then, the induced map
$$\O_U\to\O_U^{\oplus r},\qquad 1\mapsto(f_1,\ldots,f_r)$$
has the property that $(f_1,\ldots,f_r)$ generate the unit ideal in $\Gamma(U,\O_U)$.

\item $\coker(i: \L\to\EE)$ is a vector bundle on $S$.
\end{enum}
\end{exercise}

\begin{theorem}
Let $X\in\Sch$ and $\EE\in\Vect(X)$. Then, $\P(\EE)$ is a scheme.
\end{theorem}

\begin{proof}
Let $\U$ be a trivializing affine open covering of $\EE$ over $X$ and $r\geq1$ the rank of $\EE$. Given $U\in\U$,
$$\P(\EE)\times_XU\iso\P(\EE|_U)\iso\P(\O_U^{\oplus r})\iso\P_U^{r-1}$$
and so we are reduced to showing that $\P^n$ is a scheme for $n\geq1$. Given $A\in\CRing$, elements of $\P^n(A)$ are represented by pairs $(\L,s)$ with $\L\in\Pic(\Spec A)$ and $s: \L\to\O_{\Spec A}^{\oplus(n+1)}$ everywhere nonvanishing. Writing $s=(s_0,\ldots,s_n)$ with $s_i: \L\to\O_{\Spec A}$, demanding that $s_i$ is everywhere nonvanishing defines a subspace $U_i$ of $\P^n$ (by describing $U_i(A)$). We claim that $\{U_0,\ldots,U_n\}$ is an affine open covering of $\P^n$. We will show first that $U_i\inj\P^n$ is an open embedding. Let $S=\Spec A$ with $S\to\P^n$. As stated previously this is represented by a pair $(\L,(s_0,\ldots,s_n))$. Assuming $s_i$ is everywhere nonvanishing, we obtain a closed embedding $\Theta(s_i): \Theta(\L)\inj\Theta(\O_S)\iso\A_S^1$. Pulling back $\Theta(\L)\to\A_S^1$ by the zero section map $S\to\A_S^1$, induced by 
$$A[t]\to A,\qquad t\mapsto0,$$
yields a closed embedding $Z_i\inj S$. It then follows that $S\times_{\P^n}U_i\iso S\setminus Z_i$ and so $U_i\inj\P^n$ is an open embedding. One then checks that the $U_i$ form an open covering and $U_i\iso\A^n$.
\end{proof}

\begin{exercise}
Show that $S\times_{\P^n}U_i\iso S\setminus Z_i$.\footnote{It might help to note that $S\setminus Z_i=S\setminus(S\times_{\A_S^1}\Theta(\L))\iso S\times_{\A_S^1}(\A_S^1\setminus\Theta(\L))$.}
\end{exercise}

\subsection{Group Actions}
We open this section with a simple question: What is a group action? Let's start with the simplest case. Let $G$ be a group and $X$ a set. A (left) action of $G$ on $X$ is a function $\phi: G\times X\to X$, whose action is typically denoted by $g\cdot x:=\phi(g)(x)$, such that $e\cdot x=x$ and $g\cdot(h\cdot x)=gh\cdot x$ for all $g,h\in G$ and $x\in X$. It's easy to see that $\phi(g)$ is a (set-theoretic) automorphism of $X$, and in fact the data of $\phi$ is equivalent to the data of a group homomorphism $G\to\Aut_{\Set}(X)$. The data of any $g\in G$ is itself equivalent to the left multiplication map $\l_g\in\Aut_{\Grp}(G)$. This tells us that we can view $G$ as a special kind of category $\mathbf{G}$, called a \emph{groupoid}, with one object corresponding to $G$ and a bunch of automorphisms corresponding to $\l_g$ for $g\in G$. A group action of $G$ on $X$ is then equivalent to the data of a functor $\Phi: \mathbf{G}\to\Set$ sending the one object $\bullet$ of $\mathbf{G}$ to $X$. There's nothing special about $\Set$ in this situation -- we could put any category $\CC$ and everything would still go through. This matters to us because we are interested in group actions on more than just sets. In particular, we are interested in group actions on spaces and sheaves.

Given $G\in\Grp$ acting on $X\in\Set$, it is natural to consider the fixed points 
$$X^G:=\{x\in X : g\cdot x=x\textrm{ for every }g\in G\}.$$ 
We can suggestively characterize $X^G$ as the largest subset of $X$ on which $G$ acts trivially. We would like to categorify this. The main issue is that the relation of subset inclusion is not functorial. Fortunately, there is a way around this.

\begin{definition}
Given an object $X\in\CC$, a \textbf{subobject} of $X$ is an isomorphism class of monomorphisms into $X$. Explicitly, two monomorphisms $Y\inj X$ and $Y'\inj X$ are isomorphic if we may complete the diagram
\begin{center}
\begin{tikzcd}
Y \arrow[rr, dotted, "\sim"] \arrow[rd, hookrightarrow] & & Y' \\
& X \arrow[ru, hookleftarrow] &
\end{tikzcd}
\end{center}
We think of $Y\inj X$ as picking out some collection of elements in $X$, with the passage to isomorphism classes made so that this construction is functorial (and for set-theoretic reasons).
\end{definition}

\begin{exercise}
Show that subobjects of a fixed $X\in\CC$ form a poset.\footnote{Technically this could be a proper class but we won't worry ourselves about this.}
\end{exercise}

\begin{exercise}
Show that subobjects in $\Set$ are exactly subsets.
\end{exercise}

Let now $X\in\CC$, $Y\inj X$, and $\phi\in\Aut_{\CC}(X)$. We would like to understand what it means for $\phi$ to induce an automorphism of $Y$. Consider the Cartesian square
\begin{center}
\begin{tikzcd}
X\times_{X,\phi}Y \arrow[r, "\pr_1"] \arrow[d, "\pr_2"'] & X \arrow[d, "\phi"] \\
Y \arrow[r, hookrightarrow] & X
\end{tikzcd}
\end{center}
The induced map $\pr_2: X\times_{X,\phi}Y\to Y$ is automatically a monomorphism since monomorphisms are preserved by fiber products (or, more generally, any limits) and $\phi$ is an isomorphism. This is easy to see in the case that $\CC=\Set$ since then 
$$X\times_{X,\phi}Y=\{(x,y)\in X\times Y : \phi(x)=y\}.$$
In fact, in this setting we see that $\phi$ sends $Y$ to $Y$ (and thus is an automorphism of $Y$) precisely when the map $\pr_2 X\times_{X,\phi}Y\to Y$ is an isomorphism. This tells us that, in the general case, we should say \textbf{$\phi$ acts on $Y$} if the natural map $\pr_2: X\times_{X,\sigma}Y\to Y$ is an isomorphism. Building on this, we should say that \textbf{$\phi$ acts trivially on $Y$} if the diagram
\begin{center}
\begin{tikzcd}
Y \arrow[r, hookrightarrow] \arrow[d, equals] & X \arrow[d, "\phi"] \\
Y \arrow[r, hookrightarrow] & X
\end{tikzcd}
\end{center}
is Cartesian.

\begin{remark}
In $\Set$, an isomorphism is precisely a morphism that is both monic (i.e., a monomorphism) and epic (i.e., an epimorphism). This is not the case for general categories. Categories for which this does hold are called \emph{balanced} categories. Other examples of balanced categories include all abelian categories and all topoi.
\end{remark}

With this in hand, let $\Phi: \mathbf{G}\to\CC$ be a group action with $\Phi(\bullet)=X$ and $\Phi(g)=\phi_g\in\Aut_{\CC}(X)$ for each $g\in G$. Let $Y\inj X$ be a representative for a subobject of $X$. We say that $G$ \textbf{acts on $Y$} if every $\phi_g$ acts on $Y$. Similarly, we say that $G$ \textbf{acts trivially on $Y$} if every $\phi_g$ acts trivially on $Y$. These notions do not depend on our choice of representing monomorphism $Y\inj X$. Moreover, both the subobjects of $X$ on which $G$ acts and on which $G$ acts trivially form posets and so we can apply Zorn's Lemma to get maximal subobjects.\footnote{This only works if we have genuine sets and not proper classes. I'm not sure that we don't get proper classes in general (there's a double-negative for ya!).} This is where the subobject $X^G$ of \textbf{$G$-invariants} comes from. We can also use this formalism to define orbits -- namely, an \textbf{orbit} of the action of $G$ on $X$ is a subobject $\O$ of $X$ on which $G$ acts such that $G$ does not act on any proper subobject of $\O$. In other words, $\O$ is a minimal object in the poset category of subobjects of $X$ on which $G$ acts. Building off of this, a \textbf{fixed point} of the action is an orbit on which $G$ acts trivially or, equivalently, a minimal object in the poset category of subobjects of $X$ on which $G$ acts trivially. Dual to the above we may also consider \textbf{quotient objects} of $X\in\CC$, defined to be isomorphism classes of epimorphisms from $X$. This allows us to make sense of the quotient object $X_G$ of \textbf{$G$-coinvariants} as the maximal quotient object of $X$ on which $G$ acts trivially. Of course, the key comes from thinking about pushout squares
\begin{center}
\begin{tikzcd}
X \arrow[r, twoheadrightarrow] \arrow[d, "\phi"'] & Y \arrow[d] \\
X \arrow[r] & X\coprod_{X,\phi}Y
\end{tikzcd}
\end{center}
for $\phi\in\Aut_{\CC}(X)$ and $X\surj Y$ representing a quotient object of $X$.

\begin{remark}
These may not be the right definitions morally speaking, especially in general. There is a very general construction called the \emph{Grothendieck construction} that, at least when working with concrete categories, takes a concrete category and ``unpacks'' it to make categorical sense of elements. It seems that there should be some kind of ``generalized forgetful functor'' taking categories equipped with $G$-action to their underlying categories, which has a ``generalized right adjoint'' encoded by $G$-invariants and ``generalized left adjoint'' encoded by $G$-coinvariants.
\end{remark}

\begin{exercise}
Convince yourself that subobjects and quotient objects are as expected in $\Mod_A$ for any $A\in\CRing$.
\end{exercise}

\begin{exercise}
Use this formalism to make sense of what it means for a group action $\Phi: \mathbf{G}\to\CC$ to have each of the following properties.
\begin{itemize}
\item Effective -- only $e\in G$ has completely trivial action

\item Faithful -- if $g,h\in G$ act the same then $g=h$

\item Free -- only $e\in G$ acts with fixed points

\item Transitive -- $G$ acts with a single orbit
\end{itemize}
\end{exercise}

Let's briefly return to the simple setting of a group $G$ acting on a set $X$. To this we may associate the so-called \textbf{shear map}
$$G\times X\to X\times X,\qquad (g,x)\mapsto(g\cdot x,x).$$
By definition, the action of $G$ on $X$ is free (resp., transitive) if and only if the associated shear map is injective (resp., surjective). As such, this is definitely a much more familiar way of thinking about properties of group actions. The key here (i.e., what allows us to make sense of the shear map) is that $G$ is a group object in $\Set$ (recall that $\Grp(\Set)\simeq\Grp$). Given a category $\CC$ with finite products and a group object $G\in\Grp(\CC)$, we may make sense of $G$ acting on some $X\in\CC$ by giving a morphism $\phi: G\times X\to X$ with the expected properties. We can then define the shear map $G\times X\to X\times X$ in $\CC$ as above and define our action to be free (resp., transitive) if and only if the associated shear map is monic (resp., epic). Taking things one step further, we can define a \textbf{categorical $G$-quotient} of $X$ to be the data of a map $q: X\to Y$ in $\CC$ that is $G$-invariant in the sense that
\begin{center}
\begin{tikzcd}
G\times X \arrow[r, shift left, "\phi"] \arrow[r, shift right, "\pr_2"'] & X \arrow[r, "q"] & Y
\end{tikzcd}
\end{center}
commutes and, moreover, $q: X\to Y$ is initial with respect to this property in $\CC_{X/}$. In particular, the data of $q: X\to Y$ is a coequalizer. As is always the case for universal properties this characterizes $q: X\to Y$ up to unique isomorphism. It is common practice to write $X/G$ instead of $Y$ and omit mention of the quotient map $q$. To all of this we may associate the category $G(\CC)$ of objects in $\CC$ equipped with an action by $G$.

\begin{exercise}
Is the map $q: X\to X/G$ an epimorphism?
\end{exercise}

\begin{exercise}
Look up geometric quotients and GIT quotients to start getting the geometric applications of this circulating in your head.
\end{exercise}

How do we bridge these two perspectives? As per usual the key comes from thinking about representability. Given $G\in\Grp$, can we view $G$ as a group object of $\CC$? Not in general but it is sometimes possible assuming $\CC$ is concrete. By definition, under this assumption we have a forgetful functor $\oblv: \CC\to\Set$. This induces a functor $\Pre(\Set,\Grp)\to\Pre(\CC,\Grp)$. This matters because $G$ is equivalent to its Yoneda image $h^G=\Hom_{\Grp}(\cdot,G)\in\Pre_{\rep}(\Grp,\Set)$ and, by the same argument, we have an embedding $\Grp(\CC)\xto{\sim}\Pre_{\rep}(\CC,\Grp)\inj\Pre(\CC,\Grp)$ induced by Yoneda. If $\Pre(\Set,\Grp)\to\Pre(\CC,\Grp)$ factors through $\Pre_{\rep}(\CC,\Grp)$ then we can apply it to $h^G$ and pullback by Yoneda to get $\twid{G}\in\Grp(\CC)$ which encodes the same data as $G$.

\begin{exercise}
Show that both approaches to group actions encode the same data in this situation. In particular, show that there is an equivalence of categories $\twid{G}(\CC)\simeq\Fun(\mathbf{G},\CC)$.
\end{exercise}

\begin{exercise}
Is there a relationship between $X_G$ and $X/\twid{G}$?
\end{exercise}

Part of the advantage of working with $\Grp(\CC)$ is that there is an obvious forgetful functor $\oblv: G(\CC)\to\CC$. 

\begin{exercise}
Does $\oblv: G(\CC)\to\CC$ admit a left adjoint? A right adjoint? Do the functors giving $G$-invariants and $G$-coinvariants have any role to play in this?
\end{exercise}

Let's put all of this to use in a situation that will turn out to be quite important to our understanding of vector bundles. Consider the forgetful functor $\oblv: \CAlg_A\to\Mod_A$. This functor preserves limits (as the reader can readily check) and so it is natural to wonder if it admits a left adjoint. This is where symmetric algebras enter the picture. Given $M\in\Mod_A$, recall that its tensor algebra is 
$$\T(M):=\bigoplus_{n\geq0}M^{\tensor n}\in\Alg_A.$$
Each $M^{\tensor n}$ admits a permutation action by $S_n$ and we can take the $S_n$-coinvariants 
$$\Sym_A^n(M):=(M^{\tensor n})_{S_n}.$$
To give an explicit example, $\Sym_A^2(M)$ is the quotient of $M^{\tensor2}$ by the two-sided ideal generated by the relations $m_1\tensor m_2-m_2\tensor m_1$ for $m_1,m_2\in M$. Amalgamating all of these two-sided ideals together yields a homogeneous two-sided ideal $I\normal\T(M)$ and taking the quotient $\T(M)/I$ gives us the commutative $A$-algebra $\Sym_A(M)$, which can also be described explicitly as $\bigoplus_{n\geq0}\Sym_A^n(M)$.

\begin{exercise}
Show that this induces a functor $\Sym_A: \Mod_A\to\CAlg_A$ which is left adjoint to $\oblv: \CAlg_A\to\Mod_A$.
\end{exercise}

\begin{exercise}
Let $G\in\Grp$ acting on $M\in\Mod_A$. Show that $M_G$ is isomorphic as an $A$-module to the quotient of $M$ by the $A$-submodule generated by the relations $g\cdot m-m$ for $g\in G$ and $m\in M$.
\end{exercise}

\begin{exercise}
Explicitly construct an action $\mathbf{S_n}\to\Mod_A$ with $\bullet\mapsto M^{\tensor n}$ such that $(M^{\tensor n})_{S_n}\iso\Sym_A^n(M)$.
\end{exercise}

\begin{exercise}
Given $X\in\Sch$, adapt the procedure of the previous exercise to define the functor $\Sym_{\O_X}: \Mod_{\O_X}\to\CAlg(\O_X)$.\footnote{One way to go about this is to first work with presheaves and use the universal property of the tensor product. Then, sheafify and look at stalks.} Show moreover that $\Sym_{\O_X}$ is left adjoint to $\oblv: \CAlg_{\O_X}\to\Mod_{\O_X}$ and naturally maps quasicoherent modules to quasicoherent algebras.
\end{exercise}

Part of the magic of working with the category $\Space$ of spaces is that many constructions are easy to perform. This is highly relevant in this setting since we have $\Grp(\Space)\simeq\Fun(\CRing,\Grp)$. Moreover, the data of a group action in $\Space$ is just as easy to write down.

\textcolor{red}{TO DO: Given a general group object $G\in\Grp(\CC)$, make sense of $\mathbf{G}$ using the Grothendieck construction and show that $G(\CC)\simeq\Fun(\mathbf{G},\CC)$. Specialize all of this to the settings of spaces, schemes, quasicoherent sheaves, and $\O_X$-modules.}

Its easy to see that $\GG_m$ is a group space. Recall that $\A^n\setminus0$ is described as a space via 
$$(\A^n\setminus0)(A)\iso\left\{(a_1,\ldots,a_n)\in A^n : \sum_{i=1}^na_ix_i=1\textrm{ has a solution}\right\}.$$
From this characterization it is clear that $\GG_m$ induces an action $\phi$ on $\A^n\setminus0$ via
$$\GG_m(A)\times(\A^n\setminus0)(A)\to(\A^n\setminus0)(A),\qquad (\lambda,(a_1,\ldots,a_n))\mapsto(\lambda a_1,\ldots,\lambda a_n).$$
Since $\Space$ is cocomplete we can construct the categorical $\GG_m$-quotient of $\A^n\setminus0$ encoded by $\phi$. 

\begin{exercise}
Describe the quotient space $(\A^n\setminus0)/\GG_m$.
\end{exercise}

\begin{exercise}
Investigate the categories $\GG_m(\Space)$ and $\GG_a(\Space)$.
\end{exercise}

\subsection{Gradings}
Intuitively, an algebraic structure is graded if it has an action by $\Z$ that decomposes it into more manageable chunks. Let's make this more precise. The data of a (commutative) \textbf{($\Z$-)graded ring}\footnote{This should not be confused with the notion of graded-commutative ring.} is the data of $A\in\CRing$ and a decomposition $A=\bigoplus_{n\in\Z}A_n$ in $\Ab$ such that $A_mA_n\subset A_{m+n}$ under multiplication in $A$. We immediately conclude that $A_0\in\CRing$, $A_n\in\Mod_{A_0}$ for every $n\in\Z$, and $A\in\CAlg_{A_0}$. These rings form a category $\CRing^{\gr}$ with morphisms given by \textbf{graded ring maps}, which are ring maps $\phi: A\to B$ such that $\phi(A_n)\subset B_n$ for every $n\in\Z$ (note that the induced map $\phi_n: A_n\to B_n$ equips $B_n$ with the structure of an $A_0$-module). 

Given $A\in\CRing^{\gr}$, the data of a (left) \textbf{($\Z$-)graded $A$-module} is the data of $M\in\Mod_A$ and a collection of abelian groups $\{M_n\}_{n\in\Z}$ such that $M=\bigoplus_{n\in\Z}M_n$ as abelian groups and $A_mM_n\subset M_{m+n}$ under the scalar action of $A$ on $M$. We immediately conclude that $M_n\in\Mod_{A_0}$ for every $n\in\Z$. These modules form a category $\Mod_A^{\gr}$ with morphisms given by \textbf{graded $A$-module maps}, which are $A$-module maps $f: M\to N$ such that $f(M_n)\subset N_n$ for every $n\in\Z$ (note that the induced map $f_n: M_n\to N_n$ is automatically a map of $A_0$-modules). 

Note that any graded ring is naturally a graded module over itself. We refer to elements of a graded module contained in a single graded piece as \textbf{homogeneous} elements. We say that $M=\bigoplus_{n\in\Z}M_n\in\Mod_A^{\gr}$ is \textbf{concentrated} or \textbf{supported} in degrees $[a,b]$ (with $-\infty\leq a\leq b\leq\infty$) if $M_n=0$ for $n\not\in[a,b]$. Similarly, we say that $M$ is \textbf{generated} in degrees $[a,b]$ if it is generated by homogeneous elements with degrees in $[a,b]$.

\begin{remark}
Note that we can define commutative graded rings with respect to any abelian semigroup, and graded modules with respect to the ``groupification'' of said semigroup.
\end{remark}

In practice it is common to consider mainly $\Z^{\geq0}$-graded rings rather than general $\Z$-graded rings. We can make any $\Z^{\geq0}$-graded ring into a $\Z$-graded ring by taking the negatively graded pieces to be zero. In other words, $\Z^{\geq0}$-graded rings are exactly $\Z$-graded rings supported in degrees $[0,\infty)$.

\begin{example}
The main reason in applications why we usually restrict attention to $\Z^{\geq0}$-graded rings comes from the main (nontrivial) example of a graded ring. Namely, consider the polynomial ring $R:=A[t_1,\ldots,t_r]$ for any $A\in\CRing$ and let $R_n:=\{f\in R : \deg f=n\}$. This makes $R$ into a graded ring generated in degree $1$.
\end{example}

\textbf{\un{Question}:} How much commutative algebra can we transfer to the graded setting?

There is an obvious forgetful functor $\oblv: \CRing^{\gr}\to\CRing$ which has a left adjoint $\CRing\to\CRing^{\gr}$ given by considering any ring as a graded ring concentrated in degree $0$. 

\begin{exercise}
Given $A,B\in\CRing^{\gr}$, how does $\Hom_{\CRing^{\gr}}(A,B)$ compare with $\Hom_{\CRing^{\gr}}(A_0,B)$, thinking of $A_0$ as a graded ring concentrated in degree zero?
\end{exercise}

Standard adjunction properties show that $\oblv$ takes limits in $\CRing^{\gr}$ to limits in $\CRing$ while the degree $0$ concentration functor takes colimits in $\CRing$ to colimits in $\CRing^{\gr}$ (this latter result is good news for geometric applications). A similar story relates $\Mod_A^{\gr}$ and $\Mod_A$. Part of what this means is that, in order to compute a limit in $\Mod_A^{\gr}$, we compute the corresponding limit in $\Mod_A$ and then attempt to attach a (natural) grading. For example, the kernel of any map of graded modules is canonically graded and, in particular, the kernel of any map of graded rings is canonically graded (note that $0$ is naturally a graded ring). In fact, given $f: M\to N$ a map of graded $A$-modules we have $(\ker f)_n=\ker(f_n: M_n\to N_n)$. This submodule is an example of a very special submodule of $M$, called a \textbf{homogeneous} submodule (defined as you would expect).

\begin{exercise}
Show that a submodule of a graded module is homogeneous if and only if it is generated by homogeneous elements.
\end{exercise}

Applying this to a graded ring itself gives the notion of a homogeneous ideal. 

\begin{exercise}
Let $A=\bigoplus_{n\in\Z}A_n$ be a graded ring.
\begin{enum}{\alph}
\item Let $N\subset M$ be an inclusion of graded $A$-modules (so $N$ is homogeneous). Show that $M/N$ is naturally a graded $A$-module.

\item Let $I\normal A$ be a homogeneous ideal. Show that $A/I$ is naturally a graded ring.
\end{enum}
\end{exercise}

\begin{remark}
Let $f\in\Hom_{\Mod_A^{\gr}}(M,N)$. We've already seen that $\ker f\in\Mod_A^{\gr}$. It's easy to see that $\im f$ and thus $\coker f$ are graded $A$-modules as well. In the classical setting we know that every submodule is obtained as the kernel of its associated quotient map. This transfers over to the graded setting.
\end{remark}

\begin{exercise}
Let $A,B\in\CRing^{\gr}$ and $M,N\in\Mod_A^{\gr}$.
\begin{enum}{\alph}
\item Show that the forgetful functor induces a natural map $\Isom_{\Mod_A^{\gr}}(M,N)\to\Isom_{\Mod_A}(M,N)$.

\item Show that the natural map $\Isom_{\Mod_A^{\gr}}(M,N)\to\Isom_{\Mod_A}(M,N)$ is injective.

\item Show that the natural map $\Isom_{\CRing^{\gr}}(A,B)\to\Isom_{\CRing}(A,B)$ is injective.
\end{enum}
\end{exercise}

Two key processes that we would like to make sense of in the graded world are tensor products and localization. We will have cause to work with (commutative) graded $A$-algebras given $A\in\CRing^{\gr}$, which span a category $\CAlg_A^{\gr}$ defined analogously to above.

\begin{example}
Let $A=\bigoplus_{n\in\Z}A_n\in\CRing^{\gr}$. Regarding $A_0$ as a graded ring, we naturally have $A\in\CAlg_{A_0}^{\gr}$. At the same time, $A$ is naturally a commutative graded algebra over itself -- in fact, $A$ is the initial object in $\CAlg_A^{\gr}$. More simply, every commutative graded ring is naturally a commutative graded $\Z$-algebra.
\end{example}

With that said, fix a graded ring $A$. Given $M,N\in\Mod_A^{\gr}$, we want to make sense of the ``graded tensor product'' $M\tensor N\in\Mod_A^{\gr}$. This operator $\tensor$ should equip $\Mod_A^{\gr}$ with a strict symmetric monoidal structure. For inspiration we see that the tensor product $M\tensor_{A_0}N\in\Mod_{A_0}$ is naturally \emph{$\Z$-bigraded} since 
\begin{align*}
M\tensor_{A_0}N
\iso\bigoplus_{i\in\Z}M_i\tensor_{A_0}\bigoplus_{j\in\Z}N_j
\iso\bigoplus_{i,j\in\Z}M_i\tensor_{A_0}N_j
\end{align*}
as $A_0$-modules.\footnote{Beware that there are various bigrading conventions based on how we choose to simultaneously account for ``horizontal'' and ``vertical'' information.} We can make things $\Z$-graded by considering $\bigoplus_{i+j=n}M_i\tensor_{A_0}N_j$ for every $n\in\Z$. Let's call this $(M\tensor N)_n$. 

\begin{exercise}
Show that the graded abelian group $M\tensor N:=\bigoplus_{n\in\Z}(M\tensor N)_n$ is naturally a graded $A$-module and that we obtain a bifunctor $\tensor: \Mod_A^{\gr}\times\Mod_A^{\gr}\to\Mod_A^{\gr}$ equipping $\Mod_A^{\gr}$ with a strict symmetric monoidal structure.
\end{exercise}

\begin{exercise}
Show that applying the forgetful functor to $M\tensor N$ recovers the $A$-module $M\tensor_AN$.\footnote{For extra clarity one could write $\tensor_A^{\gr}$ instead of just $\tensor$.} How is $M\tensor N$ related to the $A_0$-module $M\tensor_{A_0}N$? What about $M\tensor_{\Z}N$?
\end{exercise}

\begin{exercise}
Let $M\in\Mod_A^{\gr}$ and $I\normal A$ a homogeneous ideal. We always have a canonical isomorphism of $A$-modules $A/I\tensor_AM\iso M/IM$. Show the graded analogue -- i.e., show that there is a canonical isomorphism $A/I\tensor M\iso M/IM$ of graded $A$-modules compatible with the previous isomorphism under the forgetful functor.
\end{exercise}

\begin{exercise}
Does $\Mod_A^{\gr}$ admit an internal $\Hom$? How about $\CAlg_A^{\gr}$?
\end{exercise}

\begin{exercise}
\hfill
\begin{enum}{\alph}
\item Show that $\tensor_A^{\gr}$ is an endofunctor on $\CAlg_A^{\gr}$.

\item Given $B\in\CAlg_A^{\gr}$, show that $B$ induces base change functors $\Mod_A^{\gr}\to\Mod_B^{\gr}$ and $\CAlg_A^{\gr}\to\CAlg_B^{\gr}$.
\end{enum}
\end{exercise}

What about localization? Viewing any $M\in\Mod_A^{\gr}$ simply as an $A$-module (so forgetting the grading), we can of course make sense of the localization $M_f$ for $f\in A$ \emph{as an $A$-module}. If we want to have any chance of describing this as a graded module then we definitely need $f$ to be homogeneous -- $f\in A_n$ for some $n\in\Z$. In $\Mod_A$ the localization $M_f$ is given by 
$$M_f\iso\colim(M\xto{f}M\xto{f}M\xto{f}\cdots).$$
We can't immediately port this description to the graded setting since the multiplication map $f: M\to M$ is not a graded homomorphism as we have defined it. Instead, we have $f: M_i\to M_{i+n}$ for every $i\in\Z$. The way around this is to consider shifts. Given $m\in\Z$, define the shift endofunctor $(m): \Mod_A^{\gr}\to\Mod_A^{\gr}$ via $N(m)_i:=N_{i+m}$. This functor is evidently an equivalence of categories, with inverse functor $(-m)$, and more generally we have $(m)\circ(m')=(m+m')$.\footnote{More appropriately this should be written as $(m)\circ(m')=(m'+m)$ considering order of operations.} Note that $N$ and $N(m)$ are the same as $A$-modules. It follows that $f: M\to M(n)$ and 
\begin{center}
\begin{tikzcd}
M \arrow[r, "f"] & M(n) \arrow[r, "f"] & M(2n) \arrow[r, "f"] & \cdots
\end{tikzcd}
\end{center}
is a sequence in $\Mod_A^{\gr}$.

\begin{exercise}
Show that the colimit of this sequence exists -- i.e., we can naturally place a graded $A$-module structure on $M_f$. Show moreover that $M_f$ satisfies the expected universal property.
\end{exercise}

\begin{remark}
At the same time, we can consider the modified category $\Mod_A^{\gr,n}$ whose objects are the same as $\Mod_A^{\gr}$ but the morphisms satisfy $f(M_i)\subset N_{i+n}$ instead of $f(M_i)\subset N_i$. The localization $M_f$ then corresponds to the colimit of the sequence above (with no shifts) taken in $\Mod_A^{\gr,n}$. One major disadvantage of this approach is that it can only really handle localizing at a single homogeneous element of $A$ instead of an arbitrary multiplicative subset of homogeneous elements.
\end{remark}

\begin{exercise}
Let $M\in\Mod_A^{\gr}$ and $S\subset A$ a multiplicative subset of homogeneous elements.
\begin{enum}{\alph}
\item Show that the collection of $M_f$ for $f\in S$ naturally forms a filtered system.

\item Show that $\fcolim_{f\in S}M_f$ exists and satisfies the universal property to be the localization $S^{-1}M$. Show moreover that $S^{-1}M$ is naturally a graded $S^{-1}A$-module.

\item Show that $S^{-1}A\tensor M\iso S^{-1}M$.
\end{enum}
\end{exercise}

Let $A:=A_0[t_1,\ldots,t_n]$, viewed as a commutative $\Z^{\geq0}$-graded $A_0$-algebra with the polynomial grading mentioned earlier. Our goal is to understand $\Mod_A^{\gr}$. To do this we first need to understand $\Mod_A$, starting with the case $n=1$. In this case the data of an $A$-module is the data of an $A_0$-module equipped with an $A_0$-module endomorphism (specified by the action of $t_1$). In the case $n=2$ the data of an $A$-module certainly includes the data of an $A_0$-module equipped with a pair of $A_0$-module endomorphisms. The only relation that $t_1,t_2$ satisfy is $t_1t_2=t_2t_1$ and so we need to specify that our pair of endomorphisms commutes. Extrapolating from this to the case of general $n$, the data of an $A$-module is the data of an $A_0$-module together with a collection of $n$ pairwise commuting $A_0$-module endomorphisms. 

What about the gradings? From the above we see that the data of a graded $A$-module is the data of an $A_0$-module $M$ with commuting endomorphisms $\phi_1,\ldots,\phi_n\in\End_{A_0}(M)$ and decomposition $M\iso\bigoplus_{j\in\Z}M_j$ such that $\phi_i(M_j)\subset M_{j+1}$ for $1\leq i\leq n$ and $j\in\Z$. Let $f\in A$ be nonzero homogeneous of degree $n$. Then, 
$$M_{(f)}:=(M_f)_0\iso\{f^{-i}m : m\in M_{in}\},$$
which is naturally an $A_{(f)}$-module. The action of general $f$ can be difficult to describe, but it is related to $\phi_1\ldots,\phi_n$. In particular, localizing $M$ at $t_i$ corresponds to ``formally inverting'' the action of $\phi_i$.\footnote{Which, of course, will only yield something nonzero if $\phi_i$ is injective.}

\subsection{The Proj Construction}
The following result explains why we care so much about graded rings.

\begin{theorem}
The equivalence $\Gamma: \Aff\Sch^{\op}\to\CRing$ restricts to an equivalence from the opposite of $\GG_m(\Aff\Sch)$ (i.e., affine schemes equipped with a (left) $\GG_m$-action) to $\CRing^{\gr}$.
\end{theorem}

Let's begin by unpacking the structure of an object $X\in\GG_m(\Aff\Sch)$. 

\begin{exercise}
Since $\GG_m\iso\Spec\Z[t^{\pm1}]$, the ring $\Z[t^{\pm1}]$ naturally carries the structure of a Hopf $\Z$-algebra. Show that this structure is encoded by 
\begin{align*}
\eps: \Z[t^{\pm1}]\to\Z,\qquad & t\mapsto1, \\
\mu: \Z[t^{\pm1}]\to\Z[x^{\pm1},y^{\pm1}],\qquad & t\mapsto xy, \\
\iota: \Z[t^{\pm1}]\to\Z[t^{\pm1}],\qquad & t\mapsto t^{-1}.
\end{align*}
Note that we have implicitly made the identification $\Z[t^{\pm1}]\tensor_{\Z}\Z[t^{\pm1}]\iso\Z[x^{\pm1},y^{\pm1}]$ via $t\tensor1\mapsto x$ and $1\tensor t\mapsto y$.
\end{exercise}

Write $X=\Spec A$ and let $\phi: \GG_m\times X\to X$ be the action morphism, which corresponds to a ring map $\psi: A\to\Z[t^{\pm1}]\tensor_{\Z}A\iso A[t^{\pm1}]$. Note that the projection $\pr_2: \GG_m\times X\to X$ corresponds to the inclusion $A\inj A[t^{\pm1}]$. The requirement that $\phi$ be an action translates into the commutative diagrams
\begin{center}
\begin{tikzcd}
\GG_m\times X \arrow[rr, "e_{\GG_m}\times\id_X"] \arrow[rd, "\pr_2"'] & & \GG_m\times X \arrow[ld, "\phi"] \\
& X &
\end{tikzcd}
\end{center}
and
\begin{center}
\begin{tikzcd}
\GG_m\times(\GG_m\times X) \arrow[rr, "\sim"] \arrow[rd, "\id_{\GG_m}\times\phi"'] & & (\GG_m\times\GG_m)\times X \arrow[ld, "m\times\id_X"] \\
& G\times X \arrow[d, "\phi"] & \\
& X &
\end{tikzcd}
\end{center}
where $e_{\GG_m}$ is the composition $\GG_m\to\Spec\Z\xto{e}\GG_m$ (note that $\Spec\Z$ is the terminal object in $\Aff\Sch$). Translating to the world of rings, the composition $\eps_A\circ\psi$ must be the inclusion $A\inj A[t^{\pm1}]$ and the diagram 
\begin{center}
\begin{tikzcd}
A \arrow[r, "\psi"] & A{[}t^{\pm1}{]} \arrow[r, shift left, "\mu_A"] \arrow[r, shift right] & A{[}x^{\pm1},y^{\pm1}{]}
\end{tikzcd}
\end{center}
must commute, where the unlabeled bottom arrow sends $t$ to $y$ and $a\in A$ to $\psi(a)$ (written in terms of $x$ rather than $t$).

\begin{exercise}
Let $A=\bigoplus_{n\in\Z}A_n\in\CRing^{\gr}$. Show that the ring homomorphism $\psi: A\to A[t^{\pm1}]$ determined by sending homogeneous $a\in A_n$ to $at^n$ defines a natural action of $\GG_m$ on $\Spec A$.
\end{exercise}

$\Proj X=X^+/\G_m$

We should have a $\GG_m$-invariant open subscheme $X^+\inj X$ (maximal in an appropriate sense), that recovers $\A^n\setminus0$ if $X=\A^n$.

What does all of this have to do with quotient and Hilbert schemes? What about Grothendieck's philosophy that we should study quotients rather than subs? What does it all mean?

\subsection{Graded Spaces and Schemes}
One of the advantages of graded rings is that they allow us to adapt the basic theory of (Zariski) schemes. Recall that spaces are simply functors from $\CRing$ to $\Set$. We can adapt this to consider instead the category $\Space^{\gr}:=\Fun(\CRing^{\gr},\Set)$ of \textbf{graded spaces}. Given $A\in\CRing^{\gr}$, define $\Proj A:=\Hom_{\CRing^{\gr}}(A,\cdot)\in\Space^{\gr}$. These objects span a full subcategory $\Aff\Sch^{\gr}\subset\Space^{\gr}$ of \textbf{affine graded schemes}. From this we obtain the functor $\Proj: (\CRing^{\gr})^{\op}\to\Aff\Sch^{\gr}$, which is an equivalence of categories by construction. The forgetful functor $\oblv: \CRing^{\gr}\to\CRing$ induces a functor $\oblv^*: \Space\to\Space^{\gr}$ via precomposition. At the same time, the left adjoint functor $\ms{G}: \CRing\to\CRing^{\gr}$ induces a functor $\ms{G}^*: \Space^{\gr}\to\Space$ via precomposition. In fact, the adjunction $\ms{G}\dashv\oblv$ induces an adjunction $\oblv^*\dashv\ms{G}^*$.\footnote{This is just the result of general abstract nonsense.} This restricts to an adjunction between $\Aff\Sch$ and $\Aff\Sch^{\gr}$. 

Our goal is to make sense of graded schemes and show that affine graded schemes are themselves graded schemes. In order to do this we first need to carefully consider the above collection of functors. 

\begin{exercise}
Show that $\ms{G}: \CRing\to\CRing^{\gr}$ is fully faithful and reflects isomorphisms. Show that these results remain true if we instead consider modules or commutative algebras (assuming that the base ring $A$ is itself concentrated in degree zero).
\end{exercise}

Taking things in the other direction loses a lot more information. 

\begin{example}
Consider the polynomial ring $\Z[t]$. We can view this a graded ring concentrated in degree zero or use the polynomial grading generated in degree one. Both gradings are definitely not equivalent.
\end{example}

The discrepancy in information becomes even more apparent when looking at the shift endofunctor $(n)$ for $n\in\Z$, which acts on $\Mod_A^{\gr}$.

\begin{remark}
The shift $(n)$ is not an endofunctor on $\CAlg_A^{\gr}$ or even on $\CRing^{\gr}$, even though we may naturally view $A$ as graded $A$-module. The reason is simply that $A(n)_iA(n)_j=A_{i+n}A_{j+n}\subset A_{i+j+2n}$ while $A(n)_{i+j}=A_{i+j+n}$.
\end{remark}

\begin{exercise}
Let $M,N\in\Mod_A^{\gr}$.
\begin{enum}{\alph}
\item Fix $n\in\Z$. Show that there is a canonical isomorphism $M\iso M(n)$ in $\Mod_A$. Show moreover that $M$ and $M(n)$ need not be isomorphic in $\Mod_A^{\gr}$.

\item Suppose that $M$ and $N$ are isomorphic in $\Mod_A$. Is it necessarily true that $M\iso N(n)$ in $\Mod_A^{\gr}$ for some $n\in\Z$?
\end{enum}
\end{exercise}

\begin{remark}
We evidently have a family of endofunctors $(n)$ on $\Mod_A^{\gr}$ for $n\in\Z$. Each of these is an auto-equivalence, indicating that there is some kind of generalized action of $\Z$ on $\Mod_A^{\gr}$ itself. Once again stacks are lurking in the background.
\end{remark}

$D_+(f):=\Spec(A_f)_0$.

$\P^n=\P_{\Z}^n:=\Proj\Z[t_0,\ldots,t_n]$

\begin{theorem}
Affine graded schemes are graded schemes.
\end{theorem}

\begin{theorem}
Graded schemes are schemes.
\end{theorem}

Fixing $f\in A$ homogeneous, the operation that first localizes at $f$ and then takes the degree $0$ piece defines functors $\Mod_A^{\gr}\to\Mod_A$ and $\CAlg_A^{\gr}\to\CAlg_A$. If $f$ is nilpotent then both of these functors vanish identically (i.e., they send everything to zero) and so we typically only think about $f$ non-nilpotent. Suppose now that $A$ is $\Z^{\geq0}$-graded and $M\in\Mod_A^{\gr}$. The abelian group $M_+:=\bigoplus_{n\geq0}M_n$ is a graded $A$-submodule of $M$. Applying this same process to $A$ yields the homogeneous \textbf{irrelevant ideal} $A_+\normal A$. Note that general $\Z$-graded things are still necessary to consider in this case since if $f$ has positive degree then $M_f$ necessarily contains nonzero homogeneous elements of negative degree (again assuming $f$ is non-nilpotent).

\textcolor{red}{Most of what's written in this section is nonsense. This is not how we want to do the Proj construction. It is interesting to ask ``how much'' data certain graded rings encode about certain schemes.}

\section{Faux Topology}
Let $X\in\Sch$. We say $X$ is \textbf{topologically Noetherian} if every descending chain $Z_0\supset Z_1\supset\cdots$ of closed subschemes $Z_i\subset X$ stabilizes.

\begin{exercise}
Suppose $X=\Spec A$ is affine. Show that $X$ is topologically Noetherian if and only if $A$ is Noetherian.
\end{exercise}

Going off of this, we say $X$ is \textbf{locally Noetherian} if there exists an affine open covering of $X$ by $\Spec A$ with $A$ Noetherian. We say $X$ is \textbf{Noetherian} if it is locally Noetherian and qc.

\begin{exercise}
Show that general $X\in\Sch$ is Noetherian if and only if it is topologically Noetherian.
\end{exercise}

\begin{exercise}
Show that open subschemes of locally Noetherian (resp., Noetherian) schemes are themselves locally Noetherian (resp., Noetherian).
\end{exercise}

\begin{exercise}
Give an example of a locally Noetherian scheme that is not Noetherian. Give an example of a Noetherian scheme that is not affine.
\end{exercise}

We say $X$ is \textbf{connected} if any open covering $\{U,V\}$ of $X$ with $U\cap V=\emptyset$ necessarily satisfies $U=X$ or $V=X$. Suppose $X=\Spec A$ and $\{U,V\}$ is such an open covering. Without loss of generality we may assume $U=D(I),V=D(J)$ for $I,J\normal A$. Then, $U\cap V\iso D(IJ)$ and so $U\cap V=\emptyset$ is equivalent to $IJ=0$. We obtain the following result.

\begin{proposition}
$\Spec A$ is connected if and only if $IJ=0$ implies that $I=0$ or $J=0$ for any pair of ideals $I,J\normal A$.
\end{proposition}

\begin{example}
An abundant source of connected schemes comes from taking $\Spec A$ for $A$ an integral domain.
\end{example}

\begin{exercise}
Give an example of a connected scheme that is not affine.
\end{exercise}

Let $f\in\Hom_{\Sch}(X,Y)$ be qcqs (so that $f_*: \QCoh(X)\to\QCoh(Y)$ is well-defined). We define the \textbf{scheme-theoretic image} of $f$ to be 
$$\ov{f(X)}:=\Spec_Y\O_Y/\ker(\O_Y\to f_*\O_X),$$
which is canonically a closed subscheme of $Y$. 

\begin{exercise}
Suppose that $X=\Spec B$ and $Y=\Spec A$ so that $f$ corresponds to a ring map $\phi: A\to B$. Show that there is a canonical isomorphism $\ov{f(\Spec B)}\iso\Spec A/\ker\phi$.
\end{exercise}

\begin{exercise}
Show that $\ov{f(X)}$ has the universal property that $\ov{f(X)}\inj Y$ is the initial closed embedding into $Y$ through which $f: X\to Y$ factors.
\end{exercise}

This universal property characterizes $\ov{f(X)}$ for arbitrary $f\in\Hom_{\Sch}(X,Y)$. We can still take $\ker(\O_Y\to f_*\O_X)\in\Mod_{\O_Y}$, though this may not be quasicoherent. Fortunately, it is possible to find a maximal quasicoherent $\O_X$-submodule $\J\inj\ker(\O_Y\to f_*\O_X)$ (exercise!\footnote{This is [Stacks, Tag 01QZ].}) and then define $\ov{f(X)}:=\Spec_Y\O_Y/\J$.

\begin{remark}
There is a different notion of the image of $f: X\to Y$ that might at first seem more natural. Let us denote this by $f(X)$ and call it the \textbf{classical image} of $f$. This is the subfunctor $f(X)\subset f$ which sends $C\in\CRing$ to $f(X)(C):=\im(f(C): X(C)\to Y(C))$. If $X=\Spec B$ and $Y=\Spec A$ so that $f$ corresponds to some ring map $\phi: A\to B$ then $f(X)(C)=\{\psi\circ\phi : \psi\in\Hom_{\CRing}(B,C)\}$ -- i.e., it consists of factorizable morphisms. By contrast, $\ov{f(X)}(C)=\Hom_{\CRing}(A/\ker\phi,C)$ as noted earlier. It follows that there is a natural map $f(X)\to\ov{f(X)}$.
\end{remark}

\begin{exercise}
Show that there is a natural map $f(X)\to\ov{f(X)}$ for arbitrary $f\in\Hom_{\Sch}(X,Y)$. Investigate what this map looks like in various cases. What data does the classical image encode?
\end{exercise}

\begin{definition}
We say $f\in\Hom_{\Sch}(X,Y)$ is \textbf{dominant} if $\ov{f(X)}=Y$ or, equivalently, the natural map $\O_Y\to f_*\O_X$ is a monomorphism.
\end{definition}

\begin{example}
Let $A\in\CRing$ be an integral domain and $f\in A$ nonzero. Then, the open embedding $D(f)\inj\Spec A$ is dominant. In particular, $\A^1\setminus0\inj\A^1$ is dominant.
\end{example}

\begin{example}
The natural map $X\to\ov{f(X)}$ induced by $f: X\to Y$ is dominant.
\end{example}

Given an open embedding $i: U\inj X$, it is common to write $\ov{U}$ instead of $\ov{i(U)}$. We say that $U$ is \textbf{dense} in $X$ if $i$ is dominant or, equivalently, $\ov{U}=X$.\footnote{Does anything funny happen if $U$ is not qc?}

\begin{exercise}
Let $U\inj\Spec A$ be an open embedding, so $U\iso D(I)$ for some ideal $I\normal A$.
\begin{enum}{\alph}
\item Show that $\ov{U}\iso\Spec A/\ker(A\to A[I^{-1}])$.

\item Show that $U$ is dense if and only if $I$ contains no NZDs of $A$.

\item Show that $A$ is an integral domain if and only if every nonempty open subspace of $\Spec A$ is dense.
\end{enum}
\end{exercise}

Loc. closed = closed inside open (same as inducing isomorphism with closed subscheme of open subscheme)
Complement of closed is always open but this does not account for all opens

\section{Appendix}
Here is a more complete list of important categories.
\begin{center}
\begin{tabular}{|l|l|l|}
\hline
Category & Objects & Morphisms \\
\hline
$\Ab$ & abelian groups & group homomorphisms \\
$\Aff\Op(X)$ & affine (Zariski) open subspaces of $X$ & over-morphisms \\
$\Aff\Sch$ & affine schemes & functors \\
$\Aff\Sch_S$ & $S$-affine schemes & over-morphisms \\
$\Aff\Sch_{/S}$ & affine schemes over $S$ & over-morphisms \\
$\CAlg_A$ & commutative (associative, left, unital) $A$-algebras & $A$-algebra homomorphisms \\
$\Coh(X)$ & coherent sheaves (on $X$) & $\O_X$-module homomorphisms \\
$\CRing$ & commutative (unital) rings & ring homomorphisms \\
$\Grp$ & groups & group homomorphisms \\
$\Mod_A$ & (left) $A$-modules & $A$-module homomorphisms \\
$\Mod_{\O_X}$ & (left) $\O_X$-modules & $\O_X$-module homomorphisms \\
$\Op(X)$ & (Zariski) open subspaces of $X$ & over-morphisms \\
$\Pre(\CC,\EE)$ & presheaves on $\CC$ valued in $\EE$ & functors \\
$\QCoh(X)$ & quasicoherent sheaves (on $X$) & $\O_X$-module homomorphisms\footnote{This is the perspective that results from viewing $\QCoh(X)$ as a full subcategory of $\Mod_{\O_X}$.} \\
$\Sch$ & schemes & functors \\
$\Sch_S$ & $S$-schemes\footnote{These differ from schemes over $S$ via a fiber product criterion.} & over-morphisms \\
$\Sch_{/S}$ & schemes over $S$ & over-morphisms \\
$\Set$ & sets & functions \\
$\Shv(\CC,\EE)$ & sheaves on $\CC$ valued in $\EE$ & functors \\
$\Space$ & spaces & functors \\
$\Space_S$ & $S$-spaces\footnote{These are the same thing as spaces over $S$.} & over-morphisms \\
\hline
\end{tabular}
\end{center}

Given $R\in\CC$ a ring object (the category of such objects is typically denoted $\CAlg(\CC)$), $\Mod_R(\CC)$ and $\CAlg_R(\CC)$ respectively denote the categories of (left) $R$-modules and commutative (associative, left, unital) $R$-algebras.

List of references consulted so far:
\begin{itemize}
\item \emph{Abstract Algebra} by D. Dummit and R. Foote
\item \emph{Algebra: Chapter 0} by P. Aluffi
\item \emph{Algebraic Geometry and Arithmetic Curves} by Q. Liu
\item \emph{Classical Motivation for the Riemann-Hilbert Correspondence} by B. Conrad
\item \emph{Commutative Algebra -- with a View Toward Algebraic Geometry} by D. Eisenbud
\item \emph{Higher Algebra} by J. Lurie
\item \emph{M392C Notes: Algebraic Geometry} lectures by S. Raskin (with notes taken by A. Debray)
\item \emph{nLab} by various authors (\texttt{https://ncatlab.org/nlab/show/HomePage})
\item \emph{Stacks Project} by various authors (\texttt{https://stacks.math.columbia.edu/})
\item \emph{The Rising Sea: Foundations of Algebraic Geometry} by R. Vakil
\end{itemize}
\end{document}

\section{Zoology of Morphisms - Algebraic Conditions}

We've already discussed some natural ``topological'' conditions to place on schemes and maps between them. Now it's time to bring in some algebraic conditions. Recall from earlier that we can often transfer a property of schemes to a property of maps of schemes via affine pullback. Conversely, we can often transfer a property of maps of schemes to a property of schemes by using the fact that $\Spec\Z$ is terminal in $\Sch$. What are some ways that we can bring algebra into the mix to get properties of schemes?
\begin{enum}{\arabic}
\item Put a property on $X$ itself.

\item Put a property on $\O_X(U)$ for every $U$ in some class of objects in $\Op(X)$, typically $\Aff\Op(X)$ so that $\O_X(U)$ is a ring. Note that this is a weaker version of looking at affine pullbacks since we are instead looking at affine open pullbacks.

\item Put a property on the stalks of $\O_X$.
\end{enum}

Let's now collect a bunch of definitions and facts from commutative algebra that we can mine for geometric content.

Show that $\rho: Y\to X$ is flat if and only if $\rho^*$ is exact.

\section{Differential Forms}
Let $A\in\CRing$, $B\in\CAlg_A$, and $M\in\Mod_B$. Viewing both $B$ and $M$ as $A$-modules, we may define the set $\Der_A(B,M)$ of $A$-linear \textbf{derivations} from $B$ to $M$ to be the set of $\delta\in\Hom_{\Mod_A}(B,M)$ satisfying the \emph{Leibniz rule}:
$$\delta(fg)=g\delta(f)+f\delta(g)\textrm{ for all }f,g\in B.$$

\begin{exercise}
Let $\phi: A\to B$ be the structure map and $\delta\in\Der_A(B,M)$. Show that $\delta$ kills $\phi(A)$ and that $\delta=0$ if $\phi$ is surjective or a localization map.
\end{exercise}

This determines a functor $\Der_A(\cdot,\cdot): \CAlg_A^{\op}\times\Mod_A\to\Set$ which in fact factors through $\Mod_A$ (i.e., we can add derivations and scale them by elements of $A$). This tells us what happens if we change the inputs $B$ and $M$, but what happens if we change $A$ itself? Let $A'\to A$ be a ring map. This equips both $B$ and $M$ with the structure of $A'$-modules. We obtain a map $\Der_A(B,M)\to\Der_{A'}(B,M)$ which is the identity on the level of sets. The latter set $\Der_{A'}(B,M)$ carries no natural $A$-module structure so there is clearly something more going on here.

\begin{exercise}
Let $A\to B\to C$ be a sequence of ring maps and $M\in\Mod_C$. Show that there is a natural exact sequence
\begin{center}
\begin{tikzcd}
0 \arrow[r] & \Der_B(C,M) \arrow[r] & \Der_A(C,M) \arrow[r] & \Der_A(B,M)
\end{tikzcd}
\end{center}
of $A$-modules that is functorial in $M$.
\end{exercise}

Define the \textbf{cotangent module} $\Omega_{B/A}^1\in\Mod_B$ via the universal property that $\Hom_{\Mod_B}(\Omega_{B/A}^1,\cdot)\iso\Der_A(B,\cdot)$.\footnote{This determines $\Omega_{B/A}^1$ up to unique isomorphism by an application of Yoneda's Lemma.} The data of $\Omega_{B/A}^1$ is entirely encoded by $\id\in\Hom_{\Mod_B}(\Omega_{B/A}^1,\Omega_{B/A}^1)$, which corresponds to a derivation $d=d_{B/A}\in\Der_A(B,\Omega_{B/A}^1)$. This derivation is \emph{universal} in the sense that, given any $M\in\Mod_B$ and $\delta\in\Der_A(B,M)$, there is a unique $B$-module map $\Omega_{A/B}^1\to M$ such that the diagram 
\begin{center}
\begin{tikzcd}
B \arrow[r, "\delta"] \arrow[rd, "d"'] & M \\
& \Omega_{B/A}^1 \arrow[u, dotted, "\exists!"]
\end{tikzcd}
\end{center}
commutes. Setting aside the matter of existence for the moment, let's deduce some properties of cotangent modules.

\begin{exercise}
Fix a ring map $\phi: A\to B$. Make use of the universal property of localization for the following problems.
\begin{enum}{\alph}
\item Let $S\subset B$ be a multiplicative set. Show that there is a canonical isomorphism $\Omega_{S^{-1}B/A}^1\iso S^1\Omega_{B/A}^1$ of $S^{-1}B$-modules.

\item Let $S\subset A$ be a multiplicative set such that $\phi(S)\subset B^{\times}$. Show that there is a canonical isomorphism $\Omega_{B/S^{-1}A}^1\iso\Omega_{B/A}^1$ of $B$-modules.
\end{enum}
\end{exercise}

\begin{proposition}
Let $A\to B\to C$ be a sequence of ring maps and $M\in\Mod_C$. Then, there is a canonical isomorphism 
$$\Der_A(B,M)\iso\Hom_{\Mod_C}(C\tensor_B\Omega_{B/A}^1,M)$$
of $A$-modules functorial in $M$.
\end{proposition}

\begin{proof}
We have
\begin{align*}
\Hom_{\Mod_C}(C\tensor_B\Omega_{B/A}^1,M)
&\iso\Hom_{\Mod_B}(C\tensor_B\Omega_{B/A}^1,M) \\
&\iso\Hom_{\Mod_B}(\Omega_{B/A}^1,\Hom_{\Mod_B}(C,M)) \\
&\iso\Der_A(B,\Hom_{\Mod_B}(C,M)).
\end{align*}
The latter module is a subset of $\Hom_{\Mod_A}(B,\Hom_{\Mod_B}(C,M))$, which by tensor-Hom adjunction is isomorphic to $\Hom_{\Mod_A}(C\tensor_BB,M)$. Further identifications give 
$$\Hom_{\Mod_A}(C\tensor_BB,M)\iso\Hom_{\Mod_A}(C,M)\iso\Hom_{\Mod_A}(B,M).$$
One can then show that the image of $\Der_A(B,\Hom_{\Mod_B}(C,M))$ in $\Hom_{\Mod_A}(B,M)$ is precisely $\Der_A(B,M)$. This construction is evidently functorial in $M$.
\end{proof}

\begin{corollary}
Let $A\to B\to C$ be a sequence of ring maps and $M\in\Mod_C$. Then, there is a canonical exact sequence
\begin{center}
\begin{tikzcd}
C\tensor_B\Omega_{B/A} \arrow[r] & \Omega_{C/A} \arrow[r] & \Omega_{C/B} \arrow[r] & 0
\end{tikzcd}
\end{center}
of $C$-modules.
\end{corollary}

\begin{exercise}
Let $B_1,B_2\in\CAlg_A$ and $C:=B_1\tensor_AB_2$ (which we view as a commutative algebra over both $B_1$ and $B_2$ in the usual manner). Show that there is a canonical isomorphism 
$$(C\tensor_{B_1}\Omega_{B_1/A}^1)\oplus(C\tensor_{B_2}\Omega_{B_2/A}^1)\iso\Omega_{C/A}^1$$
of $C$-modules.
\end{exercise}

Let $A\in\CRing$ and $M\in\Mod_A$. We equip the abelian group $A\oplus M$ with a commutative $A$-algebra structure as follows. Define
$$(a,m)\cdot(b,n):=(a+b,an+bm).$$
We call the resulting algebra a \textbf{split square-zero extension}.\footnote{Half of the name comes from the fact that $A\oplus M$ is a split extension of $A$-modules.} We immediately see that the projection $\pr_1: A\oplus M\to A$ is a ring map equipping $A$ with an ($A\oplus M$)-module structure and allowing us to view $A\oplus M$ as an ($A\oplus M$)-module.\footnote{Note that the identity map on $A\oplus M$ need not equip $A\oplus M$ with the structure of a module over itself since coordinate-wise operations need not make $A\oplus M$ into a ring -- indeed, multiplication on $M$ need not even be defined!} If it floats your boat, $A\oplus M$ is naturally an object of $\CAlg_{A//A}$, the category of commutative rings equipped with a ring map both from and to $A$.

\begin{exercise}
Show that the projection $\pr_2: A\oplus M\to M$ is an $A$-linear derivation.
\end{exercise}

\begin{exercise}
Identify $M$ with the subset $\{0\}\times M\subset A\oplus M$. Show that $M$ is an ideal of $A\oplus M$ with $M^2=0$. This explains half of the name.
\end{exercise}

Where does this sort of construction come from?

\begin{example}
Let $k$ be a field and consider the $k$-algebra $k[x]/(x^2)$ called the \textbf{dual numbers} over $k$. This may equivalently be viewed as the $k$-algebra $k[\eps]$ for $\eps$ a formal element such that $\eps^2=0$. As a $k$-vector space we have $k[\eps]\iso k\times k\eps$, with multiplication $(a+m\eps)(b+n\eps)=ab+(an+bm)\eps$. It follows that $k[\eps]\iso k\oplus k\eps$ as $k$-algebras.
\end{example}

We can repeat the above construction given any $B\in\CAlg_A$ to get $B\oplus M\in\CAlg_A$. 

\begin{exercise}
Show that elements of $\Der_A(B,M)$ correspond canonically to $A$-algebra sections of $\pr_1: B\oplus M\to B$ -- i.e., $A$-algebra maps $\sigma: B\to B\oplus M$ such that $\pr_1\circ\sigma=\id_B$. 
\end{exercise}

Just as before we may view $M$ as an ideal of $B\oplus M$ with $M^2=0$. 
\end{document}

Thanks to:
Keerthi Madapusi Pera
Kevin Yeh
Ehsan Shahoseini
Siddharth Mahendraker
Sam Raskin
David Ben-Zvi