\documentclass[11pt]{article}

\usepackage{kernel_of_truth}

\renewcommand{\A}{\mathcal{A}}
\newcommand{\B}{\mathcal{B}}
\newcommand{\E}{\mathcal{E}}
\newcommand{\F}{\mathcal{F}}
\renewcommand{\L}{\mathcal{L}}
\newcommand{\M}{\mathcal{M}}
\renewcommand{\N}{\mathcal{N}}
\renewcommand{\P}{\mathcal{P}}

\newcommand{\Int}[3]{\int_{#1}{#2}\;d{#3}}
\renewcommand{\phi}{\oldphi}

\newcommand{\llim}{\displaystyle\lim}
\newcommand{\lliminf}{\displaystyle\liminf}
\newcommand{\llimsup}{\displaystyle\limsup}

\newcommand{\ssum}{\displaystyle\sum}

\DeclareMathOperator{\esssup}{ess\;sup} % essential supremum
\DeclareMathOperator{\supp}{supp} % support

\begin{document}
\title{Measure Theory Notes}
\author{Zachary Gardner}
\date{}
\maketitle

\tableofcontents
\newpage

\section{Measures}
Throughout this section, unless otherwise stated, $X$ denotes a nonempty set.

\subsection{Algebras}
\begin{definition}
Let $\A\subset\P(X)$. $\A$ is a \textbf{Boolean algebra} if it is closed under finite unions and complements. $\A$ is a \textbf{$\sigma$-algebra} if it is also closed under countable unions or, equivalently, just countable \emph{disjoint} unions. 
\end{definition}

We will often use the symbol $\M$ to notationally distinguish $\sigma$-algebras from Boolean algebras. We immediately deduce that if $\A$ is a Boolean (resp., $\sigma$-) algebra then $\emptyset,X\in\A$ and $\A$ is closed under finite (resp., countable) intersections.

\begin{example}
The collection of countable and co-countable subsets of $X$ forms a $\sigma$-algebra.
\end{example}

Any $\E\subset\P(X)$ generates a minimal Boolean (resp., $\sigma$-) algebra formed by taking the intersection of all Boolean (resp., $\sigma$-) algebras containing $\E$ (note that both collections are nonempty since they contain $\P(X)$). 

\begin{definition}
Let $X$ be a topological space. The \textbf{Borel $\sigma$-algebra} of $X$ is the minimal $\sigma$-algebra $\B_X$ generated by the open subsets of $X$. The elements of $\B_X$ are called \textbf{Borel sets}. Inside of $\B_X$ there are two specially distinguished types of sets: $G_{\delta}$ sets obtained as countable intersections of open sets and $F_{\sigma}$ sets obtained as countable unions of closed sets. 
\end{definition}

\begin{definition}
Let $\{(X_i,\M_i)\}$ be a collection of nonempty sets equipped with $\sigma$-algebras. Let $X:=\prod_iX_i$ with associated projection maps $\pi_i: X\to X_i$. The \textbf{product $\sigma$-algebra} $\M:=\bigotimes_i\M_i$ is the $\sigma$-algebra on $X$ generated by $\{\pi_i^{-1}(E_i) : E_i\in\M_i\}$.
\end{definition}

The following result gives some important properties of product algebras.

\begin{proposition}
Assume the setup of the previous definition and let $I$ be the relevant indexing set. 
\begin{enum}{\alph}
\item Suppose $I$ is countable. Then, $\M$ is generated by $\{\prod_{i\in I}E_i : E_i\in\M_i\}$.

\item Suppose $I$ is countable. Then, $\M$ is generated by $\{\prod_{i\in I}E_i : E_i\in\M_i\}$ only sampling over a generating set of each $\M_i$.

\item Suppose $I$ is finite and each $X_i$ is a metric space. Then,
$$\bigotimes_{i\in I}\B_{X_i}\subset\B_X$$
with equality when each $X_i$ is separable (i.e., has a countable dense subset).
\end{enum}
\end{proposition}

\begin{definition}
A subset $\E\subset\P(X)$ is an \textbf{elementary family} if
\begin{enum}{\arabic}
\item $\emptyset\in\E$;
\item $E,F\in\E\implies E\cap F\in\E$; and
\item $E\in\E\implies E^c$ is a finite disjoint union of members of $\E$.
\end{enum}
\end{definition}

Elementary families are important because they can be used to build Boolean algebras: the subset of $\P(X)$ obtained by taking finite disjoint unions of members of an elementary family $\E$ is a Boolean algebra.

\subsection{Measures}
\begin{definition}
Let $(X,\M)$ be a set equipped with a $\sigma$-algebra $\M$. Let $\mu: \M\to[0,\infty]$ and consider the following potential properties of $\mu$.
\begin{enum}{\arabic}
\item $\mu(\emptyset)=0$.
\item $\mu$ is additive on countable disjoint unions.
\item $\mu$ is additive on finite disjoint unions.
\end{enum}
If $\mu$ satisfies \textup{(1)} and \textup{(2)} then $\mu$ is a \textbf{measure}. If $\mu$ satisfies \textup{(1)} and \textup{(3)} then $\mu$ is a \textbf{finitely additive measure}. If $(X,\M)$ admits a measure $\mu$ then it is called a \textbf{$\mu$-measurable space} or simply \textbf{measurable space}. The triple $(X,\M,\mu)$ is called a \textbf{measure space}. Members of $\M$ are said to be \textbf{$\mu$-measurable} or simply \textbf{measurable}.

Let now $(X,\M,\mu)$ be a measure space. 
\begin{itemize}
\item $E\in\M$ is \textbf{$\mu$-finite} if $\mu(E)<\infty$.
\item $E\in\M$ is \textbf{$\sigma$-finite} if $E$ is a countable union of $\mu$-finite sets.
\item $\mu$ is \textbf{finite} if $X$ is $\mu$-finite (and thus every $E\in\M$ is $\mu$-finite).
\item $\mu$ is \textbf{$\sigma$-finite} if $X$ is $\sigma$-finite (and thus every $E\in\M$ is $\sigma$-finite).
\item $\mu$ is \textbf{semi-finite} if every $E\in\M$ with nonzero measure contains a measurable subset of finite positive measure.\footnote{The word \textbf{positive} is important here since every set contains the empty set.}
\end{itemize}
\end{definition}

\begin{proposition}
Let $(X,\M,\mu)$ be a semi-finite measure space with $\mu(X)=\infty$. Then, $\M$ contains sets of arbitrarily large finite positive measure.
\end{proposition}

\begin{theorem}
Let $(X,\M,\mu)$ be a measure space. Let $\{E_i\}_{i=1}^{\infty}\subset\M$.
\begin{enum}{\alph}
\item Let $E,F\in\M$ such that $E\subset F$. Then, $\mu(E)\leq\mu(F)$.
\item $\mu(\bigcup_{i=1}^{\infty}E_i)\leq\ssum_{i=1}^{\infty}\mu(E_i)$.
\item Suppose $E_1\subset E_2\subset\cdots$. Then, $\mu(\bigcup_{i=1}^{\infty}E_i)=\llim_{i\to\infty}\mu(E_i)$.
\item Suppose $E_1\supset E_2\supset\cdots$ and $\mu(E_1)<\infty$. Then, $\mu(\bigcap_{i=1}^{\infty}E_i)=\llim_{i\to\infty}\mu(E_i)$.
\end{enum}
\end{theorem}

The above properties have names:
\begin{enum}{\alph}
\item Monotonicity
\item Subadditivity
\item Continuity from below
\item Continuity from above
\end{enum}

\begin{definition}
Let $(X,\M,\mu)$ be a measure space. $E\in\M$ is \textbf{null} if $\mu(E)=0$. For our purposes, such sets will be considered ``small.'' The measure $\mu$ is \textbf{complete} if 
$$E\in\M\textrm{ null with }F\subset E\implies F\in\M,$$
which, by monotonicity, implies $F$ is null. Intuitively, if $E$ is a small set in $\M$ then $\M$ \emph{should} contain all strictly smaller sets.
\end{definition}

\begin{theorem}
Let $(X,\M,\mu)$ be a measure space and $\mc{N}\subset\M$ the collection of null sets. Let 
$$\ov{\M}:=\{E\cup F : E\in\M,F\subset N\textrm{ for some }N\in\mc{N}\}.$$
Then, $\ov{\M}$ is a $\sigma$-algebra and there exists a unique extension $\ov{\mu}$ of $\mu$ to $\ov{\M}$ which is complete.
\end{theorem}

The measure $\ov{\mu}$ is called the \textbf{completion} of $\mu$.

\subsection{Outer Measures}
\begin{definition}
An \textbf{outer measure} on $X$ is a function $\mu^*: \P(X)\to[0,\infty]$ satisfying
\begin{enum}{\arabic}
\item $\mu^*(\emptyset)=0$;
\item monotonicity; and 
\item subadditivity.
\end{enum}
\end{definition}

The following result provides a common method of constructing outer measures (as well as justification for the name).

\begin{proposition}
Let $\E\subset\P(X)$ containing $\emptyset,X$ and let $\rho: \E\to[0,\infty]$ such that $\rho(\emptyset)=0$. Define $\mu^*: \P(X)\to[0,\infty]$ by 
$$\mu^*(A):=\inf\left\{\ssum_{i=1}^{\infty}\rho(E_i) : E_i\in\E\textrm{ such that }A\subset\bigcup_{i=1}^{\infty}E_i\right\}.$$
Then, $\mu^*$ is an outer measure.
\end{proposition}

How do we pass from outer measures to measures? The key stepping stone is the following definition.

\begin{definition}
Let $X$ be a nonempty set equipped with an outer measure $\mu^*$. A subset $A\subset X$ is \textbf{$\mu^*$-measurable} if 
$$\mu^*(E)=\mu^*(E\cap A)+\mu^*(E\cap A^c)$$
for every $E\subset X$ or, equivalently, 
$$\mu^*(E)\geq\mu^*(E\cap A)+\mu^*(E\cap A^c)$$
for every $E\subset X$ such that $\mu^*(E)<\infty$.
\end{definition}

\begin{theorem}[Carath\'{e}odory]
Let $X$ be a nonempty set equipped with an outer measure $\mu^*$ and $\M\subset\P(X)$ the collection of all $\mu^*$-measurable sets. Then, $\M$ is a $\sigma$-algebra and $\mu^*|_{\M}$ is a complete measure.
\end{theorem}

\begin{remark}
Note that this does \emph{\textbf{not}} say that $\mu^*$ extends $\rho$, for $\rho$ any suitable function inducing $\mu^*$. It is easy to cook up examples where extension fails. The following notion remedies this issue.
\end{remark}

\begin{definition}
Let $\A\subset\P(X)$ be a Boolean algebra. A \textbf{premeasure} on $(X,\A)$ is a function $\mu_0: \A\to[0,\infty]$ satisfying
\begin{enum}{\arabic}
\item $\mu_0(\emptyset)=0$; and
\item $\mu_0$ is additive on countable disjoint unions for which it is defined.\footnote{Note that $\A$ is not assumed to be a $\sigma$-algebra and so may not be closed under countable disjoint unions.}
\end{enum}
\end{definition}

All of the above terminology for measures makes sense for premeasures as well -- e.g., $(X,\A)$ is a \textbf{premeasurable space} and $(X,\A,\mu_0)$ is a \textbf{premeasure space}.

\begin{proposition}
Let $(X,\A,\mu_0)$ be a premeasure space and $\mu^*$ the induced outer measure.
\begin{enum}{\alph}
\item $\mu^*|_{\A}=\mu_0$.
\item Every set in $\A$ is $\mu^*$-measurable.
\end{enum}
\end{proposition}

\begin{theorem}
Let $(X,\A,\mu_0)$ be a premeasure space, $\mu^*$ the induced outer measure, and $\M$ the $\sigma$-algebra generated by $\A$.
\begin{enum}{\alph}
\item $\mu:=\mu^*|_{\M}$ is a measure on $\M$ extending $\mu_0$.
\item Let $\nu$ be a measure on $\M$ extending $\mu_0$. Then, $\nu(E)\leq\mu(E)$ for every $E\in\M$, with equality when $\mu(E)<\infty$.
\item Suppose $\mu_0$ is $\sigma$-finite. Then, $\mu$ is the unique measure on $\M$ extending $\mu_0$.
\end{enum}
\end{theorem}

There is a dual perspective to outer measures that is sometimes useful. Let $(X,\A,\mu_0)$ be a finite premeasure space and $\mu^*$ the induced outer measure. Define the \textbf{inner measure} 
$$\mu_*: \P(X)\to[0,\infty),\qquad E\mapsto\mu_0(X)-\mu^*(E^c).$$
Then, $E$ is $\mu^*$-measurable if and only if $\mu^*(E)=\mu_*(E)$. This captures the intuitive notion of ``inner'' and ``outer'' approximations agreeing.

\subsection{Borel Measures}
\begin{definition}
Recall that $\B_X$ denotes the Borel $\sigma$-algebra of a topological space $X$. A measure on $\B_X$ is called a \textbf{Borel measure} on $X$. 
\end{definition}

Consider now a finite Borel measure $\mu$ on $\R$. There is an associated \textbf{distribution function} $F: \R\to[0,\infty)$ given by $F(x):=\mu((-\infty,x])$ which is increasing and right continuous with $\mu([a,b])=F(b)-F(a)$ for $b>a$. We want to reverse this process, building a measure from a function with such properties. To that end, define an \textbf{h-interval} to be an interval of either the form $[a,b)$ or $(a,\infty)$. Such sets generate $\B_{\R}$.\footnote{Indeed, $\B_{\R}$ is generated by each of the collections of open, closed, left half open, and right half open intervals.}

\begin{proposition}
Let $\A$ be the Boolean algebra on $\R$ consisting of finite disjoint unions of h-intervals. Let $F: \R\to[0,\infty)$ be an increasing right continuous function. Define $\mu_0: \A\to[0,\infty]$ by $\mu_0(\emptyset):=0$ and 
$$\mu_0\paren{\bigcup_{i=1}^{\infty}(a_i,b_i]}:=\ssum_{i=1}^{\infty}(F(b_i)-F(a_i)).$$
Then, $\mu_0$ is a premeasure.
\end{proposition}

\begin{theorem}
Let $F: \R\to\R$ be an increasing right continuous function. 
\begin{enum}{\alph}
\item There exists a unique Borel measure $\mu_F$ on $\R$ such that $\mu_F([a,b])=F(b)-F(a)$ for all $b>a$.

\item Let $G: \R\to\R$ be an increasing right continuous function. Then, $\mu_F=\mu_G$ if and only if $F-G$ is constant.

\item Let $\mu$ be a Borel measure on $\R$ that is finite on bounded Borel sets. Define $H: \R\to\R$ by 
\begin{equation*}
H(x):=
\begin{cases}
\mu((0,x]), & x>0, \\
0, & x=0, \\
-\mu((x,0]), & x<0.
\end{cases}
\end{equation*}
Then, $H$ is an increasing right continuous function and $\mu=\mu_H$.
\end{enum}
\end{theorem}

Given an increasing right continuous function $F: \R\to\R$ we obtain not just $\mu_F$ but also a complete measure $\ov{\mu_F}$ defined on a $\sigma$-algebra containing $\B_{\R}$. This complete measure is often also denoted $\mu_F$ and called the \textbf{Lebesgue-Stieltjes measure} (LS measure for short) associated to $F$. 

For the remainder of this subsection, let $\mu$ be an LS measure associated to some $F$ as above. Let $\M=\M_{\mu}$ be the domain of $\mu$. Given $E\in\M$, we have 
\begin{align*}
\mu(E)
=\inf\left\{\ssum_{i=1}^{\infty}(F(b_i)-F(a_i)) : E\subset\bigcup_{i=1}^{\infty}(a_i,b_i]\right\}
=\inf\left\{\ssum_{i=1}^{\infty}\mu((a_i,b_i]) : E\subset\bigcup_{i=1}^{\infty}(a_i,b_i]\right\}.
\end{align*}

\begin{lemma}
We can replace h-intervals by open intervals in the above formula.
\end{lemma}

\begin{theorem}
Let $E\in\M$. Then,
$$\inf_{U\supset E\textrm{ open}}\mu(U)=\mu(E)=\sup_{K\cc E}\mu(K).$$
\end{theorem}

\textbf{\underline{Slogan}:} General measurable sets can be approximated by ``simple'' sets.

\begin{definition}
Let $\mu$ be a measure on a topological space $X$ defined on the Borel sets. Given $E$ an element of the domain of $\mu$, $\mu$ is \textbf{outer regular} on $E$ if $\mu(E)=\inf_{U\supset E\textrm{ open}}\mu(U)$ and \textbf{inner regular} on $E$ if $\mu(E)=\sup_{K\cc E}\mu(K)$. The measure $\mu$ is \textbf{regular} if it is simultaneously outer and inner regular on all such $E$.
\end{definition}

The concept of regularity is most often applied to Borel measures. We will see a weaker condition later in the form of Radon measure. The above theorem says that LS measures on $\R$ are both outer and inner regular.

\begin{theorem}
Let $E\subset\R$. TFAE:
\begin{enum}{\roman}
\item $E\in\M$.
\item $E=V\setminus N_1$ for $V$ a $G_{\delta}$ set and $N_1$ null.
\item $E=H\cup N_2$ for $H$ an $F_{\sigma}$ set and $N_2$ null.
\end{enum}
\end{theorem}

\begin{proposition}
Let $E\in\M$ be $\mu$-finite. Then, for every $\eps>0$ there is a set $A$ obtained as the union of finitely many open intervals such that $\mu(E\Delta A)<\eps$.
\end{proposition}

The LS measure associated to $F=\id_{\R}$ is called \textbf{Lebesgue measure} and is denoted $m$ (as is its restriction to $\B_{\R}$). The domain of $m$ is denoted $\L$ and consists of the \textbf{Lebesgue measurable sets} of $\R$.

\begin{theorem}
Lebesgue measure is compatible with translation and dilation.
\end{theorem}

\section{Integration}
\subsection{Measurable Functions}
Given $f: X\to Y$ a map of sets, we have $f^{-1}: \P(Y)\to\P(X)$ which preserves unions, intersections, and complements and so pulls back $\sigma$-algebras to $\sigma$-algebras. For the sake of convenience, $\B$ refers to either $\B_{\R}$ or $\B_{\C}$ unless otherwise stated.

\begin{definition}
Let $(X,\M),(Y,\N)$ be measurable spaces and $f: X\to Y$ a map of sets. The map $f$ is \textbf{$(\M,\N)$-measurable} or simply \textbf{measurable} if $\N$ pulls back to (a subset of) $\M$ (note that it suffices to check on a generating set of $\N$). This is precisely the content of a map of measurable spaces, denoted $f: (X,\M)\to(Y,\N)$. From this we get the category of measurable spaces.

For the sake of convenience, we define some shorthand for maps to either $\R$ or $\C$.
\begin{itemize}
\item $(\M,\B)$-measurable is \textbf{$\M$-measurable} or simply \textbf{measurable}.\footnote{We use $\B$ in place of $\L$ because general Lebesgue sets are too difficult to work with in practice.}
\item $(\B_{\R},\B)$-measurable is \textbf{Borel measurable}.
\item $(\L,\B)$-measurable is \textbf{Lebesgue measurable}.
\end{itemize}

A set map $f: X\to Y$ between topological spaces is \textbf{measurable} if it is $(\B_X,\B_Y)$-measurable.
\end{definition}

\begin{proposition}
Let $f: X\to Y$ be a continuous map between topological spaces. Then, $f$ is measurable.
\end{proposition}

\begin{remark}
Given $f,g: \R\to\R$ Lebesgue measurable, $f\circ g$ need not be Lebesgue measurable even if $g$ is continuous because of the difference between $\B$ and $\L$.
\end{remark}

\begin{definition}
Let $(X,\M)$ be a measurable space and $f$ a function on $X$. Given $E\in\M$, $f$ is \textbf{measurable on $E$} if $f^{-1}(B)\cap E\in\M$ for every $B\in\B$. Equivalently, $f|_E$ is $\M_E$-measurable for $\M_E:=\{F\cap E : F\in\M\}$.
\end{definition}

Given a family of functions $\{f_{\alpha}: X\to(Y_{\alpha},\N_{\alpha})\}_{\alpha\in A}$, there is a minimal $\sigma$-algebra $\M$ on $X$ such that each $f_{\alpha}$ is measurable. $\M$ is obtained as the $\sigma$-algebra generated by $f_{\alpha}^{-1}(E_{\alpha})$ for $E_{\alpha}\in\N_{\alpha}$ (as usual we need only sample from generating sets). This construction recovers the product $\sigma$-algebra given a family of projection maps.

\begin{proposition}
Products in the category of measurable spaces are given by products in $\Set$ equipped with the product $\sigma$-algebra.
\end{proposition}

\begin{proposition}
Let $(X,\M)$ be a measurable space and $f,g: X\to\C$ functions on $X$.
\begin{enum}{\alph}
\item $f$ is measurable if and only if $\Re f,\Im f$ are both measurable.
\item Suppose $f,g$ are both measurable. Then, $f+g,fg$ are measurable.
\end{enum}
\end{proposition}

Let $\ov{\R}$ denote the set of completed reals $\R\cup\{\pm\infty\}$ with its usual structure (note the convention that $0\cdot(\pm\infty):=0$). The notion of measurability makes sense for $\ov{\R}$-valued functions.

\begin{proposition}
The set of $\ov{\R}$-valued functions on a measurable space $(X,\M)$ is closed under pointwise limits. More precisely, given $\{f_j\}$ a sequence of $\ov{\R}$-valued measurable functions on $X$, each of the functions
\begin{itemize}
\item $g_1(x):=\sup_jf_j(x)$;
\item $g_2(x):=\inf_jf_j(x)$;
\item $g_3(x):=\llimsup_{j\to\infty}f_j(x)$;
\item $g_4(x):=\lliminf_{j\to\infty}f_j(x)$
\end{itemize}
is measurable. Moreover, if $f(x):=\llim_{j\to\infty}f_j(x)$ exists for every $x\in X$ then the resulting function $f$ is measurable.
\end{proposition}

If the pointwise limit of $\C$-valued measurable functions on $X$ exists then it is also measurable.

\begin{corollary}
Let $f,g: X\to\ov{\R}$ be measurable. Then, $\min(f,g)$ and $\max(f,g)$ are both measurable.
\end{corollary}

\begin{definition}
Let $X$ be a set and $E\subset X$. The \textbf{characteristic function} or \textbf{indicator function} of $E$ is $\chi_E: X\to\{0,1\}$ given by
\begin{equation*}
\chi_E(x):=
\begin{cases}
1, & x\in E, \\
0, & x\not\in E.
\end{cases}
\end{equation*}
A \textbf{simple function} is a finite $\C$-linear combination of characteristic functions. Every simple function has a unique \textbf{standard representation} as a sum of characteristic functions of disjoint sets.
\end{definition}

If $f,g$ are simple then is it immediate that $f+g,fg$ are simple. Given $(X,\M)$ a measurable space and $E\subset X$, one can show that $\chi_E$ is measurable if and only if $E\in\M$. Unless otherwise stated, we will take simple functions to be measurable.

\begin{theorem}\label{increasing_simple_approximation}
Let $(X,\M)$ be a measurable space and $f: X\to[0,\infty]$ a measurable function. Then, there exists a sequence $\{\phi_n\}$ of simple functions $\phi_n$ on $X$ such that
\begin{enum}{\arabic}
\item $0\leq\phi_1\leq\phi_2\leq\cdots\leq f$;
\item $\phi_n\to f$ pointwise; and
\item $\phi_n\to f$ uniformly on any set on which $f$ is bounded.
\end{enum}
The same holds true if $[0,\infty]$ is replaced by $\C$ and in \textup{(1)} we replace $\phi_n$ by $|\phi_n|$ and $f$ by $|f|$.
\end{theorem}

As this is an approximation result, it is crucial to note that the proof is constructive (i.e., it cooks up the sequence of simple functions in a relatively computable way).

\begin{proposition}
Let $(X,\M,\mu)$ be a measure space. TFAE:
\begin{enum}{\roman}
\item The measure $\mu$ is complete.
\item Given functions $f,g$ such that $f$ is measurable and $f=g$ a.e., the function $g$ is measurable.
\item Let $\{f_n\}$ be a sequence of measurable functions on $X$ converging pointwise a.e. to a function $f$ on $X$. Then, $f$ is measurable.
\end{enum}
\end{proposition}

\begin{proposition}\label{complete_approximation}
Let $(X,\M,\mu)$ be a measure space, $(X,\ov{\M},\ov{\mu})$ its completion, and $f$ an $\ov{\M}$-measurable function on $X$. Then, there exists an $\M$-measurable function $g$ on $X$ such that $f=g$ a.e. (with respect to $\ov{\mu}$).
\end{proposition}

\subsection{Integration of Nonnegative Functions}
Fix a measure space $(X,\M,\mu)$ and define
$$L^+:=\{f: X\to[0,\infty] \mid f\textrm{ is }\M\textrm{-measurable}\}.$$

\begin{definition}
Let $\phi$ be a simple function in $L^+$ with standard representation $\ssum_{j=1}^na_j\chi_{E_j}$. The \textbf{integral} of $\phi$ with respect to $\mu$ (over $X$) is the sum $\ssum_{j=1}^na_j\mu(E_j)$. Given $A\in\M$, the integral of $\phi$ with respect to $\mu$ over $A$ is written and calculated via
$$\Int{A}{\phi}{\mu}=\int_A\phi=\Int{A}{\phi(x)}{\mu(x)}=\int\phi\chi_A\;d\mu,\qquad\int=\int_X.$$
\end{definition}

\begin{proposition}
Let $\phi,\psi$ be simple functions in $L^+$.
\begin{enum}{\alph}
\item $c\geq0\implies\int c\phi=c\int\phi$.
\item $\int(\phi+\psi)=\int\phi+\int\psi$.
\item $\phi\leq\psi\implies\int\phi\leq\int\psi$.
\item $A\mapsto\int_Ad\mu$ is a measure on $\M$.
\end{enum}
\end{proposition}

Given $f\in L^+$, define 
$$\int f\;d\mu:=\sup\left\{\Int{}{\phi}{\mu} : \phi\in L^+\textrm{ simple such that }0\leq\phi\leq f\right\}.$$
The above proposition then clearly extends to this generalized integral.

\begin{theorem}[Monotone Convergence Theorem]
Let $\{f_n\}$ be a sequence in $L^+$ such that $f_1\leq f_2\leq\cdots$ and $f\in L^+$ the pointwise limit of this sequence (which necessarily exists). Then, 
$$\int f=\llim_{n\to\infty}\int f_n.$$
\end{theorem}

A direct consequence of this is that $\int f$ is actually computable for any $f\in L^+$. By Theorem \ref{increasing_simple_approximation}, there is an increasing family $\{\phi_n\}$ of simple functions whose pointwise limit is $f$ and so the Monotone Convergence Theorem implies that $\int f=\llim_{n\to\infty}\int\phi_n$.

\begin{proposition}
Integration of functions in $L^+$ commutes with finite and infinite sums.
\end{proposition}

\begin{proposition}
Let $f\in L^+$. Then, $\int f=0$ if and only if $f=0$ a.e.
\end{proposition}

\begin{corollary}
Let $f\in L^+$ and $\{f_n\}$ a sequence in $L^+$ such that $f_n(x)$ increases to $f(x)$ for a.e. $x\in X$. Then, $\int f=\llim_{n\to\infty}\int f_n$.
\end{corollary}

\begin{theorem}[Fatou's Lemma]
Let $\{f_n\}$ be a sequence in $L^+$. Then,
$$\int\lliminf_{n\to\infty}f_n\leq\lliminf_{n\to\infty}\int f_n.$$
\end{theorem}

\begin{corollary}
Let $f\in L^+$ and $\{f_n\}$ a sequence in $L^+$ such that $f_n\to f$ pointwise a.e. Then, $\int f\leq\lliminf_{n\to\infty}\int f_n$.
\end{corollary}

\begin{proposition}
Let $f\in L^+$ such that $\int f<\infty$. Then, $\{x\in X : f(x)=\infty\}$ is null and $\{x\in X : f(x)>0\}$ is $\sigma$-finite.
\end{proposition}

\subsection{Integration of Complex Functions}
Let $(X,\M,\mu)$ be a measure space and $f\in L^+$ everywhere finite. Then, there are positive and negative parts $f^+,f^-\in L^+$ that are everywhere finite such that $f=f^+-f^-$. We say $f$ is \textbf{integrable} if $\int f^+,\int f^-$ are both finite and, in this case, define $\int f:=\int f^+-\int f^-$. Clearly, since $|f|=f^++f^-$, $f$ is integrable if and only if $\int|f|<\infty$.

\begin{proposition}
The set of $\R$-valued integrable functions on $X$ is an $\R$-vector space with integration as a linear functional.
\end{proposition}

Suppose now that $f$ is a $\C$-valued measurable function on $X$. Given $E\in\M$, we say $f$ is \textbf{integrable on $E$} if $\int_E|f|<\infty$. Clearly, $f$ is integrable if and only if $\Re f,\Im f$ are both integrable. In either case, we define $\int f:=\int\Re f+i\int\Im f$. This give rise to a $\C$-vector space of $\C$-valued integrable functions on $X$ variously denoted 
$$L^1=L^1(X,\mu)=L^1(X)=L^1(\mu).$$

\begin{proposition}
Let $f\in L^1$. Then, $\abs{\int f}\leq\int|f|$.
\end{proposition}

\begin{proposition}
Let $f,g\in L^1$.
\begin{enum}{\alph}
\item The set $\{x\in X : f(x)\neq0\}$ is $\sigma$-finite.
\item TFAE:
\begin{enum}{\roman}
\item $\int_Ef=\int_Eg$ for every $E\in\M$.
\item $\int|f-g|=0$.
\item $f=g$ a.e.
\end{enum}
\end{enum}
\end{proposition}

By the above, nothing is lost if we identify $L^1(\mu)$ with the corresponding space of a.e. equivalence classes. This upgrades the $L^1$ pseudo-norm to a bona fide norm. By Proposition \ref{complete_approximation}, we may identify $L^1(\mu)$ with $L^1(\ov{\mu})$ and so assume without loss of generality that $\mu$ is complete.

\begin{theorem}[Lebesgue Dominated Convergence Theorem]
Let $\{f_n\}$ be a sequence in $L^1$ converging pointwise a.e. to a function $f$. Suppose there exists nonnegative $g\in L^1$ such that $|f_n|\leq g$ a.e. for every $n$. Then, $f\in L^1$ and $\int f=\llim_{n\to\infty}\int f_n$.
\end{theorem}

The above theorem is named for the fact that the function $g$ ``dominates'' the sequence $\{f_n\}$.

\begin{corollary}
Let $\{f_n\}$ be a sequence in $L^1$ such that $\ssum_{n=1}^{\infty}\int|f_n|<\infty$. Then, $\ssum_{n=1}^{\infty}f_n$ converges a.e. to a function $f\in L^1$ and $\int f=\ssum_{n=1}^{\infty}\int f_n$.
\end{corollary}

The next result says that simple functions are dense in $L^1$ in a somewhat controllable way.
\begin{theorem}
Let $f\in L^1(\mu)$ and $\eps>0$. 
\begin{enum}{\alph}
\item There exists an integrable simple function $\phi$ such that $\int|f-\phi|<\eps$.

\item Suppose $\mu$ is an LS measure on $\R$. Then, $\phi$ as above can be obtained as a linear combination of indicators of sets each obtained as a finite union of disjoint open intervals. Moreover, there exists a continuous function $g$ on $\R$ vanishing outside a bounded interval such that $\int|f-g|\;d\mu<\eps$.
\end{enum}
\end{theorem}

Specifying all of this to the Lebesgue measure on $\R$ yields the theory of \textbf{Lebesgue integration}.

\begin{theorem}
Let $f: X\times[a,b]\to\C$ such that $f(\cdot,t): X\to\C$ is integrable for every $t\in[a,b]$. Define $F: [a,b]\to\C$ by 
$$F(t):=\int_Xf(x,t)\;d\mu(x).$$
\begin{enum}{\alph}
\item Suppose $t_0\in[a,b]$ such that $\llim_{t\to t_0}f(x,t)=f(x,t_0)$ for all $x$. Further, suppose there exists $g\in L^1$ such that $|f(x,t)|\leq g(x)$ for all $x,t$. Then, $\llim_{t\to t_0}F(t)=F(t_0)$ and so $F$ is continuous if each $f(x,\cdot): [a,b]\to\C$ is continuous.

\item Suppose $\df{\partial f}{\partial t}$ exists and there is some $g\in L^1$ such that $\abs{\df{\partial f}{\partial t}(x,t)}\leq g(x)$ for all $x,t$. Then, $F$ is differentiable and 
$$F'(t)=\int_X\df{\partial f}{\partial t}(x,t)\;d\mu(x,t).$$ 
\end{enum}
\end{theorem}

It is easy to see that Riemann integrable functions are Lebesgue measurable thus Lebesgue integrable. The following result characterizes the Riemann integrable functions among the Lebesgue integrable functions.

\begin{theorem}
Let $f$ be a bounded $\R$-valued function on $[a,b]$. Then, $f$ is Riemann integrable if and only if its set of discontinuities has Lebesgue measure $0$.
\end{theorem}

\subsection{Modes of Convergence}
Let $(X,\M,\mu)$ be a measure space, $\{f_n\}$ a sequence of $\C$-valued measurable functions on $X$, and $f: X\to\C$ some measurable function. Recall the following notions of convergence.
\begin{itemize}
\item $f_n\to f$ uniformly if $\norm{f-f_n}_{\infty}\to0$.\footnote{The map $\norm{\cdot}_{\infty}$ here denotes the uniform norm. Later the same notation will be used for the canonical norm on the space $L^{\infty}$.}
\item $f_n\to f$ pointwise if $f_n(x)\to f(x)$ for every $x\in X$.
\item $f_n\to f$ (pointwise) a.e. if $f_n(x)\to f(x)$ for a.e. $x\in X$.
\item $f_n\to f$ in $L^1$ if $f_n,f\in L^1$ and $\int|f-f_n|\to0$.
\end{itemize}
Consider the following sequences that illustrate how these notions of convergence differ.
\begin{enum}{\roman}
\item $f_n:=n^{-1}\chi_{(0,n)}\implies f_n\to0$ uniformly but not in $L^1$. 
\item $f_n:=\chi_{(n,n+1)}\implies f_n\to0$ pointwise but not uniformly or in $L^1$.
\item $f_n:=n\chi_{[0,1/n]}\implies f_n\to0$ a.e. but not pointwise or in $L^1$.
\item $f_n:=\chi_{[j/2^k,(j+1)/2^k]}$ for $n=2^k+j$ with $0\leq j<2^k\implies f_n\to0$ in $L^1$ but not a.e.
\end{enum}

\begin{definition}
$\{f_n\}$ is \textbf{Cauchy in measure} if for every $\eps>0$ we have 
\begin{center}
$\mu(\{x\in X : |f_m(x)-f_n(x)|\geq\eps\})\to0$ as $m,n\to\infty$.
\end{center}
Define \textbf{convergence in measure} similarly.
\end{definition}

In the above, \textup{(i,iii,iv)} converge to $0$ in measure while \textup{(ii)} is not even Cauchy in measure. 

\begin{proposition}
Suppose $f_n\to f$ in $L^1$. Then, $f_n\to f$ in measure.
\end{proposition}

\begin{theorem}
Let $\{f_n\}$ be Cauchy in measure. Then, there exists $f: X\to\C$ measurable such that $f_n\to f$ in measure and a subsequence $\{f_{n_j}\}$ such that $f_{n_j}\to f$ a.e. Moreover, $f$ is unique up to a.e. equivalence among such a.e. limits.
\end{theorem}

\begin{corollary}
Suppose $f_n\to f$ in $L^1$. Then, there exists a subsequence $\{f_{n_j}\}$ such that $f_{n_j}\to f$ a.e.
\end{corollary}

\begin{theorem}[Egoroff]
Suppose $\mu(X)<\infty$ and $f_n\to f$ a.e. Then, $f_n\to f$ \textbf{almost uniformly} -- i.e., for every $\eps>0$ there exists $E\in\M$ with $\mu(E)<\eps$ such that $f_n\to f$ uniformly on $E^c$.
\end{theorem}

Almost uniform convergence is \textbf{not} the same as uniform convergence a.e.

\begin{proposition}
Suppose $f_n\to f$ almost uniformly. Then, $f_n\to f$ a.e. and in measure.
\end{proposition}

\begin{theorem}[Lusin]
Suppose $f: [a,b]\to\C$ is Lebesgue measurable and $\eps>0$. Then, there exists $E\cc[a,b]$ with $\mu(E^c)<\eps$ such that $f|_E$ is continuous.
\end{theorem}

\begin{remark}
This is one of Littlewood's three principles of measure theory. 
\end{remark}

Lusin's theorem does \textbf{not} say that $f$ is continuous a.e.

\subsection{Product Measures}
Let $(X,\M,\mu)$ and $(Y,\N,\nu)$ be measure spaces. A (measurable) \textbf{rectangle} is a set of the form $A\times B$ for $A\in\M$ and $B\in\N$. The collection $\A$ of finite disjoint unions of such rectangles is a Boolean algebra whose induced $\sigma$-algebra is $\M\tensor\N$. Let $A\times B=\bigsqcup_j(A_j\times B_j)$ be an at most countable partition of a rectangle into rectangles. Given $x\in X$ and $y\in Y$, we have
\begin{align*}
\chi_A(x)\chi_B(y)
=\chi_{A\times B}(x,y)
=\ssum_j\chi_{A_j\times B_j}(x,y)
=\ssum_j\chi_{A_j}(x)\chi_{B_j}(y).
\end{align*}
Integrating over $X$ and exchanging the order of summation and integration gives
$$\mu(A)\chi_B(y)=\ssum_j\mu(A_j)\chi_{B_j}(y).$$
Integrating over $Y$ then gives
$$\mu(A)\nu(B)=\ssum_j\mu(A_j)\nu(B_j)$$
and so we obtain a well-defined premeasure 
$$\A\to[0,\infty],\qquad\bigsqcup_{j=1}^n(A_j\times B_j)\mapsto\ssum_{j=1}^n\mu(A_j)\nu(B_j)$$
which induces the \textbf{product measure} $\mu\times\nu$ on $(X\times Y,\M\tensor\N)$. If $\mu$ and $\nu$ are both $\sigma$-finite then the same holds true for $\mu\times\nu$ and, in fact, $\mu\times\nu$ is then the unique measure $\lambda$ on $\M\tensor\N$ such that $\lambda(A\times B)=\mu(A)\nu(B)$ for all rectangles.

\begin{remark}
The notion of product measure and the above results extend to any finite product of measure spaces.
\end{remark}

\begin{definition}
Let $E\subset X\times Y$. Given $x\in X$, define $E_x:=\{y\in Y : (x,y)\in E\}$. Similarly, given $y\in Y$, define $E^y:=\{x\in X : (x,y)\in E\}$. In the same vein, let $f$ be a function on $X\times Y$. We obtain induced functions $f_x$ on $Y$ and $f^y$ on $X$ for every $x\in X$ and $y\in Y$.
\end{definition}

Using the above setup, it is a quick check that $(\chi_E)_x=\chi_{E_x}$ and $(\chi_E)^y=\chi_{E^y}$.

\begin{proposition}
Let $(X,\M)$ and $(Y,\N)$ be measurable spaces.
\begin{enum}{\alph}
\item Let $E\in\M\tensor\N$. Then, $E_x\in\N$ and $E^y\in\M$ for every $x\in X$ and $y\in Y$.
\item Let $f$ be a $(\M\tensor\N)$-measurable function on $X\times Y$. Then, $f_x$ is $\N$-measurable and $f^y$ is $\M$-measurable for every $x\in X$ and $y\in Y$.
\end{enum}
\end{proposition}

\begin{definition}
Let $X$ be a nonempty set. A subset $\mc{C}\subset\P(X)$ is a \textbf{monotone class} if it is closed under countable increasing unions and countable decreasing intersections.
\end{definition}

The intersection of monotone classes is a monotone class, so any subset of $\P(X)$ generates a monotone class.

\begin{lemma}[Monotone Class Lemma]
Let $\A$ be a Boolean algebra on a nonempty set $X$. Let $\mc{C}$ be the induced monotone class and $\M$ the induced $\sigma$-algebra. Then, $\mc{C}=\M$.
\end{lemma}

A priori we have $\mc{C}\subset\M$. The main use of the Monotone Class Lemma comes from allowing us to check countable decreasing intersections instead of complements.

\begin{theorem}
Let $(X,\M,\mu)$ and $(Y,\N,\nu)$ be $\sigma$-finite measure spaces and $E\in\M\tensor\N$. Then, $x\mapsto\nu(E_x)$ is a measurable function on $X$ and $y\mapsto\mu(E^y)$ is a measurable function on $Y$. Moreover, 
$$\int\nu(E_x)\;d\mu(x)=(\mu\times\nu)(E)=\int\mu(E^y)\;d\nu(y).$$
\end{theorem}

\begin{theorem}[Tonelli]
Let $f\in L^+(X\times Y)$ for $\sigma$-finite measure spaces $(X,\M,\mu)$ and $(Y,\N,\nu)$. Define functions $g$ on $X$ and $h$ on $Y$ by 
$$g(x):=\int f_x\;d\nu,\qquad h(y):=\int f^y\;d\mu.$$
Then, $g\in L^+(X)$, $h\in L^+(Y)$, and 
\begin{align*}
\int\sqbrac{\int f(x,y)\;d\mu(x)}d\nu(y)
=\int f\;d(\mu\times\nu)
=\int\sqbrac{\int f(x,y)\;d\nu(y)}d\mu(x).
\end{align*}
\end{theorem}

The double integrals appearing in the above theorem are often written as 
$$\int\sqbrac{\int f(x,y)\;d\mu(x)}d\nu(y)=\iint f(x,y)\;d\mu(x)d\nu(y)=\iint f\;d\mu d\nu.$$

\begin{theorem}[Fubini]
Let $f\in L^1(\mu\times\nu)$ for $\sigma$-finite measure spaces $(X,\M,\mu)$ and $(Y,\N,\nu)$. Then, $f_x\in L^1(\nu)$ for a.e. $x\in X$ and $f^y\in L^1(\mu)$ for a.e. $y\in Y$. Moreover, the induced a.e. functions $g$ and $h$ defined as above reside in $L^1(\mu)$ and $L^1(\nu)$, respectively, and we have 
$$\iint f\;d\mu d\nu=\int f\;d(\mu\times\nu)=\iint f\;d\nu d\mu.$$
\end{theorem}

Note that $\mu\times\nu$ is almost never complete even if $\mu,\nu$ are both complete. This can be remedied as follows.

\begin{theorem}
Let $(X,\M,\mu)$ and $(Y,\N,\nu)$ be $\sigma$-finite measure spaces. Let $(X\times Y,\L,\lambda)$ be the completion of $(X\times Y,\M\tensor\N,\mu\times\nu)$ and $f$ a $\L$-measurable function.
\begin{enum}{\alph}
\item Suppose $f\geq0$. Then, $f_x$ is $\N$-measurable for a.e. $x\in X$ and $f^y$ is $\M$-measurable for a.e. $y\in Y$. Moreover, the functions $g,h$ defined as above are measurable.

\item Suppose $f\in L^1(\lambda)$. Then, the same results as in \textup{(a)} hold true and we can say more. The functions $f_x,f^y$ are integrable for a.e. $x\in X$ and $y\in Y$. Moreover, the functions $g,h$ are integrable and we have
$$\iint f\;d\mu d\nu=\int f\;d(\mu\times\nu)=\iint f\;d\nu d\mu.$$
\end{enum}
\end{theorem}

\subsection{The $n$-Dimensional Lebesgue Integral}
Lebesgue measure $m$ is the completion of the product measure $m^n$ on either $\B_{\R}^{\tensor n}=\B_{\R^n}$ or $\L^{\tensor n}$. We write $\L^n$ for the domain of $m$. The product constituents for a rectangle in $\R^n$ are called its \textbf{sides}.

\begin{theorem}
The Lebesgue measure on $\R^n$ is outer and inner regular.
\end{theorem}

\begin{theorem}
Let $f\in L^1(m)$ and $\eps>0$. Then, there exists a simple function $\phi=\sum_{j=1}^Na_j\chi_{R_j}$ with each $R_j$ a product of intervals such that $\int|f-\phi|<\eps$. Moreover, there exists a continuous function $g$ on $\R^n$ vanishing outside a bounded set such that $\int|f-g|<\eps$.
\end{theorem}

\begin{theorem}[Translation Invariance]
Fix $a\in\R^n$ and consider the translation map 
$$\tau_a: \R^n\to\R^n,\qquad x\mapsto x+a.$$
\begin{enum}{\alph}
\item Let $E\in\L^n$. Then, $\tau_a(E)\in\L^n$ and $m(\tau_a(E))=m(E)$.
\item Let $f$ be a Lebesgue measurable function on $\R^n$. Then, $f\circ\tau_a$ is a Lebesgue measurable function on $\R^n$. Moreover, if $f$ is in $L^+$ or $L^1$ then $f\circ\tau_a$ is as well and $\int f\circ\tau_a\;dm=\int f\;dm$.
\end{enum}
\end{theorem}

\begin{theorem}
Let $T\in\GL_n(\R)$. 
\begin{enum}{\alph}
\item Let $f$ be a Lebesgue measurable function on $\R^n$. Then, $f\circ T$ is a Lebesgue measurable function on $\R^n$. Moreover, if $f$ is in $L^+$ or $L^1$ then $f\circ T$ is as well and 
$$\int f\;dm=\abs{\det T}\int f\circ T\;dm.$$
\item Let $E\in\L^n$. Then, $T(E)\in\L^n$ and $m(T(E))=\abs{\det T}m(E)$.
\end{enum}
\end{theorem}

\begin{corollary}
Rotation preserves Lebesgue measure.
\end{corollary}

\begin{theorem}[Change of Variables]
Let $\Omega\in\R^n$ be an open set and $G=(g_1,\ldots,g_n)\in C^1(\Omega,\R^n)$ with pointwise differential $D_xG$. 
\begin{enum}{\alph}
\item Let $f$ be a Lebesgue measurable function on $G(\Omega)$. Then, $f\circ G$ is a Lebesgue measurable function on $\Omega$. Moreover, if $f$ is in $L^+$ or $L^1$ then $f\circ G$ is as well and 
$$\int_{G(\Omega)}f(x)\;dm(x)=\int_{\Omega}(f\circ G)(x)\abs{\det D_xG}\;dm(x).$$
\item Let $E\in\L^n$ such that $E\subset\Omega$. Then, $G(E)\in\L^n$ and $m(G(E))=\int_E\abs{\det D_xG}\;dm(x)$.
\end{enum}
\end{theorem}

\section{Signed Measures and Differentiation}
\subsection{Signed Measures}
Fix a measurable space $(X,\M)$. 

\begin{definition}
A \textbf{signed measure} on $X$ is a function $\nu: \M\to[-\infty,\infty]$ satisfying
\begin{enum}{\arabic}
\item $\nu(\emptyset)=0$;
\item $\nu$ assigns at most one of $\pm\infty$; and
\item given $\{E_j\}\subset\M$ a countable disjoint collection, $\nu(\bigcup_{j\geq1}E_j)=\sum_{j\geq1}\nu(E_j)$, the latter converging absolutely if the former is finite.
\end{enum}
\end{definition}

We sometimes use the term \textbf{positive measure} to distinguish ordinary measures from general signed measures.

\begin{example}
\hfill
\begin{enum}{\arabic}
\item Let $\mu_1,\mu_2$ be measures on $X$ with at least one finite. Then, $\mu_1-\mu_2$ is a signed measure on $X$.
\item Let $\mu$ be a measure on $X$ and $f: X\to[-\infty,\infty]$ an \textbf{extended $\mu$-integrable function}, so that at least one of $\int f^+\;d\mu,\int f^-\;d\mu$ is finite. Then, $\nu_f$ given by $E\mapsto\int_Ef\;d\mu$ is a signed measure. We informally write $d\nu_f=fd\mu$.
\end{enum}
\end{example}

We will soon see that the above examples essentially encapsulate all examples of signed measures.

\begin{proposition}
Signed measures are continuous from above and below.
\end{proposition}

\begin{definition}
Let $\nu$ be a signed measure on $X$. A set $E\in\M$ is \textbf{positive} if $\nu(F)\geq0$ for every $F\in\M$ such that $F\subset E$. We say $\nu$ is \textbf{negative} or \textbf{null} under the appropriate similar conditions.
\end{definition}

\begin{lemma}
Measurable subsets and countable unions of positive sets are positive.
\end{lemma}

\begin{theorem}[Hahn Decomposition]
Let $\nu$ be signed measure on $X$. Then, there exists a \textbf{Hahn decomposition} of $\nu$ -- i.e., a pair $(P,N)$ with $P$ positive and $N$ negative such that $P\cup N=X$ and $P\cap N=\emptyset$. Moreover, if $(P',N')$ is another such pair then $P\Delta P'=N\Delta N'$ is null.
\end{theorem}

\begin{definition}
Two signed measures $\mu,\nu$ are \textbf{mutually singular} (or singular with respect to each other) if there exist $E,F\in\M$ such that $E\cup F=X$, $E\cap F=\emptyset$, $E$ is $\mu$-null, and $F$ is $\nu$-null. We write $\mu\perp\nu$. Informally, $\mu$ and $\nu$ ``live on disjoint sets.''
\end{definition}

\begin{theorem}[Jordan Decomposition]
Let $\nu$ be a signed measure on $X$. Then, there exists a unique \textbf{Jordan decomposition} of $\nu$ -- i.e., a pair $(\nu^+,\nu^-)$ of positive measures such that $\nu=\nu^+-\nu^-$ and $\nu^+\perp\nu^-$.
\end{theorem}

The measures $\nu^{\pm}$ are called the \textbf{positive} and \textbf{negative variations} of $\nu$. The positive measure $|\nu|:=\nu^++\nu^-$ is called the \textbf{total variation}.

\begin{definition}
Let $\nu$ be a signed measure on $X$. Define $L^1(\nu):=L^1(\nu^+)\cap L^1(\nu^-)$. Given $f\in L^1(\nu)$, define 
$$\int f\;d\nu:=\int f\;d\nu^+-\int f\;d\nu^-.$$
We say that $\nu$ is ($\sigma$-)\textbf{finite} if $|\nu|$ is ($\sigma$-)\textbf{finite}.
\end{definition}

\subsection{The Lebesgue-Radon-Nikodym Theorem}
Fix a measurable space $(X,\M)$. Unless otherwise stated, $\nu$ denotes a signed measure and $\mu$ a positive measure on $X$.

\begin{definition}
We say $\nu$ is \textbf{absolutely continuous} with respect to $\mu$ and write $\nu\ll\mu$ if every $\mu$-null set is $\nu$-null.
\end{definition}

\begin{proposition}
TFAE:
\begin{enum}{\roman}
\item $\nu\ll\mu$.
\item $|\nu|\ll\mu$.
\item $\nu^{\pm}\ll\mu$.
\end{enum}
\end{proposition}

\begin{proposition}
Suppose $\nu\perp\mu$ and $\nu\ll\mu$. Then, $\nu=0$.
\end{proposition}

\begin{theorem}
We have $\nu\ll\mu$ if and only if for every $\eps>0$ there exists $\delta>0$ such that $|\nu(E)|<\eps$ for every $E\in\M$ such that $\mu(E)<\delta$.
\end{theorem}

\begin{proposition}
Let $f$ be an extended $\mu$-integrable function on $X$. Then, $\nu_f$ is absolutely continuous with respect to $\mu$. Moreover, $\nu_f$ is finite if and only if $f\in L^1(\mu)$.
\end{proposition}

\begin{corollary}
Let $f\in L^1(\mu,\C)$. Then, for every $\eps>0$ there exists $\delta>0$ such that $|\nu_f(E)|<\eps$ for every $E\in\M$ such that $\mu(E)<\delta$.
\end{corollary}

\begin{lemma}
Suppose $\mu,\nu$ are both finite measures on $X$. Then, either $\nu\perp\mu$ or there exists $\eps>0$ and $E\in\M$ with $\mu(E)>0$ such that $\nu\geq\eps\mu$.
\end{lemma}

\begin{theorem}[Radon-Nikodym-Lebesgue]
Let $\nu$ be a $\sigma$-finite signed measure and $\mu$ a $\sigma$-finite positive measure on $X$. 
\begin{enum}{\alph}
\item There exists a unique \textbf{Lebesgue decomposition} -- i.e., a pair $(\lambda,\rho)$ of $\sigma$-finite signed measures such that $\lambda\perp\mu$, $\rho\ll\mu$, and $\nu=\lambda+\rho$.

\item There exists a \textbf{Radon-Nikodym derivative} -- i.e., an extended $\mu$-integrable function $f: X\to\R$ such that $d\rho=fd\mu$. Moreover, $f$ is unique up to $\mu$-a.e. equivalence.
\end{enum}
\end{theorem}

Assuming $\nu\ll\mu$, we often write $d\nu/d\mu$ in place of $f$ and have the relation
$$d\nu=\df{d\nu}{d\mu}d\mu.$$

\begin{corollary}
Let $\nu$ be a $\sigma$-finite signed measure and $\mu$ a $\sigma$-finite positive measure on $X$ such that $\nu\ll\mu$. Then, the Radon-Nikodym derivative $d\nu/d\mu$ exists is unique up to $\mu$-a.e. equivalence.
\end{corollary}

\begin{proposition}[Chain Rule]
Let $\nu$ be a $\sigma$-finite signed measure and $\mu,\lambda$ $\sigma$-finite positive measures on $X$ such that $\nu\ll\mu$ and $\mu\ll\lambda$.
\begin{enum}{\alph}
\item Let $g\in L^1(\nu)$. Then, $g\cdot(d\nu/d\mu)\in L^1(\mu)$ and 
$$\int g\;d\nu=\int g\df{d\nu}{d\mu}\;d\mu.$$

\item We have $\nu\ll\lambda$ and 
$$\df{d\nu}{d\lambda}=\df{d\nu}{d\mu}\df{d\mu}{d\lambda}$$
$\lambda$-a.e.
\end{enum}
\end{proposition}

\section{Topological Preliminaries}
Recall that a topological space is called normal if disjoint closed subsets can be separated by disjoint open neighborhoods of each closed set. A topological space is called locally compact or LCH if it is Hausdorff and every point has a compact neighborhood (i.e., a compact subset containing an open neighborhood of the point).

\begin{theorem}[Urysohn's Lemma]
Let $X$ be a normal topological space and $A,B\subset X$ disjoint closed. Then, there exists $f\in C(X,[0,1])$ such that $f|_A=0$ and $f|_B=1$.
\end{theorem}

\begin{theorem}[Tietze Extension Theorem]
Let $X$ be a normal topological space, $A\subset X$ closed, and $f\in C(A)$. Then, there exists $F\in C(X)$ such that $F|_A=f$.
\end{theorem}

\begin{theorem}[Urysohn's Lemma, LCH Version]
Let $X$ be an LCH space and $K\subset U\subset X$ with $K$ compact and $U$ open. Then, there exists $f\in C_c(X,[0,1])$ such that $K\subset\supp(f)\subset U$ and $f|_K=1$.
\end{theorem}

\begin{theorem}[Tietze Extension Theorem, LCH Version]
Let $X$ be an LCH space, $K\cc X$, and $f\in C(K)$. Then, there exists $F\in C_c(X)$ such that $F|_K=f$.
\end{theorem}

A function $f: X\to\C$ is said to \textbf{vanish at infinity} if for every $\eps>0$ the set $\{x\in X : |f(x)|\geq\eps\}$ is compact. We let $C_0(X)$ denote the set of continuous functions $f: X\to\C$ vanishing at infinity.

\begin{proposition}
Let $X$ be an LCH space. Then, $C_0(X)$ is precisely the closure of $C_c(X)$ in the uniform metric.
\end{proposition}

It is sometimes useful to think about the topology of uniform convergence and of uniform convergence on compacta.

\begin{theorem}[Arzel\`{a}-Ascoli, Version I]
Let $X$ be a compact Hausdorff space and $A\subset C(X)$ equicontinuous and pointwise bounded. Then, $A$ is precompact (i.e., its closure is compact) and totally bounded in the uniform metric (i.e., it can be covered by finitely many balls of radius $\eps$ for every $\eps>0$).
\end{theorem}

\begin{theorem}[Arzel\`{a}-Ascoli, Version II]
Let $X$ be a $\sigma$-compact LCH space (i.e., $X$ is a countable union of compact subspaces) and $\{f_n\}\subset C(X)$ equicontinuous and pointwise bounded. Then, there exists $f\in C(X)$ and a subsequence $\{f_{n_k}\}$ such that $f_{n_k}\to f$ uniformly on compacta.
\end{theorem}

\section{Elements of Functional Analysis}
\subsection{Normed Linear Spaces}
\begin{theorem}
A normed linear space is complete if and only if every absolutely convergent series converges.
\end{theorem}

Let $X$ be a normed linear space and $Y\subset X$ a closed linear subspace. Then, $X/Y$ is naturally a normed linear space with norm
$$\norm{x+Y}:=\inf_{y\in Y}\norm{x+y}.$$

\begin{theorem}
Let $X,Y$ be normed linear spaces with $Y$ Banach and $L(X,Y)$ the space of bounded linear maps from $X$ to $Y$, equipped with the operator norm. Then, $L(X,Y)$ is Banach and so, in particular, the continuous dual $X^*$ is Banach.
\end{theorem}

\subsection{Linear Functionals}
\begin{proposition}
Let $X$ be a $\C$-vector space. If $f$ is a $\C$-linear functional on $X$ and $u=\Re f$ then $u$ is an $\R$-linear functional on $X$ and $f(x)=u(x)-iu(ix)$ for every $x\in X$. Conversely, if $u$ is an $\R$-linear functional on $X$ then $f$ given by $x\mapsto u(x)-iu(ix)$ is a $\C$-linear functional on $X$. In the normed case we have $\norm{u}=\norm{f}$.
\end{proposition}

\begin{definition}
Let $X$ be an $\R$-vector space. A function $p: X\to\R$ is a \textbf{sublinear functional} on $X$ if $p(x+y)\leq p(x)+p(y)$ and $p(\lambda x)=\lambda p(x)$ for every $x,y\in X$ and $\lambda\in\R^{\geq0}$.
\end{definition}

\begin{theorem}[Hahn-Banach, Real Version]
Let $X$ be an $\R$-vector space, $p$ a sublinear functional on $X$, $Y\subset X$ a linear subspace, and $f$ a linear functional on $Y$ such that $f(x)\leq p(x)$ for every $x\in Y$. Then, there exists a linear functional $F$ on $X$ such that $F(x)\leq p(x)$ for every $x\in X$ and $F|_Y=f$.
\end{theorem}

\begin{theorem}[Hahn-Banach, Complex Version]
Let $X$ be a $\C$-vector space, $p$ a seminorm on $X$, $Y\subset X$ a linear subspace, and $f$ a linear functional on $Y$ such that $|f(x)|\leq p(x)$ for every $x\in Y$. Then, there exists a linear functional $F$ on $X$ such that $|F(x)|\leq p(x)$ for every $x\in X$ and $F|_Y=f$.
\end{theorem}

An immediate corollary of this is that every element $f\in Y^*$ admits an extension $F\in X^*$ such that $\norm{F}_{X^*}=\norm{f}_{Y^*}$, using the seminorm $x\mapsto\norm{f}_{Y^*}\norm{x}$.

\begin{theorem}
Let $X$ be a normed linear space over $\C$.
\begin{enum}{\alph}
\item Let $Y\subset X$ be a closed linear subspace and $x\in X\setminus Y$. Then, there exists $f\in X^*$ such that $f(x)\neq0$ and $f|_Y=0$. In fact, $f$ can be taken to satisfy $\norm{f}=1$ and $f(x)=\inf_{y\in Y}\norm{x-y}$.
\item Let $x\in X$ be nonzero. Then, there exists $f\in X^*$ such that $\norm{f}=1$ and $f(x)=\norm{x}$.
\item $X^*$ separates points.
\item The evaluation map $X\to X^{**}$ is a linear isometry.
\end{enum}
\end{theorem}

\subsection{The Baire Category Theorem and Its Consequences}
Let $X$ be a topological space and $A\subset X$. Recall that $A$ is \textbf{nowhere dense} in $X$ if the closure $\ov{A}$ of $A$ in $X$ has empty interior or, equivalently, $X\setminus\ov{A}$ is dense in $X$.

\begin{theorem}[Baire Category Theorem]
Let $X$ be a topological space homeomorphic to a complete metric space.
\begin{enum}{\alph}
\item Let $\{U_n\}$ be a sequence of open dense subsets of $X$. Then, $\bigcap_{n\geq1}U_n$ is dense in $X$.
\item $X$ is \textbf{of the second category} -- i.e., $X$ is not the countable union of nowhere dense subsets.
\end{enum}
\end{theorem}

A topological space which is the countable union of nowhere dense subsets is said to be \textbf{of the first category} or \textbf{meager}. The complement of a meager set is called \textbf{residual}; note that this is not the same thing as being nonmeager. Meagerness provides a measure of smallness.

\begin{theorem}[Open Mapping Theorem]
Let $X,Y$ be Banach spaces and $T\in L(X,Y)$ surjective. Then, $T$ is open. Hence, if $T$ is also injective then $T$ is an isomorphism of normed linear spaces (i.e., $T^{-1}$ is bounded).
\end{theorem}

\begin{theorem}[Closed Graph Theorem]
Let $X,Y$ be Banach spaces and $T: X\to Y$ closed linear. Then, $T$ is bounded.
\end{theorem}

\begin{theorem}[Uniform Boundedness Principle]
Let $X,Y$ be normed linear spaces, $A\subset L(X,Y)$, and $Z\subset X$ nonmeager such that
$$\sup_{T\in A}\norm{Tx}<\infty$$
for every $x\in Z$. Then, $\sup_{T\in A}\norm{T}<\infty$.
\end{theorem}

By the Baire category theorem, if $X$ is Banach then we can take $Z=X$.

\subsection{Topological Vector Spaces}
\begin{definition}
Let $X$ be a normed linear space. The \textbf{weak topology} on $X$ is the minimal topology such that every element of $X^*$ is continuous -- a net $\{x_{\alpha}\}\subset X$ converges weakly to $x\in X$ if and only if $f(x_{\alpha})\to f(x)$ for every $f\in X^*$. The \textbf{weak* topology} on $X^*$ is characterized by a net $\{f_{\alpha}\}\subset X^*$ converging weak* to $f\in X^*$ if and only if $f_{\alpha}(x)\to f(x)$ for every $x\in X$. 
\end{definition}

Let $X,Y$ be Banach spaces. Using the above, we obtain notions of \textbf{weak} and \textbf{strong operator topology} on $L(X,Y)$.

\begin{remark}
How do these various topologies compare?
\begin{itemize}
\item On $X$: weak $\prec$ norm
\item On $X^*$: weak* $\prec$ weak $\prec$ dual norm
\item On $L(X,Y)$: weak operator $\prec$ strong operator $\prec$ operator norm
\end{itemize}
\end{remark}

\begin{theorem}[Banach-Alaoglu]
Let $X$ be a normed linear space and $B$ the closed unit ball in $X^*$. Then, $B$ is compact in the weak* topology.
\end{theorem}

The above theorem, which is a consequence of Tychonoff's theorem, does \textbf{not} say that $B$ is compact in the dual norm topology. Indeed, $B$ is compact in the dual norm topology if and only if $X$ (hence $X^*$) is finite dimensional.

\subsection{Hilbert Spaces}
We use the terms \textbf{inner product space}, \textbf{IPS}, and \textbf{pre-Hilbert space} interchangeably. Given an IPS $(H,\ip{\cdot,\cdot})$, we define 
$$\norm{\cdot}: H\to\R^{\geq0},\qquad x\mapsto\ip{x,x}^{1/2}.$$

\begin{proposition}[Cauchy-Schwarz Inequality]
Let $H$ be an IPS. Then, $\abs{\ip{x,y}}\leq\norm{x}\cdot\norm{y}$ for every $x,y\in H$, with equality if and only if $x,y$ are linearly dependent.
\end{proposition}

\begin{corollary}
$\norm{\cdot}$ is a norm on $H$.
\end{corollary}

\begin{corollary}
Suppose $x_n\to x$ and $y_n\to y$ in $H$. Then, $\ip{x_n,y_n}\to\ip{x,y}$.
\end{corollary}

\begin{corollary}[Parallelogram Law]
$\norm{x+y}^2+\norm{x-y}^2=2(\norm{x}^2+\norm{y}^2)$ for all $x,y\in H$.
\end{corollary}

\begin{corollary}[Pythagorean Theorem]
Let $x_1,\ldots,x_n\in H$ be pairwise orthogonal. Then, 
$$\norm{\sum_{j=1}^nx_j}^2=\sum_{j=1}^n\norm{x_j}^2.$$
\end{corollary}

For the rest of the section, we assume that $H$ is a Hilbert space (i.e., a complete IPS).

\begin{theorem}
Let $Y\subset H$ be a closed linear subspace. Then, $H=Y\oplus Y^{\perp}$. Moreover, given a decomposition $x=y_0+z_0$ for $y_0\in Y$ and $z_0\in Y^{\perp}$, we have
$$\norm{x-y_0}=\inf_{y\in Y}\norm{x-y},\qquad\norm{x-z_0}=\inf_{z\in Y^{\perp}}\norm{x-z}.$$
\end{theorem}

\begin{theorem}
The map $H\to H^*$ given by $y\mapsto\ip{\cdot,y}$ is a conjugate-linear isomorphism.
\end{theorem}

We use ``ON'' as shorthand for ``orthonormal.''

\begin{theorem}[Bessel's Inequality]
Let $\{u_{\alpha}\}_{\alpha\in A}\subset H$ be an ON set. Then,
$$\sum_{\alpha\in A}\abs{\ip{x,u_{\alpha}}}^2:=\sup_{F\subset A\textup{ finite}}\sum_{\alpha\in F}\abs{\ip{x,u_{\alpha}}}^2\leq\norm{x}^2$$
for every $x\in H$. In particular, $\{\alpha\in A : \ip{x,u_{\alpha}}\neq0\}$ is countable.
\end{theorem}

\begin{theorem}
Let $\{u_{\alpha}\}_{\alpha\in A}\subset H$ be an ON set. TFAE:
\begin{enum}{\roman}
\item For every $x\in H$, $\ip{x,u_{\alpha}}=0\implies x=0$.

\item For every $x\in H$, $\norm{x}^2=\sum_{\alpha\in A}\abs{\ip{x,u_{\alpha}}}^2$.

\item For every $x\in H$, $\sum_{\alpha\in A}\ip{x,u_{\alpha}}u_{\alpha}$ is a countable sum converging absolutely (i.e., in the norm topology) to $x$.
\end{enum}
\end{theorem}

A set satisfying any of the above equivalent conditions is called an \textbf{ON basis} for $H$. Note that, for $H$ infinite dimensional, an ON basis for $H$ need not be an algebraic basis. Condition (ii) is Parseval's identity.

\begin{proposition}
Every Hilbert space has an ON basis.
\end{proposition}

\begin{proposition}
A Hilbert space is separable (i.e., has a countable dense subset) if and only if it has a countable ON basis, in which case every ON basis is countable.
\end{proposition}

\begin{proposition}
Let $\{u_{\alpha}\}_{\alpha\in A}\subset H$ be an ON basis for $H$. Then,
$$\phi: H\to\l^2(A),\qquad x\mapsto(\alpha\mapsto\ip{x,u_{\alpha}})$$
is a unitary $\C$-linear isomorphism.
\end{proposition}

\section{$L^p$ Spaces}
Throughout this section, $(X,\M,\mu)$ denotes a fixed measure space unless otherwise stated.

\subsection{Basic Theory of $L^p$ Spaces}
\begin{lemma}
Let $a,b\geq0$ and $\lambda\in(0,1)$. Then, $a^{\lambda}b^{1-\lambda}\leq\lambda a+(1-\lambda)b$, with equality if and only if $a=b$.
\end{lemma}

Given $1\leq p\leq\infty$, let $1\leq q\leq\infty$ be its \textbf{conjugate exponent} satisfying $1/p+1/q=1$ and thus given by $q=p/(p-1)$. Unless otherwise stated, $(p,q)$ should be considered a pair of conjugate exponents.

\begin{proposition}[H\"{o}lder Inequality]
Let $f,g$ measurable functions on $X$. Then, $\norm{fg}_1\leq\norm{f}_p\norm{g}_q$. In particular, if $f\in L^p$ and $g\in L^q$ then $fg\in L^1$. If in addition $1<p,q<\infty$ then the above inequality is an equality if and only if $\alpha|f|^p=\beta|g|^q$ $\mu$-a.e. for some nonzero constants $\alpha,\beta$.
\end{proposition}

\begin{proposition}[Minkowski Inequality]
Let $1\leq p<\infty$ and $f,g\in L^p$. Then, $\norm{f+g}_p\leq\norm{f}_p+\norm{g}_p$.
\end{proposition}

\begin{theorem}
Let $1\leq p<\infty$. Then, $L^p$ is a Banach space and the set of simple functions arising from $\mu$-finite sets is dense in $L^p$.
\end{theorem}

Given a measurable $f$ on $X$, define its \textbf{essential supremum} to be 
\begin{align*}
\norm{f}_{\infty}
=\esssup_{x\in X}|f(x)|
:=\inf\{a\geq0 : \mu(\{x\in X : |f(x)|>a\})=0\},
\end{align*}
with the convention that $\inf\emptyset:=\infty$.

\begin{theorem}
Let $(X,\M,\mu)$ be a measure space.
\begin{enum}{\alph}
\item Let $f,g$ be measurable functions on $X$. Then, $\norm{fg}_1\leq\norm{f}_1\norm{g}_{\infty}$. Moreover, if $f\in L^1$ and $g\in L^{\infty}$ then $\norm{fg}_1=\norm{f}_1\norm{g}_{\infty}$ if and only if $|g(x)|=\norm{g}_{\infty}$ a.e. on the set where $f(x)\neq0$.

\item The map $\norm{\cdot}_{\infty}$ is a norm on $L^{\infty}$.

\item $\norm{f-f_n}_{\infty}\to0$ if and only if there exists $E\in\M$ such that $\mu(E^c)=0$ and $f_n\to f$ uniformly on $E$.

\item The space $L^{\infty}$ is Banach.

\item The simple functions are dense in $L^{\infty}$.
\end{enum}
\end{theorem}

\begin{remark}
The norm $\norm{\cdot}_{\infty}$ coincides with the uniform norm assuming that $\mu$ is Borel taking positive values on nonempty open sets and the function in question is continuous.
\end{remark}

\begin{proposition}
Let $0<p<q<r\leq\infty$.
\begin{enum}{\alph}
\item $L^q\subset L^p+L^r$.

\item $L^p\cap L^r\subset L^q$. In fact, given $f\in L^p\cap L^r$, we have $\norm{f}_q\leq\norm{f}_p^{\lambda}\norm{f}_r^{1-\lambda}$ for $\lambda\in(0,1)$ defined by $q^{-1}=\lambda p^{-1}+(1-\lambda)r^{-1}$ -- i.e., 
$$\lambda=\df{q^{-1}-r^{-1}}{p^{-1}-r^{-1}}.$$
\end{enum}
\end{proposition}

\begin{proposition}
Let $0<p<q\leq\infty$.
\begin{enum}{\alph}
\item Let $A$ be a nonempty set. Then, $\l^p(A)\subset\l^q(A)$ and $\norm{f}_q\leq\norm{f}_p$.

\item Assume $\mu(X)<\infty$. Then, $L^p(\mu)\supset L^q(\mu)$ and $\norm{f}_p\leq\norm{f}_q\mu(X)^{1/p-1/q}$.
\end{enum}
\end{proposition}

\subsection{The Dual of $L^p$}
Let $g\in L^q$. Consider the linear map $\phi_g: L^p\to\C$ given by $f\mapsto\int fg$, which is bounded since $\norm{\phi_g}\leq\norm{g}_q$.

\begin{proposition}
Suppose either $1\leq q<\infty$ \textbf{or} $q=\infty$ and $\mu$ is semi-finite. Then,
\begin{align*}
\norm{g}_q
=\norm{\phi_g}
=\sup\left\{\abs{\int fg} : \norm{f}_p=1\right\}.
\end{align*}
\end{proposition}

\begin{theorem}
Let $\Sigma$ be the space of simple functions on $X$ vanishing outside a set of finite measure and $g$ a measurable function on $X$ such that $fg\in L^1$ for every $f\in\Sigma$. Suppose that 
$$M_q(g):=\sup\left\{\abs{\int fg} : f\in\Sigma\textrm{ and }\norm{f}_p=1\right\}<\infty$$
and either $S_g:=\{x\in X : g(x)\neq0\}$ is $\sigma$-finite \textbf{or} $\mu$ is semi-finite. Then, $g\in L^q$ and $M_q(g)=\norm{g}_q$.
\end{theorem}

\begin{theorem}
Suppose either $1<p<\infty$ \textbf{or} $p=1$ and $\mu$ is semi-finite. Then, the linear map $\Phi: L^q\to(L^p)^*$ given by $g\mapsto\phi_g$ is an isometric isomorphism.
\end{theorem}

If $p=2$ and we change the definition of $\phi_g$ to integrating against $\ov{g}$ then $\Phi$ defines a Hilbert space automorphism of $L^2$.

\begin{corollary}
Given $1<p<\infty$, the space $L^p$ is reflexive.
\end{corollary}

\subsection{Some Useful Inequalities}
\begin{proposition}[Chebyshev Inequality]
Let $0<p<\infty$, $f\in L^p$, and $\alpha>0$. Then,
$$\mu(\{x\in X : |f(x)|>\alpha\})\leq\paren{\df{\norm{f}_p}{\alpha}}^p.$$
\end{proposition}

\begin{theorem}
Let $(X,\M,\mu),(Y,\N,\nu)$ be $\sigma$-finite measure spaces, $K$ an $(\M\tensor\N)$-measurable function on $X\times Y$, and $1\leq p\leq\infty$. Suppose there exists $C>0$ such that 
\begin{align*}
\int|K(x,y)|\;d\mu(x)\leq C&\textrm{ for a.e. }y\in Y, \\
\int|K(x,y)|\;d\nu(y)\leq C&\textrm{ for a.e. }x\in X.
\end{align*}
Given $f\in L^p(\nu)$, define 
$$Tf(x):=\int K(x,y)f(y)\;d\nu(y).$$
Then, $Tf(x)$ converges absolutely for a.e. $x\in X$, $Tf\in L^p(\mu)$, and $\norm{Tf}_p\leq C\norm{f}_p$.
\end{theorem}

The resulting bounded linear map $T: L^p(\nu)\to L^p(\mu)$ is called an \textbf{integral operator} and $K$ its \textbf{kernel}. 

\begin{remark}
The proof of the above result shows that only the first equality is needed if $p=1$ and only the second inequality is needed if $p=\infty$.
\end{remark}

\begin{proposition}[Minkowski Inequality for Integrals]
Let $(X,\M,\mu),(Y,\N,\nu)$ be $\sigma$-finite measure spaces and $f$ an $(\M\tensor\N)$-measurable function on $X\times Y$. 
\begin{enum}{\alph}
\item Suppose $f\geq0$ and $1\leq p\leq\infty$. Then,
\begin{align*}
\sqbrac{\int\paren{\int f(x,y)\;d\nu(y)}^p\;d\mu(x)}^{1/p}
\leq\int\sqbrac{\int f(x,y)^p\;d\mu(x)}^{1/p}\;d\nu(y).
\end{align*}

\item Suppose $1\leq p\leq\infty$, $f(\cdot,y)\in L^p(\mu)$ for a.e. $y\in Y$, and $y\mapsto\norm{f(\cdot,y)}_p$ is an element of $L^1(\nu)$. Then,
\begin{enum}{\arabic}
\item $f(x,\cdot)\in L^1(\nu)$ for a.e. $x\in X$;
\item $x\mapsto\int f(x,y)\;d\nu(y)$ is an element of $L^p(\mu)$; and
\item $\norm{\int f(\cdot,y)\;d\nu(y)}_p\leq\int\norm{f(\cdot,y)}_p\;d\nu(y)$.
\end{enum}
\end{enum}
\end{proposition}

\begin{theorem}
Let $K$ be a Lebesgue measurable function on $(0,\infty)\times(0,\infty)$ such that
\begin{itemize}
\item $K(\lambda x,\lambda y)=\lambda^{-1}K(x,y)$ for every $\lambda>0$ and $x,y\in(0,\infty)$; and
\item there exists some $1\leq p\leq\infty$ such that $C:=\int|K(x,1)|x^{-1/p}\;dx<\infty$.
\end{itemize}
Given $f\in L^p$ and $g\in L^q$, define
\begin{align*}
Tf(y)&:=\int_0^{\infty}K(x,y)f(x)\;dx, \\
Sg(x)&:=\int_0^{\infty}K(x,y)g(y)\;dy.
\end{align*}
Then,
\begin{enum}{\alph}
\item $Tf$ and $Sg$ are defined a.e.;
\item $\norm{Tf}_p\leq C\norm{f}_p$; and
\item $\norm{Sg}_q\leq C\norm{g}_q$.
\end{enum}
\end{theorem}

\begin{corollary}
Let $f\in L^p$ and $g\in L^q$. Define
\begin{align*}
Tf(y)&:=y^{-1}\int_0^yf(x)\;dx, \\
Sg(x)&:=\int_x^{\infty}y^{-1}g(y)\;dy.
\end{align*}
\begin{enum}{\alph}
\item Suppose $1<p\leq\infty$. Then, $\norm{Tf}_p\leq\df{p}{p-1}\norm{f}_p$.
\item Suppose $1\leq q<\infty$. Then, $\norm{Sg}_q\leq q\norm{g}_q$.\footnote{The inequalities in \textup{(a)} and \textup{(b)} hold trivially if $p=1$ or $q=\infty$.}
\end{enum}
\end{corollary}

\section{Radon Measures}
Throughout this section, let $X$ denote an LCH space.

\subsection{Positive Linear Functionals on $C_c(X)$}
A linear functional $I$ on $C_c(X)$ is called \textbf{positive} if $f\geq0\implies I(f)\geq0$. 

\begin{proposition}
Let $I$ be a positive linear functional on $C_c(X)$. Then, for every $K\cc X$ there exists $C_K>0$ such that $|I(f)|\leq C_K\norm{f}_u$ for every $f\in C_c(X)$ supported on $K$.
\end{proposition}

Let $\mu$ be a Borel measure on $X$ that is finite on compacta. Then, there is an embedding $C_c(X)\subset L^1(\mu)$ and the map $f\mapsto\int f\;d\mu$ is a positive linear functional on $C_c(X)$.

\begin{definition}
A \textbf{Radon measure} $\mu$ on $X$ is a Borel measure that is
\begin{itemize}
\item finite on compacta;
\item outer regular on Borel sets; and
\item inner regular on opens.
\end{itemize}
\end{definition}

Given $U\subset X$ open and $f\in C_c(X)$, we write $f\prec U$ if $0\leq f\leq1$ and $\supp(f)\subset U$.

\begin{theorem}[Riesz Representation Theorem, Positive Version]
Let $I$ be a positive linear functional on $C_c(X)$. Then, there exists a unique Radon measure $\mu$ on $X$ such that $I(f)=\int f\;d\mu$ for every $f\in C_c(X)$. Moreover,
$$\mu(U)=\sup\{I(f) : f\in C_c(X),f\prec U\}$$
for every $U\subset X$ open and
$$\mu(K)=\inf\{I(f) : f\in C_c(X),f\geq\chi_K\}$$
for every $K\cc X$.
\end{theorem}

\subsection{Regularity and Approximation Theorems}
\begin{proposition}
Every Radon measure is inner regular on $\sigma$-finite sets.
\end{proposition}

\begin{corollary}
Every $\sigma$-finite Radon measure on $X$ is regular. In particular, if $X$ is $\sigma$-compact then every Radon measure on $X$ is regular.
\end{corollary}

\begin{proposition}
Let $\mu$ be a $\sigma$-finite Radon measure on $X$ and $E$ a Borel set.
\begin{enum}{\alph}
\item For every $\eps>0$ there exist $U,F\subset X$ with $U$ open and $F$ closed such that $F\subset E\subset U$ and $\mu(U\setminus F)<\eps$.

\item There exist an $F_{\sigma}$ set $A$ and $G_{\delta}$ set $B$ such that $A\subset E\subset B$ and $\mu(B\setminus A)=0$.
\end{enum}
\end{proposition}

\begin{theorem}
Suppose every open set in $X$ is $\sigma$-compact (this holds, e.g., if $X$ is second-countable). Let $\mu$ be a Borel measure on $X$ that is finite on compacta. Then, $\mu$ is regular hence Radon. 
\end{theorem}

\begin{proposition}
Let $\mu$ be a Radon measure on $X$ and $1\leq p<\infty$. Then, $C_c(X)\subset L^p(\mu)$ is dense.
\end{proposition}

\begin{theorem}[Lusin]
Let $\mu$ be a Radon measure on $X$, $f: X\to\C$ measurable vanishing outside a set of finite measure, and $\eps>0$. Then, there exists $\phi\in C_c(X)$ such that $\phi=f$ except on a set of measure $<\eps$. Moreover, if $f$ is bounded then $\phi$ may be taken to satisfy $\norm{\phi}_u\leq\norm{f}_u$.
\end{theorem}

\subsection{The Dual of $C_0(X)$}
Let $M(X)$ denote the set of complex Radon measures on $X$.\footnote{A complex Radon measure is a complex measure whose real and imaginary parts are Radon. To say that the real and imaginary parts are Radon in turn is to say that their associated positive and negative variations are Radon.} This a normed linear space with norm given by $\norm{\mu}:=|\mu|(X)$, where $|\mu|$ denotes the total variation of $\mu$ uniquely characterized by the fact that if $d\mu=f\;d\rho$ for $\rho$ a positive measure then $d|\mu|=|f|\;d\rho$.\footnote{We may obtain $|\mu|$ as the sum of the total variations of the real and imaginary parts of $\mu$.}

\begin{theorem}[Riesz Representation Theorem, Complex Version]
Given $\mu\in M(X)$, define
$$I_{\mu}: C_0(X)\to\C,\qquad f\mapsto\int f\;d\mu.$$
Then, the map $\mu\mapsto I_{\mu}$ defines an isometric isomorphism from $M(X)$ to $C_0(X)^*$. In particular, if $X$ is compact then $M(X)\iso C(X)^*$.
\end{theorem}
\end{document}