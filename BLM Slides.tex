\documentclass[aspectratio=1610]{beamer}
\usetheme{Boadilla}

\usepackage{kernel_of_truth}

\newcommand{\Cart}{\textup{Cart}}
\newcommand{\DA}{\mathsf{DA}}
\newcommand{\DC}{\mathsf{DC}}
\newcommand{\sat}{\textup{sat}}
\newcommand{\Sat}{\textup{Sat}}
\newcommand{\str}{\textup{str}}
\newcommand{\W}{\mathcal{W}}
\newcommand{\X}{\mathfrak{X}}

\renewcommand{\phi}{\varphi}

\newtheorem{conjecture}{Conjecture}
\newtheorem{remark}{Remark}
\newtheorem{proposition}{Proposition}

\title{BLM Basics}
\author{Zachary Gardner}
\date{}

\begin{document}
\begin{frame}
\titlepage
\end{frame}

\begin{frame}{Introduction}
Let $X$ be an algebraic variety defined over a perfect field $k$ of characteristic $p$. Let $W\Omega_X^{\bullet}$ denote the classical de Rham-Witt complex of $X$, which when $X=\Spec R$ is characterized up to unique isomorphism as the initial object of the category of so-called $R$-framed $V$-pro-complexes. Here are two key results.

\begin{theorem}
Suppose $X$ is smooth. Then, there exists a canonical map $W\Omega_X^{\bullet}\to\Omega_X^{\bullet}$ inducing a qis $W\Omega_X^{\bullet}/pW\Omega_X^{\bullet}\to\Omega_{X/k}^{\bullet}$.
\end{theorem}

\begin{theorem}
Let $\X$ be a smooth formal scheme over $\Spf(W(k))$ with special fiber $X:=\Spec k\times_{\Spf(W(k))}\X$. Suppose that the Frobenius map $\phi_X: X\to X$ extends to a map of formal schemes $\phi_{\X}: \X\to\X$. Then, there exists a natural qis $\Omega_{\X/W(k)}^{\bullet}\to W\Omega_X^{\bullet}$ which depends on the choice of $\phi_{\X}$ but is independent of this choice on the level of derived categories.
\end{theorem}
\end{frame}

\begin{frame}{Introduction}
Our goal is to construct a complex $\W\Omega_X^{\bullet}$, called the saturated de Rham-Witt complex, which agrees with $W\Omega_X^{\bullet}$ when $X$ is smooth and more naturally satisfies the above results. The key comes from extrapolating properties of $\Omega_{R/k}^{\bullet}$ when $X=\Spec R$ is smooth. The construction of $\W\Omega_R^{\bullet}$ will proceed in two stages, the first stage providing us with a relevant Verschiebung map $V$ through a saturation process and the second stage roughly forcing completeness with respect to $V$. In the interest of time, we will skip right to the algebra.
\end{frame}

\begin{frame}{Dieudonn\'{e} Complexes}
\begin{definition}
Let $\DC$ denote the category of \textbf{Dieudonn\'{e} complexes}, which are triples $(M,d,F)$ with $(M,d)$ a cochain complex of abelian groups and $F: M\to M$ the \textbf{Frobenius} map, a homomorphism of graded abelian groups satisfying 
$$dF(x)=pF(dx)$$
for every $x\in M$. In practice, we will often omit $d$ and $F$ when they are clear from context. A morphism $f: (M,d,F)\to(M',d',F')$ of Dieudonn\'{e} complexes is a map of cochain complexes $f: (M,d)\to(M',d')$ such that $F'\circ f=f\circ F$. Note that $\DC$ has a natural symmetric monoidal structure given by tensor product.
\end{definition}

\begin{remark}
The function $F$ is not a map of complexes from $(M/pM,d)$ to itself. It does, however, give a map of complexes from $(M/pM,0)$ to $(M/pM,d)$. This becomes part of the initial inspiration for considering Dieudonn\'{e} complexes when paired with the Cartier isomorphism (to be described later).
\end{remark}
\end{frame}

\begin{frame}{D\'{e}calage}
Any $p$-torsion-free complex $(M,d)$ may be regarded as a subcomplex of the localization $M[p^{-1}]$. Given such a complex, define the \textbf{d\'{e}calage} $\eta_pM\subset M[p^{-1}]$ via
$$(\eta_pM)^n:=\{x\in p^nM^n : dx\in p^{n+1}M^{n+1}\}.$$
Given $(M,d,F)\in\DC$ a $p$-torsion-free complex, we have a well-defined cochain map $\alpha_F: M\to\eta_pM$ defined on $n$th components by $x\mapsto p^nF(x)$. Conversely, given $(M,d)$ simply a $p$-torsion-free complex and $\alpha\in\Hom(M,\eta_pM)$, we may define a Frobenius map $F: M\to M$ on $n$th components via $x\mapsto p^{-n}\alpha(x)$. These two constructions are inverse to one another.
\end{frame}

\begin{frame}{Saturated Dieudonn\'{e} Complexes}
\begin{definition}
$(M,d,F)\in\DC$ is \textbf{saturated} if $M$ is $p$-torsion-free and for every $n\in\Z$ the map $F$ induces a group isomorphism $M^n\xto{\sim}\{x\in M^n : dx\in pM^{n+1}\}$ (equivalently, $\alpha_F$ is an isomorphism). We obtain a full subcategory $\DC_{\sat}\subset\DC$. Since $F: M\to M$ is necessarily injective with image containing $pM$, we obtain a unique \textbf{Verschiebung} map $V: M\to M$ satisfying $F(Vx)=px$ for every $x\in M$.
\end{definition}

\begin{proposition}
Given $M\in\DC_{\sat}$, the Verschiebung map $V: M\to M$ is injective and satisfies
\begin{itemize}
\item $F\circ V=V\circ F=p\cdot\id$;

\item $F\circ d\circ V=d$;

\item $p\cdot(d\circ V)=V\circ d$.
\end{itemize}
\end{proposition}
\end{frame}

\begin{frame}{Saturated Dieudonn\'{e} Complexes}
\begin{definition}
A morphism $f\in\Hom_{\DC}(M,N)$ \textbf{exhibits $N$ as a saturation of $M$} if $N$ is saturated and, for every $K\in\DC_{\sat}$, $f$ induces a bijection $\Hom_{\DC}(N,K)\xto{\sim}\Hom_{\DC}(M,K)$. If it exists, the data of $(N,f)$ is unique up to unique isomorphism. We call it the \textbf{saturation} of $M$ and write $\Sat(M)$.
\end{definition}

\begin{proposition}
Saturations exist. 
\end{proposition}
\end{frame}

\begin{frame}{Saturated Dieudonn\'{e} Complexes}
\begin{proof}
Let $(M,d,F)\in\DC$. Consider the subcomplex $M[p^{\infty}]\subset M$ defined by 
$$M[p^{\infty}]:=\{x\in M : p^nx=0\textrm{ for }n\gg0\}.$$
Replacing $M$ by $M/M[p^{\infty}]$ if necessary, we may assume WLOG that $M$ is $p$-torsion-free. The obstruction to $M$ being saturated is measured by $\alpha_F: M\to\eta_pM$ failing to be an isomorphism. So, we just force things:
$$\Sat(M):=\colim(M\xto{\alpha_F}\eta_pM\xto{\eta_p(\alpha_F)}\eta_p(\eta_pM)\xto{\eta_p(\eta_p(\alpha_F))}\cdots).$$
Then, $\Sat(M)$ is saturated since $\eta_p$ commutes with filtered colimits. Moreover, the tautological map $M\to\Sat(M)$ exhibits $\Sat(M)$ as a saturation of $M$.
\end{proof}
\end{frame}

\begin{frame}{Cartier Complexes}
Given $M\in\DC$ and $x\in M$, we have $d(Fx)=pF(dx)$ and so the image of $Fx$ in $M/pM$ is a cycle. Hence, we obtain a graded map $M\to H^{\bullet}(M/pM)$ which necessarily factors through $M/pM$.

\begin{definition}
$M\in\DC_{\sat}$ is of \textbf{Cartier type} if $M$ is $p$-torsion-free and $F$ induces a graded isomorphism $M/pM\xto{\sim}H^{\bullet}(M/pM)$. We will often abbreviate this by simply saying that $M$ is \textbf{Cartier}.
\end{definition}

As it turns out, the Cartier type condition is the right condition to place on $M$ so that we can control the behavior of the saturated de Rham-Witt complex (which is not yet defined). The first inkling of this is the following.

\begin{theorem}[Cartier Criterion]
Let $M$ be Cartier. Then, the canonical map $M\to\Sat(M)$ induces a qis 
$$M/pM\to\Sat(M)/p\Sat(M).$$
\end{theorem}
\end{frame}

\begin{frame}[fragile]{Cartier Complexes}
How do we prove this result? The key is d\'{e}calage. Given $M$ a $p$-torsion-free complex, there is a map $\ov{\gamma}: \eta_pM\to H^{\bullet}(M/pM)$ of graded abelian groups defined on $n$th components by $x\mapsto[p^{-n}x]$. This factors uniquely as 
\begin{center}
\begin{tikzcd}
\eta_pM \arrow[r] & (\eta_pM)/p(\eta_pM) \arrow[r, "\gamma"] & H^{\bullet}(M/pM)
\end{tikzcd}
\end{center}
As we saw in BMS I, $\gamma$ is a qis when $H^{\bullet}(M/pM)$ is equipped with the cochain complex structure coming from the Bockstein operator $\beta: H^{\bullet}(M/pM)\to H^{\bullet+1}(M/pM)$ induced by the short exact sequence
\begin{center}
\begin{tikzcd}
0 \arrow[r] & M/pM \arrow[r, "p"] & M/p^2M \arrow[r] & M/pM \arrow[r] & 0
\end{tikzcd}
\end{center}
From this we conclude that, given $f: M\to N$ a map of $p$-torsion-free complexes that is a qis mod $p$, the induced map $(\eta_pM)/p(\eta_pM)\to(\eta_pN)/p(\eta_pN)$ is a qis. 
\end{frame}

\begin{frame}[fragile]{Cartier Complexes}
Using the presentation
$$\Sat(M)=\colim(M\xto{\alpha_F}\eta_pM\xto{\eta_p(\alpha_F)}\eta_p(\eta_pM)\xto{\eta_p(\eta_p(\alpha_F))}\cdots),$$
to show that the natural map $M/pM\to\Sat(M)/p\Sat(M)$ is a qis it suffices to show that 
$$(\eta_p^kM)/p(\eta_p^kM)\to(\eta_p^{k+1}M)/p(\eta_p^{k+1}M)$$
is a qis for each $k\geq0$. But by what just showed it suffices to check the case $k=0$, which follows since
\begin{center}
\begin{tikzcd}
M/pM \arrow[r, "\alpha_F"] \arrow[d, "\iso", "F"'] & (\eta_pM)/p(\eta_pM) \arrow[ld, "\gamma", "\textrm{qis}"'] \\
H^{\bullet}(M/pM)
\end{tikzcd}
\end{center}
commutes and so $\alpha_F$ is a qis.
\end{frame}

\begin{frame}[fragile]{Completion}
Given $M\in\DC_{\sat}$ and $r\in\Z^{\geq0}$, define 
$$\W_r(M):=M/(\im(V^r)+\im(dV^r)),$$
which is a quotient complex of $M$. This comes equipped with natural restriction maps 
$$\Res: \W_{r+1}(M)\to\W_r(M).$$
Using this we define the \textbf{completion} of $M$ to be 
$$\W(M):=\flim_{r\geq0}\W_r(M).$$
We may uniquely complete each diagram
\begin{center}
\begin{tikzcd}
M \arrow[r, "F"] \arrow[d] & M \arrow[d] \\
\W_r(M) \arrow[r, dotted, "\exists!\,F"'] & \W_{r-1}(M)
\end{tikzcd}
\end{center}
and together these induce the Frobenius $F: \W(M)\to\W(M)$.
\end{frame}

\begin{frame}{Completion}
Similarly, we obtain the Verschiebung $V: \W(M)\to\W(M)$ from unique maps $V: \W_r(M)\to\W_{r+1}(M)$. We now enumerate a number of properties of the completion $\W(M)$.

\begin{itemize}
\item $\W(M)$ is naturally an object of $\DC$.

\item The association $M\mapsto\W(M)$ is functorial in $M$.

\item The canonical map $\rho_M: M\to\W(M)$ of Dieudonn\'{e} complexes is functorial in $M$. 

\item $\W(M)$ is $p$-adically complete. Hence, if $M$ is strict then $M$ itself is $p$-adically complete. It follows that $M$ is $\l$-torsion-free for $\l\neq p$ and thus that $M$ is torsion-free since being $p$-torsion-free is part of being saturated.
\end{itemize}
\end{frame}

\begin{frame}{Strict Dieudonn\'{e} Complexes}
\begin{definition}
$M\in\DC_{\sat}$ is \textbf{strict} if $\rho_M$ is an isomorphism. We obtain a full subcategory $\DC_{\str}\subset\DC_{\sat}$.
\end{definition}

\begin{example}
Regard $M\in\Mod_{\Z}$ as a complex concentrated in degree zero. Then, any endomorphism $F: M\to M$ endows $M$ with the structure of a Dieudonn\'{e} complex. Using this, we deduce that $M$ is saturated if and only if $M$ is $p$-torsion-free and $F$ is an automorphism. Assuming $M$ is saturated, $M$ is strict if and only if it is $p$-adically complete.
\end{example}

It isn't hard to see that $\W(M)$ is saturated given $M\in\DC_{\sat}$. In fact, though, more is true.

\begin{theorem}
Let $M\in\DC_{\sat}$. Then, $\W(M)\in\DC_{\str}$.
\end{theorem}
\end{frame}

\begin{frame}{Strict Dieudonn\'{e} Complexes}
Completion also satisfies a useful universal property.

\begin{proposition}
Let $M,N\in\DC_{\sat}$ with $N$ strict. Then, $\rho_M$ induces a natural bijection 
$$\Hom_{\DC}(\W(M),N)\xto{\sim}\Hom_{\DC}(M,N).$$
\end{proposition}

Where the theorem comes from is, given $M\in\DC_{\sat}$ and $r\in\Z^{\geq0}$, $F^r: M\to M$ induces an isomorphism of graded abelian groups $\W_r(M)\xto{\sim}H^{\bullet}(M/p^rM)$ and so the associated quotient map $M/p^rM\to\W_r(M)$ is a qis. From this, we conclude that, given $f\in\Hom_{\DC_{\sat}}(M,N)$, the induced map $M/p^rM\to N/p^rN$ is a qis if and only if the induced map $\W_r(M)\to\W_r(N)$ is an isomorphism. Moreover, we need only check at the $r=1$ level to get equivalence for every $r$.
\end{frame}

\begin{frame}{Getting Somewhere}
\begin{proposition}
Let $M\in\DC_{\sat}$. 
\begin{enumerate}
\item Given $r\in\Z^{\geq0}$, $\rho_M$ induces a qis $M/p^rM\to\W(M)/p^r\W(M)$.

\item $\rho_M$ exhibits $\W(M)$ as the $p$-completion of $M$ in $D(\Z)$.
\end{enumerate}
\end{proposition}

Applying the Cartier Criterion, we conclude the following.

\begin{corollary}
Let $M$ be Cartier with each $M^n$ $p$-adically complete. Then, the canonical map $M\to\W\Sat(M)$ is a qis.
\end{corollary}

With the module-theoretic stuff figured out, let's introduce some multiplicative structure into the mix.
\end{frame}

\begin{frame}{Dieudonn\'{e} Algebras}
\begin{definition}
Let $\DA$ denote the category of \textbf{Dieudonn\'{e} algebras}, which are triples $(A,d,F)$ with $(A,d)$ a cdga\footnote{Recall that this means that $(A,d)$ is a cochain complex with graded ring structure such that multiplication on $A$ is graded-commutative and $d$ satisfies the Leibniz rule. Additionally, we require that if $x\in A^n$ is homogeneous of odd degree then $x^2=0$ in $A^{2n}$.} and \textbf{Frobenius} $F: M\to M$ a homomorphism of graded abelian groups such that 
\begin{itemize}
\item $A^n=0$ for $n>0$;

\item given $x\in A^0$, $Fx\equiv x^p\pmod{p}$;

\item given $x\in A$, $dF(x)=pF(dx)$.
\end{itemize}
Morphisms in $\DA$ are cochain maps with the expected compatibilities.
\end{definition}

Roughly speaking, we think of Dieudonn\'{e} algebras as Dieudonn\'{e} complexes equipped with a ring structure compatible with Frobenius and the differential.
\end{frame}

\begin{frame}[fragile]{Dieudonn\'{e} Algebra Constructions}
\begin{definition}
$A\in\DA$ is \textbf{saturated} if it saturated as a Dieudonn\'{e} complex. In this case, $\W(A)\in\DC_{\str}$ is naturally a Dieudonn\'{e} algebra and the tautological map $\rho_A: A\to\W(A)$ is a morphism in $\DA$. We say that $A$ is \textbf{strict} if $\rho_A$ is an isomorphism in $\DA$, which holds if and only if $A$ is strict regarded as an object of $\DC$. From this we obtain full subcategories $\DA_{\str}\subset\DA_{\sat}\subset\DA$.
\end{definition}

As with Dieudonn\'{e} complexes, saturations and completions of Dieudonn\'{e} algebras make sense and exist (and have the expected universal properties). In fact, they are obtained by placing suitable algebra structures on the relevant constructions for Dieudonn\'{e} complexes.
\end{frame}

\begin{frame}[fragile]{Witt Vectors}
\begin{example}
Let $R\in\CAlg_{\F_p}$ and regard $W(R)$ as a cdga concentrated in degree zero. Then, $W(R)$ equipped with its Witt vector Frobenius $F$ is naturally a Dieudonn\'{e} algebra which is saturated if and only if $R$ is perfect. In the saturated case, $W(R)$ is automatically strict. Recall that, in the case that $R$ is perfect, $W(R)$ is $p$-adically complete with canonical isomorphism $W(R)/pW(R)\xto{\sim}R$ and satisfies the universal property that, given $p$-adically complete $A\in\CRing$ equipped with a ring map $R\to A/pA$, we may uniquely complete the diagram
\begin{center}
\begin{tikzcd}
W(R) \arrow[r, "\exists!"] \arrow[d, twoheadrightarrow] & A \arrow[d, twoheadrightarrow] \\
R \arrow[r] & A/pA
\end{tikzcd}
\end{center}
\end{example}

Closely related to the Witt vector story is the following. Fix $p$-torsion-free $R\in\CRing$. We will see shortly that $\Omega_R^{\bullet}$ is naturally a Dieudonn\'{e} algebra (under an additional constraint). Before that, we briefly discuss the notion of $\delta$-rings. 
\end{frame}

\begin{frame}{$\delta$-Rings}
\begin{definition}
Let $\CRing_{\delta}$ denote the category of \textbf{(commutative) $\delta$-rings}, which are pairs $(A,\delta)$ with $A\in\CRing$ and $\delta: A\to A$ a \textbf{$p$-derivation} satisfying
\begin{itemize}
\item $\delta(xy)=x^p\delta(y)+\delta(x)y^p+p\delta(x)\delta(y)$;

\item $\delta(x+y)=\delta(x)+\delta(y)-\displaystyle\sum_{0<i<p}\df{(p-1)!}{i!(p-i)!}x^iy^{p-i}$;

\item $\delta(1)=0$
\end{itemize}
for all $x,y\in A$. A morphism $f: (A,\delta_A)\to(B,\delta_B)$ in $\CRing_{\delta}$ is a ring map $f: A\to B$ such that $f\circ\delta_A=\delta_B\circ f$.
\end{definition}

Roughly speaking, the point of the second property above is to capture the identity 
$$p(\delta(x+y)-\delta(x)-\delta(y))=x^p+y^p-(x+y)^p$$
even when there is $p$-torsion. The category $\CRing_{\delta}$ is nicely behaved (in ways that we won't make precise) and is the right place to study lifts of Frobenius.
\end{frame}

\begin{frame}{Frobenius Lifts}
Given $(A,\delta)\in\CRing_{\delta}$, the map
$$\phi: R\to R,\qquad x\mapsto x^p+p\delta(x)$$
is a mod $p$ lift of Frobenius. Conversely, given $p$-torsion-free $A\in\CRing$ with $\phi: A\to A$ a mod $p$ lift of Frobenius, the map
$$\delta: A\to A,\qquad x\mapsto\df{\phi(x)-x^p}{p}$$
is a $p$-derivation on $A$ and so $(A,\delta)\in\CRing_{\delta}$.

Returning to $\Omega_R^{\bullet}$, suppose that we have chosen $\phi: R\to R$ a mod $p$ lift of Frobenius (which need not exist in general). Then, as above we have an associated $p$-derivation $\delta_{\phi}: R\to R$. 
\end{frame}

\begin{frame}
\begin{proposition}
There exists a unique ring homomorphism $F: \Omega_R^{\bullet}\to\Omega_R^{\bullet}$ such that
\begin{itemize}
\item $F=\phi$ on $R=\Omega_R^0$;

\item given $x\in R$,
$$F(dx)=\underbrace{x^{p-1}dx+d\delta_{\phi}(x)}_{=:\rho_{\phi}(x)}.$$
\end{itemize}
Moreover, the triple $(\Omega_R^{\bullet},d,F)$ is a Dieudonn\'{e} algebra.
\end{proposition}

\begin{proof}
Uniqueness is clear since $\Omega_R^{\bullet}$ is generated by elements of the form $x$ and $dx$ for $x\in R$. The main idea behind proving existence is to obtain $F$ using the universal property of $\Omega_R^{\bullet}$. The first step is to verify that $\rho_{\phi}: R\to\Omega_R^1$ is a group homomorphism and a $\phi$-linear derivation in the sense that
$$\rho_{\phi}(xy)=\phi(y)\rho_{\phi}(x)+\phi(x)\rho_{\phi}(y)$$
for all $x,y\in R$. The universal property of $\Omega_R^{\bullet}$ then gives that there is a unique $\phi$-semilinear map $F: \Omega_R^1\to\Omega_R^1$ such that $F\circ d=\rho_{\phi}$. This extends uniquely to a ring homomorphism $F: \Omega_R^{\bullet}\to\Omega_R^{\bullet}$ satisfying the desired properties and one checks $(\Omega_R^{\bullet},d,F)\in\DA$.
\end{proof}
\end{frame}

\begin{frame}{de Rham Complexes as Dieudonn\'{e} Algebras}
Unless otherwise stated, we will hereafter assume that $\Omega_R^{\bullet}$ is equipped with the Dieudonn\'{e} algebra structure described by the proposition. This structure possesses a useful universal property.

\begin{proposition}
Given $p$-torsion-free $A\in\DA$, the restriction map
$$\Hom_{\DA}(\Omega_R^{\bullet},A)\to\Hom_{\CRing}(R,A^0)$$
is injective with image $\{f\in\Hom_{\CRing}(R,A^0) : f\circ\phi=F_A\circ f\textrm{ on }R\}$.
\end{proposition}

Our aim now is to introduce the Cartier isomorphism in order to tackle our original goal. We start by expanding upon the above construction.
\end{frame}

\begin{frame}{The Completed de Rham Complex}
Fix $R\in\CRing$. The \textbf{completed de Rham complex} of $R$ is 
$$\w{\Omega}_R^{\bullet}:=\flim_n\Omega_R^{\bullet}/p^n\Omega_R^{\bullet}\iso\flim_n\Omega_{R/p^nR}^{\bullet},$$
which naturally satisfies $\w{\Omega}_R^{\bullet}/p^n\w{\Omega}_R^{\bullet}\iso\Omega_{R/p^nR}^{\bullet}$ and is $p$-adically complete. The completed de Rham complex admits a unique multiplication such that the tautological map $\Omega_R^{\bullet}\to\w{\Omega}_R^{\bullet}$ is a homomorphism of dga's. Return now to the assumption that $R$ is $p$-torsion-free with $\phi$ a choice of mod $p$ lift of Frobenius (which we encapsulate by saying that $(R,\phi)$ or simply $R$ is \textbf{good}). Then, the morphism $F: \Omega_R^{\bullet}\to\Omega_R^{\bullet}$ induces $F: \w{\Omega}_R^{\bullet}\to\w{\Omega}_R^{\bullet}$ endowing $\w{\Omega}_R^{\bullet}$ with the structure of a Dieudonn\'{e} algebra that has a universal property similar to the one for $\Omega_R^{\bullet}$.

\begin{proposition}
Given $p$-torsion-free and $p$-adically complete $A\in\DA$, the restriction map
$$\Hom_{\DA}(\w{\Omega}_R^{\bullet},A)\to\Hom_{\CRing}(R,A^0)$$
is injective with image $\{f\in\Hom_{\CRing}(R,A^0) : f\circ\phi=F_A\circ f\textrm{ on }R\}$.
\end{proposition}
\end{frame}

\begin{frame}{The Cartier Map}
\begin{proposition}
Let $A\in\CAlg_{\F_p}$. Then, there exists a unique homomorphism of graded algebras $\Cart=\Cart_A: \Omega_A^{\bullet}\to H^{\bullet}(\Omega_A^{\bullet})$, called the \textbf{Cartier map}, such that 
$$\Cart(x)=[x^p],\qquad \Cart(dy)=[y^{p-1}dy]$$
for all $x,y\in A$.
\end{proposition}

This map is necessarily given by 
$$x_0dx_1\wedge\cdots\wedge dx_n\mapsto[x_0^p(x_1^{p-1}dx_1)\wedge\cdots\wedge(x_n^{p-1}dx_n)]$$
for $x_0,x_1,\ldots,x_n\in R$.

\begin{theorem}[Cartier Isomorphism]
Let $k\in\CAlg_{\F_p}$ be perfect and $A\in\CAlg_k$ smooth. Then, $\Cart: \Omega_A^{\bullet}\to H^{\bullet}(\Omega_A^{\bullet})$ is an isomorphism.\footnote{This is a nontrivial result that requires some work done later in the BLM paper.}
\end{theorem}
\end{frame}

\begin{frame}[fragile]{Cartier Pays Off}
What does this have to do with our previous setting? The map $F: \w{\Omega}_R^{\bullet}\to\w{\Omega}_R^{\bullet}$ induces the composition
\begin{center}
\begin{tikzcd}
\Omega_{R/pR}^{\bullet} \arrow[r, "\sim"] & \w{\Omega}_R^{\bullet}/p\w{\Omega}_R^{\bullet} \arrow[r, "F"] & H^{\bullet}(\w{\Omega}_R^{\bullet}/p\w{\Omega}_R^{\bullet}) \arrow[r, "\sim"] & H^{\bullet}(\Omega_{R/pR}^{\bullet})
\end{tikzcd}
\end{center}
One can check that this agrees with the Cartier map and so we see that the composition is independent of the choice of $\phi$.

\begin{corollary}
Let $R\in\CRing$ be good and suppose there exists $k\in\CAlg_{\F_p}$ perfect such that $R/pR$ is a smooth $k$-algebra. Then, $\w{\Omega}_R^{\bullet}$ is Cartier and the canonical map $\w{\Omega}_R^{\bullet}\to\W\Sat(\w{\Omega}_R^{\bullet})$ is a qis.
\end{corollary}

The only non-obvious part of the above result is that $\w{\Omega}_R^{\bullet}$ is $p$-torsion-free, which follows from the fact that each $\w{\Omega}_R^i$ is a projective module of finite rank over the completion $\w{R}=\flim_nR/p^nR$.
\end{frame}

\begin{frame}{The Saturated de Rham-Witt Complex}
\begin{definition}
Let $A\in\DA_{\str}$ and $R\in\CAlg_{\F_p}$. We say that $f\in\Hom_{\CAlg_{\F_p}}(R,A^0/VA^0)$ \textbf{exhibits $A$ as a saturated de Rham-Witt complex of $R$} if, given $B\in\DA_{\str}$, 
$$\Hom_{\DA}(A,B)\xto{\sim}\Hom_{\CAlg_{\F_p}}(R,B^0/VB^0).$$
The pair $(A,f)$ is unique up to unique isomorphism if it exists, so we call $A$ the \textbf{saturated de Rham-Witt complex} of $R$ and write $\W\Omega_R^{\bullet}$.
\end{definition}

\begin{proposition}
Saturated de Rham-Witt complexes exist!
\end{proposition}

We construct $\W\Omega_R^{\bullet}$ by first replacing $R$ by $R^{\red}$ if necessary (which changes nothing) and then taking $\W\Sat(\Omega_{W(R)}^{\bullet})$.
\end{frame}

\begin{frame}{Comparison with a Smooth Lift}
Returning to the completed de Rham complex, we have the following result.

\begin{proposition}
Let $R\in\CRing$ be good and $B\in\DA_{\str}$. Then, there is a natural bijection
$$\Hom_{\DA}(\w{\Omega}_R^{\bullet},B)\xto{\sim}\Hom_{\CAlg_{\F_p}}(R,B^0/VB^0).$$
\end{proposition}

This ties into what we just did assuming we can ``bridge characteristics.''

\begin{theorem}
Let $R\in\CRing$ be good and suppose there exists $k\in\CAlg_{\F_p}$ perfect such that $R/pR$ is a smooth $k$-algebra. Then, there is a canonical morphism $\mu: \w{\Omega}_R^{\bullet}\to\W\Omega_{R/pR}^{\bullet}$ of Dieudonn\'{e} algebras which is a qis.
\end{theorem}
\end{frame}

\begin{frame}[fragile]{Comparison with the de Rham Complex}
Given $R\in\CAlg_{\F_p}$, consider 
$$\W_1\Omega_R^{\bullet}:=\W\Omega_R^{\bullet}/(\im(V)+\im(dV))$$
which comes equipped with a tautological ring map $e: R\to\W\Omega_R^0$ extending uniquely to a map $\nu: \Omega_R^{\bullet}\to\W_1\Omega_R^{\bullet}$ of dga's.

\begin{theorem}
Suppose there exists $k\in\CAlg_{\F_p}$ perfect such that $R$ is a smooth $k$-algebra. Then, $\nu$ is an isomorphism.
\end{theorem}

Under the above setup, the composition
\begin{center}
\begin{tikzcd}
\W\Omega_R^{\bullet} \arrow[r] & \W_1\Omega_R^{\bullet} \arrow[r, "\nu^{-1}"] & \Omega_R^{\bullet}
\end{tikzcd}
\end{center}
is a map of cdga's which induces a qis $\W\Omega_R^{\bullet}/p\W\Omega_R^{\bullet}\to\Omega_R^{\bullet}$.
\end{frame}
\end{document}