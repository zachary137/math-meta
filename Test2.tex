\documentclass[11pt]{article}

\usepackage{kernel_of_truth}
\usepackage{normal_setup}

\newcommand{\DA}{\mathsf{DA}} % Dieudonn\'{e} algebra(s)
\newcommand{\DC}{\mathsf{DC}} % Dieudonn\'{e} complex(es)
\newcommand{\free}{\textup{free}} % free
\newcommand{\Isom}{\textup{Isom}} % isomorphism(s)
\newcommand{\pf}{\textup{pro-free}} % free
\newcommand{\sat}{\textup{sat}} % saturated
\newcommand{\Sat}{\textup{Sat}} % saturation
\newcommand{\str}{\textup{str}} % strict
\newcommand{\W}{\mathcal{W}}

\renewcommand{\L}{\mathbb{L}}
\renewcommand{\phi}{\varphi}

\begin{document}
\section{Introduction}
Our goal is to describe different types of Dieudonn\'{e} complexes in terms of fixed points. This is relatively easy to accomplish for saturated complexes but requires a bit more work for strict complexes. As a reminder, recall that $M\in\DC$ is saturated if it is $p$-torsion-free and 
$$\alpha_F: M\to\eta_pM,\qquad x\mapsto p^nF(x)$$
is an isomorphism of Dieudonn\'{e} complexes (where $x\in M^n$). Note that the data of a map of complexes $M\to\eta_pM$ is equivalent to a choice of Frobenius on $M$ making it into a Dieudonn\'{e} complex. In more detail, $\alpha: M\to\eta_pM$ induces
$$F_{\alpha}: M\to M,\qquad x\mapsto p^{-n}\alpha(x)$$
a map of graded abelian groups (where $x\in M^n$). Recall also that $M\in\DC_{\sat}$ is strict if the canonical map $\rho_F: M\to\W M$ is an isomorphism, noting that $\W M$ is always strict. 

\section{D\'{e}calage}
As we already know, the d\'{e}calage process determines an endofunctor $\eta_p: \Ch(\Z)^{\tf}\to\Ch(\Z)^{\tf}$. One of the key properties of d\'{e}calage is that it kills off $p$-torsion in cohomology -- given $M\in\Ch(\Z)^{\tf}$, there is a canonical isomorphism 
$$H^{\bullet}(M)/H^{\bullet}(M)[p]\xto{\sim}H^{\bullet}(\eta_pM)$$
of graded abelian groups.\footnote{We can upgrade this to an isomorphism of complexes if the RHS is equipped with the differential induced by the Bockstein operator.} Hence, $\eta_p$ sends qis's to qis's and we obtain the following result.

\begin{proposition}
There is an essentially unique functor $L\eta_p: D(\Z)\to D(\Z)$ such that
\begin{center}
\begin{tikzcd}
\Ch(\Z)^{\tf} \arrow[r, "\eta_p"] \arrow[d] & \Ch(\Z)^{\tf} \arrow[d] \\
D(\Z) \arrow[r, dotted, "\exists!\; L\eta_p"'] & D(\Z)
\end{tikzcd}
\end{center}
commutes up to natural isomorphism.
\end{proposition}

Thinking of $D(\Z)$ as $\Ch(\Z)^{\tf}$ with qis's inverted, we obtain $L\eta_pX$ for $X\in D(\Z)$ by choosing a representative for $X$ in $\Ch(\Z)^{\tf}$, applying $\eta_p$, and taking the corresponding qis class in $D(\Z)$.\footnote{By using the term `class' here I don't mean to suggest that we are performing some kind of quotient process. Instead, I mean that the result is well-defined up to qis (which is isomorphism in $D(\Z)$). In particular, passing to the skeleton of $D(\Z)$ gives something unique (I think).} Similar comments apply if one were to choose a different model of $D(\Z)$, in particular the homotopical model $h\Ch(\Z)^{\free}$.

\section{Completion}
\begin{definition}
\textbf{Classical $p$-completion} is the functor
$$\w{\cdot}: \Mod_{\Z}\to\Mod_{\Z},\qquad X\mapsto\flim_{n\geq1}X/p^nX.$$
We say $X\in\Mod_{\Z}$ is \textbf{classically $p$-complete} if the natural map $X\to\w{X}$ is an isomorphism. On a somewhat related note, $X\in D(\Z)$ is \textbf{derived $p$-complete} if $\Hom_{D(\Z)}(Y,X)=0$ for every $Y\in D(\Z)$ such that $p: Y\xto{\sim}Y$. Such objects span a full subcategory $D_p(\Z)\subset D(\Z)$.
\end{definition}

\begin{proposition}
The inclusion $D_p(\Z)\inj D(\Z)$ admits a left adjoint $\w{\cdot}: D(\Z)\to D_p(\Z)$ called the \textbf{derived $p$-completion} given by choosing a representative in $\Ch(\Z)^{\tf}$ and applying classical $p$-completion in each degree.\footnote{Part of the content of this result is that the choice of representative does not matter (up to qis). In particular, $D_p(\Z)$ is invariant under this process.}
\end{proposition}

In line with the above, we extend derived notions to $\Ch(\Z)$ by passing to qis classes. This in turn allows us to extend derived notions to $\Mod_{\Z}$ by thinking of abelian groups as complexes concentrated in degree $0$. Given $X\in\Mod_{\Z}$, the classical $p$-completion of $X$ represents the derived $p$-completion of $X$ and so we may identify the two. In this simple case, to check that $X$ is derived $p$-complete we need only verify that $\Hom_{D(\Z)}(\Z[p^{-1}],X)=0$.

\begin{proposition}
Let $X\in\Mod_{\Z}$. Then, $X$ is \textbf{pro-free} (i.e., the $p$-completion of a free abelian group) if and only if it is derived $p$-complete and $p$-torsion-free.
\end{proposition}

Complexes of pro-free abelian groups span a full subcategory $\Ch(\Z)^{\pf}\subset\Ch(\Z)$. This category is clearly linked to $D_p(\Z)$ by the above, and in fact the connection is strong.

\begin{theorem}
The functor $\Ch(\Z)^{\pf}\to D(\Z)$ obtained by passing to qis classes has essential image $D_p(\Z)$ and induces an equivalence $h\Ch(\Z)^{\pf}\xto{\sim}D_p(\Z)$. In more detail, given $X,Y\in\Ch(\Z)^{\pf}$, $\Hom_{\Ch(\Z)}(X,Y)\surj\Hom_{D(\Z)}(X,Y)$ and $f,g\in\Hom_{\Ch(\Z)}(X,Y)$ have the same image if and only if $f\simeq g$.
\end{theorem}

Before discussing fixed points, we mention two supplementary results that will be important soon. The first result concerns compatibility of d\'{e}calage and $p$-completion.

\begin{proposition}
Suppose that $M\to N$ in $D(\Z)$ exhibits $N$ as a derived $p$-completion of $M$. Then, the induced map $L\eta_pM\to L\eta_pN$ exhibits $L\eta_pN$ as a derived $p$-completion of $L\eta_pM$. Hence, $L\eta_p$ restricts to an endofunctor of $D_p(\Z)$.
\end{proposition}

The second result concerns completion of Dieudonn\'{e} complexes. 

\begin{proposition}
Given $M\in\DC_{\sat}$, the canonical map $\rho_F: M\to\W M$ exhibits $\W M$ as a derived $p$-completion of $M$. Moreover, $\rho_F$ is a qis if and only if $M$ is derived $p$-complete.
\end{proposition}

\section{Fixed Points}
\begin{definition}
Let $\mc{C}$ be a category and $T: \mc{C}\to\mc{C}$ an endofunctor. The \textbf{fixed point} category $\mc{C}^T$ of $\mc{C}$ with respect to $T$ is the category whose objects are pairs $(X,\phi)$ with $X\in\mc{C}$ and $\phi\in\Isom_{\mc{C}}(X,TX)$. The data of a morphism $f: (X,\phi)\to(X',\phi')$ is $f\in\Hom_{\mc{C}}(X,X')$ such that
\begin{center}
\begin{tikzcd}
X \arrow[r, "f"] \arrow[d, "\phi"'] & X' \arrow[d, "\phi'"] \\
TX \arrow[r, "Tf"'] & TX'
\end{tikzcd}
\end{center}
commutes.
\end{definition}

\begin{remark}
Let $(\mc{C},T)$ and $(\mc{C}',T')$ be categories equipped with endofunctors that are intertwined in the sense that there is a functor $\mc{C}\to\mc{C}'$ intertwining $T$ and $T'$ up to specified natural isomorphism. Then, there is a natural induced functor $\mc{C}^T\to(\mc{C}')^{T'}$.
\end{remark}

Basically by definition, we immediately see that there is an equivalence 
$$\DC_{\sat}\xto{\sim}(\Ch(\Z)^{\tf})^{\eta_p},\qquad (M,F)\mapsto(M,\alpha_F).$$ 
Because of the earlier commutative diagram for d\'{e}calage, we obtain a functor $\theta$ via
$$\DC_{\str}\inj\DC_{\sat}\xto{\sim}(\Ch(\Z)^{\tf})^{\eta_p}\to D(\Z)^{L\eta_p}.$$

\begin{theorem}
The composite functor $\theta: \DC_{\str}\to D(\Z)^{L\eta_p}$ induces an equivalence $\DC_{\str}\xto{\sim}D_p(\Z)^{L\eta_p}$.
\end{theorem}

We begin by showing that the essential image of $\theta$ is $D_p(\Z)^{L\eta_p}$. To that end, choose an object of $D_p(\Z)^{L\eta_p}$. On the level of representatives, this amounts to choosing $X\in\Ch(\Z)^{\tf}$ and a qis $\alpha: X\to\eta_pX$. Using $\alpha$, we endow $X$ with the structure of a Dieudonn\'{e} module as discussed earlier. Each of the arrows in the diagram 
\begin{center}
\begin{tikzcd}
X \arrow[r, "\alpha"] & \eta_pX \arrow[r, "\eta_p\alpha"] & \eta_p^2X \arrow[r, "\eta_p^2\alpha"] & \cdots
\end{tikzcd}
\end{center}
is a qis and so the induced map $X\to\Sat(X)$ is a qis. Since $X$ is derived $p$-complete, $\Sat(X)$ is derived $p$-complete and so the canonical map $\Sat(X)\to\W\Sat(X)$ is a qis. Hence, the completed saturation map $X\to\W\Sat(X)$ is a qis. This fits into a commutative diagram
\begin{center}
\begin{tikzcd}
X \arrow[r] \arrow[d, "\alpha"'] & \W\Sat(X) \arrow[d, "\W\Sat(\alpha)"] \\
\eta_pX \arrow[r] & \eta_p\W\Sat(X)
\end{tikzcd}
\end{center}
The right vertical map is an isomorphism since $\W\Sat(X)$ is saturated. It follows that $(X,\alpha)$ and $(\W\Sat(X),\W\Sat(\alpha))$ represent isomorphic objects in $D_p(\Z)^{L\eta_p}$ and so $(X,\alpha)$ lies in the essential image of $\W\Sat(X)\in\DC_{\str}$ under $\theta$. To finish seeing that the essential image of $\theta$ is $D_p(\Z)^{L\eta_p}$, note that $X\in\DC_{\str}$ satisfies $X\iso\W X$ and the latter is derived $p$-complete (which means $\theta$ factors through $D_p(\Z)^{L\eta_p}$). 

Our aim now is to show that $\theta$ is fully faithful. To that end, choose $X,Y\in\DC_{\str}$ (from which we get that $X,Y$ are both pro-free by earlier comments) and consider the natural map
$$\Theta_{X,Y}: \Hom_{\DC_{\str}}(X,Y)\to\Hom_{D_p(\Z)^{L\eta_p}}(X,Y)$$
induced by $\theta$. Recall from earlier that we have an equivalence $h\Ch(\Z)^{\pf}\xto{\sim}D_p(\Z)$ obtained by passing to qis classes. We wish to understand what $\Hom_{D_p(\Z)^{L\eta_p}}(X,Y)$ looks like under this equivalence. With this in mind, we introduce the following definition.

\begin{definition}
Suppose $X,Y\in\Ch(\Z)^{\tf}$ are equipped with the structure of Dieudonn\'{e} modules and $f\in\Hom_{\Ch(\Z)}(X,Y)$. We say that $f$ is \textbf{weakly $F$-compatible} if the diagram
\begin{center}
\begin{tikzcd}
X \arrow[r, "f"] \arrow[d, "\alpha_F"'] & Y \arrow[d, "\alpha_F"] \\
\eta_pX \arrow[r, "\eta_p(f)"'] & \eta_pY
\end{tikzcd}
\end{center}
commutes up to homotopy. In the case that $Y$ is saturated this is the same as requiring that 
$$\alpha_F^{-1}\circ\eta_pf\circ\alpha_F=F^{-1}\circ f\circ F\simeq f$$
as maps of complexes. This notion clearly extends to homotopy classes of maps in $[X,Y]$.
\end{definition}

It follows that the (functorial) bijection $\Hom_{D_p(\Z)}(X,Y)\longleftrightarrow[X,Y]$ induces a (functorial) bijection
$$\Hom_{D_p(\Z)^{L\eta_p}}(X,Y)\longleftrightarrow\{f\in[X,Y] : f\textrm{ is weakly }F\textrm{-compatible}\}.$$
The matter of whether $\Theta_{X,Y}$ is bijective therefore boils down to the following lemma.

\begin{lemma}
Let $X,Y\in\Ch(\Z)^{\tf}$ equipped with the structure of Dieudonn\'{e} modules such that $Y$ is strict. Let $f\in\Hom_{\Ch(\Z)}(X,Y)$ be weakly $F$-compatible. Then, there exists a unique natural choice of $\twid{f}\in\Hom_{\DC}(X,Y)$ such that $\twid{f}\simeq f$.
\end{lemma}

\begin{proof}
We first prove existence. By hypothesis there exists a map $h: X^{\bullet}\to Y^{\bullet-1}$ of graded abelian groups such that $F^{-1}\circ f\circ F=f+dh+hd$. We seek a homotopy $u: X^{\bullet}\to Y^{\bullet-1}$ such that taking $\twid{f}:=f+du+ud$ gives $F^{-1}\circ\twid{f}\circ F=\twid{f}$. No matter how we choose $u$, the identity $FdV=d$ gives
\begin{align*}
F^{-1}\circ(du+ud)\circ F
=d(VuF)+(VuF)d
\end{align*}
and so 
\begin{align*}
F^{-1}\circ\twid{f}\circ F
&=F^{-1}\circ f\circ F+F^{-1}\circ\twid{f}\circ F \\
&=f+dh+hd+d(VuF)+(VuF)d \\
&=f+d(h+VuF)+(h+VuF)d.
\end{align*}
Thus, the condition we want is $u=h+VuF$ and so we take
$$u:=\sum_{r\geq0}V^ruF^r.$$
Now to prove uniqueness. Let $g\in\Hom_{\DC}(X,Y)$ such that $g\simeq0$, so $g=dh+hd$ for some homotopy $h: X^{\bullet}\to Y^{\bullet-1}$. Given $r\geq0$,
\begin{align*}
g
&=F^{-r}\circ g\circ F^r \\
&=F^{-r}(dh+hd)\circ F^r \\
&=d(V^rhF^r)+V^r(hF^rd).
\end{align*}
Hence, the composition
\begin{center}
\begin{tikzcd}
X \arrow[r, "g"] & Y \arrow[r] & \W_rY
\end{tikzcd}
\end{center}
vanishes and so $g$ vanishes since $Y$ is strict.
\end{proof}
\end{document}